\documentclass[11pt,a4paper]{article}
\usepackage{a4wide,url,graphicx}
\usepackage[utf8]{inputenc}
\usepackage[english]{babel}

\newcommand{\ccnotice}{\vfill
  \begin{minipage}{.22\textwidth}\centering
    \includegraphics[width=\textwidth]{by.png}
  \end{minipage}\hfill\begin{minipage}{.77\textwidth}
  This text can be reused under the terms of the Creative Commons CC-BY
  License \url{https://creativecommons.org/licenses/by/3.0}.
  \end{minipage}
}

\parindent0pt
\parskip4pt
\title{Handout for a Seminar Paper \\[4pt] on the topic \emph{Patent Search}}

\author{Hans-Gert Gr\"abe, Leipzig}
\date{July 24, 2019}

\begin{document}
\maketitle

\section{General}

In the seminar paper on the topic \emph{patent search} five selected patent
documents are to be analysed to trace Altschuller's way to his basic
structural statements about the creative process in the engineering invention
process.

The core of the approach is the transition from a \emph{special problem} to a
\emph{special solution} (as described in the patent) as an abstraction
process, that
\begin{itemize}
\item assigns the specific problem to a general problem class,
\item suggests solutions to the general problem from analogy considerations   
\item and finally breaks down these solutions to a special solution of the
  special problem.
\end{itemize}
In contrast to Altshuller's beginnings, today exist elaborated,
experience-based TRIZ tools and concepts to propose solutions to the general
problem.

The aim of the analysis of the patent specification is to determine the
suitability of this general principles on the context of the specific solution
of the specific problem described in the patent claim. For this, first the
specific problem and the specific solution are to be extracted in sufficiently
general and meaningful terms.  Based on such a description the specific
solution process has to be related to appropriate more general TRIZ
principles, wherein the elaboration of a \emph{main contradiction} in the
specific problem definition has a guiding function.

Such a contradiction always exists only \emph{relative to the commomly
  accepted state of the art}\footnote{German: „Stand der Technik“.}, since it
is the goal of the various TRIZ principles, to resolve such contradictions and
thus to define a \emph{new state of the art}, in which the solution strategies
for \emph{this} contradiction have become commonplace. So it is important to
describe the respective commonly accepted state of the art, against which the
further argumentation is performes. This \emph{commonly accepted state of the
  art} assumed to be known to a „skilled person“ is always a temporal relative
phenomenon. On the contrary, TRIZ studies the historic development of ideas.

In some cases, such a major contradiction may be difficult identify. Here it
would be necessary to investigate if the patented solution is a \emph{standard
  solution} by a non-obvious parameter optimization is. This is often an
indication that the modeling of the context goes too far and the resolution of
the contradiction is already subsumed within the commonly accepted state of
the art.

Therefore to process the patent analysis the following steps required:
\begin{itemize}
\item [1.] Extract the patent metadata.
  
\item [2.] Describe the commonly accepted state of the art with respect to the
  problem setting.

  A patent must always have an \emph{inventiveness} level compared to the
  commonly accepted state of the art, that is not obvious to a skilled
  person. The patent claim to be described is understandable only in such a
  context of previous technical developments, that are known to a skilled
  person, but not to the readers of the seminar work. This context of
  historical development of ideas (that is important also to recognize for the
  TRIZ analysis) must therefore first \emph{sufficiently accurate} to be
  described.

\item [3.] Description of the functional model.

  After marking this context, it is necessary to explain the concepts in more
  detail that are relevant to describe the problem and solution presented in
  the patent claim.

  For this purpose, first the system has to be delimited within a super
  system.  Often a system fulfills a specific function within such a super
  system that needs to be worked out at this point.
  
  For this, the \emph{Innovation Checklist} [KS: Chap. 5.1] may be a good
  guidance. Furthermore, it can be useful to identify
  \begin{itemize}
  \item functions -- main function(s), secondary functions [KS: Chap. 4.1.]
    and
  \item resources -- type, properties, behavior [KS: Chap. 4.2]
  \end{itemize}
  within the system.

\item [4.] Formulation of the \emph{miniproblem}: elaboration of the basic
  problem as \emph{special problem} formulated as contradiction.

\item [5.] Description of the solution to the problem in the patent claim as
  \emph{special solution}.

\item [6.] Classification of the problem in the TRIZ classification as
  \emph{general problem}.

\item [7.] Description of the relation of the special solution to the general
  TRIZ structures.
\end{itemize}

This methodology is prototypically described in the following example.

\section{Example: Patent EP0066122B1}

\subsection{Metadata}
\begin{itemize} \itemsep0pt
\item Title: Differential gearings
\item Source: \url{https://patents.google.com/patent/EP0066122B1}
\item Patentee: Theodoros Tsiriggakis
\item Patent Data: publication 1982-12-08, notice of the Patent Grant
  1985-05-15
\item Classification by Google:
  \begin{itemize}
  \item F16 H 48/147 Differential gearings without gears with driven cam
    followers or balls engaging two opposite cams
  \end{itemize}
\end{itemize}

\subsection{Description of the Commonly Accepted State of the Art}

When a vehicle drives through a bend, the outer wheels put one further way
than the inner ones and therefore have to turn faster.  For this purpose, the
rotational speed of the drive axle has to be adjusted during transfer to the
driven shafts with different, adapted to the particular driving situation
ratios.

This is usually done with a differential gearing on the basis of gears. The
two driven shafts sit on a common axle, sprockets are mounted and
interconnected through one or more pinions. These pinions convey the
differential speed between both shafts. The drive energy can be conducted
directly to one of the two shafts or via a crown wheel in various ways into
the gear design, e.g., -- as comparable to this patent -- by fixing the pinion
with the crown wheel. In this case, the pinion not only mediates the speed
compensation, but also the power transmission. A more detailed explanations
can be found in the Wikipedia [DG].

The class F16H48 of the [IPC] collects patents with different versions of such
differential gearings. In a class of differential gearings instead of a
pinions rolling elements are used to mediate the speed compensation between
two plates that are mounted on the driven shafts. This principle was patented
in 1932 by Andrew Francis Ford (patent US1897555) and is in the present patent
claim referred to as a „class-forming state of technology“, that is also
referred as patent class F16 H 48/147 in the [IPC]. So it can be assumed that
at the time of filing the patent this basic type of execution of a
differential gear belonged to the cmmonly accepted state of the art.

As previous problems mainly friction losses of the rolling elements are
listed, that lead to restless running and rapid wear. Furthermore a
conventional differential gearing has the problem that on black ice one wheel
spins may run at double speed while the other does not move.

\subsection{Functional Model}

\paragraph{Super System.}
Submission of rotational energy via the drive shaft, transmission of
rotational energy on the two driven shafts adjusting their individual speeds
to drive the wheels accordingly.

\paragraph{Components.}
\begin{itemize}
\item \emph{Cam track} -- concentric with the axis of the cam track plate
  implemented guiding structure for the rolling elements on the cam track
  plate, used to control the movement of the rolling elements. A rolling
  element is guided on the corresponding cam tracks of the opposite plates and
  rotates with the differential speed of the driven wheels.
\item \emph{Cam plate} -- transmission unit of movement on one of the driven
  shafts. A splined shaft permits the cam track plates to move in the
  direction of the system axis.  The required back-adjusting force by a
  spring, that the cam plate presses against the rolling elements, is not
  listed and probably silently subsumed as part of the commonly accepted state
  of the art.
\item \emph{Powered shafts} -- shafts connected to the wheels, mounted on a
  common axis, the \emph{system axis}.
\item \emph{Rolling elements} -- an in this patent complex transmission
  element for speed adaptation and simultaneous power transmission from the
  crown wheel to the cam track plates and further to each of the two driven
  shafts.
\item \emph{System axis} -- common axis of rotation of the cam track plates
  and the crown wheel.
\item \emph{Transmission element}, \emph{carrier} (also cage or basket) --
  submission of the driving power into the system via the driving shaft,
  relative to which the speed of the two driven shafts must be adjusted. In
  the patent claim specified as \emph{crown wheel}, which is parallel to the
  cam track plates, so that a fixed connection between the crown gear and cam
  track plates would lead to a uniform rotation of the driven axles.
\end{itemize}

\paragraph{How the system operates.}
From the driving shaft, the rotation is rigidly transmitted on the crown wheel
and thus on the structure of the rolling elements fixed within the crown
wheel, such that the main rotation is transferred via the cam structures on
the two cam track plates and at the same time the corresponding self-rotation
of the rolling elements compensates the velocity difference of the driven
shafts according to the registered speed difference.

The rolling elements are fixed in „cage structures“ on the crown wheel such
that they move rigidly with the crown wheel and both the power transfer and
also the speed compensation to the cam track plates are conveyed by limited
movements relative to the crown wheel. The rolling elements can move
perpendicular to the direction of rotation of the crown wheel as well as turn
around its own axis.
  
\subsection{Formulation of the Miniproblem}

For good grip (power transmission), the rolling elements must be as strong as
possible connected with the cam track plates, for the compensation of the
speeds they must roll freely on the cam tracks.

The rolling elements should turn (for speed compensation) and should not turn
(for power transmission) at the same time.

In the pinion solution spinning is in the foreground, which is the reason,
why under extreme operating conditions a wheel spins at double speed whereas
the other one stands still.

\subsection{Description of the Solution to the Problem in the Patent
  Claim} 

By a sinusoidal design of the cam tracks -- in the minima the grip is
particularly high, since the potential energy is minimized, in the maxima, the
rolling motion is particularly favorable, since there is no potential
resistance to be overcome -- positions with high power transmission alternate
with positions with low rolling resistance.

Twisting the two cam tracks against each other achieves that the rolling
elements on one track are in the minimum and thus give maximum grip when the
rolling elements on the other track go through the maximum.

Rolling elements carry both straight-line movements (perpendicular to the
plane of rotation, according to the state on the cam track) as well as
rotational movements. The vertical displacement movements are triggered by the
structure of the two cam tracks, the rotational movements by that of the speed
differential triggered by the driven axes.  Both movements show different
friction behavior. A layered structure of the rolling element separates these
movements -- in the first layer, the „linear part“ is provided along the
system axis in a V-shaped groove, and in the second layer the „rotational
part“ is mounted to a conical recess. The total movement is thus separated
into these two movements to realize for each movement the most advantageous
structural form.

The cam track plates are symmetrical to each other, that enables a
particularly simple production.

The cam track plates transmit the same torque in the basic position of the
axles and slip in at a difference in the speeds repeatedly in this basic
position, so that the above-described „black ice effect“ does not occur.

\subsection{Classification of the Problem in the TRIZ Systematics}

The solution does not fit directly into \emph{Altshuller's contradiction
  matrix}, because here the power transmission and the rolling resistance
stand in conflict with each other.

In the \emph{contradiction matrix 2003} the problem can only be roughly
classified as conflict between power and speed, of the four proposed
principles only the principle of dynamization is applicable.

Overall, the solution has more the character of a further optimization of a
principally already known solution (design of the cam track, more accurate
structure the rolling elements).

\subsection{Relation of the Special Solution to the General TRIZ Structures}

The solution uses dynamization as well as spatial separation on both the cam
track as well as in the structure of the rolling elements.

Due to the sinusoidal design (odp:P19, principle of the periodic effect) areas
with good grip (minima) of rolling elements change with areas of better role
property (odp:P15, principle of dynamization).

Due to the constructional division of the rolling elements their complex
movement is separated into a linear and a rotational motion, to enable optimal
constructive solutions (odp:P01, principle of decomposition).

With the division into two cam tracks with offset by $45^\circ$ (odp:P01,
principle of decomposition) is also achieved that one of the rolling elements
is always in the area with good grip and thus sufficient power transmission on
\emph{both} driven shafts is always guaranteed.

Already integrated in the commonly accepted state of the art is the
concentration of two functions (speed compensation and power transmission) in
the rolling elements (odp:P06, principle of universality), the cage structure,
about the crown wheel and rolling elements are connected to each other
(odp:P07, principle of nesting), as well as the rotationally symmetrical
design of the rolling elements (odp:P14, principle of the ball similarity) as
well as the use of rolling elements (odp:P24, principle of the mediator).

\section{Literature}
\raggedright
\begin{itemize}
\item{[DG]} Wikipedia entry „Differentialgetriebe“ (differential
  transmission).  \url{https://de.wikipedia.org/wiki/Differentialgetriebe}
  (20.07.2019).
\item{[IPC]} International Patent Classification.
  \url{https://www.wipo.int/classifications/ipc/ipcpub} (20.07.2019).
\item{[KS]} Karl Koltze, Valeri Souchkov: Systematische Innovation (Systematic
  Innovation). 2nd, revised edition Hanser Verlag, Munich 2017.
\item{[PS]} Patent EP0066122B1, published on 15.05.1985 by the
  European Patent Office.
\item{[ZH]} Dietmar Zobel, Rainer Hartmann: Erfindungsmuster (Invention
  Pattern). 2nd, revised edition, Expert Verlag, Renningen 2016.
\end{itemize}

\vfill
\ccnotice
\end{document}
