\documentclass[11pt,a4paper]{article}
\usepackage[ngerman]{babel}
\usepackage{ls}
\usepackage[utf8]{inputenc}

\title{Handout zu Goldovsky (2017)}
\author{Tarek Stelzle}
\date{15. Dezember 2020}

\newcommand\descitem[1]{\item{\bfseries #1}}

\begin{document}
\maketitle
%\tableofcontents
%\newpage

\section{Einleitung}
Im folgenden wird auf die übersetzte Ausarbeitung \emph{Über die Gesetze der
  Konstruktion technischer Systeme} von B.I. Goldovsky aus dem Jahr 2017
eingegangen.  Hierbei werden die interessantesten und wichtigsten Aspekte
aufgezeigt.

\section{Frührere Arbeit des Autors}
Der Autor beschäftigte sich mit dem Thema „Entwicklung eines Systems von
Gesetzmäßig\-keiten der Konstruktion und Entwicklung technischer Systeme“.
Dabei wollte er das Thema genauer verstehen.  Aus dieser Arbeit gingen drei
Aussagen hervor.

\begin{enumerate}[noitemsep]
\item System der Gesetzmäßgikeit ist komplexer als die Einteilung von
  G.S. Altshuller (\ref{Altshuller_liste}).
\item Teile der Gesetzmäßgikeit lassen sich deduktiv begründen.
\item Die Gesetze der Konstruktion technischer Systeme, welche die
  Arbeitssicherheit treffen, sind hervorzuheben.
\end{enumerate}

\subsection{Anmerkungen zur Klassifizierung der Gesetze}
In dem Artikel weist B.I. Goldovsky auf die folgenden Klassifizierungen der
Entwichlichung technischer Systeme hin.

Von G.S. Altshuller \label{Altshuller_liste} werden die Gesetze in die drei
Bereiche „Statik“, „Kinematik“ und „Dynamik“ unterteilt.

Eine natürliche Klassifizierung wurde von V.M. Petrov vorgenommen.  Dieser
fasste die Gesete der „Kinematik“ und „Dynamik“ zu den „Evolutionsgesetzen“
zusammen.  Die Gesetze der „Statik“ wurden zu den „Organistations-Gesetzen“
(Gesetze der Konstruktion) umformuliert.

Weiterhin bezeichnete N.A. Shpakowski die Gesetze der Konstruktion als
zentrale Gesetze.  Die Evolutionsgesetze sind dabei nur unterstützend.

\section{Besonderheiten der Gesetze der Konstruktion technischer Systeme}

Die Unterteilung in zentrale und unterstützende Gesetze unterstützt auch
Goldovsky.  Er erläutert, dass die Gesetze der Konstruktion bei
Nichteinhaltung bedingungslose Folgen haben, welche zum Stillstand des
technischen Systems führen.  Andererseits schlägt bei Missachtung der
Entwicklungsgesetze die Entwicklung und Ausführung der technischen Systeme
eine andere Bahn ein.  Das technische Systeme würde, wenn auch ineffizienter,
dennoch funktionieren.

\subsection{Unterteilung der Gesetze der Konstruktion (PNF, ENF)}

Der wichtigeste Faktor eines technischen System ist das Bedürfnis der
Allgemeinheit.  Dieses wird in den primär nützlichen Funktionen (PNF)
formuliert.  In der zweiten Ebene sind die elementar nützlichen Funktionen
(ENF).  Diese bedingen die Ausführbarkeit der PNF.

Für die Umsetzung der ENF müssen verschieden Teilsysteme vorhanden sein.  In
anderen Worten muss die \textbf{funktionale Vollständigkeit des technischen
  Systems} gewährleistet sein.

\subsection{Gesetz der strukturellen Vollständigkeit eines technischen
  Systems} 

\begin{quote}
\textbf{Die Gesamtheit der Elemente der Struktur und die Wechselwirkungen
  zwischen ihnen müssen den Durchsatz der natürlichen Flüsse (Stoff, Energie
  und/oder Information) zu den notwendigen Teilen des Systems sichern sowie
  eine solche Umwandlung dieser Ströme, dass alle elementaren Funktionen des
  Systems erfüllt werden.}
\end{quote}

Diese Definition bezieht sich auf dynamische Systeme, welche für den Menschen
notwendige Prozesse erfüllen.  Im Gegensatz dazu existieren auch statische
Systeme, auch \emph{Anlagen} genannt.  Zur besseren Einteilung werden zur
Unterscheidung wesentliche oder unwesentliche dynamischische Prozesse
verwendet.  Dennoch sind auch in Anlagen Flüsse zu erkennen.  Demnach kann der
Fluss-Ansatz und deswegen auch der Begriff der strukturellen Vollständigkeit
auf Anlagen angewendet werden.

\section{Wirkprinzipien}

Die Wirkprinzipien werden im Artikel als die natürlichen Prozesse, Effekte und
Erscheinungen der Teilsysteme bezeichnet, welche die Ausführung der nützlichen
Systemfunktion sichern.  Da die Parameter für ein System die Funktion deutlich
ändern können, kann man eine funktionale Nische in funktionell-parametrische
Nischen unterteilen.  Als Beispiel werden das System Energiespeicher genannt.
Ein Untersystem davon wäre das Speichern von Energie.  Je nachdem, wie viel
Energie gespeichert wird, ändert sich die Funktion des Systems.

\subsection{Erweiterung der PNF um Wirkprinzipien}

Da die Parameter, wie oben beschrieben, eine Auswirkung auf die Funktionaliät
des Systems haben, wird zusätzlich zu der PNF nun das zentrale Wirkprinizip
mit angegeben.

\section{Verallgemeinerung der Bedinungen für die Anwendung eines technischen
  Systems}

Diese vier Punkte beschreiben die Bedinungen, die gewährleistet sein müssen,
damit ein technisches System als Anwendung für die Allgemeinheit funktioniert.

\begin{enumerate}[noitemsep]
\item Die PNF muss qualitativ und qunatitaiv den Anfordungen der Allgemeinheit
  und/oder des technischen Umfelds entsprechen.
\item Gewährleistung der Stabilität des Funktionierens.
\item Gewährleistung des erforderlichen Grads der Steuerbarkeit.
\item Gewährleistung der Benutzerfreundlichkeit des technischen Systems.
\end{enumerate}

\section{Entwicklung technischer Systeme}

\subsection{Weiterentwicklung}

Hierbei werden die quantitativen Parameter im Arbeitszustand festgelegt.

\subsection{Erstentwicklung}

Bei der Erstentwicklung eines technischen Systems sollten die
\textbf{parametrischen Schwellenwerte} überwunden werden.  Eine physikalische
Schwelle sind Werte, die zum Funktionieren des technischen Systems überwunden
werden müssen.  Funktionelle paramterische Schwellen nennt man das Niveau,
welches überschritten werden muss, damit ein technisches System nicht mehr als
Prototyp gewertet wird.

\subsection{Entwicklungstsstufen}

Damit ein technisches System arbeitsfähig ist, ist eine Abstimmung der
Struktur erforderlich.  Die Abstimmmung eines technischen Systems an seine
Struktur kann man in drei Etappen einteilen.

\begin{enumerate}
\item \textbf{Anfangsetappe}\\ In dieser Stufe soll die Arbeitsfähigkeit des
  technischen Systems hergestellt werden.
\item \textbf{Abstimmung}
\begin{enumerate}
\item \textbf{Schwellenabstimmung (Konjugation).} Diese Phase ist
  abgeschlossen, sobald das technische System als betriebsfähig erklärt
  ist. Hierbei werden Formeln mit „nicht weniger als“ und „nicht mehr als“
  verwendet.
\item \textbf{Optimierungsabstimmung.} In dieser Etappe werden die
  quantitativen Werte des technischen System abgestimmt. Dies kann sich über
  den kompletten restlichen Lebenszyklus erstrecken. Zur Optimierung werden
  Gleichungen verwendet.
\end{enumerate}
\end{enumerate}

\section{Konformität von Struktur und Funktion}

Dies ist das Ergebnis der vorher erwähnten Konjugation.  In anderen Artikeln
ist es unter dem Begriff „Entsprechung von Funktion und Struktur“ zu finden.
Das Gesetz legt die Übereinstimmung der Funktion des technischen Systems mit
der Struktur fest.  Der Nachweis dazu führt aber zu Widersprüchen, da die
Synthese einerseits kein trivaler Vorgang ist und deswegen das Ergebnis nicht
eindeutig bestimmt werden kann.  Andererseits ist diese Zusammensetzung ein
analytisches Verfahren, was zu einem eindeutigen Ergebnis führen muss.

\subsection{Entsprechung zwischen der Komplexität der Funktion und der
  Struktur}

Dies erklärt, dass bei Anstieg der Komplexität des technischen Systems auch
die Komplexität der Steuerung ansteigen muss.

\subsection{Gesetz der Aufrechterhaltung}

Als Schlussfolgerung aus dem vorherigen Absatz ist erkenntlich, dass eine
Vereinfachung des Systems nicht willkürlich vorgenommen werden kann.  Eine
Vereinfachung kann mittels drei verschiedener Ansätze erreicht werden.

\begin{enumerate}[noitemsep]
\item Vereinfachung der Funktion des Systems.
\item Übertragung der Komplexität in die Teilsysteme (funktional-ideales
  Trimmen).
\item Übertragen der Komplexität auf die Mikroebene (Änderung des
  Funktionsprinzipes der Teilsysteme).
\end{enumerate}

\subsection{Welle der Idealität}

Die Verbesserung der Idealität eines Systems geht einerher mit einem
komplizierteren System.  Ab einem Punkt ist das System so komplex, dass eine
Reduktion der Zuverlässigkeit des Funktionierens voranschreitet.  Das Phänomen
„Welle der Idealität“ beschreibt, dass dadurch eine Vereinfachung des Systems
folgt.

\section{Steuerbarkeit des Systems}

Weiterhin muss bei der Konjugation eines technischen Systems auf die
Steuerbarkeit des Systems geachtet werden.  Nur dynamische Systeme sind
steuerbar.  Dennoch muss eine Steuerung nur gegeben sein, wenn folgende Punkte
erfüllt sind:

\begin{enumerate}[noitemsep]
\item Einige Zustände des Systems sind dynamisch.
\item Notwendigkeit, das System in bestimmte Zustände zu versetzen
\item Durch das Funktionieren des Basisprozesses ist ein Wechsel in diesen
  Zustand nicht möglich
\end{enumerate}

Die Steuerbarkeit ist nur ein Mittel.  Dies sollte nur verwendet werden, wenn
das System ohne Steuerung nicht funktionieren würde.

Zu den wichtigsten Steuereinheiten gehört das An- und Ausschalten des
technischen Systems.  Weitere werden durch Funktionsweise und -prinzipien
festgelegt.

Bei der Synthese mit einem Subsystem der Steuerung sollte als erstes die
Wirkung beachtet werden, mit der das System in den Zielzustand versetzt werden
soll.  Aufgrund dieser wird das Wirkprinzip des Steuersubsytems aufgrund der
gegebenen Resourcen festgelegt.

Die Zustandsänderung durch eine Steuerung muss in dem Fluss als Strukturglied
dargestellt werden.  Dieses sichert die Einwirkung auf den Fluss und ist für
Steuereinwirkung empfänglich.

\section{Gesetz der Symmetrie technischer Objekte}

Die Einwirkung der Umwelt auf ein System bildet eine Symmetrie ab.  Diese ist
bedingt durch die Kombination dieser.  Das Gesetz sollte aber als Tendenz
verstanden werden, da in anderen Artikeln ein Abweichen von dieser Symmetrie
zu einer Optimierung des technischen Systems geführt hat.  Demnach ist die
gegebene Symmetrie eine Einschränkung der Vielfalt des technischen Systems.

\section{Diskussionsthemen}
\begin{itemize}
\item Konformität von Struktur und Funktion.
\begin{itemize}
\item Zusammensetzung des Systems nicht trivial, aber definitiv machbar.
\end{itemize}
\item Wirkprinizip
\begin{itemize}
\item Uneindeutigkeit der Beschreibung des technischen Systems.
\item Definition beruht auf dem Blickwinkel der Person.
\end{itemize}
\end{itemize}
\end{document}
