\documentclass{beamer}
\usepackage{lsfolien}
\usepackage[german]{babel}
\usepackage[utf8]{inputenc}

\myfootline{Systemmodellierung und Semantic Web -- WS 20/21}{Hans-Gert Gräbe}

\title{Modellierung nachhaltiger Systeme\\ und Semantic Web\\[6pt] \Large
  Modellierung widersprüchlicher Anforderungen\\ in der TRIZ
  \vskip1em}

\subtitle{Vorlesung im Modul 10-202-2330\\ im Master und Lehramt Informatik\\
  sowie im Modul 10-202-2309 im Master Informatik}

\author{Prof. Dr. Hans-Gert Gräbe\\
\url{http://www.informatik.uni-leipzig.de/~graebe}}

\date{Wintersemester 2020/21}
\begin{document}

{\setbeamertemplate{footline}{}
\begin{frame}
  \titlepage
\end{frame}}

\section{Grundlagen}
\begin{frame}{Begriff eines Technischen Systems}

(V. Petrov, 2020) \vskip1em

Ein \textbf{System} ist eine Menge von \emph{Elementen}, die
\emph{untereinander verbunden} sind und \emph{miteinander interagieren}, die
ein \emph{einheitliches Ganzes} bilden, das \emph{Eigenschaften} besitzt, die
nicht bereits in den konstituierenden Elementen, einzeln betrachtet, enthalten
sind.\vskip1em

Eine solche Eigenschaft wird als \textbf{Systemeffekt}, \textbf{Synergie} oder
\textbf{Emergenz} bezeichnet.\vskip1em

Als \textbf{Synergie} bezeichnet man den summierenden Effekt der
Wechselwirkung von zwei oder mehr Faktoren, dadurch charakterisiert, dass ihre
Wirkung deutlich über die Wirkung jeder einzelnen der Komponenten und deren
einfache Summe hinausgeht.
\end{frame}

\begin{frame}{Technische Systeme als Reduktion auf Wesentliches}
  \begin{block}{Die Reduktion auf das Wesentliche ... }
    ... fokussiert auf die folgenden drei Dimensionen:
    \begin{itemize}
    \item[(1)] Abgrenzung des TS nach außen gegen eine \emph{Umwelt},
      Reduktion dieser Beziehungen auf Input/Output-Beziehungen und
      garantierten Durchsatz (Zweckbestimmtheit und Arbeitsfähigkeit).
    \item[(2)] Abgrenzung des TS nach innen durch Gruppierung von Teilen als
      \emph{Komponenten}, deren Funktionieren auf eine „Verhaltenssteuerung“
      über deren Schnittstellen reduziert wird.
    \item[(3)] Reduktion der Beziehungen im TS selbst auf \emph{kausal
      wesentliche}.
    \end{itemize}
  \end{block}
\end{frame}

\begin{frame}{Technische Systeme und Vorgängigkeit}
  \begin{block}{Das TS in der Welt der technischen Systeme}
    Die Beschreibung des TS selbst ist nur auf der Grundlage von
    Beschreibungen anderer (explizit oder implizit gegebener) TS möglich.  Der
    Beschreibung vorgängig sind:
    \begin{itemize}
    \item[(1)] Eine vage Vorstellung der (funktionierenden)
      Input/Output-Charakteristika der Umwelt.
    \item [(2)] Ein klares Bild von der Funktionsweise der Komponenten über
      die reine Spezifikation hinaus.
    \item [(3)] Eine vage Vorstellung von Ursache-Wirkungs-Beziehungen im
      System selbst, das der detaillierten Modellierung vorausgeht.
    \end{itemize}
  \end{block}
\end{frame}

\begin{frame}{Komponenten und Objekte}
  (Szyperski 2002)
  \begin{itemize}
  \item Komponenten sind wieder Systeme.
  \item Die können selbst entwickelt oder von Dritten erworben sein. 
  \item Man muss nicht die ganze Komponente erwerben, es reicht aus, den
    \emph{Dienst} zu nutzen.
  \end{itemize}
  So ist es auch in vielen Fällen: Eine Komponente ist im System mit ihrer PNF
  über deren Spezifikation als Black Box verfügbar, der Betrieb der Komponente
  (Bereitstellung der Funktion) erfolgt durch Dritte, aus \emph{deren}
  Verantwortungsbereich heraus, die Funktion wirkt auf „meinen“ Objekte in
  \emph{meinem} Verantwortungsbereich.
  \begin{itemize}
  \item Damit die Unterscheidung nach Szyperski: Komponenten kapseln
    Funktionalität, Objekte kapseln Systemzustände. 
  \end{itemize}
\end{frame}

\begin{frame}{Das minimale technische System in der TRIZ}

\begin{center}
  \begin{tikzpicture}[line width=1pt,
      pfeil/.style={shorten <=4pt, shorten >=4pt, line width=1.5pt},
      mytext/.style={text width=2.5cm,align=center}]
    \node[draw] at (0,1.3) [rectangle] (A0) {Werkzeug};
    \node[draw] at (5,1.3) [rectangle] (A2) {Objekt};
    \node[text width=2cm,align=center] at (8,1.3) [rectangle] (A3) {nützliches Produkt};
    \node[draw,mytext] at (0,3.5) [rectangle] (A4) {Betrieb durch Dritte};
    %\node[mytext,below of = A0] {Old state of the library};
    %\node[mytext,below of = A1] {New state of the library};
    \draw[pfeil,->] (A0)--(A2) ;
    \draw[pfeil,->] (A2)--(A3) ;
    \draw[pfeil,->,dashed] (A4)--(A0) ;
    \draw[dashed] (-1.5,0) -- (6,0) -- (6,2) -- (-1.5,2) -- cycle ;
    \node[fill=white] at (2.7,1.3) [rectangle] {wirkt auf};
    \node[] at (2.7,.3) {mein Verantwortungsbereich};
  \end{tikzpicture}
\end{center}
Gestrichelter Rahmen = das minimale technische System

Gestrichelter Pfeil = wird bei Szyperski, nicht aber in der TRIZ thematisiert 
\end{frame}

\begin{frame}{Komponenten und Umwelt}
  Komponenten (insbesondere solche, die durch Dritte betrieben werden) sind
  damit Pointer an andere Stellen in der \emph{Welt technischer Systeme} und
  stellen somit nur eine andere Form der „Beziehung eines Systems zur Umwelt“
  dar.\vskip1em

  Es entsteht die Frage, ob die Aspekte (1) und (3) in der Liste der
  „Reduktionen auf Wesentliches“ unter diesem Ansatz vereinigt werden können.
  \vskip1em

  Andererseits entsteht die Frage, wie der Objektbegriff in die
  Gesamtlogik einzubauen ist.\vskip1em

  Beide Fragen lassen wir an dieser Stelle erst einmal offen. 
\end{frame}

\begin{frame}{Modellierung von Systemen}
  Zwei Fragestellungen:
  \begin{itemize}
  \item[(1)] Neues System bauen
  \item[(2)] Bestehendes System umbauen
  \end{itemize}\vskip2em
  (1) kann als Spezialfall von (2) aufgefasst werden, da jeder Bedarf nach
  einem neuen System mit wenigstens \emph{groben Vorstellungen} über jenes
  neue System kommt, also auch unter (1) eine wenigstens \emph{grobe
    Beschreibungsform} des zu schaffenden Systems existiert.
\end{frame}

\begin{frame}{Modellierung von Systemen}
\begin{center}
  \begin{tikzpicture}[line width=1pt,
      pfeil/.style={shorten <=4pt, shorten >=4pt, line width=1.5pt},
      mytext/.style={text width=2.5cm,align=center}]
    \node[draw,mytext] at (0,0) [rectangle] (A0) {System, wie es ist};
    \node[draw,mytext] at (7,0) [rectangle] (A2) {System, wie es sein soll};
    \draw[pfeil,->] (A0)--(A2) ;
    \node[fill=white] at (3.5,.3) [rectangle] {Transformation};
  \end{tikzpicture}
\end{center}
  Dieses grundlegende Schema passt nicht nur auf technische Systeme, sondern
  auch auf die Modellierung sozialer, sozio-ökologischer und kultureller
  Systeme, ist also hinreichend universell.   
\end{frame}

\end{document}
