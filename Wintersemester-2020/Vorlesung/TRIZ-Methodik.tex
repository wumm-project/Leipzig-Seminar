\documentclass{beamer}
\usepackage{lsfolien}
\usepackage[german]{babel}
\usepackage[utf8]{inputenc}

\myfootline{Systemmodellierung und Semantic Web -- WS 20/21}{Hans-Gert Gräbe}

\title{Modellierung nachhaltiger Systeme\\ und Semantic Web\\[6pt] \Large
  Modellierung widersprüchlicher Anforderungen\\ in der TRIZ
  \vskip1em}

\subtitle{Vorlesung im Modul 10-202-2330\\ im Master und Lehramt Informatik\\
  sowie im Modul 10-202-2309 im Master Informatik}

\author{Prof. Dr. Hans-Gert Gräbe\\
\url{http://www.informatik.uni-leipzig.de/~graebe}}

\date{Wintersemester 2020/21}
\begin{document}

\section{Grundlagen}
\begin{frame}{TRIZ und ARIZ}

TRIZ ist nicht nur eine Theorie, sondern schlägt als anzuwendende Methodik ein
genaues algorithmisches Vorgehen vor.\vskip1em

Von diesem Algorithmus ARIZ (Algorithmus zur Lösung erfinderischer Probleme)
gibt es mehrere Varianten, die „offizielle“ ist ARIZ-85C, die auf eine Version
von Altschuller aus dem Jahr 1985 zurückgeht. Andere (D. Zobel) sehen wenig
Fortschritt gegenüber ARIZ-77 und empfehlen diesen etwas einfacheren
Zugang.\vskip1em

Wir orientieren uns an AIPS-2015 (Algorithmus zur Korrektur von
Problemsituationen), einer Version in der Tradition des OTSM-TRIZ, die auch im
Minsker TRIZ-Trainer zum Einsatz kommt.
\end{frame}

\begin{frame}{TRIZ-Trainer -- die erste Phase des Lösungsprozesses}
  
Die erste Phase des Lösungsprozesses liefert ein genaues Modell des „Systems,
wie es ist“, das zur Lösung der Problematik transformiert werden muss. Sie
besteht aus drei Schritten
\begin{itemize}
\item [(A)] Kontextualisierung der Aufgabenstellung. Das System als Black
  Box. 
\item [(B)] Analyse und Modellierung von Aufbau- und Ablauforganisation des
  „Systems, wie es ist“ -- der „Maschine“ in der Terminologie des
  TRIZ-Trainers
\item [(C)] Identifizierung und Lokalisierung des zentralen Widerspruchs,
  Bestimmung der operativen Zone und der operativen Zeit, wo und wann der
  Widerspruch auftritt, sowie Aufstellen möglicher Konflikthypothesen.  
\end{itemize}
Aus diesen Hypothesen wird eine Aufgabe formuliert, die in der zweiten Etappe
genauer analysiert werden soll. 
\end{frame}

\begin{frame}{TRIZ-Trainer -- die erste Phase des Lösungsprozesses}
  
Erster Abschnitt „Präzisierung der Umstände“:
\begin{itemize}
\item [1.] Identifiziere das zu untersuchende System als Black Box und gib ihm
  einen „sprechenden Namen“, aus dem sich die Semantik des Systems bereits
  grob erschließt -- was ist das „nützliches Produkt“? 
\item [2.] Identifiziere die \emph{primär nützliche Funktion} (PNF) des
  Systems.

  Untersuche dazu ggf., welchem \emph{Zweck} das System im Obersystem dient
  und bestimme ggf. den für den Betrieb des Systems erforderlichen Durchsatz
  (Leistung des Obersystems für das Funktionieren des Systems).
\item [3.] Formuliere das bestehende Problem, welches dem spezifikationskonformen
  Verhalten des Systems im Wege steht -- den „unerwünschten Effekt“.
\end{itemize}

\end{frame}

\begin{frame}{TRIZ-Trainer -- die erste Phase des Lösungsprozesses}
  
Zweiter Abschnitt „Systemkonflikt“: 
\begin{itemize}
\item [4.] Bestimme die Bestandteile der Maschine (deren Aufbauorganisation)
  sowie deren Arbeitsweise (deren Ablauforganisation). Oft reicht es aus, den
  Schwerpunkt auf eine der beiden Fragen zu legen.\vskip1em

  Folge dabei dem allgemeinen Aufbaumuster „Energiequelle, Antrieb,
  Transmission, Werkzeug, Aktion, bearbeitetes Objekt, nützliches Produkt plus
  Steuerung“.\vskip1em
  
  Hierbei ist es wichtig, die primär nützliche Funktion (PNF) des Systems zu
  beschreiben, auch wenn das Problem durch eine Nebenfunktion ausgelöst wird,
  da sich um die PNF herum die im System genutzten Ressourcen gruppieren.
\end{itemize}
\end{frame}

\begin{frame}{TRIZ-Trainer -- die erste Phase des Lösungsprozesses}
  
\begin{itemize}
\item [5.] Diese PNF steht in gewissem Verhältnis zur „Wirkung, die nicht ohne
  Probleme abgeschlossen werden kann“.\vskip1em

  Diese Wirkung sowie ihr Verhältnis zur PNF sind nun genauer zu bestimmen als
  Kern des zu lösenden Konflikts.\vskip1em

  Bei dieser Analyse sind insbesondere Ort und Zeit des Konflikts genauer
  einzugrenzen, um mögliche spätere Separationen nach Ort oder Zeit als
  grundlegende Lösungsmethoden vorzubereiten.
\end{itemize}

\end{frame}

\begin{frame}{TRIZ-Trainer -- die erste Phase des Lösungsprozesses}
  
Dritter Abschnitt „Aufstellen einer Hypothese“: 
\begin{itemize}
\item Durch genauere Analyse der „Konfliktursachen“ werden eine oder mehrere
  Hypothesen allgemeinen Charakters aufgestellt, welche Maßnahmen im Sinne
  eines \emph{idealen Endresultats} das Problem lösen würden.\vskip1em
  
  Einer dieser Ansätze wird als „Aufgabe“ für die zweiten Phase des
  Lösungsprozesses genauer formuliert, um diese dann mit passenden
  TRIZ-Werkzeugen zu bearbeiten.
\end{itemize}

\end{frame}

\begin{frame}{Das Ideale Endresultat}
  
Das \textbf{Ideale Endresultat} (IER) beschreibt das „System, wie es sein
soll“ als \emph{Ziel} der Transformation, ohne sich zunächst darum zu scheren,
ob das formulierte Resultat praktisch realisiert werden kann. Im weiteren
Lösungsprozess werden die Hindernisse identifiziert, die auf dem Weg zum IER
zu überwinden sind, und auf Basis der TRIZ-Methodik Strategien entwickelt, wie
diese Hindernisse praktisch überwunden werden können.\vskip1em

Das IER ist einer der Grundbegriffe der TRIZ. Das IER ist eine
Orientierungsgröße im Sinne einer „konkreten Utopie“, die wesentlich den
Zielkorridor bestimmt, auf den sich der weitere Lösungsprozess in der zweiten
Phase konzentriert.

\end{frame}

\end{document}
