\documentclass[11pt,a4paper]{article}
\usepackage{a4wide,url,enumitem}
\usepackage[utf8]{inputenc}
\usepackage[german]{babel}

\parindent0pt
\parskip4pt
\setcounter{secnumdepth}{-2}

\title{Vorlesungsplan \\[1em] \emph{Modellierung nachhaltiger Systeme und
    Semantic Web} \\[1em] im Wintersemester 2020/21}

\author{Hans-Gert Gr\"abe}

\date{27. Oktober 2020}

\begin{document}
\maketitle
\tableofcontents
\newpage
\subsection{Allgemeines}

Die Vorlesung findet synchron online statt (jeweils donnerstags 11-13 Uhr) und
orientert sich am Flipped Classroom Konzept. Die Vorlesung besteht aus drei
Teilen.

Im ersten Teil werden wir das Konzept eines \emph{Technischen Systems}
erschließen und uns mit der TRIZ als wichtigster Systematischer
Innovationsmethodik beschäftigen.  Im Gegensatz zu anderen Kreativitäts- und
Innovationsmethodiken setzt die TRIZ auf die Systematisierung
ingenieur-technischer Erfahrungen. 

Im zweiten Teil befassen wir uns genauer mit Aspekten der Erstellung von
Begriffsnetzen für Datenmodelle auf der Basis des \emph{Resource Description
  Frameworks} (RDF), der \emph{Linked Open Data Cloud}, dem dabei entstehenden
\emph{Giant Global Graph} und der Bedeutung dieser Entwicklungen für die
Organisation kooperativer Handlungszusammenhänge.

Im dritten Teil untersuchen wir schließlich die Rolle von Daten und
Informationen sowie die Erzeugung neuer Sprache für die Entwicklung
technischer Systeme in einer bürgerlichen Gesellschaft.

Neben einer allgemeinen Literaturliste wird dazu zu jeder Vorlesung Literatur
zur Vorbereitung angegeben, die \textbf{vor der Vorlesung zu studieren ist},
um den Ausführungen folgen zu können. In der Vorlesung wird das Thema nur
kursorisch beleuchtet, es besteht aber die Möglichkeit, Fragen zur Literatur
zu stellen und einzelne Aspekte gemeinsam zu diskutieren.

Die meisten Materialien zur Vorlesung sind im öffentlichen Materialordner im
github Projekt \emph{Leipzig-Seminar} (im Weiteren \textbf{LS-Materialordner})
\begin{center}
  \url{https://github.com/wumm-project/Leipzig-Seminar}
\end{center}
im Verzeichnis \texttt{Wintersemester-2020/Material} verfügbar oder sind
anderweitig im Internet leicht aufzufinden. Auf klassische gedruckte Literatur
und Ihre Fähigkeiten, diese zu beschaffen, wird dennoch nicht verzichtet.

\subsection{Datenschutz}

Wir folgen nicht nur theoretisch, sondern auch praktisch einem Open Culture
Ansatz und stellen Kursmaterialien öffentlich zur Verfügung.  Dies gilt auch
für die (kommentierten) Chatverläufe der Vorlesung, in denen auch Ihre Namen
genannt werden.  Wir gehen von Ihrem Einverständnis mit diesem Vorgehen aus,
wenn Sie dem nicht explizit widersprechen.  Die Diskussionen selbst werden
\textbf{nicht} aufgezeichnet.

\subsection{Allgemeine Literaturliste}
\begin{itemize}[noitemsep]
\item Robert Adunka (2020). TRIZ Anwendungsbeispiele. \\
  \url{https://www.triz-consulting.de/ueber-triz/triz-anwendungsbeispiele-2/} 
\item Iouri Belski (2020). Tools of TRIZ. A web repository of TRIZ materials
  on 12 simple TRIZ heuristics.
  \url{https://emedia.rmit.edu.au/triz/content/tools-triz}
\item Karl Koltze, Valeri Souchkov (2017). Systematische Innovationsmethoden.
  Hanser Verlag, München. ISBN 9783446451278
\item Andrei Kuryan, Dmitri Kucharavy (2018). The OTSM-TRIZ Heritage of
  Nikolai N. Khomenko. A General Theory of Powerful Thinking. Folien eines
  Vortrags auf dem TDS 2018 in St. Petersburg. Als \texttt{OTSM-Folien.pdf} im
  LS-Materialordner.
\item Nikolai Khomenko, John Cooke (2007). Inventive problem solving using the
  OTSM-TRIZ “TONGS” model.  Als \texttt{tongs-en.pdf} im LS-Materialordner.
\item Alex Lyubomirskiy, Simon Litvin, Sergei Ikovenko et al. (2018). Trends
  of Engineering System Evolution (TESE).  TRIZ Consulting Group. ISBN
  9783000598463.
\item Dietmar Zobel (2007). Kreatives Arbeiten. Expert Verlag, Renningen.\\
  ISBN 9783816927136.
\item Dietmar Zobel (2020). TRIZ für alle. Expert Verlag, Renningen. ISBN
  9783816985105.
\end{itemize}

\subsection{29.10.: Einführung. Technikbegriff}

\textbf{Lernziel:}
Technikbegriff als Einheit von 
\begin{itemize}[noitemsep]
\item gesellschaftlich verfügbarem Verfahrenswissen,
\item institutionalisierten Verfahrensweisen (state of the art) und
\item privatem Verfahrenskönnen.
\end{itemize}

\subsection{5.11.: Technische Systeme}

\textbf{Literaturliste:}
\begin{itemize}[noitemsep]
\item Hans-Gert Gräbe (2020). Die Menschen und ihre Technischen Systeme. LIFIS
  Online, 19. Mai 2020. \url{http://dx.doi.org/10.14625/graebe_20200519}.
\end{itemize}

\textbf{Lernziele:}
\begin{itemize}[noitemsep]
\item Unterscheidung von Beschreibungs- und Vollzugformen Technischer Systeme
\item Technische Systeme und Komplexitätsreduktion als „Reduktion auf das
  Wesentliche“
\item Infrastrukturelle Einbettung Technischer Systeme in die „Welt der
  technischen Systeme“
\end{itemize}

\subsection{12.11.: Modellierung widersprüchlicher Anforderungen und die\\
  TRIZ-Methodik nach AIPS-2015}

\textbf{Literaturliste:}
\begin{itemize}[noitemsep]
\item Hans-Gert Gräbe (2020). Zur Methodik des TRIZ-Trainers und AIPS-2015.\\ 
  Als \texttt{TRIZ-Methodik.pdf} im LS-Materialordner.
\item (Kuryan, Kucharavy 2018), ab Folie 31.
\end{itemize}
Zusatzliteratur:
\begin{itemize}[noitemsep]
\item (Koltze, Souchkov 2017)
\item (Adunka 2020)
\end{itemize}

\textbf{Lernziele:}
\begin{itemize}[noitemsep]
\item Kontextualisierung von Anforderungssituationen durch den Systemoperator 
\item Technische Systeme als Black Box, Zwecke und die Bestimmung der PNF
  (primär nützliche Funktion, core concern)
\item Bestimmung von Aufbau- und Ablauforganisation eines Systems nach
  AIPS-2015 
\item Nützliche und schädliche Effekte, operative Zone und operative Zeit von
  widersprüch\-lichen Systemeigenschaften
\end{itemize}

\subsection{19.11.: Modellierung natürlicher Systeme und die Theorie Offener
  Systeme}

\textbf{Literaturliste:}
\begin{itemize}
\item Hans-Gert Gräbe (2020).  Zum Systembegriff in der Theorie dynamischer
  Systeme.  Als \texttt{TDS.md} im LS-Materialordner.
\end{itemize}
Zusatzliteratur:
\begin{itemize}[noitemsep]
\item Ludwig von Bertalanffy (1950). An outline of General System Theory, The
  British Journal for the Philosophy of Science, Volume I.2, 134–165.\\
  \url{https://doi.org/10.1093/bjps/I.2.134}
\item Erich Jantsch (1992). Die Selbstorganisation des Universums. Hanser,
  München.
\item Ilya Prigogine, Isabelle Stengers (1993). Das Pardox der Zeit. Piper,
  München, Kap. 3--5.
\end{itemize}

\textbf{Lernziele:}
\begin{itemize}[noitemsep]
\item Der AIPS-2015 Workflow als Ausprägung einer allgemeineren Theorie
  Dynamischer Systeme 
\item Grundlagen der Modellierung natürlicher Systeme
\end{itemize}

\subsection{26.11.: Nachhaltigkeit und Entwicklungszyklen von Systemen}

\textbf{Literaturliste:}
\begin{itemize}[noitemsep]
\item Frank W. Geels, Johan Schot (2007). Typology of Sociotechnical
  Transition Pathways. In: Research Policy 36 (2007), 399–417.\\
  \url{https://doi.org/10.1016/j.respol.2007.01.003} 
\item Hans-Gert Gräbe (2020). TRIZ und Transformationen sozio-technischer und
  sozio-ökolo\-gischer Systeme. LIFIS Online, 27.06.2020.\\
  \url{http://dx.doi.org/10.14625/graebe_20200627}.
\item C.S. Holling (2000). Understanding the Complexity of Economic,
  Ecological, and Social Systems. In: Ecosystems (2001) 4, 390–405.\\
  {\scriptsize \url{https://www.esf.edu/cue/documents/Holling_Complexity-EconEcol-SocialSys_2001.pdf}}
\end{itemize}

Zusatzliteratur:
\begin{itemize}[noitemsep]
\item Mike Davis (2009).  Wer wird die Arche bauen? Das Gebot utopischen
  Denkens im Zeitalter der Katastrophen.  In: Das Ende des
  Kasino-Kapitalismus?  Globalisierung und Krise. Blätter
  Verlagsges., Berlin, S. 267-285. ISBN 3980492559. 
\item Hans-Gert Gräbe (2012). Wie geht Fortschritt? LIFIS Online,
  12.11.2012.\\ 
  \url{http://www.leibniz-institut.de/archiv/graebe_12_11_12.pdf}
\item Rainer Thiel (2000). Die Allmählichkeit der Revolution. Blick in sieben
  Wissenschaften.  LIT-Verlag, Münster. ISBN 9783825849457. 
\end{itemize}

\textbf{Lernziele:}
\begin{itemize}[noitemsep]
\item Entwicklung von Systemen, Grenzübergänge und Rekontextualisierungen 
\item Evolution und Revolution
\item Systemische Transformationsprozesse
\end{itemize}

\subsection{3.12.: Digitales Universum. Der digitale Wandel als
  techno-kultureller Umbruch}

\textbf{Literaturliste:}
\begin{itemize}[noitemsep]
\item Michael Schetsche: Die digitale Wissensrevolution – Netzwerkmedien,
  kultureller Wandel und die neue soziale Wirklichkeit. In: zeitenblicke 5
  (2006), Nr. 3, [2006-12-03]. urn:nbn:de:0009-9-6419
\end{itemize}
Zusatzliteratur:
\begin{itemize}[noitemsep]
\item Felix Stalder. Kultur der Digitalität. Suhrkamp 2016.
\end{itemize}

\textbf{Lernziele:}
\begin{itemize}[noitemsep]
\item Begriffe \emph{Digitales Universum}, \emph{Digitale Identitäten}
\item Authentifizierung und Autorisierung als sozio-technische Prozesse
\item Digitale Identitäten und Rollen
\end{itemize}

\subsection{10.12.: Datenmodellierung und Modellierung von Begriffswelten.}

\textbf{Literaturliste:}
\begin{itemize}[noitemsep]
\item Marc Zobrist (2018). Ontologien und Datenmodelle. \\
  \url{https://histhub.ch/ontologien-und-datenmodelle/}
\item Hans-Gert Gräbe, Thomas Riechert, Michael Martin (2011).  OD.FMI –
  Engineering operativ-administrativer Daten für die universitäre Lehre.
  Manuskript, eingereicht und abgelehnt bei der Delfi 2011, September 2011, TU
  Dresden. \\
  \url{http://www.informatik.uni-leipzig.de/~graebe/Texte/delfi-11.pdf}
\item Hans-Gert Gräbe (seit 2007). Das OD.FMI Projekt. \\
  {\small \url{http://www.informatik.uni-leipzig.de/~graebe/OLAT/LVPlanung/ODQuelle.html}}
\end{itemize}

\textbf{Lernziele:}
\begin{itemize}[noitemsep]
\item Modellierung von Begriffswelten als Teil der Modellierung Technischer
  Systeme
\item RDF-Grundbegriffe URI, Namensraum, Subjekt, Prädikat, Objekt
\end{itemize}

\subsection{17.12.: RDF. Basics, Konzepte, Werkzeuge}

\textbf{Literaturliste:}
\begin{itemize}
\item Open Data Support (2013). Trainingsmodul 1.3 Einführung
  in RDF \& SPARQL.  Als \texttt{RDF-Einfuehrung.pdf} im LS-Materialordner.
\end{itemize}

\textbf{Lernziele:}
\begin{itemize}[noitemsep]
\item RDF Konzepte. RDF-Graphen, Mustersuche, SPARQL Anfragesprache
\item RDF Werkzeuge. RDF Stores, SPARQL Endpunkte
\end{itemize}

\subsection{7.1.: Die Linked Open Data Cloud. Ontologien und Begriffswelten.
Der Giant Global Graph}

\textbf{Literaturliste:}
\begin{itemize}
\item TBA
\end{itemize}

\textbf{Lernziele:}
\begin{itemize}[noitemsep]
\item Modellierung gemeinsamer Begriffswelten als kooperativer sozialer
  Prozess 
\item Bedeutung gemeinsamer Begriffswelten für das geordnete Zusammenleben 
\end{itemize}

\subsection{14.1.: Daten und Information}

\textbf{Literaturliste:}
\begin{itemize}
\item TBA
\end{itemize}

\textbf{Lernziele:}
\begin{itemize}[noitemsep]
\item \emph{Fiktionen} als normale Form der Komplexitätsreduktion in
  interpersonalen Handlungszusammenhängen
\item Das OSI-7-Schichtenmodell und Sprachschöpfung beim Design Technischer
  Systeme  
\end{itemize}

\subsection{21.1.: Kooperatives Handeln in der bürgerlichen Gesellschaft}

\textbf{Literaturliste:}
\begin{itemize}
\item TBA
\end{itemize}

\textbf{Lernziele:}
\begin{itemize}[noitemsep]
\item Menschliche Praxen als widersprüchliches Verhältnis von begründeten
  Erwartungen und erfahrernen Ergebnissen
\item Sozialisierungsbedingungen von Wissen und von Handeln
\item Kooperative Verfahrensweisen und deren Weiterentwicklung
\item Bedingungen der Möglichkeit des Handels in der bürgerlichen Gesellschaft 
\end{itemize}

\subsection{28.1.: Technikentwicklung und die Entwicklung kapitalistischer
  Unternehmensformen. Netzkooperation} 

\textbf{Literaturliste:}
\begin{itemize}
\item TBA
\end{itemize}

\textbf{Lernziele:}
\begin{itemize}[noitemsep]
\item Grundformen kapitalistischer Unternehmen und deren Verhältnis zu den
  Konzepten Innovation, Genialität und geistigem Eigentum
\item Bedingtheiten und Notwendigkeit von Open Culture
\end{itemize}

\subsection{4.2.: Herausbildung einer Open Culture als notwendiger
  Begleitprozess technischer Höherentwicklung} 

\textbf{Literaturliste:}
\begin{itemize}
\item TBA
\end{itemize}

\textbf{Lernziele:}
\begin{itemize}[noitemsep]
\item Open Culture und Infrastrukturentwicklung
\item Services und Geschäftsmodelle unter den Bedingungen von Open Culture
\end{itemize}

\subsection{5.2.: Optional: Interdisziplinäres Gespräch} 

Optional: Interdisziplinäres Gespräch „Gesetze der Entwicklung technischer
Systeme“ als Blockseminar.

\end{document}
