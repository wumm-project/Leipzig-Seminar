\documentclass[11pt,a4paper]{article}
\usepackage{a4wide,url,enumitem}
\usepackage[utf8]{inputenc}
\usepackage[german]{babel}

\parindent0pt
\parskip4pt
\setcounter{secnumdepth}{-2}

\title{Vorlesungsplan „Modellierung nachhaltiger Systeme und Semantic Web“ im
  Wintersemester 2020/21}

\author{Hans-Gert Gr\"abe}

\date{10. Oktober 2020}

\begin{document}
\maketitle
\subsection{Allgemeines}

Die Vorlesung findet synchron online statt (jeweils do 11-13 Uhr) und
orientert sich am Flipped Classroom Konzept. Neben einer allgemeinen
Literaturliste wird dazu zu jeder Vorlesung Literatur zur Vorbereitung
angegeben. In der Vorleung wird das Thema nur kursorisch beleuchtet, es
besteht aber die Möglichkeit, Fragen zur Literatur zu stellen und einzelne
Aspekte gemeinsam zu diskutieren.

\textbf{Allgemeine Literaturliste:}
\begin{itemize}[noitemsep]
\item Karl Koltze, Valeri Souchkov. Systematische Innovationsmethoden. Hanser,
  München 2017.
\item Lyubomirski u.a.: Theory of Evolution of Technical Systems.
  Sulzbach-Rosenberg, 2018.
\item Dietmar Zobel. TRIZ für alle. Expert Verlag, Renningen 2018.
\item Dietmar Zobel. Kreatives Denken. Expert Verlag, Renningen 2007.
\end{itemize}
\newpage
\tableofcontents
\newpage
\subsection{29.10.: Einführung. Technikbegriff}

\textbf{Lernziele:}
\begin{itemize}[noitemsep]
\item Technikbegriff als Einheit von 
  \begin{itemize}[noitemsep]
  \item gesellschaftlich verfügbarem Verfahrenswissen,
  \item institutionalisierten Verfahrensweisen (state of the art) und
  \item privatem Verfahrenskönnen.
  \end{itemize}
\end{itemize}

\subsection{5.11.: Technische Systeme}

\textbf{Literaturliste:}
\begin{itemize}[noitemsep]
\item Hans-Gert Gräbe. Menschen und ihre Technischen Systeme
\end{itemize}

\textbf{Lernziele:}
\begin{itemize}[noitemsep]
\item Unterscheidung von Beschreibungs- und Vollzugformen Technischer Systeme
\item Technischer Systeme und Komplexitätsreduktion als „Reduktion auf das Wesentliche“
\item Infrastrukturelle Einbettung Technischer Systeme in die „Welt der technischen Systeme“ 
\end{itemize}

\subsection{12.11.: Modellierung widersprüchlicher Anforderungen und die\\
  TRIZ-Methodik nach AIPS-2015}

\textbf{Literaturliste:}
\begin{itemize}[noitemsep]
\item Hans-Gert Gräbe. TRIZ-Methodik.pdf
\item {}[Koltze, Souchkov]
\end{itemize}

\textbf{Lernziele:}
\begin{itemize}[noitemsep]
\item Kontextualisierung von Anforderungssituationen durch den Systemoperator 
\item Technischer Systeme als Black Box, Zwecke und die Bestimmung der PNF (primär nützlichen
  Funktion, core concern) 
\item Bestimmung von Aufbau- und Ablauforganisation eines Systems nach
  AIPS-2015 
\item Nützliche und schädliche Effekte, operative Zone und operative Zeit von
  widersprüch\-lichen Systemeigenschaften
\end{itemize}

\subsection{19.11.: Modellierung natürlicher Systeme und die Theorie Offener
  Systeme}

\textbf{Literaturliste:}
\begin{itemize}
\item Hans-Gert Gräbe. Linkliste aus dem Seminar im Vorjahr
\end{itemize}

\textbf{Lernziele:}
\begin{itemize}[noitemsep]
\item Der AIPS-2015 Workflow als Ausprägung einer allgemeineren Theorie
  Dynamischer Systeme 
\item Grundlagen der Modellierung natürlicher Systeme
\end{itemize}

\subsection{26.11.: Nachhaltigkeit und Entwicklungszyklen von Systemen}

\textbf{Literaturliste:}
\begin{itemize}[noitemsep]
\item Holling, Fox, Gehrs, Gräbe aus dem Vorjahr
\end{itemize}

Zusatzliteratur:
\begin{itemize}[noitemsep]
\item Mike Davis. Arche
\item Hans-Gert Gräbe. Wie geht Fortschritt?
\item Rainer Thiel. Allmählichkeit der Revolution.
\end{itemize}

\textbf{Lernziele:}
\begin{itemize}[noitemsep]
\item Entwicklung von Systemen, Grenzübergänge und Rekontextualisierungen 
\item Evolution und Revolution
\item Systemische Transformationsprozesse
\end{itemize}

\subsection{3.12.: Digitales Universum. Der digitale Wandel als
  techno-kultureller Umbruch}

\textbf{Literaturliste:}
\begin{itemize}[noitemsep]
\item Setsche
\end{itemize}

Zusatzliteratur:
\begin{itemize}[noitemsep]
\item Felix Stalder. Kultur der Digitalität.
\end{itemize}

\textbf{Lernziele:}
\begin{itemize}[noitemsep]
\item Begriffe \emph{Digitales Universum}, \emph{Digitale Identitäten}
\item Authentifizierung und Autorisierung als sozio-technische Prozesse
\item Digitale Identitäten und Rollen
\end{itemize}

\subsection{10.12.: Datenmodellierung und Modellierung von Begriffswelten.
  Das Resource Description Framework (RDF)}

\textbf{Literaturliste:}
\begin{itemize}[noitemsep]
\item TBA
\end{itemize}

\textbf{Lernziele:}
\begin{itemize}[noitemsep]
\item Modellierung von Begriffswelten als Teil der Modellierung Technischer Systeme
\item RDF-Grundbegriffe URI, Namensraum, Subjekt, Prädikat, Objekt
\end{itemize}

\subsection{17.12.: RDF. Basics, Konzepte, Werkzeuge}

\textbf{Literaturliste:}
\begin{itemize}
\item TBA
\end{itemize}

\textbf{Lernziele:}
\begin{itemize}[noitemsep]
\item RDF Konzepte. RDF-Graphen, Mustersuche, SPARQL Anfragesprache
\item RDF Werkzeuge. RDF Stores, SPARQL Endpunkte
\end{itemize}

\subsection{7.1.: Die Linked Open Data Cloud. Ontologien und Begriffswelten.
Der Google Knowledge Graph}

\textbf{Literaturliste:}
\begin{itemize}
\item TBA
\end{itemize}

\textbf{Lernziele:}
\begin{itemize}[noitemsep]
\item Modellierung gemeinsamer Begriffswelten als kooperativer sozialer
  Prozess 
\item Bedeutung gemeinsamer Begriffswelten für das geordnete Zusammenleben 
\end{itemize}

\subsection{14.1.: Daten und Information}

\textbf{Literaturliste:}
\begin{itemize}
\item TBA
\end{itemize}

\textbf{Lernziele:}
\begin{itemize}[noitemsep]
\item \emph{Fiktionen} als normale Form der Komplexitätsreduktion in
  interpersonalen Handlungszusammenhängen
\item Das OSI-7-Schichtenmodell und Sprachschöpfung beim Design Technischer
  Systeme  
\end{itemize}

\subsection{21.1.: Kooperatives Handeln in der bürgerlichen Gesellschaft}

\textbf{Literaturliste:}
\begin{itemize}
\item TBA
\end{itemize}

\textbf{Lernziele:}
\begin{itemize}[noitemsep]
\item Menschliche Praxen als widersprüchliches Verhältnis von begründeten
  Erwartungen und erfahrernen Ergebnissen
\item Sozialisierungsbedingungen von Wissen und von Handeln
\item Kooperative Verfahrensweisen und deren Weiterentwicklung
\item Bedingungen der Möglichkeit des Handels in der bürgerlichen Gesellschaft 
\end{itemize}

\subsection{28.1.: Technikentwicklung und die Entwicklung kapitalistischer
  Unternehmensformen. Netzkooperation} 

\textbf{Literaturliste:}
\begin{itemize}
\item TBA
\end{itemize}

\textbf{Lernziele:}
\begin{itemize}[noitemsep]
\item Grundformen kapitalistischer Unternehmen und deren Verhältnis zu den
  Konzepten Innovation, Genialität und geistigem Eigentum
\item Bedingtheiten und Notwendigkeit von Open Culture
\end{itemize}

\subsection{4.2.: Herausbildung einer Open Culture als notwendiger
  Begleitprozess technischer Höherentwicklung} 

\textbf{Literaturliste:}
\begin{itemize}
\item TBA
\end{itemize}

\textbf{Lernziele:}
\begin{itemize}[noitemsep]
\item Open Culture und Infrastrukturentwicklung
\item Services und Geschäftsmodelle unter den Bedingungen von Open Culture
\end{itemize}

\end{document}
