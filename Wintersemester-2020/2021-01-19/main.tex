\documentclass[11pt,a4paper]{article}
\usepackage[ngerman]{babel}
\usepackage{ls}
\usepackage[utf8]{inputenc}
\usepackage{hyperref}

\title{Handout - Evolutionbäume technischer Systeme bei Nikolay Shpakovsky}
\author{Tom Strempel}
\date{19.01.2021}

\begin{document}

\maketitle

\section{Einleitung}
% Objekt (übt für uns allein keine nützliche Funktion aus) <-> Komponete

Das Buch Tree of Technology Evolution erschien 2016 als Übersetzung des russischen Originals von 2010, welches von Nikolay Shpakovsky verfasst wurde. Darin wird werden die Konzepte des Evolutionsmusters und Evolutionsbaumes eingeführt um Informationsfelder zu strukturieren.

Shpakovsky hat diese Konzepte anhand der Entwicklung einer großen Menge realer technischer Systeme entwickelt. Die Gesetze der TRIZ sind auf die selbe Weise entstanden und Altshullers Arbeit stellt auch die Grundlage für Shpakovsky dar. 

Die ersten drei Kapitel des Buches beschäftigen sich mit mit den nötigen Grundlagen zum Erstellen der Evolutionsbäume. In den darauf folgenden Kapiteln wird die Erstellung und Anwendung von Evolutionsbäumen thematisiert.

\section{Begriffsdefinitionen}

Es erfolgt eine Schärfung des Idealitätsbegriffs eines technischen Systems. Laut Shpakovsky ist ein ideales System, ein System welches seine Funktion mit vernachlässigbaren Kosten in Bezug auf den Nutzen ausführen kann. Die Ausführungskosten $C$ können aber nicht Null betragen. Das Ziel der technischen Evolution ist es ein möglichst ideales System zu erzeugen.

Im Buch wird die verrichtete Arbeit $F$ einer Funktion als \glqq Performance\grqq{} bezeichnet. Dieser Begriff ist aus meiner Sicht schon durch das Leistungsverhalten einer Software belegt, demnach wird folgend $F$ als verrichtete Arbeit einer Funktion definiert.

\begin{equation*}
    I = \frac{F}{C}
\end{equation*}
Mit:
\begin{itemize}
    \item $I$: Idealität
    \item $F$: Verrichtete Arbeit der Funktion
    \item $C$: Ausführungskosten der Funktion
\end{itemize}

Eine Funktion ist definiert durch die Ausübung einer Aktion durch ein Arbeitswerkzeug an einem Arbeitsobjekt.



Es werden drei Problemtypen im modernen Ingenieurwesen identifiziert:

\begin{enumerate}
    \item Lösung dringender technischer Probleme
    \item Vorhersage der zukünftigen Entwicklung technischer Systeme
    \item Patente umgehen bzw. absichern
\end{enumerate}

Die Behandlung von Patenten und deren Umgehung ist ein wichtiges Motiv im Buch von Altshuller, da er in seiner Karriere oft darauf zurückgreifen musste. Dies unterscheidet ihn auch von den bisher behandelten Autoren der TRIZ.
Für die Lösung dieser drei Problemtypen wird eine strukturierte Wissensbasis bzw. ein strukturiertes Informationsfeld benötigt. Für ein solches Feld werden fünf Anforderungen von Shpakovsky genannt:

\begin{enumerate}
    \item Objektivität der Klassifikationskriterien
    \item Vollständigkeit (Das Vorhandensein aller sich signifikant unterscheidenden Versionen)
    \item Geeigneter Generalisierungsgrad
    \item Visualisierbarkeit
    \item Ausreichende Beschreibung bzw. Voraussage noch nicht existierender Versionen
\end{enumerate}

\subsection{Keine Gesetze sondern Empfehlungen}
Shpakovsky spricht bei den aufgestellten Konzepten des Evolutionsmusters und -baums nie von Gesetzen. Sondern von Anforderungen, Regeln und insbesondere bei Konstruktionsanleitungen von Empfehlungen. Dabei weicht er die Objektivität eigenhändig auf. Eine wirklich explizite Erklärung dieses Wandels vom Gesetz zur Empfehlung findet nicht statt, der Umstand kann aber anhand des erstellten Evolutionsbaums des Bildschirms nachvollzogen werden.
Der Stamm eines Evolutionsbaumes sollte beispielsweise nur aus einem Evolutionsmuster bestehen (vgl. \cite[S. 122f]{evo}), es wird aber deutlich das beim Bildschirm zwei Evolutionsmuster als Stamm dienen, nämlich das Trimmen und das Segmentieren.
Hier ist es, aufgrund der Beschaffenheit des betrachteten technischen Systems, angebracht die Empfehlung nicht zu befolgen. Dies wäre bei einem Gesetz nicht möglich bzw. es dürfte gar nicht aufgrund der Natur eines Gesetzes vorkommen.


\section{Evolutionsmuster}

\begin{enumerate}
    \item Mono-Bi-Poly
    \item Trimmen
    \item Expandieren und Trimmen
    \item Segmentierung
    \item Evolution von Oberflächeneigenschaften
    \item Evolution von inneren Strukturen
    \item Geometrische Evolution
    \item Dynamisierung
    \item Erhöhung der Kontrollierbarkeit
    \item Erhöhung der Koordination der Aktionen
\end{enumerate}

Aus diesen zehn grundlegenden Evolutionsmustern können spezifischere Evolutionsmuster erstellt werden. Die Evolutionsmuster von eins bis vier sind Muster, welche Ressourcen für andere Evolutionsmuster bereitstellen. Es ist z. B. nicht möglich einen unsegmentierten Monolith zu dynamisieren. Die Struktur des Objektes wird durch die Muster fünf bis sieben vorgegeben. An sinnvoll erscheinenden Punkten werden Muster zur Dynamisierung, Kontrollierbarkeit und Koordination eingefügt. Diese hierarchische Struktur der Transformationen ist in Abb. \ref{fig:basic_evo} verdeutlicht.

\begin{figure*}[htb]
	\centering
	\includegraphics[width=0.9\linewidth]{figures/Basisevolutionsbaum2.png}
	\caption{\small Grundlegender Evolutionsbaum \cite{evo}}
	\label{fig:basic_evo}
\end{figure*}

Eine Transformation ist die Anwendung eines Evolutionsmusters auf ein Objekt, welches darauf folgend zu einer transformierten Version des Objektes wird.

\section{Evolutionsbaum}

Es existieren grundlegende und spezifische Evolutionsbäume. Grundlegende Evolutionsbäume stellen eine in einem Baum organisierte Menge von Evolutionsmuster generalisierter Eigenschaften technischer Objekte dar. Der Start erfolgt von der simpelsten Version des Objektes z. B. ein Monolith. Die Hauptachse der Entwicklung wird auch als Stamm bezeichnet, als Stamm werden Ressourcen bereitstellende Evolutionsmuster wie die Segmentierung bevorzugt, da diese die Bedingung sind für eine spätere Dynamisierung.


\subsection{Erfüllung der Anforderungen an ein strukturiertes Informationsfeld}

Die behandelnden Konzepte decken alle gestellten Anforderungen an ein strukturiertes Informationsfeld ab. Objektivität ist gewährleistet durch die Herleitung von einer großen Menge realer Systeme ähnlich wie in der TRIZ. Durch die Verwendung des grundlegenden Baums für das Finden aller Basisversionen eines Objektes ist die Vollständigkeit gewährleistet. Für eine geeignete Abstraktion btw. Spezifizität wird je nach Anforderung ein grundlegender oder spezifischer Evolutionsbaum verwendet. Die Visualisierbarkeit ist gewährleistet durch die Baumstruktur. Lücken bzw. nicht vollendende Evolutionsmuster können durch einen Vergleich der grundlegenden und spezifischen Evolutionsbäume gefunden und beschrieben werden.

\subsection{Ermittlung nach nicht bekannter Versionen}



\begin{figure*}[htb]
	\centering
	\includegraphics[width=0.9\linewidth]{figures/removable display.PNG}
	\caption{\small Ausschnitt des spezifischen Evolutionsbaums vom Bildschirm \cite{evo}, siehe \url{http://www.target-invention.com/} für den kompletten Baum}
	\label{fig:spez_evo}
\end{figure*}


Für die Analyse eines Objektes muss sowohl der grundlegende (siehe Abb. \ref{fig:basic_evo}) als auch der spezifische Evolutionsbaum (siehe Abb. \ref{fig:spez_evo}) erstellt werden. Durch den Vergleich der beiden Bäume können sowohl Lücken als auch nicht nicht zu Ende geführte Evolutionsmuster entdeckt werden. Die höchste Stufe des Musters der Dynamisierung besteht aus einer kompletten Entkopplung der einzelnen Bestandteile. Bei einem Laptop würde dies bedeuten, dass man Bildschirm und Peripheriegeräte trennen kann. Zur Zeit der Erstellung des Evolutionsbaums des Bildschirms um 2002 gab es diese Version des Objektes noch nicht. Durch Erkennung dieser Lücke wurde eine sinnvolle neue Version gefunden. Die komplette Dynamisierung wird heutzutage erreicht, indem die Rechentechnik in den Bildschirm integriert wird und die Peripherie über Bluetooth verbunden wird. Damit wurde gezeigt, dass Evolutionsbäume in der Lage sind zukünftige Entwicklungen abzubilden.

\section{Patentumgehung}

Es besteht oft das Problem, dass es für ein gewünschtes Produkt bereits ein Patent gibt. In dieser Situation ist es entweder möglich hohe Lizenzkosten an den Patentinhaber zu zahlen oder das Patent zu umgehen.
Die juristische Methode der Patentumgehung, welche daraus besteht Schlupflöcher und fehlerhafte Patentbeschreibungen zu nutzen um ein Patent zu invalidieren, ist nicht immer anwendbar.
Es ist alternativ möglich das untersuchte Objekt abzuändern um ein besseres Produkt zu entwickeln. Diese erfinderische Methode hat den Nachteil, dass man große Entwicklungskosten hat und den grundlegenden Aufbau ändern muss. Andererseits ist es nicht möglich ohne Abänderung ein alternatives Patent zu bekommen.
Aus diesem für die TRIZ typischen Konflikt geht eine Synthese in Form der juristisch-erfinderischen Methode hervor. Diese neue Methode ziel darauf ab durch Evolutionsbäume noch nicht von Patenten abgedeckte Transformationsversionen zu finden.
Die Suche nach existierenden Patenten kann zusätzlich erleichtert werden, indem man die Objekt- und Transformationsnamen als Schlüsselworte verwendet.


\section{Diskussion}

Folgende Punkte können weiterführend diskutiert werden:
\begin{itemize}
    \item Sind Evolutionsbäume auch für Organisationsstrukturen geeignet?
    \item Eigene Ideen zum Evolutionsbaum
    \item Shpakovsky nutzt Empfehlungen statt Konzepte, steht damit das Konzept der Evolutionsbäume auf wackeligen Füßen?
\end{itemize}

Erläuterung zum ersten Punkt: Shpakovsky verwendet bei der Beschreibung des Mono-Bi-Poly Evolutionsmusters das Beispiel eines (vgl. \cite[S. 77]{evo}) Schiffgeschwaders. Ein Geschwader bestehend aus mehreren Schiffen unter einem gemeinsamen Befehlshaber stellt ein Poly-System dar. Da es sich dabei aber nicht definitiv um ein technisches System, sondern um eine Organisationsform, handelt, stellt sich die Frage ob Evolutionsbäume auch geeignet sind um Organisationsstrukturen darzustellen. 

% 2D --> Baum
% Schärfung des Idealitätsbegriffs

\bibliographystyle{unsrt}
\bibliography{references.bib}

\end{document}
