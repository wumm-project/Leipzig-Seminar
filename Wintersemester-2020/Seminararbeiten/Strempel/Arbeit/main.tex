\documentclass[11pt,a4paper]{article}
\usepackage{ls}
\usepackage[main=english,german,russian]{babel}
\usepackage[utf8]{inputenc}
\usepackage{hyperref}

\newenvironment{code}{\tt \begin{tabbing}
\hskip12pt\=\hskip12pt\=\hskip12pt\=\hskip12pt\=\hskip5cm\=\hskip5cm\=\kill}
{\end{tabbing}}
\def\dq{{\char34}}

\title{A Proposal for Modelling TRIZ System Evolution Concepts}

\author{Tom Strempel}

\date{March 31, 2021}

\begin{document}
\maketitle

\section{Aim of the work}

The aim of this paper is to elaborate a proposal for an ontological modelling
of the areas of \emph{TRIZ System Evolution Concepts} based on the approaches
in \cite{TESE2018} and \cite{Shpakovsky2016} and further own investigations.
The work fits into the activities of the \emph{WUMM Ontology Project}
\cite{WUMM} to model core TRIZ concepts using modern semantic web means.  The
work consists of two parts -- a \emph{turtle file}, in which the semantic
modelling is performed based on the SKOS framework \cite{SKOS}, and \emph{this
  elaboration}, in which the backgrounds and motivations of the concrete
modelling decisions are detailed.

\section{Starting point} 

The central concern of practical TRIZ applications is the analysis, evaluation
and transformation of systems in order to improve their operational behaviour.
As in the lecture, the term transformation is understood in a broad sense and
also includes the planning and design of new systems as the transformation of
a system that is only available as vague conceptual requirements into a system
that operates in the real world. TRIZ provides a whole methodological toolbox
that can be used together with domain-specific concepts for the systematic
planning and implementation of such a transformation task. In the seminar we
observed that this TRIZ toolkit is embedded in broader reasoning contexts in
which engineering experience and scientific knowledge are systematised and
generalised.

One of the aspects examined in this context is the evolution of classes of
engineering systems in a historical context in order 
\begin{enumerate}
\item to extract repeating patterns of engineering procedures as «laws»
  \cite{Altshuller1979}, «laws, evolutionary lines and trends» \cite{KS} or
  just «engineering trends» \cite{TESE2018} or
\item to identify evolutionary connections in the unfolding of the history of
  technology \cite{Shpakovsky2016}.
\end{enumerate}

Exploring this aspect, the focus on the exact form of the transformation of a
single system described above was left and, in the style of \emph{distant
  reading}, a variety of information about historical transformations in
different classes of systems has to be analysed in order to extract
transformation patterns from it.  If, for example, the «development of
display» \cite[p. 22]{TESE2018}, \cite[ch. 5]{Shpakovsky2016} is analysed,
this is based on a much stronger abstraction of the system concept compared to
the system concept of classical TRIZ modelling, even if this more
comprehensive abstraction is only rarely explicated in the relevant works --
for example as a \emph{class of systems} in the narrower sense. In the rest of
this paper, the concept of system is used in the same vague generality of an
intuitive understanding as an externally given (metaphysical) concept as in
the referenced works, without attempting to go into more details.

The central to TRIZ understanding, that engineering achievements can be
conceptualised as system transformations, leads in the analysis of historical
technology development to the structure of a directed graph with the
prototypical link
\begin{center}\tt
  OldSystem \textrm{---}\fbox{isTransformedInto}$\to$ NewSystem
\end{center}
In the first approach, this graph is considered as a set of such links to be
classified. The graph structure plays a subordinate role, because even in the
concept of \emph{development line} a rather linear progression is postulated
(e.g. \cite[Figure 4.104]{KS}, but see \cite[4.8.4 and Figure 4.72]{KS}). In
the second approach \cite{Shpakovsky2016}, the graph structure is considered
more consistently, but also with the aim to classify the links in more detail.

The aim of these conceptualisations is on the one hand to develop the
methodology of \emph{evolutionary potential analysis} \cite[4.8.7]{KS} and on
the other hand to consolidate and improve the central TRIZ tools such as the
40 application standards («principles») or the 76 inventive standards.

\section{The Conceptualisations}

The conceptualisations to be developed follow the basic assumptions and
positings that are elaborated in more detail in \cite{Graebe2021}. In
particular, the following namespace prefixes are used:
\begin{itemize}[noitemsep]
\item \texttt{ex:} -- the namespace of a special system to be modelled. 
\item \texttt{tc:} -- the namespace of the TRIZ concepts (RDF subjects).
\item \texttt{od:} -- the namespace of WUMM's own concepts (RDF predicates,
  general concepts). 
\end{itemize}
Furthermore the \texttt{SKOS} ontology is used to model labels and definitions
of the object.

Our central task is to model the links in concrete evolutionary trees. The
full evolution tree as an edge-marked graph then can be conceptualised as a
set of such links in the usual way.

A link in such a concrete evolution graph has the typical shape
\begin{center}\tt
  ex:TVWithLargePixels ex:decreasePixelSize ex:TVWithMediumPixels .
\end{center}
where the transformation predicate \texttt{ex:decreasePixelSize} is assigned
to certain evolution patterns (even several).
\begin{code}\tt
ex:decreasePixelSize \\
\> a rdf:Property, skos:Concept ; \\
\> od:usesPattern tc:SegmentationPattern ; \\
\> skos:prefLabel "Decrease pixel size"@en ; \\
\> skos:definition """Decrease pixel size by segmentation \\
\>\> of one big pixel in several smaller ones"""@en .
\end{code}

\section{Basic concepts}

In this section the concepts developed by Shpakovsky in \cite{Shpakovsky2016}
are introduced and discussed.

\subsection{Structured information field}

There are three types of problems in modern engineering: the solution of
urgent technical problems, the forecast of the evolution of technical system
and patent protection or circumvention. While the importance of the first two
types are very easy to understand for non-engineers, the third type requires
some explanation. If a technical system is patented one must either pay the
owner of the patent or develop a competing system which is not covered by the
patent, both options are money and labor intensive. If such measures are not
taken a company risks to be sued out of contracts.

An example for that would be the ongoing case between Heckler \& Koch (HK) and
C. G. Haenel over the production of the new standard issue weapon of the
German army. The case is centered around the over-the-beach capability of the
weapon, which means that the weapon can still be fired after being submerged
in water for a short time. If the water is not removed the weapon can jam or
missfire. HK patented a solution for this problem in
\href{https://patents.google.com/patent/EP2018508B1/en}{EP2018508B1} by adding
a fluid passage to the recoil spring mechanism. HK argues that the system used
in the Haenel Mk556 is a violation of it's patent, while Haenel says it is
distinct from this patent. HK managed to kick Haenel out of the procurement
process with this argumentation. Haenel thus lost a big order because of a
(perceived) patent violation. This example illustrates very well why patent
protection and circumvention is an essential part of modern engineering.

Conventional solution methods such as trial and error and brainstorming are
not suited for the requirements of these three problems. A higher level system
the \textit{science of invention} must be created to adequately tackle these
problems. This system must obey the five requirements listed below:

\begin{enumerate}[noitemsep]
\item Objective classification criteria (objectiveness)
\item The presence of all significantly different versions (fullness)
\item Suitable degree of generalisation and specificity
\item Visualization (to find gaps for patent circumvention)
\item Sufficient description or prediction of not yet existing versions
  (informativity)
\end{enumerate}

In the following sections the concept of the evolution pattern and tree are
described which will fulfill these requirements.

\subsection{Evolution patterns and trees}
% scope is object, not the technical system

\begin{enumerate}[noitemsep]
\item Mono-Bi-Poly
\item Trimming
\item Expanding-trimming
\item Segmentation
\item Geometrical evolution
\item Object structure evolution
\item Evolution of surface properties
\item Dynamization
\item Increasing the controllability
\item Increasing the coordination of the elements
\end{enumerate}

From these ten basic evolution patterns, more specific evolution patterns can
be created. The evolution patterns from one to four are patterns that provide
resources for other evolution patterns. For example, there is no possibility
for dynamization on an unsegmented monolith. The structure of the object is
given by patterns five to seven. Patterns for dynamization, controllability,
and coordination are inserted at points that seem reasonable. This
hierarchical structure of transformations is shown in
Fig. \ref{fig:basic_evo}.  It is not required to follow an evolution pattern
to it's end before applying a different one. The direction of evolution is
also not strictly given.

\begin{figure*}[htb]
  \centering
  \includegraphics[width=\textwidth]{figures/basictree.png}
  \caption{\small Basic evolution tree \cite{Shpakovsky2016}}
  \label{fig:basic_evo}
\end{figure*}

A basic principle in TRIZ is the interaction of a tool and an (work) object:

\begin{center}\tt
  Tool \textrm{---}\fbox{interactsWith}$\to$ Object
\end{center}

Analogous to this a transformation is the application of an evolutionary
pattern to an object, which subsequently becomes a transformed version of the
object. Equivalents of these Patterns can be found in the TRIZ principles
e. g. the Segmentation Pattern corresponds to the first TRIZ principle
\textit{Principle of decomposition or segmentation}.

An evolution tree is build out of multiple patterns which are combined at
certain locations (see Fig. \ref{fig:basic_evo}). The positions of these
locations are specific for each specialized evolution tree and are only shown
in approximation here.

The Evolution Tree is a self-similar concept, e. g. an object is approximately
similar to itself. An Evolution tree can thus contain another evolution tree,
e. g. the evolution tree of the screen contains the evolution tree of a plasma
screen, which could be analysed further.

\subsection{Outside influence}

The transition between some steps of evolution patterns require outside
development, e. g. the transition from a changeable image (flip-book cinema)
to the cinematographer was a joint product of many inventors. Outside
involvement is so required for adding and evolving objects e. g. in the
mono-bi-poly, segmentation and expanding-trimming pattern.

In a discussion with Shpakovsky it was clarified that this outside influence
can be seen as taking the same and new components from a super system, which
is outside the scope of the model. Components are selected on the basis of
their benefit in increasing productivity and other quality parameters.

\subsection{Fulfillment of the requirements for a structured information
  field} 

The concepts covered cover all requirements for a structured information
field. Objectivity is guaranteed by derivation from a large set of real
systems similar to TRIZ. Completeness is guaranteed by using the basic tree
for finding all basic versions of an object. For a suitable abstraction
btw. specificity a basic or specific evolution tree is used depending on the
requirements. Visualizability is guaranteed by the tree structure. Gaps or not
completing evolution patterns can be found and described by comparing the
basic and specific evolution trees.

\section{Further concepts}

\subsection{Not laws but recommendations}

Shpakovsky never calls his concepts of the evolution pattern and tree laws but
uses the terms requirements, rules and, in context with construction
instructions, recommendations. Thereby he himself softens the objectivity of
his concepts. A really explicit explanation of this change from law to
recommendation does not take place, but the circumstance can be understood on
the basis of the created evolution tree of the screen.

The trunk of an evolutionary tree, for example, should consist of only one
evolutionary pattern (cf. \cite[p. 122f]{Shpakovsky2016}), but it becomes
clear that in the case of the screen two evolutionary patterns serve as the
trunk, namely trimming and segmenting.

Here it is appropriate, due to the nature of the object, not to follow the
recommendation. This would not be possible with a law or it should not occur
at all due to the nature of a law.

\subsection{Determination of not known versions}

\begin{figure*}[htb]
  \centering
  \includegraphics[width=\textwidth]{figures/removable display.PNG}
  \caption{\small Section of the specific evolution tree of the screen
    \cite{Shpakovsky2016}, see \url{http://www.target-invention.com/} for the
    complete tree}
	\label{fig:spec_evo}
\end{figure*}

For the analysis of an object, both the basic (see Fig. \ref{fig:basic_evo})
and the specific evolutionary tree (see Fig. \ref{fig:spec_evo}) must be
created. By comparing the two trees, gaps as well as unfinished evolutionary
patterns can be discovered. The highest level of the pattern of dynamization
consists of a complete decoupling of the individual components. For a laptop,
this would mean separating the screen and peripherals. At the time the
evolutionary tree of the screen was created around 2002, this version of the
object did not yet exist. By recognizing this gap, a useful new version was
found. Nowadays, complete dynamization is achieved by integrating the
computing technology into the screen and connecting the peripherals via
Bluetooth. Thus it was shown that evolution trees are able to map future
developments.

\subsection{Patent circumvention}

There is often the problem that a patent already exists for a desired
product. In this situation it is either possible to pay high license fees to
the patent owner or to circumvent the patent.

The legal method of patent circumvention, which consists of using loopholes
and erroneous patent descriptions to invalidate a patent, is not always
applicable.

It is alternatively possible to modify the object under investigation to
develop a better product. This inventive method has the disadvantage of having
large development costs and having to change the basic design. On the other
hand, it is not possible to obtain an alternative patent without modification.

From this conflict, typical for TRIZ, a synthesis emerges in the form of the
legal-inventive method. This new method aims at finding transformation
versions not yet covered by patents through evolution trees.

The search for existing patents can be additionally facilitated by using the
object and transformation names as keywords.  

\section{Modelling}

Ontologies are based on the modelling on models. This is done over several
levels, where the lower levels are used to model real-world examples and thus
have a high level of specificity. Higher levels are used to model concepts and
even more general concepts. In this work three levels are used for modelling
which results in three RDF namespaces: 

\begin{itemize}[noitemsep]
\item \texttt{ex:} -- Level 1 (Real world examples and patterns)
\item \texttt{tc:} -- Level 2 (Subjects and concepts)
\item \texttt{od:} -- Level 3 (Predicates) 
\end{itemize}

\subsection{Modelling the evolution tree concepts}

File \textit{EvolutionTree.ttl} contains the description for the concepts for
evolution trees presented in \cite{Shpakovsky2016}.

All RDF subjects or nodes in the corresponding graph are part of the
\texttt{tc:} namespace. New predicates in \texttt{od:} didn't need to be
modelled because the already existing ones covered every need. Every chapter
and subsection in the table of contests is modelled with at least one subject
or triple, e. g. the segmentation pattern has it's own RDF subject
\texttt{tc:SegmentationPattern} and is described in it.

\begin{code}\tt
tc:SegmentationPattern \\
\> od:subConceptOf tc:BasicEvolutionPattern ; \\
\> od:hasSubConcept tc:Monolith, tc:TwoParts, tc:ManyParts, tc:Granules, \\
\> tc:Powder, tc:Paste, tc:Liquid, tc:Foam, tc:Fog, tc:Gas, tc:Plasma, \\
\> tc:Field, tc:Vacuum, tc:IdealObject ; \\
\> a skos:Concept, od:AdditionalConcept ; \\
\> skos:prefLabel "Segmenting objects and substances"@en ; \\
\> skos:example "Segmentation of an aircraft propulsion unit"@en ; \\
\> skos:broader tc:Segmentation, tc:TrendofTransitiontoMicrolevel .
\end{code}

\begin{code}\tt
tc:Liquid \\
\> od:subConceptOf tc:SegmentationPattern ; \\
\> a skos:Concept, od:AdditionalConcept ; \\
\> skos:prefLabel "Liquid"@en . \\
\end{code}

Relations or edges between subjects are modelled by the predicates
\texttt{od:subConceptOf} and \texttt{od:hasSubConcept}, e. g. as
\texttt{tc:SegmentationPattern} is a \texttt{tc:BasicEvolutionPattern} there
are linked together. Different transformation versions like
\texttt{tc:Monolith} are also referenced that way. As TRIZ trends are used as
the basic evolution patterns they must be referenced from the modelled
patterns, e. g. \texttt{tc:Segmentation} and
\texttt{tc:TrendofTransitiontoMicrolevel} are referenced by
\texttt{tc:SegmentationPattern} via \texttt{skos:broader}. Evolution pattern
are more specific than their corresponding TRIZ principles because they are
seen in the context of the evolution tree and thereby \texttt{skos:broader} is
used instead of \texttt{skos:narrower}.  

\begin{code}\tt
tc:Segmentation \\
\> od:hasRecommendation tc:Segmentation\_1, tc:Segmentation\_2,
tc:Segmentation\_3 ; \\ 
\> od:hasAltshuller73Id "01" ; \\
\> od:hasAltshuller84Id "01" ; \\
\> a od:Principle ; \\
\> rdfs:label "Principle of decomposition or segmentation"@en .
\end{code}

Triples usually consist of the RDF subject (TRIZ concept), the referenced
subjects via predicates, the \texttt{a} predicate, and further skos labels,
examples and definitions. All information about one subject is modelled inside
a single triple. The file also contains comments marked with \texttt{\#} where
the currently modelled part is marked or further described to keep track.

\begin{code}\tt
tc:FrontalSearch \\
\> od:subConceptOf tc:InformationFieldSearch ; \\
\> a skos:Concept, od:AdditionalConcept ; \\
\> skos:prefLabel "Frontal search of an Information Field"@en ; \\
\> skos:definition """Search starts from random points. \\
\>\> Search of the whole information field for relevant information."""@en .
\end{code}

\subsection{Modelling the screen evolution tree}

Shpakosvky had the example of the 

\subsection{Modelling the ship propulsion evolution tree}

Souchkov in \cite{KS} describes the evolution tree using the example of the
boat. This is done via the standard TRIZ methodology and can be partly
implemented in Shpakovsky's more specific concept of an evolution tree. A boat
as a technical system has a very wide array of transformations. Thus it is
vital to specify the elemental function by looking at the main axis of the
evolution, the tree trunk.

Souchkov splits the transformations into three categories: New transformations
for delivering the main function, existing transformations that could be
developed further and completed or discontinued transformations. We are
interested in the new transformations for delivering the main function as this
is used as the main axis of development. Developments follows through the
transformations tree trunk, rowboat, sailboat, steamboat, diesel-boat,
water-jetboat and atom-boat. Hence the corresponding elemental function is
\textit{provide the boat with a power source} as they all, with one exception,
describe what the engine or power source is. A tree trunk has no power source,
a rowboat uses muscle power, a sailboat the wind, a steamboat a steam machine
and so on. As the water-jet is a means of propulsion and does not describe the
power source, but how the power is used for propulsion (e. g. propeller,
paddle wheel), it is misplaced on the evolution tree trunk.

Modelling this discrepancy is done via using a different evolution pattern for
the transition between the diesel-boat and water-jet-boat, namely the
expanding-trimming pattern. Consequently the transition is not part of the
modelled evolution tree, while all other transitions between the main
transformations are. 

The granularity of this tree is very coarse. The transitions between the main
versions e. g. from sailboat to steamboat are very big steps in the overall
technical development. First ships with sails were developed in the 2nd
millennium BCE in the South China Sea, while the first practical steamers were
build in the early 19th century, this massive time frame shows that the tree
is very coarsely build. Because of that transitions or predicates are very
specific as they must depict large developments developments. Finer grained
evolution trees have the advantage of reusing predicates, this is not
applicable here. Again we are running into the self similarity of the
evolution tree as one tree node could be easily expanded, e. g. the torpedo
boat node could be expanded to include all development on torpedo boats.

The terms boat and ship are used interchangeably in the evolution tree and
this approach was further adopted.

Thus the predicates can only applied for this example and thus not be easily
used for other objects. The branching transformations can also be seen as
evolution trees themselves as they describe another elemental function. 

Along the evolution tree trunk the segmentation pattern is used. The basic
tree trunk is a single not dynamic object and thus a \textit{Monolith}. By
making paddles out of a part of the tree and forming the rest into a boat we
get the rowboat (\textit{Two Parts} in the segmentation pattern).   

\begin{figure*}[htb]
  \centering
  \includegraphics[width=\textwidth]{figures/boat.png}
  \caption{\small Souchkov's evolution tree of the boat translated from German
    \cite{KS}}
	\label{fig:boat}
\end{figure*}
\begin{tabular}{r@{: }l}
\textbf{Bold font} & new transformations for delivering the main function\\
Normal font & transformations that can be developed further\\
\textit{Italic font} & discontinued transformations
\end{tabular}

\begin{code}\tt
ex:Boat \\
\> a tc:SpecificEvolutionTree ; \\
\> skos:prefLabel "Boat"@en, "Boot"@de ; \\
\> skos:altLabel "Power source of the boat", \\
\>\> "Energiequelle bzw. Motor des Boots" ; \\
\> skos:definition "Specific evolution tree of the boat power source"@en, \\
\>\> "Spezifischer Evolutionsbaum der Energiequelle bzw. Motors des Boots"@de
.  \\[4pt]

ex:Rowboat \\
\> a ex:Boat ; \\ 
\> ex:addPaddles ex:BoatWithManyRowers ; \\
\> ex:addRecreationalInstallations ex:RecreationalRowboat ; \\
\> ex:replacePaddleWithSail ex:Sailboat ; \\
\> skos:prefLabel "Rowboat"@en ; \\
\> skos:definition "Manual labour as the power source"@en . \\[4pt]

ex:replacePaddleWithSail \\
\> a rdf:Property, skos:Concept ; \\
\> od:usesPattern tc:SegmentationPattern, tc:Gas ; \\
\> skos:prefLabel "Replace Paddle with sail"@en ; \\
\> skos:definition """Rudder and thus muscle power is replaced by sails \\
\>\> and thus wind power"""@en .
    
\end{code}

The transformations that can be developed further can be described as
incomplete evolution patterns. The expanding-trimming pattern is used to
describe these branching transformations because if one for example converts a
ship for military usage components (weapons, armour etc.) are added while
other components (excess weight, cargo bays etc.) are removed or trimmed. 

\subsection{Evolution tree of the aircraft propulsion device}

\begin{figure*}[htb]
  \centering
  \includegraphics[width=0.75\textwidth]{figures/aircraft.png}
  \caption{\small Shpakovsky's evolution tree of the aircraft propulsion
    device \cite{Shpakovsky2003}}
	\label{fig:aircraft}
\end{figure*}

As a further example Shpakovsky provides is the evolution of the aircraft
propulsion device. Airplanes don't include sailplanes as it is stated that an
airplane must use an engine. This engine drives the propulsion device, in it's
earliest form a propeller, which in turn moves the aircraft forward. The
elemental function of the object \textit{aircraft propulsion device} is
defined as \textit{A propulsive device is what an aircraft uses to push off
  the surrounding space while flying.} \cite{Shpakovsky2003}.

The starting point of the evolution is the single-bladed propeller, which
corresponds to the monolith in the segmentation pattern. Subsequently the two
parts transformation would be a double bladed propellers. The Evolution Tree
Trunk consists here of the entire tree as the object develops along the
segmentation pattern.

For granules, liquid and vacuum exists no current transformation. These gaps
can be used to theorise about potential inventions, cost and feasibility must
also be considered to get a good solution. While for example water can be
sprayed into the propeller area to increase the thrust, this can only be done
for short periods as water is expensive to carry all the time. Throwing away
rocket stages for further acceleration of a space craft are also proposed for
the granules transformation, but fuel tanks bear little resemblance to
granules. The granules, liquid and gas transformations are thus left empty on
the tree and are thus not modelled. 

The same way as described from before is used to model the evolution tree. An
example for adding an extra propeller row is provided below. 

\begin{code}\tt
ex:AircraftPropulsionDevice \\
\> a tc:SpecificEvolutionTree ; \\
\> skos:prefLabel "Propulsion device for aircraft"@en ; \\
\> skos:definition "A propulsive device is what an aircraft \\
\>\> uses to push off the surrounding space while flying"@en .
\end{code}
\begin{code}\tt
ex:MultiBladePropeller \\
\> a ex:AircraftPropulsionDevice ; \\
\> ex:addPropellerRow ex:DoubleRowPropeller ; \\
\> skos:prefLabel "Multiblade propeller"@en .
\end{code}
\begin{code}\tt
ex:addPropellerRow \\
\> a rdf:Property, skos:Concept ; \\
\> od:usesPattern tc:SegmentationPattern, tc:ManyParts ; \\
\> skos:prefLabel "Add propeller row"@en .
\end{code}

\begin{thebibliography}{xxx}
\raggedright
\bibitem{Altshuller1979} Genrich Altshuller (1979).  Creativity as an exact
  science (in Russian). English version: Gordon and Breach, New York 1988.
\bibitem{Graebe2021} Hans-Gert Gr\"abe (2021). About the WUMM modelling
  concepts of a TRIZ ontology.  \url{https://github.com/wumm-project/Leipzig-Seminar/blob/master/Wintersemester-2020/Seminararbeiten/Anmerkungen.pdf}.
\bibitem{KS} Karl Koltze, Valeri Souchkov (2017).  Systematische
  Innovationsmethoden (in German).  Hanser, Munich. ISBN 978-3-446-45127-8.
\bibitem{TESE2018} Alex Lyubomirsky, Simon Litvin, Sergei Ikovenko et al.
  (2018). Trends of Engineering System Evolution (TESE).  TRIZ Consulting
  Group. ISBN 9783000598463.
\bibitem{Shpakovsky2016} Nikolay Shpakovsky (2016). Tree of Technology
  Evolution. English translation of the Russian original (Forum, Moscow
  2010).\\ \url{https://wumm-project.github.io/TTS.html}
\bibitem{Shpakovsky2003} One of the evolution trends of an aircraft propulsive device. \url{http://www.gnrtr.com/Generator.html?pi=211&cp=3}
\bibitem{SKOS} SKOS -- The Simple Knowledge Organization System.
  \url{https://www.w3.org/TR/skos-reference/}.  
\bibitem{WUMM} The WUMM Project. \url{https://wumm-project.github.io/} 
\end{thebibliography}

\end{document}
