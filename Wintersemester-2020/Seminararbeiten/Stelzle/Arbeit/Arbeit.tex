\documentclass[11pt,a4paper]{article}
\usepackage{ls}

\usepackage{float}
\usepackage{xcolor}
\usepackage[english]{babel}
\setlist{noitemsep}

\newenvironment{code}{\tt \begin{tabbing}
\hskip12pt\=\hskip12pt\=\hskip12pt\=\hskip12pt\=\hskip5cm\=\hskip5cm\=\kill}
{\end{tabbing}}

\title{A Proposal for Modelling TRIZ Functional Analysis}

\author{Tarek Stelzle}

\date{April 11,2021}

\begin{document}
\maketitle

\tableofcontents

\section{Aim of the work}

The aim of this paper is to elaborate a proposal for an ontological modelling
of the areas of \emph{TRIZ Functional Analysis} based on the approaches in
\cite{KS}, \cite{WebinarFunctionAnalysis}, \cite{SouchkovGlossary} and further
own investigations.  The work fits into the activities of the \emph{WUMM
  Ontology Project} \cite{WUMM} to model core TRIZ concepts using modern
semantic web means.  The work consists of two parts -- a \emph{turtle file},
in which the semantic modelling is performed based on the SKOS framework
\cite{SKOS}, and \emph{this elaboration}, in which the background and
motivation of the concrete modelling decisions are detailed.

The paper is structured in the following way.  In section
\ref{sec:starting_point} the information sources are mentioned and further
explained.  In section 3 the conceptualisation is shortly explained.  In
section \ref{sec:functional_analysis} the Functional Analysis as described in
\cite{KS} will be summarized.  The next section introduces Python tools which
have been implemented to make the \textit{turtle}-file creation easier.
Following it will be shown how and why the \textit{turtle}-file has been
created in that way.  In extension to the last section in section
\ref{sec:examples} implementations of some real world example based on the
\textit{turtle}-file for Functional Analysis are given.  In the last section
we draw some conclusions.

\section{Starting point} 
\label{sec:starting_point}

The starting point for the proposed ontology for Functional Analysis is the
book \textit{Systematische Innovationsmethoden} \cite{KS}.  The definitions,
demonstrations and explanations given there will be used to implement the
ontology.

Other definitions for Functional Analyis are picked from Valeri Souchkov's
\textit{Glossary Of TRIZ and TRIZ-Related Terms} \cite{SouchkovGlossary} as a
great summary with good definitions.

These two information sources and the following approach to build up an
ontology for Functional Analysis which is explained in more detail in a
Webinar of Nikolai Shchedrin are the starting point to build our own ontology
for Functional Analysis.

\subsection{Webinar of Nikolai Shchedrin}

Nikolai Shchedrin gave a talk about building an ontology for \textit{Function}
and \textit{Function Analysis} \cite{WebinarFunctionAnalysis} where he
presented some insights on how to structure the ontolgy.

\subsubsection{Function Classification}
\label{subsubsec:function_classification}

First he introduced a new classification for functions distinguishing three
types:
\begin{enumerate}
\item Function of the subsystem
\item Function of the upper system
\item Function of the surrounding objects
\end{enumerate}

\subsubsection{Model Of A Function}

Furthermore he introduced the \textit{Model of a function}.  With this a
function is further described.  Not only does it show the Action and the two
components which interact, but it furthermore shows the parameter, the type of
the function and the degree of execution.

The \textit{Functional Model} is a graph representation of the system which is
analysed.  Every node in the graph is a component or a subsystem of the
system.  The functions are represented by edges.  This will be explained in
more detail in \ref{subsec:functional_model}.

The \textit{Model of a function} can be helpful for building the
\textit{Functional Model}.

\subsubsection{Objectives Of A System}
\label{subsubsec:objectives_system}

As explained by Nikolai Shchedrin the objective of a system can be divided
into three classes:
\begin{enumerate}
\item Primary Objective
\item Secondary Objective
\item Auxiliary Objective
\end{enumerate}

The \emph{Primary Objective} is the objective, which the system was build for.

\emph{Secondary Objectives} of the system are functionalities which are
offered by the system, but for which it was not mainly built.

And the \emph{Auxiliary Objectives} fulfill the purpose to get the Primary
Objective to work.

\subsubsection{Primary Function}
\label{subsubsec:primary_function}

Furthermore a function can be classified in one of these three classes:
\begin{enumerate}
\item Primary Function
\item Core Function
\item Auxiliary Function
\end{enumerate}

The Primary Function is the Primary Objective and its technical execution. 

The Core Function represents a function, which directly helps executing the
Primary Function.

The Auxiliary Functions describe the set of functions which help run other
subsystems.

\section{The Conceptualisations}
\label{sec:conceptualisations}

The conceptualisations to be developed follow the basic assumptions and
settings that are elaborated in more detail in \cite{Graebe2021}. In
particular, the following namespace prefixes are used:
\begin{itemize}
\item \texttt{ex:} -- the namespace of a special system to be modelled. 
\item \texttt{tc:} -- the namespace of the TRIZ concepts (RDF subjects).
\item \texttt{od:} -- the namespace of WUMM's own concepts (RDF predicates,
  general concepts).
\end{itemize}

The concept will be to gather various definitions from the mentioned sources.
From these sources the important TRIZ triples will be implemented.

From there the necessary connection will be created for the various triples.
Furthermore missing concepts will be implemented to fully develop Functional
Analysis and related concepts.

For the organisation of the turtle file and to structure the building process
of the turtle file, first of all the Functional Model will be built.  This is
done by following the five steps, which are explained in
\ref{subsec:functional_model}.  With the given model the Functional Analysis
components can be added to attach all important concepts.

In the end the implementation will be supported by different example, which
are going to use the implemented concepts.  In figure
\ref{fig:example_conceptionalism} an implemented function of an example is
shown.

\begin{figure}[ht]
  \centering
  \begin{code}
    ex:Steer\\
    \> a tc:function ;\\
    \> tc:SubjectActionObject ex:DriverTurnSteeringwheel ;\\
    \> tc:QualityOfFunction tc:UsefulFunction ;\\
    \> skos:prefLabel "Steer"@en ;\\
    \> skos:Definition "Turning the steering wheel by the\\
    \> \> driver to change the direction of the vehicle."@en .
  \end{code}
  \caption{Example For Function \textit{Steer}}
  \label{fig:example_conceptionalism}
\end{figure}

\section{Functional Analysis}
\label{sec:functional_analysis}

In this section a summary about the terms and definitions explained in
\cite{KS} will be given.  This should help to explain in the following
sections, why the \textit{turtle}-file was modeled in that way.

\subsection{Concept}

In Functional Analysis the idea is to represent a system by various functions.
This representation helps in organising and structuring the system.  To tackle
the optimisation problem a more precise look is taken at the non-useful and
contradictory functions in the system.

The two main objectives of Functional Analysis in TRIZ are formulated in the
following.

\begin{enumerate}
\item Recognising Of Problems
  
  Hereby the objective is to find as many as possible non-useful or
  contradictory functions in the system.
\item Trimming Of Components
  
  To optimize the system some components have to be redesigned.  During this
  process the functionality of the component must not be changed.
\end{enumerate}

With these two main objectives there are five tasks to handle in Functional
Analysis.

\begin{enumerate}
\item Recognising Interactions Between Objects
\item Recognising Problems Within The System
\item Formulating Open Problems
\item Innovative Redesign
\item Optimising Systems By Trimming
\item Bypassing Patents
\end{enumerate}

\subsection{Quality Of A Function}

To further analyze the functions it is recommended to give each function a
level of quality.  This quality of a function makes it easier to find problems
within the system.

Accorind to the book there are five different quality levels.

\begin{enumerate}
\item Useful Function:
  
  A function works as intended and the result is only positive.
\item Useful, But Insufficient Function
  
  A function has a positive impact but the result is not satisfieing.
\item Useful, But Bad Controllable Function
  
  A function with a positive impact but is not satisfieing as the result
  cannot be controlled.  Hence it is wrongly timed.
\item Useful, But Excessive Function
  
  A function with a positive impact but a bigger result than necessary.
\item Harmful Function
  
  A function which has a negative impact.
\end{enumerate}

More precise definitions can be found in the book \cite{KS} or in the Glossary
of Souchkov \cite{SouchkovGlossary}.  As both of these are merged into the
\textit{turtle}-file the defintions can now also be found there.

Nikolai Shchedrin mentioned also a new function quality in his web-seminar.
This is called \textit{Useful Function With Disadvanteges}.  As mentioned on
the website \cite{ShchedrinUsefulFunctionDisadvantage} this class includes
Redundant Functions, Insufficient Functions, Bad Controllable Functions and
Missing Functions.

As there is no source mentining the Definition of a Redundant Function and a
Missing Function, these will be interpreted in the following way.

\begin{enumerate}
\item Redundant Function:
  
  A function with a positive result, which is implemented a second time in the
  system.
\item Missing Function:
  
  A useful function which is not implemented in the system and therefore
  missing.
\end{enumerate}

\subsection{Functional Model}
\label{subsec:functional_model}

The Functional Model structures the function and the components in the system.
In the book the following steps are recommended when building a Functional
Model.  It is also mentioned that steps two to four can be made in one step.

\begin{enumerate}
\item Making A List Of Components
\item Determine Interactions
\item Linking Functions To Components (Subject)
\item Determining The Direction Of Function (Arrow)
\item Define The Quality Of The Function
\end{enumerate}

With this a graph-like structure is built.  Every node in this graph
represents a component or subsystem within the system.  An edge represents a
function.  Different styles and forms of an edge represent the quality of the
function.

For easier reading and analyzing the Functional Model, it is possible to group
components according to self-defined properties.

\subsection{Positive And Negative Aspects}

Using the Functional Analysis helps you to fully understand the system you are working with.
Furthermore you can exchange knowledge between colleagues and clarify misunderstandings.

A Functional Analysis makes also sense, when no problem is known.
This is good if you want to improve your system without knowing specific complications.

A negative attitude of the Functional Analysis, is that only known systems can be analyzed.

\section{Implementation Of Tools}

\subsection{Creating Turtle File}

For easier and faster creation of the "Matrix 2003" a small python tool was
implemented.  This tool reads the matrix from a \textit{csv}-file and
generates a \textit{turtle}-file.

\subsubsection{The CSV-File}

First of alle \textit{csv} file needs to be in the following syntax, because
otherwise the matrix is wrongly interpreted.  Every line in the csv file is a
row in the matrix.  Furthermore every comma seperates a column entry in a a
row.  The different entries in the matrix field are seperated by dashes.

For reading the matrix the tool "tabula" was used \cite{tabula}.  With this
tool the Matrix2003 from the book "Systematische Innovation" was interpreted
as a txt file.  As this did not have the right syntax the small script, named
\textit{convertTxtMatrixToCsv.py}, was created.  With this the matrix is
converted into the right syntax, as mentioned above.

\paragraph{Using The Script}

Python will need to be installed on the computer.  The command to run the
script will then be:

\begin{center}
  python convertTxtMatrixToCsv.py $<$csv-file-name$>$
\end{center}

The output \textit{csv}-file will be called \textit{createdMatrix.csv}.  The
file \textit{temporary.txt} is only created for storing the information
temporary.  It can be deleted afterwards.

\subsubsection{Creating The Matrix}
\label{subsubsec:creating_matrix}

The following explains the implementation of the \textit{createMatrix.py}
script.

The script uses three other files.  The first is the \textit{principles.txt}.
In this file all the forty principles are written down, with their belonging
number at the beginning.  Hence the script can map an entry index to the
belonging principle name.

The second file, which is also used is called \textit{parameters.txt}.  In
there are all the forty-eight parameters which represent the column and rows
in the matrix.  These names are similar to those for the Altshuller Matrix.
The nine new parameters, which were introduced for the Matrix2003 are the
following.

As this matrix introduced upper-parameters, a third file for these is needed.
This has been called \textit{upperParameters.txt}.  The upper-parameters are
linked by writing the upper-parameter number with a dot before the parameters
in the \textit{parameters.txt}.  The python script then handles the rest.

\begin{enumerate}
\item QuantityOfInformation
\item FunctionEfficiency
\item Noise
\item HarmfulEmissions
\item CompatitbilityOrConnectability
\item Security
\item Vulnerability
\item AestheticsOrAppearance
\item ComplexityOfControl
\end{enumerate}

The layout for the Turtle file is the same as the layout for the "Altshuller
Matrix".  As this makes it easier for later adjustments or improvements of the
matrices.

At first the script creates the header for the Matrix2003, which is the same
as in the "Altshuller Matrix".  Then it defines the owl entry for the Matrix.
Afterwards the script iterates over the entries in the \textit{csv} file and
builds the matrix in the following way.

\begin{figure}[ht]
  \centering
  \begin{code}
    <http://opendiscovery.org/rdf/Matrix/E.06.34>\\
    \> od:upperDecreasingParameter tc:Non-Functionality ;\\
    \> od:decreasingParameter tc:EaseOfUse ;\\
    \> od:upperIncreasingParameter tc:Physical Parameters ;\\
    \> od:increasingParameter tc:SurfaceOfTheStationaryObject ;\\
    \> od:recommendedPrinciple tc:PreferredAction, tc:Asymmetry, \\
    \> \> \> tc:Mediator, tc:SelfService ;\\
    \> a od:MatrixEntry .
  \end{code}
  \caption{Entry of the Matrix2003 Turtle File}
\end{figure}

\paragraph{Using The Script}

For running the script python will need to be installed on the computer.  The
script can then be started with the following command:

\begin{center}
    python createMatrix.py $<$csv-file-name$>$
\end{center}

Afterwards there will be a file called \textit{created\_matrix.ttl}.  This
will be the result of the script.

\subsection{Translating}

For translating the various tags a python script was implemented, which takes
a turtle files as input and adds the missing language tags.  Therefore the
script checks each line which already has a language tag.  Then it makes an
api call with the first tag as the source language.  It adds the language tags
for English, German and Russian.  Each tag is a separate api call.  For
translation the Google Translator is used.

\paragraph{Using the Script}

Python has to be installed on the computer.  Furthermore does the script need
the \textit{deep\_translator} library \cite{deep_translator}.  The library can
be installed with the following command:

\begin{center}
    pip install deep-translator
\end{center}

Afterwards the script can be started.  \textit{$<$turtle-file-name$>$}
specifies the file for which the missing language tags should be added.

\begin{center}
    python translateText.py $<$turtle-file-name$>$
\end{center}

This will output a file name \textit{translated\_$<$turtle-file-name$>$.ttl}.
This is the new turtle file with the added language tags.  Furthermore all
translated words or sentences will be shown in the terminal output.

\section{Creating And Modeling the Turtle-File}
\label{sec:turtle_file}

As explained in the book "Systematische Innovationsmethoden" the Functional
Model for the Functional Analysis is built in five steps.  These steps have
been explained in \ref{subsec:functional_model}.  The explanation of the
implementation will also be structured in these five steps.

Each of these steps is also a new WUMM concept in the RDF graph. 

\begin{enumerate}
\item od:ListOfComponents
\item od:ListOfInteractions
\item od:FunctionForComponents
\item od:DirectionOfFunction
\item od:DefiningQualityOfFunction
\end{enumerate}

For each step the necessary implementation steps will be explained.

In the following all implementation examples will only include the English
language tags.  In the Turtle File itself there are also tags in Russian and
German.


\subsection{Making A List Of Components}

Within the first step for building the Functional Model all components which
are part of the system are listed.  Hence the system concept is implemented.

A system has different objectives, mentioned in
\ref{subsubsec:objectives_system}.  These objectives are implemented as TRIZ
components.  The \textit{hasPrimaryObjective}, \textit{hasSecondaryObjective}
and \textit{hasAuxiliaryObjective} WUMM concepts link a system to the
belonging objective.

The system consists of subsystems and components.  As a system can itself be a
subsystem of another system, also an upper system is implemented.

The relations between these are created via new WUMM triples.  These are
\textit{hasSubsystem}, \textit{hasUppersystem}, \textit{hasComponent} and
\textit{hasComponent}.

The objective of this step is to create a \textit{Component Model}.  In the
component model all component and subsystems of the system are listed.  As
this is a Functional Model without the functions, it is implemented as a
sub-concept of the Functional Model.  The additions to the Functional Model
will be added in the next steps.

Creating the Component Model is done with the Component Analysis.  This is
implemented as a sub-concept of the first step of Building the Functional
Model.

The link to the Component Model is done via the \textit{domain} and
\textit{range} properties.  As the Component Analysis starts with a system and
returns a Component Model.

The implementation of the Component Analysis is shown in figure
\ref{fig:implementation_component_analysis}.

\begin{figure}[ht]
  \centering
  \begin{code}
    tc:ComponentAnalysis\\
    \> a skos:Concept ;\\
    \> od:subConceptOf od:ListOfComponents ;\\
    \> od:SouchkovDefinition "A step in Function Analysis that identifies\\
    \> \> components of a technical system being analyzed and its supersystem."@en ;\\
    \> rdfs:domain tc:System ;\\
    \> rdfs:range tc:ComponentModel ;\\
    \> skos:prefLabel "Component Analysis"@en ;\\
    \> skos:altLabel "Component and Structural Analysis"@en .
  \end{code}
  \caption{Implementation Of Component Analysis}
  \label{fig:implementation_component_analysis}
\end{figure}

\subsection{Determine Interactions}

In the next step components, which in some kind interact with each other are
listed.  These are called Interactions and are implemented as a TRIZ concept.

The linking is done with a WUMM concept called \textit{hasInteraction}.  The
RDFS property domain and range are both Components and hence an interaction is
created.

All the interactions can be shown with an Interaction Matrix.  This is
mentioned in the implementation as a TRIZ concept.

The Interaction Matrix can be created with an Interaction Analysis.  The
Interaction Analysis is implemented, see figure
\ref{fig:implementation_interaction_analysis}, as a TRIZ concept.  And as this
leads to the result of the belonging interactions it is described as a
sub-concept of the second step.

\begin{figure}[ht]
  \centering
  \begin{code}
    tc:InteractionAnalysis\\
    \> a skos:Concept ;\\
    \> rdfs:domain tc:System ;\\
    \> rdfs:range tc:InteractionMatrix ;\\
    \> od:subConceptOf od:ListOfInteractions ;\\
    \> od:SouchkovDefinition "A part of Function Analysis that identifies\\
    \> \> interactions between components included in a Component Model."@en ;\\
    \> skos:prefLabel "Interaction Analysis"@en ;\\
    \> skos:altLabel "Structure Analysis"@en ;\\
    \> skos:altLabel "Funtion Analysis of the Process"@en ;\\
    \> skos:altLabel "Process analysis"@en ;\\
    \> skos:example: "The coffee production process is technically implemented\\
    \> \> by the components of the coffee machine."@en .\\
  \end{code}
  \caption{Implementation Of Interaction Analysis}
  \label{fig:implementation_interaction_analysis}
\end{figure}

\subsection{Linking Functions To Components}

In the next step functions will be linked to the belonging components.

Therefore a action is defined.  Furthermore a WUMM concept is added which
attaches an action to an interaction.

A function is defined as having a direction, but as this is going to be added
in the next step the term function is going to be used without these
functionality in this section.

The allowed values for constructing a function will be the TRIZ concept
Subject-Action-Object.  As this already includes the direction, this will be
explained in the following section (\ref{subsec:direction_function}).

The sub-concepts of a function are the various specifications of a function.
These are all summed up in the TRIZ concept \textit{QualityOfFunction}, which
is going to be explained in section \ref{subsec:quality_function}.

The following three functions are also implemented as a sub-concept, but are
not representing a quality of the function.  These three ordering where
introduced by Nikolai Shchedrin and have been explained in section
\ref{subsubsec:primary_function}.

\begin{itemize}
\item Main Function
\item Auxiliary Function
\item Core Function
\end{itemize}

Nikolai Shchedrin introduced the Model Of A Function, which will also be
implemented.  This model represents the function with two components, an
action, a parameter, the type of the function and the degree of execution.
The not already implemented \textit{Parameter} and \textit{Degree Of
  Execution} will be added as TRIZ concept.  Furthermore WUMM concepts will be
added, which attach these TRIZ concepts to a function.

As mentioned in section \ref{subsubsec:function_classification} Nikolai
Shchedrin also added three new types for classifying a function.  These types
were also added as a TRIZ concept.  With the WUMM concept
\textit{hasFunctionType} a function is connected with the according function
type.

\subsection{Determining The Direction Of The Function}
\label{subsec:direction_function}

The representation for the function is done with the concept of
Subject-Action-Object.  This is implemented as a TRIZ concept.  Therefore the
\textit{Function Carrier} and the \textit{Object of the Function} is needed.
Both are also embedded as a TRIZ concept.

To connect an object of a function with a function carrier, the new WUMM
concept \textit{hasFunctionCarrier} is introduced.  This is done similar for
the opposite direction.

An action can be attached to a direction with \textit{hasDirection}.  This
maps a action to a Subject-Action-Object triple.

\begin{figure}[ht]
  \centering
  \begin{code}
    tc:SubjectActionObject\\
    \> a skos:Concept ;\\
    \> od:subConceptOf tc:FunctionModeling ;\\
    \> od:hasSubConcept tc:FunctionCarrier, tc:ObjectOfFunction, tc:Predicate,\\
    \> \> tc:DirectionOfFunction ;\\
    \> od:SouchkovDefinition "A triad which identifies a Function Carrier,\\
    \> \> its specific Function, and Target Object."@en ;\\
    \> skos:prefLabel "Subject - Action - Object"@en ;\\
    \> skos:altLabel "Tool - action - product"@en .
  \end{code}
  \caption{Implementation Of Subject-Action-Object}
  \label{fig:implementation_subject_action_object}
\end{figure}

\subsection{Define The Quality Of The Function}
\label{subsec:quality_function}

The fifth and last step in building the Functional Model within the Functional
Analysis defines the quality of the given function.

Therefore, as mentioned before, the TRIZ concept \textit{QualityOfFunction} is
introduced.  This is a sub-concept of the function and in that way linked to
the function.

The following items define the quality of the function and are thus allowed
values in the \textit{QualityOfFunction}.

\begin{itemize}
\item Additional Function
\item Basic Function
\item Excessive Function
\item Harmful Function
\item Ideal Function
\item Insufficient Function
\item Neutral Function
\item Providing Function
\item Technical Function
\item Useful Function
\item Transport Function
\item Bad Controllable Function
\item Redundant Function
\item Missing Function
\end{itemize}

The explanation of those can be found in the book "Systematische
Innovationsmethoden" \cite{KS} and in the paper from Nikolai Shchedrin
\cite{WebinarFunctionAnalysis}.

The following two are introduced by Nikolaid Shchedrin and are a aggregation
of quality of functions from the above list.  These are also implmented as
TRIZ concept.

\begin{itemize}
\item Function Disadvantage
\item Useful Function Disadvantage
\end{itemize}

After running all five steps on the system the Functional Model is created.
As the Component Model is small part of the Functional Model, this is
mentioned as sub-concept.  Furthermore the Function Modeling describes a
process and rules for building the Function Model.  Hence this is also a
sub-concept of Function Model.

Furthermore for completion \textit{process} is added as a TRIZ concept.

\subsection{Optimising The Function}

After creating the Functional Model a Principle has to be applied to the
function which need optimisation.  This is done by choosing an increasing and
decreasing parameter from the chosen Matrix.  These parameters have been
implemented as \textit{rdf:property}.

In a next step these properties are added to a chosen function and form a new
concept, named \textit{functionWithMatrixParameter}.

With the concept \textit{ChooseMatrixEntryByFunction} the parameters are then
checked and the suitable matrix entry is returned.  As there are various
Matrix, which differ slightly in their entries a new property
\textit{chosenMatrix} is added as an allowed value to
\textit{ChooseMatrixEntryByFunction}.  With this the chosen Matrix can be
mentioned.

This results in the \textit{MatrixEntryByFunctionParameter} concept.  With
this and the implemented WUMM concept \textit{ChoosePrincipleFromMatrixEntry}
a Principle from the matrix entry is chosen.  Finally the chosen principle is
connected with the function by
\textit{PrincipleByMatrixEntryByFunctionParameter}.

\subsection{The Matrix 2003}

The Matrix 2003 has been created with the help of the python script, which has
been explained in \ref{subsubsec:creating_matrix}.  To connect the principle
id's from the Matrix 2003 with the implemented principles in the file
\textit{Principles.ttl} an id has been added.  The property has been called
\textbf{od:Matrix2003Id}.

\subsection{The Parameters}

As for the created Matrix 2003 more parameters and upper-parameter were added,
the already created \textit{Parameters.ttl} was updated.

The nine new parameters were created as TRIZ concepts.

Furthermore the six upper-parameters were also added as TRIZ concepts.  For
those the new WUMM concept \textit{od:UpperParameter} was implemented.

\section{Example For Functional Analysis}
\label{sec:examples}

In the following section for four different systems the Functional Model will
be implemented.  With the given Functional Model the problems of the system
can be obtained.  Afterwards with the help of the various Matrices and the
Principles solutions for these problems can be found.  For some optimisable
functions a principle will be looked up in the matrix, which will be visible
in the examples.  Some functions will not be optimized on purpose, as this
should help to figure out own ideas and solutions.

For the following examples always one implementation of a function is
described.  The creation of the other functions in the example works
analogously.

\subsection{Example Washing Machine}

The first examples describes the system of a washing machine.  This example
was used by Nikolai Shchedrin.

First the important system components \textit{Washing Machine Drum} and
\textit{Laundry} were defined as an component.  Furthermore the action
\textit{Turn} is implemented.  With the two components the interaction
\textit{Washing Machine Drum - Laundry} is created.

At point three, as mentioned before, the function \textit{CirculateLaundry} is
created.

In the next step the direction for the function is added.  This is done, by
defining the Subject-Action-Object triple \textit{Washingmachinedrum - Turn -
  Laundry}, with the interaction and the action.  In this case \textit{Washing
  Machine Drum} is the Subject and the \textit{Laundry} is the Object.
Afterwards this attached to the function.

In the final step we create the quality of the function.  At this point the
creator has to analyze the function and reason why this quality was chosen.
Here, the quality was chosen as a \textit{Useful Function}, as the turning of
the laundry helps the cleaning.  Furthermore it is assumed that the washing
machine works perfectly in this aspect.  Otherwise the function could have a
harmful addition.  The quality is then added as another triple to the
function.

At this point the function has all important triples.  The function in figure
\ref{fig:implementation_function_circulate_laundry} then represents one part
of the Functional Model.

\begin{figure}[ht]
  \centering
  \begin{code}
    ex:CirculateLaundry\\
    \> a tc:Function ;\\
    \> tc:SubjectActionObject ex:WashingmachinedrumTurnLaundry ;\\
    \> tc:QualityOfFunction tc:UsefulFunction ;\\
    \> skos:prefLabel "Ciruclate Laundry"@en .\\
  \end{code}
  \caption{Implementation Of The Function \textit{CirculateLaundry}}
  \label{fig:implementation_function_circulate_laundry}
\end{figure}

\subsection{Example Coffee}

The next example is mentioned in the book "Systematische Innovationsmethoden".
The example describes the system of coffee interacting with a cup when poured
into it.  A small part of the Functional Model is implemented as a RDF graph.

At first the two components \textit{Cup} and \textit{Coffee} are implemented
in \textit{ex} namespace.  With the two components the interaction
\textit{CupCoffee} is created.  Afterwards the action \textit{Rest} is added.

With the components and action defined, the function \textit{Pollute}, which
can be seen in figure \ref{fig:implementation_function_pollute} is created.
This describes that the coffee leaves residues at the cup.

As before, the direction of the function is added with the
Subject-Action-Object triple.  In this case this is named
\textit{CupRestCoffee} and the Subject is the \textit{Coffee}.

As the residue is a negative effect this would be a \textit{Harmful Function}.
Hence this function would be a starting point to find a Principles in the
Matrix to solve this problem.

With the creation of further functions the Functional Model is fulfilled.  At
last all the functions with problems can be analyzed and a suitable Matrix
Entry can be identified.  From there a Principle is chosen, which optimizes
the function.

\begin{figure}[ht]
  \centering
  \begin{code}
    ex:Pollute\\
    \> a tc:Function ;\\
    \> tc:SubjectActionObject ex:CoffeeRestCup ;\\
    \> tc:QualityOfFunction tc:HarmfulFunction ;\\
    \> skos:prefLabel "Pollute"@en .\\
  \end{code}
  \caption{Implementation Of The Function \textit{Pollute}}
  \label{fig:implementation_function_pollute}
\end{figure}

\subsection{Example Car}

In the third example the system of a car is analyzed.  Hereby the function
\textit{SitUncomfortable} is explained.

First of all the two components \textit{UncomfortableSeats} and
\textit{Inmates} are created.  The action is in this case \textit{Touch} and
the Subject-Action-Object triple will be
\textit{InmatesTouchUncomfortableSeats}.  The \textit{Inmates} are the Subject
as these are touching the seats.

The quality of the function is a \textit{Function Disadvantage}, as the
inmates are able to sit, which is a positive or useful aspect.

On the other hand the seats are uncomfortable for the inmates.  This can be
harmful as the driver can be disturbed by this.  Furthermore, as the feeling
of a uncomfortable seat differs for each person, this function is bad
controllable.

\begin{figure}[ht]
  \centering
  \begin{code}
    ex:SitUncomfortable\\
    \> a tc:Function ;\\
    \> tc:SubjectActionObject ex:InmatesTouchUncomfortableSeats ;\\
    \> tc:QualityOfFunction tc:FunctionDisadvantage ;\\
    \> skos:prefLabel "Uncomfortable Sitting"@en ;\\
    \> skos:Definition "By touching the occupants with the uncomfortable seats of\\
    \> \> the car, a useful function forms with disadvantages, since the occupants\\
    \> \> on the one hand can sit but only uncomfortably"@en .
  \end{code}
  \caption{Implementation Of The Function \textit{SitUncomfortable}}
  \label{fig:implementation_function_sit_uncomfortable}
\end{figure}

\subsection{Example Mechanical Computer Mouse}

The last example is taken from the book "Systematische Innovationsmethoden".
Picture \ref{fig:functional_model_mechanical_computer_mouse} shows an overview
over the Functional Model.  This picture was created by the book authors
\cite{KS}.  The different arrows represent different kind of qualities for the
function.  The black arrows are Useful Functions, the red ones describe
Harmful Functions and the dotted lines represent a Bad Controllable Function.

\begin{figure}[ht]
  \centering
  \includegraphics[scale=.6]{mouse.png}
  \caption{Functional Model For A Mechanical Computer Mouse In German}
  \label{fig:functional_model_mechanical_computer_mouse}
\end{figure}

The implementation of the function \textit{Scratches} is going to be
explained.  This describes the arrow from \textit{Case (Gehäuse)} to
\textit{Tabletop (Tischplatte)}.

The interaction is therefore \textit{CaseTabletop} with the two components
\textit{Case} and \textit{Tabletop}.  All of these are in the ex namespace as
explained in section \ref{sec:conceptualisations}.

Furthermore the action \textit{Move} is defined. 
And with that the direction \textit{CaseMoveTabletop}, by the Subject-Action-Object triple created.

Finally the quality of the function is added.  The book defined this as a
\textit{Harmful Function}.

As this is an Harmful Function an optimisation is necessary.  Therefore we
define \textit{ScratchesParameter} which is a
\textit{functionWithMatrixParameter} concept.  Hence we have to choose the
decreasing and increasing parameters.  As in this paper also the Matrix 2003
was implemented.  This is the one chosen to find a suitable principle.

As an increasing parameter \textit{Power} is chosen, because the moving of the
mouse increases the working possibilities of the mouse.  The decreasing
parameter is \textit{Loss Of Material}, as the tabletop looses material during
the moving of the mouse.

With the chosen parameters the matrix entry can be obtained.  In this case
this would be \textit{$<$http://opendiscovery.org/rdf/Matrix/E.18.25$>$}.

From there the principle \textit{tc:FlexibleCoversOrThinFilms} is picked.

This should lead to an optimisation for the function.  Here it could be adding
a layer between the tabletop and the mouse, like a mouse-pad.

\section{Conclusion}
\label{sec:conclusion}

This paper implemented a suggestion for an ontology for Functional Analysis.
Most of the information were taken from the book "Systematische
Innovationsmethoden" and the talk from Nikolai Shchedrin.  This gives a good
summary over all the concepts which are need for a Functional Model and the
Functional Analysis.  In a next step other sources could be added and the
ontology could be advanced.

Furthermore the examples could be made more specific, which would help for a
greater understanding on the topic.  An interesting aspect would also to see
multiple approaches on how to optimize a function with different principles.

As there is another update of the Matrix another implementation for this would
also be helpful.

In conclusion this implementation should give a good starting point for the
ontology for Functional Analysis, but could be advanced with more knowledge
and ideas from other sources.

\begin{thebibliography}{xxx}
\raggedright
\bibitem{Altshuller1979} Genrich Altshuller (1979).  Creativity as an exact
  science (in Russian). English version: Gordon and Breach, New York 1988.
\bibitem{Graebe2021} Hans-Gert Gr\"abe (2021). About the WUMM modelling
  concepts of a TRIZ ontology.
  \url{https://github.com/wumm-project/Leipzig-Seminar/blob/master/Wintersemester-2020/Seminararbeiten/Anmerkungen.pdf}.
\bibitem{KS} Karl Koltze, Valeri Souchkov (2017).  Systematische
  Innovationsmethoden (in German).  Hanser, Munich. ISBN 978-3-446-45127-8.
\bibitem{TESE2018} Alex Lyubomirsky, Simon Litvin, Sergei Ikovenko et al.
  (2018). Trends of Engineering System Evolution (TESE).  TRIZ Consulting
  Group. ISBN 9783000598463.
\bibitem{Shpakovsky2016} Nikolay Shpakovsky (2016). Tree of Technology
  Evolution. English translation of the Russian original (Forum, Moscow
  2010).\\ \url{https://wumm-project.github.io/TTS.html}
\bibitem{SKOS} SKOS -- The Simple Knowledge Organization System.
  \url{https://www.w3.org/TR/skos-reference/}.  
\bibitem{WUMM} The WUMM Project. \url{https://wumm-project.github.io/} 
\bibitem{Matrix2003} Matrix2003:
  \url{https://triz-journal.com/wp-content/uploads/2018/04/Screen-Shot-2018-04-30-at-15.20.25.png}.
\bibitem{tabula} tabulapdf: \url{https://github.com/tabulapdf/tabula}.
\bibitem{deep_translator} Deep Translator:
  \url{https://pypi.org/project/deep-translator/}.
\bibitem{WebinarFunctionAnalysis} Nikolai Shchedrin (2020). Webinare des
  TRIZ-Ontologie-Projekts "Funktion" und "Funktionsanalyse".
\bibitem{SouchkovGlossary} Valeri Souchkov (2018). Glossary of TRIZ and TRIZ-related terms.
\bibitem{ShchedrinUsefulFunctionDisadvantage} Nikolai Shchedrin (2020).
  Webinar on Function Model and Functional Analysis.
  \url{https://triz-summit.ru/onto_triz/mod/metod/triz/fa/model_fa/func_syst_model/func/func_type/disadv_f/}.
\end{thebibliography}

\end{document}

