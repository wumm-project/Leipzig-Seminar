\section{Implementation Of Tools}

\subsection{Creating Turtle File}

For easier and faster creation of the "Matrix 2003" a small python tool was implemented.
This tool reads the matrix from a \textit{csv}-file and generates a \textit{turtle}-file.

\subsubsection{The CSV-File}

First of alle \textit{csv} file needs to be in the following syntax, because otherwise the matrix is wrongly interpreted.
Every line in the csv file is a row in the matrix.
Furthermore every comma seperates a column entry in a a row.
The different entries in the matrix field are seperated by dashes.

For reading the matrix the tool "tabula" was used \cite{tabula}.
With this tool the Matrix2003 from the book "Systematische Innovation" was interpreted as a txt file.
As this did not have the right syntax the small script, named \textit{convertTxtMatrixToCsv.py}, was created. 
With this the matrix is converted into the right syntax, as mentioned above.

\paragraph{Using The Script}

Pyhton will need to be installed on the computer. 
The command to run the script will then be:

\begin{center}
    python convertTxtMatrixToCsv.py $<$csv-file-name$>$
\end{center}

The output \textit{csv}-file will be called \textit{createdMatrix.csv}.
The file \textit{temporary.txt} is only created for storing a the information temporary. 
It can be deleted afterwards.

\subsubsection{Creating The Matrix}

The following explains the implementation of the \textit{createMatrix.py} script.

The script uses two other files. 
The first ist the \textit{principles.txt}.
In this file alle the forty principles are written down, with their belonging number at the beginnning.
Hence the script can map an entry index to the belonging principle name.

The second file, which is also used is called \textit{parameters.txt}.
In there are all the forty-eight parameters which represent the column and rows in the matrix.
These names are similar to those for the Altshuller Matrix.
The nine new parameters, which were introduced for the Matrix2003 are the following. 

\begin{enumerate}[noitemsep]
    \item QuantityOfInformation
    \item FunctionEfficiency
    \item Noise
    \item HarmfulEmissions
    \item CompatitbilityOrConnectability
    \item Security
    \item Vulnerability
    \item AestheticsOrAppearance
    \item ComplexityOfControl
\end{enumerate}

The layout for the Turtle file is the same as the layout for the "Altshuller Matrix". 
As this makes it easier for later adjustments or improvements of the matrixes.

At first the script creates the header for the Matrix2003, which is the same as in the "Altshuller Matrix".
Then it defines the owl entry for the Matrix.
Afterwards the script iterates over the entries in the \textit{csv} file and builds the matrix in the following way.

\begin{figure}[H]
    \centering
    \begin{code}\tt
        <http://opendiscovery.org/rdf/Matrix/E.06.34>\\
        \> od:decreasingParameter tc:EaseOfUse ;\\
        \> od:increasingParameter tc:SurfaceOfTheStationaryObject ;\\
        \> od:recommendedPrinciple tc:PreferredAction, tc:Asymmetry, \\
        \> \> \> tc:Mediator, tc:SelfService ;\\
        \> a od:MatrixEntry .
    \end{code}
    \caption{Entry of the Matrix2003 Turtle File}
\end{figure}

\paragraph{Using The Script}

For running the script python will need to be installed on the computer.
The script can then be started with the following command:

\begin{center}
    python createMatrix.py $<$csv-file-name$>$
\end{center}

Afterwards there will be a file called \textit{created\_matrix.ttl}.
This will be the result of the script.


\subsection{Translating}

For translating the various tags a python script was implemented, which takes a turtle files as input and adds the missing language tags.
Therefore the script checks each line which already has a language tag. 
Then it makes an api call with the first tag as the source language.
It add the language tags for english, german and russian. 
Each tag is a seperate api call.
For translation the Google Translator is used.

\paragraph{Using the Script}

Python has to be installed on the computer.
Furthermore does the script need the \textit{deep\_translator} library \cite{deep_translator}. 
The library can be installed with the following command:

\begin{center}
    pip install deep-translator
\end{center}

Afterwards the script can be started. 
\textit{$<$turtle-file-name$>$} specifies the file for which the missing language tags should be added.

\begin{center}
    python translateText.py $<$turtle-file-name$>$
\end{center}

This will output a file name \textit{trasnlated\_$<$turtle-file-name$>$.ttl}.
This is the new turtle file with the added language tags.
Furthemore all translated words or sentences will be shown in the terminal output.
