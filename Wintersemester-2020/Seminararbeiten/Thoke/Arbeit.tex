\documentclass[11pt,a4paper]{article}
\usepackage{ls}
\usepackage[main=english,russian]{babel}

\newenvironment{code}{\tt \begin{tabbing}
\hskip12pt\=\hskip12pt\=\hskip12pt\=\hskip12pt\=\hskip5cm\=\hskip5cm\=\kill}
{\end{tabbing}}
\def\dq{{\char34}}

\title{A Proposal for Modelling the TRIZ Concepts\\ Flow and Flow Analysis}

\author{Immanuel Thoke}

\date{Version of March 2, 2021}

\begin{document}
\maketitle

\section{Aim of the work}

The aim of this paper is to elaborate a proposal for an ontological modelling
of the areas of \emph{TRIZ Flows} and \emph{TRIZ Flow Analysis} based on the
materials of the TRIZ Ontology Project (TOP) of the TRIZ Developer Summit
\cite{TOP} and further own investigations. The work fits into the activities
of the \emph{WUMM Project} \cite{WUMM} to accompany the TOP project and to
transfer it into the language of modern semantic concepts
\cite{WUMM-Ontology}.  The work therefore consists of two parts -- a
\emph{turtle file}, in which the semantic modelling is performed based on the
SKOS framework \cite{SKOS}, and \emph{this elaboration}, in which the
backgrounds and motivations of the concrete modelling decisions are detailed.

\section{Starting point} 

Flows and flow analysis play a rather peripheral role in TRIZ.  In the
standard reference \cite{Petrov2007} on a \emph{TRIZ Body of Knowledge} is
listed as item 2.5.8, but in the 7 references to the literature listed there
are no systematic explanations about the role of flows in TRIZ theory, but
only individual examples in which concrete flows (such as magnetic flow)
played a role in concrete modelling.  

We start from the definition of both concepts in the TOP Glossary
\cite{TOP-Glossary}: 
\begin{quote}
  \textbf{Flow} is the directional movement in space of particles of mass of
  matter, as well as the directional movement of energy or information. Flow
  has dual properties: the properties of a substance of which the flow
  consists, and the properties of a field, which is formed as a result of the
  directional movement of particles of matter. Flows can be useful, harmful
  and parasitic. The flow model contains static components: source, channel,
  receiver, control system. Flow is a special case of a process in which
  directional movement occurs in physical space.

  \textbf{Flow analysis} is a method of systems analysis to establish
  relationships in a system to flow, find resources, and determine compliance
  with existing system requirements. The result of a flow analysis is a model
  of the flow as is and a model of the flow to be. Flow modification
  techniques are used to develop systems with flows and to solve inventive
  problems in them.
\end{quote}
The use of these concepts is further included there in a single place in the
context of cause-effect analysis (ibid.):
\begin{quote}
  Cause-effect analysis is performed: 
  \begin{itemize}[noitemsep]
  \item When the causes of an undesirable effect are not clear (when we cannot
    go from an administrative contradiction to a technical one $AC\to TC$.
  \item When it is necessary to clarify the causes of an undesirable effect
    (to deepen the understanding of the causes of an undesirable effect, e.g.
    after a functional or a flow analysis).
  \end{itemize}
\end{quote}
Other sources are much more sparing in their statements on these two
individual terms, but list other flow-related concepts, for example in
\cite{Souchkov2018}:
\begin{itemize}[noitemsep]
\item Delay Zone -- A location in a flow in which the integral flow speed is
  significantly lower than local flow speed. A Delay Zone is a typical
  disadvantage identified by Flow Analysis. 
\item Flow -- A sequence of events that have the same common feature.
\item  Flow Analysis-- An analytical method and a tool which identifies
  disadvantages in flows of energy, substances, and information in a technical
  system.
\item  Flow Disadvantage -- A disadvantage of a technical system being
  analyzed identified during Flow Analysis. Examples: Bottlenecks, «Gray
  Zones», «Stagnation Zones», etc.
\item Flow Distribution Analysis-- A part of Flow Analysis that identifies
  distribution of flows and their disadvantages. 
\item Stagnation Zone -- A part of a flow in which the flow stops temporarily
  or permanently. A Stagnation Zone is a typical disadvantage identified by
  Flow Analysis.
\item Transmission -- One of the key components (subsystems) of a Complete
  Technical System which according to the Law of System Completeness of a
  technical system transmits a flow of energy required to operate a working
  unit from an engine to the working unit.
\end{itemize}

\section{On the concept of system in the TOP modelling}

A central concept in TOP modelling is the distinction between
\begin{itemize}[noitemsep]
\item [(1)] the system as it is,
\item [(2)] the model of the system as it is,
\item [(3)] the model of the system as it should be, and 
\item [(4)] the system as it should be. 
\end{itemize}
The TOP glossary \cite{TOP-Glossary} explains the differences as follows
\begin{itemize}[noitemsep]
\item [(1)] The \emph{system as is} is a system in its original state before
  it is analysed and transformed into a new «system as is».
\item [(2)] The \emph{model of the system as is} is formed from the «system as
  is» by means of various TRIZ models: component-structural and functional
  models, su-field or ele-field models, description of contradictions or of
  typical conflict schemes, etc. Depending on the chosen model type, the model
  will be transformed into the «model of the system as required».
\item [(3)] The \emph{model of the system as required} is formed from the
  «model of the system as is» by procedures which correspond to the selected
  model transformation method (functional, su-field, ele-field, solution of
  the contradictions in requirements and properties, etc.). The transition is
  performed along the line
  \begin{center}
    «System as is» $\to$ «Model of the system as is»\\ $\to$ «Model of the
    system as required» $\to$ «System as required»
  \end{center}
  in accordance with the scheme of a TRIZ Model.
\item [(4)] The \emph{system as required} is a system derived from the «system
  as is» through a transformation, based on the «model of the system as
  required».
 \end{itemize}
The referenced notions of \emph{TRIZ Model} and \emph{System} ar explained as
follows:
\begin{itemize}[noitemsep]
\item A \emph{system} is a set of elements in relationship and connection with
  each other, which forms a certain integrity, unity. The need to use the term
  «system» arises when it is necessary to emphasize that something is large,
  complex, not fully immediately understandable, yet whole, unified. In
  contrast to the notions of «set» and «aggregate», the concept of a system
  emphasises order, integrity, regularities of construction, functioning and
  development. The notion of system is part of the system and functional
  approach, and is used in the system operator.
\item A \emph{TRIZ model} is a schematic notation of gradual transition
  from the problem to TRIZ model of the problem, then to TRIZ model of the
  solution and then to the solution itself; or from the system to TRIZ model
  of the system, then to TRIZ model of the new system and then to actual
  change of the system («system as required»). The TRIZ model includes the
  basic components of inventive thinking: analysis, synthesis, evaluation.
\end{itemize}
This distinction, which is important for the planning and subsequent execution
of the transformation of the system, is applied to all sub-concepts in the TOP
modelling approach.  In particular, a distinction is made between «flow as
is», «model of flow as is», «model of flow as required» and «flow as
required».

The system definition here is to be seen critically, since no distinction
between a real-world system and its description is made, as is explained in
more detail in \cite{Graebe2020}.  In this sense also in (1) and (4) a
distinction has to be made between the real-world system and its description.
Only the descriptions are accessible for modelling, hence (1) and (4) have to
be replaced by «description of the system ...». (1) and (2) or (3) and (4)
then differ primarily in the degree of formalisation according to the
methodological guidelines of TRIZ.  Since the aim of modeling is not only to
describe the solution, but also to implement it, i.e. carry out the projected
transformation on the real-world system, a further level of description
\begin{itemize}
\item [(5)] the \emph{description of the system as it was realized},
\end{itemize}
is relevant if real-world development is also to be taken into account.

Such readjustment of the basic concepts then also allows to model iterative
system transformations, in which at the beginning of every iteration the
difference between the «system as required in the previous step» and the
«system as it was realized», can be analysed and based on this analysis also
the «model of the system as required» can be transformed.

This, however, requires a more precise distinction between the levels 
\begin{itemize}[noitemsep]
\item [(0)] of different versions of the \emph{real-world} system,
\item [(1)] of \emph{descriptions and models} of these different versions and
\item [(2)] the \emph{concepts, terms and notions} used in these descriptions.
\end{itemize}
The object of ontologisation is only level (2), in which the uniformity of the
concepts over the different versions existing on level (1) is to be secured,
which is reflected in the duality of concept (level (2)) and concept instance
(level (1)).  Since RDF is relatively tolerant about not including parts of a
concept into instances, it makes sense to make the distinctions according to
(1)--(4) only at the instance level as a methodological property of a basic
type \texttt{owl:Thing}, which is inherited by all subconcepts.  

This was implemented in the proposed modelling as \texttt{od:hasMode}
whose value range is an enumeration type \texttt{tc:ModeValue}.  See
\cite{Graebe2021} for details and an example.


\begin{thebibliography}{xxx}
\raggedright
\bibitem{Graebe2020} Hans-Gert Gräbe (2020).  Die Menschen und ihre
  Technischen Systeme. LIFIS Online.  \url{doi:10.14625/graebe_20200519}.
\bibitem{Graebe2021} Hans-Gert Gräbe (2021).  Some remarks about the WUMM
  Ontology Modeling.  Manuscript, WUMM repository «Leipzig-Seminar». 
\bibitem{Petrov2007} Vladimir Petrov, Michail Rubin, Simon Litvin (2007).
  Fundamentals of TRIZ Theory of Inventive Problem Solving (in Russian).
  Reprinted in 2020. ISBN 978-5-4496-8183-6.
\bibitem{Souchkov2018} Valeri Souchkov (2018).  Glossary of TRIZ and
  TRIZ-related terms. Version 1.2.
  \url{https://matriz.org/wp-content/uploads/2016/11/TRIZGlossaryVersion1_2.pdf}. 
\bibitem{SKOS} SKOS -- The Simple Knowledge Organization System.
  \url{https://www.w3.org/TR/skos-reference/}.  
\bibitem{TOP} The TRIZ Ontology Project (TOP)
  \foreignlanguage{russian}{Онтология ТРИЗ} of the TRIZ Developer Summit.
  \url{https://triz-summit.ru/Onto_TRIZ/}.
\bibitem{TOP-Glossary} TRIZ 100 Glossary. A short glossary of key TRIZ
  concepts and terms (in Russian).
  \url{https://triz-summit.ru/onto_triz/100/}.
\bibitem{WUMM} The WUMM Project. \url{https://wumm-project.github.io/} 
\bibitem{WUMM-Ontology} The WUMM TOP Companion Project.
  \url{https://wumm-project.github.io/Ontology.html} 
\end{thebibliography}

\end{document}

