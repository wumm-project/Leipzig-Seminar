\documentclass{beamer}
\usepackage{lsfolien,enumitem}
\usepackage[german]{babel}

\title{Modellierung nachhaltiger Systeme\\ und Semantic Web\\[6pt] Entwicklung
  Technischer Systeme\\ im TESE-Buch}

\author{Hans-Gert Gr\"abe}

\date{17. November 2020}

\begin{document}
\begin{frame}
  \maketitle
\end{frame}
\myfootline{Modellierung nachhaltiger Systeme -- WS 20/21}{Hans-Gert Gräbe}

\begin{frame}{Worüber ich nicht sprechen werde}

  Ich überlasse die Diskussion konkreter Gesetze, Trends und
  Entwicklungslinien (kurz noch immer \emph{Gesetze}) späteren
  Seminarvorträgen.

Im TESE-Buch werden 10 solche Gesetze und das \emph{Gesetz der Entwicklung
  nach S-Kurven} als Supergesetz thematisiert und ausführlich an Hand von
Beispielen erläutert.

Die Gesetze werden dabei in eine gewissen Hierarchie gebracht.

Auch darauf werde ich nicht eingehen.
\end{frame}

\newcommand{\law}[2]{\parbox{#1cm}{\centering #2}}
\begin{frame}{Die Hierarchie der Gesetze im TESE-Buch}
\begin{center}\tiny
\tikz[>={Triangle[length=1pt 3, width=1pt 1]},scale=.5] {
\node[draw] at (5,10) [rectangle]
(A0) {\law{1.5}{Trend der Evolution nach S-Kurven}};
\node[draw] at (5,8) [rectangle]
(A1) {\law{1.5}{Gesetz der Erhöhung der Idealität}};
\node[draw] at (10,3.5) [rectangle]
(A2) {\law{1.5}{Gesetz der Erhöhung der Vollständigkeit TS}};
\node[draw] at (0,4) [rectangle]
(A3) {\law{1.5}{Gesetz der Erhöhung der Abstimmung}};
\node[draw] at (10,6.5) [rectangle]
(A4) {\law{1.5}{Gesetz der Erhöhung der Effektivität der Ausnutzung von
    Flüssen}}; 
\node[draw] at (5,1.5) [rectangle]
(A5) {\law{1.5}{Gesetz der ungleich- mäßigen Entwicklung der Teile des
    Systems}};  
\node[draw] at (10,.5) [rectangle]
(A7) {\law{1.5}{Gesetz der Verdrängung des Menschen aus TS}};
\node[draw] at (0,2) [rectangle]
(A8) {\law{1.5}{Gesetz der Erhöhung der Steuerbarkeit}};
\node[draw] at (0,0) [rectangle]
(A9) {\law{1.5}{Gesetz der Erhöhung der Dynamisierung}};
\node[draw] at (0,6) [rectangle]
(A10) {\law{1.5}{Gesetz des Übergangs ins Obersystem}};
\node[draw] at (5,5.5) [rectangle]
(A12) {\law{1.5}{Gesetz der Erhöhung der Multifunktionalität}};
\draw[->] (A0) -- (A1) ;
\draw[->] (A1) -- (-2.5,8) |- (A3) ;
\draw[->] (1,8) -| (A10) ;
\draw[->] (A1) -- (7.5,8) |- (A4) ;
\draw[->] (7.5,8) |- (A2);
\draw[->] (7.5,8) |- (A4);
\draw[->] (7.5,8) |- (A5);
\draw[->] (A1) -- (A12);
\draw[->] (A2) -- (A7);
\draw[->] (A5) |- (A3);
\draw[->] (A3) -- (A8) ;
\draw[->] (A8) -- (A9) ;}
\end{center}
\end{frame}

\begin{frame}{Technische Systeme und Produktkataloge} 

Zur Abgrenzung von Kontinuitätslinien TS werden im TESE-Buch offensichtlich
\textbf{Produktgruppenbegriffe aus Handelskatalogen} wie Flugzeuge, Autos,
Werkzeugmaschinen, Computer, Telefone, Speichereinheiten, Militärfahrzeuge,
Computermäuse, Schreibwerkzeuge usw. (die Auf\-listung orientiert sich an den
Abbildungen im Buch) verwendet, die selbst wieder in verschiedenen Korngrößen
geclustert werden können.

In \emph{diesen} -- aus dem ökonomischen Nachbarsystem übernommenen --
Strukturen lassen sich in der Tat Evolutionslinien der „Produktgängigkeit“
verfolgen, die für die Entwicklung von unternehmerischen Geschäftsstrategien
\emph{praktisch} extrem relevant sind.

\end{frame}

\begin{frame}{Evolution vs. Entwicklung} 

Begriffe \emph{Evolution} und \emph{Entwicklung} entsprechend der Diskussion
am 10.11.: \emph{Aus der Sicht der Unternehmer} kann eine Geschäftsstrategie
aktiv \emph{entwickelt} werden, während die Technik in einem naturwüchsigen,
vom Unternehmen nicht beeinflussbaren Prozess als „Umweltleistung“
\emph{evolviert}.

Damit wird auch verständlich, was eine S-Kurve ist. Es geht dabei \emph{nicht}
um die Entwicklung technischer Systeme nach einer Logik der
Technikentwicklung, was das auch immer sei, sondern um die Dynamik von
Produktzyklen.

\end{frame}

\begin{frame}{Technische und ökonomische Entwicklung} 

Ein solches Gesetz gibt es bei Altschuller schon deshalb nicht, weil eine
solche Perspektive \emph{nicht} seine Perspektive ist.

Es ist aber eine aus \emph{praktischer} Sicht sehr relevante Perspektive des
\emph{unternehmerischen Innovationsmanagements}, die jedoch auch die
\emph{Welt Technischer Systeme} in dem in der Vorlesung präzisierten
Verständnis verlässt.  Im TESE-Buch wird deshalb auch von „S-curve analysis
and pragmatic S-curve analysis“ (ebenda, Überschrift Kapitel 3) geschrieben.
\end{frame}

\begin{frame}{Konzeptualisierung eines „Innovationsmanagements“}

Dort (ebenda, Kapitel 1) wird für ein solches Innovationsmanagement das (neue)
Konzept eines \emph{Technology Pull} als über die Konzepte \emph{Technology
  Push} und \emph{Market Pull} hinausgehend entwickelt.

Gegenstand technologischer Evolution sind damit aber diese technologischen
Produktionsbedingungen selbst, insbesondere die Organisation entsprechender
Innovationsprozesse in Unternehmen.

Damit wird als Obersystem die strategischen Führungsstrukturen von Unternehmen
relevant, in denen die Innovationsprozesse praktisch gestaltet werden.

\end{frame}

\begin{frame}{Konzeptualisierung eines „Innovationsmanagements“}
  
Auch hierbei haben wir es mit der Dualität von System-Template -- gängigen
gesellschaftlichen Verfahrensweisen zur Organisation von Innovationsprozessen
-- und konkreten realweltlichen Systemausprägungen in den einzelnen
Unternehmen zu tun.

Die \emph{Hauptfunktion} jener Strukturen im Unternehmen ist die Organisation
des Innovationsprozesses in enger Verbindung mit der allgemeinen
Geschäftsstrategie.  Dieser Prozess selbst wird vom strategischen Management
entschieden und verantwortet, das hierzu die \emph{widersprüchlichen
  Anforderungen} verschiedener Unternehmensteile (R\&D, Vertrieb, Finanzen,
Controlling, SCM, CRM) unter einen Hut zu bringen hat.

\end{frame}

\newcommand{\bbox}[2]{\parbox{#1cm}{\centering #2}}
\begin{frame}{Konzeptualisierung eines „Innovationsmanagements“}
  
Die in (TESE 2018) zusammengetragenen Empfehlungen sind \emph{ein} Aspekt in
diesem komplexen Abwägungsprozess.

Mit „state of the art in science and technology“ neben den „needs of society
and marketplace“ ein weiteres Obersystem in Stellung gebracht.

\begin{center}\tiny
  \begin{tikzpicture}[
      >={Triangle[length=0pt 3,width=0pt 3]},
      rounded corners=1pt,line width=.5pt,scale=.6] 
  \node[draw=green] at (0,2) [circle] (A0)
       {\bbox{1}{Idea Generation}};
  \node[draw=red] at (3,2) [rectangle] (A1)
       {\bbox{1}{Develop- ment}};
  \node[draw=red] at (6,2) [rectangle] (A2)
       {\bbox{1}{Manufac- turing}};
  \node[draw=red] at (9,2) [rectangle] (A3)
       {\bbox{1}{Marketing and Sales}};
  \node[draw=green] at (12,2) [circle] (A4)
       {\bbox{1}{Commer- cial Product}};
  \node[draw=green] at (6,4.3) [rectangle] (A5)
       {\bbox{5}{State of the art\\[12pt] in science and technology}}; 
  \node[draw=green] at (6,-.3) [rectangle] (A6)
       {\bbox{5}{Needs of society\\[12pt] and the marketplace}}; 
  \draw[color=red] (-.5,4.4) node {\bbox{2}{New\\ Technology}};
  \draw[color=red] (-.5,-.5) node {\bbox{2}{New\\ Ideas}};  
  \draw[<->] (1.2,2) -- (2,2) ;
  \draw[<->] (4,2) -- (5,2) ;
  \draw[<->] (7,2) -- (8,2) ;
  \draw[<->] (10,2) -- (10.8,2) ;
  \draw[<->] (3,2.5) -- (3,3.5) ;
  \draw[<->] (6,2.5) -- (6,3.5) ;
  \draw[<->] (9,2.5) -- (9,3.5) ;
  \draw[<->] (3,1.5) -- (3,0.5) ;
  \draw[<->] (6,1.5) -- (6,0.5) ;
  \draw[<->] (9,1.5) -- (9,0.5) ;
  \draw[<->] (-.4,4) -- (-.4,3.3) ;
  \draw[<->] (.4,4.5) -- (1.4,4.5) ;
  \draw[<->] (-.4,0) -- (-.4,0.7) ;
  \draw[<->] (.4,-.5) -- (1.4,-.5) ;
  \draw[->,dashed] (1,3) arc(120:60:10) ;
  \draw[<-,dashed] (1,1) arc(-120:-60:10) ;  
  \draw (1.5,0.3) node[fill=green] {\bbox{1.5}{\sc{Market Pull}}};
  \draw (10.5,4) node[fill=green] {\bbox{2.3}{\sc{Technology Push}}};
\end{tikzpicture}\\\normalsize
Eine Reproduktion von Fig. 3 in (Preez 2006).
\end{center}
\end{frame}
\end{document}
