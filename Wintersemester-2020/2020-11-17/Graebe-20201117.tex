\documentclass[11pt,a4paper]{article}
\usepackage{ls}
\usepackage[main=german,russian]{babel}

\newcommand{\bbox}[2]{\parbox{#1cm}{\small\centering #2}}

\title{Modellierung nachhaltiger Systeme und Semantic Web\\[6pt] Technische
  Systeme im TESE-Buch \\[6pt]\Large Diskussionsgrundlage für das
  Seminar am 17.11.2020}

\author{Hans-Gert Gr\"abe}

\date{15. November 2020}

\begin{document}
\maketitle

\section{Vorbemerkungen}

Ich überlasse die Diskussion konkreter Gesetze, Trends und Entwicklungslinien
(kurz noch immer \emph{Gesetze} -- siehe die Anmerkungen zum Seminar am
10.11.) späteren Seminarvorträgen.

In \cite{TESE2018} werden 10 solche Gesetze und das \emph{Gesetz der
  Entwicklung nach S-Kurven} als Supergesetz thematisiert und ausführlich an
Hand von Beispielen erläutert. Die Gesetze werden dabei in eine gewissen
Hierarchie gebracht.  Auch darauf werde ich nicht eingehen.
\begin{center}
\newcommand{\law}[2]{\parbox{#1cm}{\small\centering #2}}
\tikz[>={Triangle[length=3pt 9, width=3pt 3]}] {
  
\node[draw] at (5,10) [rectangle]
(A0) {\law{4}{Trend der Evolution nach S-Kurven}};

\node[draw] at (5,8) [rectangle]
(A1) {\law{4}{Gesetz der Erhöhung der Idealität}};

\node[draw] at (10,4) [rectangle]
(A2) {\law{4}{Gesetz der Erhöhung der Vollständigkeit TS}};

\node[draw] at (0,4) [rectangle]
(A3) {\law{4}{Gesetz der Erhöhung der Abstimmung}};

\node[draw] at (10,6) [rectangle]
(A4) {\law{4}{Gesetz der Erhöhung der Effektivität der Ausnutzung von
    Flüssen}}; 

\node[draw] at (5,4) [rectangle]
(A5) {\law{4}{Gesetz der ungleich- mäßigen Entwicklung der Teile des Systems}}; 

\node[draw] at (10,2) [rectangle]
(A7) {\law{4}{Gesetz der Verdrängung des Menschen aus TS}};

\node[draw] at (0,2) [rectangle]
(A8) {\law{4}{Gesetz der Erhöhung der Steuerbarkeit}};

\node[draw] at (0,0) [rectangle]
(A9) {\law{4}{Gesetz der Erhöhung der Dynamisierung}};

\node[draw] at (0,6) [rectangle]
(A10) {\law{4}{Gesetz des Übergangs ins Obersystem}};

\node[draw] at (5,6) [rectangle]
(A12) {\law{4}{Gesetz der Erhöhung der Multifunktionalität}};

\draw[->] (A0) -- (A1) ;
\draw[->] (A1) -- (-2.5,8) |- (A3) ;
\draw[->] (1,8) -| (A10) ;
\draw[->] (A1) -- (7.5,8) |- (A4) ;
\draw[->] (7.5,8) |- (A2);
\draw[->] (7.5,8) |- (A4);
\draw[->] (7.5,8) |- (A5);
\draw[->] (A1) -- (A12);
\draw[->] (A2) -- (A7);
\draw[->] (A5) -- (A3);
\draw[->] (A3) -- (A8) ;
\draw[->] (A8) -- (A9) ;}
\end{center}

\section{Technische Systeme und Produktkataloge}

Ich gehe stattdessen auf die hier relativ gut erkennbaren, wenn auch nicht
explizierten Prinzipien ein, nach denen technische Systeme abgegrenzt und in
Kontinuitätslinien gruppiert werden.

Hierfür werden offensichtlich Produktgruppenbegriffe aus Handelskatalogen wie
Flugzeuge, Autos, Werkzeugmaschinen, Computer, Telefone, Speichereinheiten,
Militärfahrzeuge, Computermäuse, Schreibwerkzeuge usw. (die Auf\-listung
orientiert sich an den Abbildungen im Buch) verwendet, die selbst wieder in
verschiedenen Korngrößen geclustert werden können.

In \emph{diesen} -- aus dem ökonomischen Nachbarsystem übernommenen --
Strukturen lassen sich in der Tat Evolutionslinien der „Produktgängigkeit“
verfolgen, die für die Entwicklung von unternehmerischen Geschäftsstrategien
\emph{praktisch} extrem relevant sind.

Ich hoffe, dabei die Begriffe \emph{Evolution} und \emph{Entwicklung}
entsprechend der Diskussion am 10.11. korrekt verwendet zu haben: \emph{Aus
  der Sicht der Unternehmer} kann eine Geschäftsstrategie aktiv
\emph{entwickelt} werden, während die Technik in einem naturwüchsigen, vom
Unternehmen nicht beeinflussbaren Prozess als „Umweltleistung“
\emph{evolviert}.

Damit wird auch verständlich, was eine S-Kurve ist. Es geht dabei \emph{nicht}
um die Entwicklung technischer Systeme nach einer Logik der
Technikentwicklung, was das auch immer sei, sondern um die Dynamik von
Produktzyklen.

Ein solches Gesetz gibt es bei Altschuller schon deshalb nicht, weil eine
solche Perspektive \emph{nicht} seine Perspektive ist.

Es ist aber eine aus \emph{praktischer} Sicht sehr relevante Perspektive des
\emph{unternehmerischen Innovationsmanagements}, die jedoch auch die
\emph{Welt Technischer Systeme} in dem in der Vorlesung präzisierten
Verständnis verlässt.  In \cite{TESE2018} wird deshalb auch von „S-curve
analysis and pragmatic S-curve analysis“ (ebenda, Überschrift Kapitel 3)
geschrieben. 

\section{Konzeptualisierung eines „Innovationsmanagements“ in \cite{TESE2018}} 

Dort (ebenda, Kapitel 1) wird für ein solches Innovationsmanagement das (neue)
Konzept eines \emph{Technology Pull} als über die Konzepte \emph{Technology
  Push} und \emph{Market Pull} hinausgehend entwickelt.  Diesen Ansatz habe
ich in \cite{Graebe2020} genauer analysiert, indem ich die TRIZ-Instrumente
der Widerspruchsanalyse auf diesen Gegenstand selbst angewendet habe.  Daraus
ein längeres Zitat.
\begin{quote}  
  Gegenstand technologischer Evolution sind damit aber diese technologischen
  Produktionsbedingungen selbst. Dies wird auch in \cite{TESE2018} so gesehen,
  denn die im Buch beschriebenen Handlungsoptionen beziehen sich auf die
  Organisation entsprechender Innovationsprozesse in Unternehmen. Damit kann
  aber das Obersystem in unserer TRIZ-Analyse der begriff"|lichen Grundlagen
  von \cite{TESE2018} weiter eingeschränkt werden auf die strategischen
  Führungsstrukturen von Unternehmen, in denen die Innovationsprozesse
  praktisch gestaltet werden. Auch hierbei haben wir es mit der Dualität von
  System-Template -- gängigen gesellschaftlichen Verfahrensweisen zur
  Organisation von Innovationsprozessen -- und konkreten realweltlichen
  System"|ausprägungen in den einzelnen Unternehmen zu tun. Die
  \emph{Hauptfunktion} jener Strukturen im Unternehmen ist die Organisation
  des Innovationsprozesses in enger Verbindung mit der allgemeinen
  Geschäftsstrategie.  Dieser Prozess selbst wird vom strategischen Management
  entschieden und verantwortet, das hierzu die \emph{widersprüchlichen
    Anforderungen} verschiedener Unternehmensteile (R\&D, Vertrieb, Finanzen,
  Controlling, SCM, CRM) unter einen Hut zu bringen hat. Die in
  \cite{TESE2018} zusammengetragenen Empfehlungen sind \emph{ein} Aspekt in
  diesem komplexen Abwägungsprozess. Eine Methodik zwischen „technology push“
  und „market pull“ ist dabei mit Blick etwa auf die Ausführungen in
  \cite{Preez2006} eher auf dem Niveau der 1960er Jahre anzusiedeln. Ebenda
  wird in Fig.~3 mit „state of the art in science and technology“ neben den
  „needs of society and marketplace“ ein weiteres Obersystem in Stellung
  gebracht, das auch bei Patenterteilungen mit den Begriffen „Stand der
  Technik“ und „Erfindungshöhe“ eine wichtige Rolle spielt.
\end{quote}
\begin{center}
  \begin{tikzpicture}[
      >={Triangle[length=0pt 6,width=0pt 5]},
      rounded corners=2pt,line width=1pt] 
  \node[draw=green] at (0,2) [circle] (A0)
       {\bbox{2}{Idea Generation}};
  \node[draw=red] at (3,2) [rectangle] (A1)
       {\bbox{1.8}{Develop- ment}};
  \node[draw=red] at (6,2) [rectangle] (A2)
       {\bbox{1.8}{Manufac- turing}};
  \node[draw=red] at (9,2) [rectangle] (A3)
       {\bbox{1.8}{Marketing and Sales}};
  \node[draw=green] at (12,2) [circle] (A4)
       {\bbox{2}{Commercial Product}};
  \node[draw=green] at (6,4.3) [rectangle] (A5)
       {\bbox{8}{State of the art\\[12pt] in science and technology}}; 
  \node[draw=green] at (6,-.3) [rectangle] (A6)
       {\bbox{8}{Needs of society\\[12pt] and the marketplace}};   
  \draw[color=red] (1,.3) node
       {\bbox{2}{\footnotesize \sc{Market Pull}}};
  \draw[color=red] (11.3,3.6) node
       {\bbox{2.3}{\footnotesize \sc{Technology Push}}};
  \draw[color=red] (-.5,4.4) node {\bbox{2}{New\\ Technology}};
  \draw[color=red] (-.5,-.5) node {\bbox{2}{New\\ Ideas}};
  
  \draw[<->] (1.2,2) -- (2,2) ;
  \draw[<->] (4,2) -- (5,2) ;
  \draw[<->] (7,2) -- (8,2) ;
  \draw[<->] (10,2) -- (10.8,2) ;
  \draw[<->] (3,2.5) -- (3,3.5) ;
  \draw[<->] (6,2.5) -- (6,3.5) ;
  \draw[<->] (9,2.5) -- (9,3.5) ;
  \draw[<->] (3,1.5) -- (3,0.5) ;
  \draw[<->] (6,1.5) -- (6,0.5) ;
  \draw[<->] (9,1.5) -- (9,0.5) ;
  \draw[<->] (-.4,4) -- (-.4,3.3) ;
  \draw[<->] (.4,4.5) -- (1.4,4.5) ;
  \draw[<->] (-.4,0) -- (-.4,0.7) ;
  \draw[<->] (.4,-.5) -- (1.4,-.5) ;
  \draw[->,dashed] (1,3) arc(120:60:10) ;
  \draw[<-,dashed] (1,1) arc(-120:-60:10) ;
\end{tikzpicture}\\
Eine Reproduktion von \cite[Fig. 3]{Preez2006}, die dort von
\cite{Galanakis2006} übernommen ist.
\end{center}

In \cite[Fig. 1]{Preez2006} werden sieben Generationen von
Innovationsmanagement-Modellen aufgelistet, das in \cite{TESE2018} als
Neuerung verkaufte gehört in die Kategorie der Modelle der dritten Generation,
die in den 1970er Jahren entwickelt wurden.  Moderne
Innovationsmanagementansätze wie etwa das in \cite{Preez2006} genauer
beschriebene \emph{Fugle Innovation Process Model} gehen in ihrer Komplexität
weit über derartige Ansätze hinaus.

\begin{thebibliography}{xxx}
\bibitem{Galanakis2006} Kostas Galanakis (2006).  Innovation process. Make
  sense using systems thinking.  In: Technovation Volume 26, Issue 11,
  p. 1222--1232.
\bibitem{Graebe2020} Hans-Gert Gräbe (2020). Die Menschen und ihre Technischen
  Systeme. LIFIS Online, 19.05.2020.
  \url{http://dx.doi.org/10.14625/graebe_20200519}
\bibitem{TESE2018} Alex Lyubomirsky, Simon Litvin, Sergei Ikovenko u.a.
  (2018).  Trends of Engineering System Evolution (TESE).  TRIZ Consulting
  Group. ISBN 9783000598463.
\bibitem{Preez2006} Niek D Du Preez, Louis Louw, Heinz Essmann (2006). An
  innovation process model for improving innovation capability.  Journal of
  high technology management research, vol 17, 1--24.
\end{thebibliography}
\end{document}
