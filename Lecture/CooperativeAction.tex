\documentclass{beamer}
\usepackage{lsfolien}
\usepackage[english]{babel}
\usepackage[utf8]{inputenc}

\myfootline{System Modelling and Semantic Web -- Winter term
  2021/22}{Hans-Gert Gräbe}

\newcommand{\ueberschrift}[1]{\begin{center}\bf #1\end{center}}

\title{Modelling Sustainable Systems\\ and Semantic Web\\[6pt] Cooperative
  Action in Digital Change \vskip1em}

\subtitle{Lecture in the Module 10-202-2309\\ for Master Computer Science}

\author{Prof. Dr. Hans-Gert Gräbe\\
\url{http://www.informatik.uni-leipzig.de/~graebe}}

\date{January 2022}
\begin{document}

{\setbeamertemplate{footline}{}
\begin{frame}
  \titlepage
\end{frame}}

\section{Cooperative Action}
\begin{frame}{Cooperative Action. Practical Examples}

\textbf{Examples of cooperative structures}
\begin{itemize}
\item OEIS -- The online encyclopedia of number sequences
\item The Debian Project -- \url{http://www.debian.org/index.de.html} 
\item The Apache Project -- \url{http://www.apache.org/}
\item Java Community Process -- \url{https://www.jcp.org}
\item Wolfram Alpha -- \url{http://www.wolframalpha.com/}
\end{itemize}

Theoretical considerations: The GNU Manifesto

\url{https://www.gnu.org/gnu/manifesto.en.html}

\end{frame}

\begin{frame}{Cooperative Action. Practical Examples}
\begin{itemize}
\item What similarities can be seen?
\item Which priorities characterise internal and external relationship?
\item Which hints for a \emph{theory of forms of cooperation} can be derived?
\item How does this relate to the considerations of the \emph{1985 GNU
  Manifesto}?
\end{itemize}

Example: OEIS -- The Online Encyclopedia of Integer Sequences

\url{https://oeis.org/?language=german}
\end{frame}

\begin{frame}{OEIS -- The Online Encyclopedia of Integer Sequences}
\textbf{Observations:}
\begin{itemize}
\item In the internal relationship, power structures have emerged that are
  based on well-known academic reputation structures.
\item Central moments of an internal personal structuring are Bylaws, Board of
  Trustees, Advisory Board, Editorial Board.
\item There is a "History of the OEIS". Today's structures can only be
  understood on the background of this historical development.
\item Four "goals" are defined
  \begin{itemize}
  \item To own the intellectual property known as "The Online Encyclopedia of
    Integer Sequences" (or "OEIS").
  \item To maintain the OEIS as a service that is freely accessible by the
    general public.
  \item To act so as to maintain its own existence indefinitely.
  \item To collect and distribute funds in order to carry out the first three
    goals.
  \end{itemize}
\end{itemize}\vskip2em
\end{frame}

\begin{frame}{The 5-Level Model}
\begin{enumerate}
\item User: Uses the given options without having to take part in its
  extension. Interested in the \emph{existence} of the platform.
\item Contributor: Posts own content. Contribution to the \emph{development}
  of the platform content.
\item Editorial Board: Review of submissions. Contribution to the
  \emph{quality assurance of the content} of the platform.
\item Platform operator: Reproduction of conditions for the platform to be
  running smoothly (in a comprehensive socio-technical sense) as management of
  the internal relationship.
\item The core of the OIES Foundation: Reproduction of conditions that ensure
  that running the platform is even possible. (Management of the external
  relationship).
\end{enumerate}
What is the relationship between the individual levels and the 5-level model?
\vskip2em
\end{frame}

\newcommand{\abox}[1]{\parbox{4.5cm}{\vskip2pt\centering #1\vskip2pt}}
\begin{frame}{The 5-Level Model}
  \begin{center}
    \begin{tabular}{|c|}\hline
      Users \\\hline
      Editors \\\hline
      Office \\ \hline
      Platform Operator \\\hline
      Financing \\\hline
    \end{tabular}\hfill
    \begin{tabular}{|c|}\hline
      Users\\\hline
      Content Provider \\\hline
      \abox{Organisation and quality assurance of the platform}\\ \hline
      \abox{Socio-technical organisation of the
      infrastructure}\\\hline
      Financing\\\hline
    \end{tabular}
  \end{center}

This relates to five system levels -- the coupling between the system elements
is organised in the system of the next level.
\end{frame}

\begin{frame}{The 5-Level Model}
\textbf{Observation:}
\begin{itemize}
\item The model is typical for today's platform structures and can be found in
  different forms.
\item E.g. Amazon:
  \begin{itemize}
  \item Level 2: Different shop owners.
  \item Level 3: Organisation of the shop operator by Amazon, establishing an
    institutionalised code of conduct and its monitoring as a social level of
    the infrastructure.
  \item Level 4: Technical level of the infrastructure. Research and further
    development of the algorithmic basis as requirement for level 3.
  \item Level 5: Amazon as a private capitalist company.
  \end{itemize}
\end{itemize}
\end{frame}

\begin{frame}{Forms of Cooperative Action}
\textbf{Observations:}
\begin{itemize}
\item The (legal as well as economic) functional logic of civic capitalist
  relationships shapes the internal relationship.
\item Level $i$ creates the infrastructural prerequisites for the level $i-1$.
\item From level 1 to level 5, the scope of personal involvement in the
  cooperative project increases.
\item It is not a relation between equals: From level 1 to level 5 the
  possibility to influence the development of the cooperative project
  increases.
\item There are fluctuations of staff between these levels: Intensive users
  become contributors, hard-working contributors participate in the editorial
  board, etc.
  \begin{itemize}
  \item In the example, reputation and power structures are formed that are
    heavily oriented at academic reputation patterns or, conversely, are
    influenced by them.
  \end{itemize}
\end{itemize}\vskip2em
\end{frame}

\begin{frame}{Forms of Cooperative Action}
\textbf{Observation:} Prosumer approaches can be observed at all levels; there
is no typical division into producers and consumers.
\begin{itemize}
\item The transition from level $i$ to level $i+1$ means to move from a
  \emph{user} of the service of the infrastructure to a \emph{producer} of
  this service within the framework of the cooperative context.
\item Every contributor remains a user, every editor remains a contributor
  etc., and brings in the knowledge about the "what?"
\item Hence the question of the identification of "Customer needs" (what?)
  move in the background in favour of questions of the implementation (how?)
  of cooperative goals on the respective level.
\end{itemize}
\end{frame}

\begin{frame}{Forms of Cooperative Action}\small
The internal structure of capitalist companies follows a similar "top-bottom
logic". From such a perspective the following forms can be distinguished.
\vspace{-1em}
\begin{itemize}
\item[1.] The classical owner-managed company.
  \begin{itemize}
  \item With the notions “ingenious inventor” and “wage labourer”.
    “Intellectual Property” is a right of a person and basis for the
    expropriation of the wage labourer.
  \end{itemize}
\item[2.] Stakeholder-driven company forms such as Stock Corporation.
  \begin{itemize}
  \item With the notion “legal person”. Copyright as economically useful legal
    title in the \emph{external relationship} and basis for expropriation of
    the "ingenious inventor". Copyright, Closed Culture.
  \end{itemize}
\item[3.] Network cooperation.
  \begin{itemize}
  \item Copyright law as a functional basis of the \emph{internal
    relationship} required to reproduce the infrastructure. Copyleft, Open
    Culture.
  \end{itemize}
\item[4.] (Hypothetical?) Free cooperation.
\end{itemize}\vskip2em
\end{frame}
\end{document}
