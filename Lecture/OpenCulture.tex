\documentclass{beamer}
\usepackage{lsfolien}
\usepackage[english]{babel}
\usepackage[utf8]{inputenc}

\myfootline{System Modelling and Semantic Web -- Winter term
  2021/22}{Hans-Gert Gräbe}

\newcommand{\ueberschrift}[1]{\begin{center}\bf #1\end{center}}

\title{Modelling Sustainable Systems\\ and Semantic Web\\[6pt] Open Culture
  \vskip1em}

\subtitle{Lecture in the Module 10-202-2309\\ for Master Computer Science}

\author{Prof. Dr. Hans-Gert Gräbe\\
\url{http://www.informatik.uni-leipzig.de/~graebe}}

\date{February 2022}
\begin{document}

{\setbeamertemplate{footline}{}
\begin{frame}
  \titlepage
\end{frame}}

\begin{frame}{Open Culture as Phenomenon}  
In the lectures and seminars we were faced in many places with “Open Culture“
(Open Source, Open Design, Open Access etc.) as a phenomenon of digital
change. In the last lecture we developed an interpretation framework for this
phenomenon in the context of cooperate action.
\begin{itemize}
\item Digital change is a change \emph{within} the bourgeois civil society.
  \begin{itemize}
  \item Essential constituent elements of a civil society -- privacy,
    prohibition of penetration, right to free speech, personal rights,
    property, monetary system -- are not questioned.
  \end{itemize}
\item These constituent elements stand in tension to one another and need to
  be rebalanced socially under the new conditions.
\end{itemize}
\end{frame}

\begin{frame}{Open Culture as Phenomenon} 
\begin{itemize} 
\item Concept of the \emph{author's work} as individually attributable
  intellectual output that is publicly available. 
  \begin{itemize} 
  \item At the legal level, this means to weigh up between the legitimate
    particular interests of owners and the public interest in free
    accessibility.
  \end{itemize}
\item With the technological simplification of access to digital works the
  field of tension between the consequences of individual attributability of
  intellectual output (“intellectual property”) and their public accessibility
  of this output shifts at the center of the controversy about the further
  development of the civil society.
\item With a broader concept of \emph{Open Culture} a new "cease-fire line"
  started \emph{practically} to establish itself since around 2005.
\end{itemize}
\end{frame}

\begin{frame}{Open Culture as Phenomenon} 
\begin{itemize} 
\item These \emph{practical} changes started with the transition from the
  "Free Software" concept to the "Open Source" concept initiated around 2000.
\item The visionary beginnings of Free Software in the 1980s and its forms of
  institutionalisation prepared the ground for these developments, even if not
  everyone of the activists from the first hour is satisfied with that further
  development.
\item The practical activities and social experiences in the GNU project and
  the GPL as a first \emph{legal technical instrument} played a particularly
  important role.
\end{itemize}
In the following, some aspects of the historical genesis of the term
\emph{intellectual property} and the practical struggle for a related
balancing facts between the aforementioned poles are shown.
\end{frame}

\begin{frame}{On the Way to "Intellectual Property"}
\textbf{Invention of printing.}
\begin{itemize} 
\item The book as \emph{opus} leads to a stronger alignment of content and
  form.
\item The haptic perception of books as artifacts enhances the perception of
  knowledge as a \emph{thing}.
\item The new medium also creates new craftsmanship and professions closely
  related to the formation of the relationships of a civil society.
\item It emerges a new symbioses of technology and power.
  \begin{itemize}
  \item 15th century: Copyright as a monopoly right of the book printers'
    guild -- copying rights, secured by the crown
  \item In mutual interest -- economic interests of the book printers and
    control of "public opinion" by the power.
  \end{itemize}
\end{itemize}
\end{frame}

\begin{frame}{On the Way to "Intellectual Property"}
Two “cultures of knowledge” form the poles of a field of tension.
\begin{itemize}
\item Perception of ideas as individual performance, as a result of creativity
  and ingenuity.
  \begin{itemize}
  \item Basis for the formation of the term “work” and its embedding in the
    (civil) right of personality.
  \end{itemize}
\item Panta rhei -- knowledge as a procedural element of a changing world.
  \begin{itemize}
  \item Newton: "Standing on the shoulders of giants"
  \item Ideas as permanent recombination. Flow of ideas as inherently societal
    achievement.
  \item The safeguarding of the conditions of creative productivity is in the
    foreground.
  \end{itemize}
\end{itemize}
\end{frame}

\begin{frame}{On the Way to "Intellectual Property"}
The tension between these two cultures manifests itself as a field of
tension between two pillars of the civil law:
\begin{itemize}
\item Level of action execution $\to$ \emph{property} as the basis of
  responsibility.
\item Level of action planning $\to$ \emph{freedom} (free as in free speech;
  liberty, freedom of contract) of combinability
\end{itemize}
Development of the legal constitution of a civil society in the 19th century.
\begin{itemize}
\item United States declare their Constitution (Bill of Rights. September 17,
  1787) as an important result of the American Independence War.
\item Civil Code (BGB. January 1, 1900) as the first codification of private
  law in the German Empire.
\end{itemize}
\end{frame}

\begin{frame}{On the Way to "Intellectual Property"}
The beginnings cannot be presented here in a comprehensive way.
\begin{itemize}
\item 1790: Copyright becomes part of the American Constitution (regular
  14-year period of protection).
\item Major differences between Anglo-American and continental European legal
  systems. 
\item Berne Convention for the Protection of Literary and Artistic Works.
\item 1886 first version, 1908 revised Berne Convention.
\item Protection period of at least 50 years after the death of the author. 
\item Harmonization of property rights, equality between nationals and
  foreigners.
\end{itemize}
\end{frame}

\begin{frame}{On the Way to "Intellectual Property"}
\textbf{The spiritual fathers.}
\begin{itemize}
\item Significant increase of the economic importance of science and knowledge
  in the 20th century.
\item 1950s: Fourastié claims that in the tertiary sector will be the most
  important sphere of future value creation.
\item 1960s and 1970s: Milton Friedman and the Chicago School -- Theoretical
  foundation for neoliberalism.
\item Late 1970s: Daniel Bell and the post-industrial society.
\end{itemize}
\end{frame}

\begin{frame}{On the Way to "Intellectual Property"}
The roadmap: Revised Berne Convention
\begin{itemize}
\item Other versions Rome 1928, Brussels 1948, Stockholm 1967
\item 1952 UNESCO Universal Copyright Convention (UCC), in order to involve
  also the USA.
\item In 1967 these topics are united under the roof of the World Intellectual
  Property Organization (WIPO).
\item RBC, Paris version of July 24, 1971 with amendment of September 29, 1979
  -- the version which is valid today.
\item 1973 -- The Soviet Union joins the RBC.
\item 1989 -- The USA joins the RBC.
\item Today (2020) 179 states joined the RBC.\\
  \url{https://www.wipo.int/treaties/en/ip/berne/}
\end{itemize}
\end{frame}

\begin{frame}{On the Way to "Intellectual Property"}
\textbf{The roadmap: the supporters are joining forces.}
\begin{itemize}
\item 1967 founding of WIPO as an umbrella organization for worldwide
  administration of intellectual property rights.
\item 1974 upgrading of WIPO to a sub-organization of the UN
  \begin{itemize}
  \item Manages today RBC, trademark protection agreement, harmonization of
    the patent system and the handling of rights on industrial designs.
  \end{itemize}
\item 1984 the International Intellectual Property Alliance (IIPA) was founded
  for the worldwide implementation of the concept of \emph{Intellectual
    Property} as legal term.
\item 1986 the Intellectual Property Committee (IPC) was founded as industry
  lobby organization complementary to the IIPA to fix “Intellectual Property”
  in the course of the Uruguay Round in GATT.
\end{itemize}
\end{frame}

\begin{frame}{On the Way to "Intellectual Property"}
\begin{itemize}
\item 1980s -- US policy develops various penal mechanisms against countries
  with insufficient IPR legalisation.
\item 1995 TRIPS-1 -- Trade Related Aspects of Intellectual Property Rights --
  as partial result of the GATT negotiations which resulted in the
  establishment of the WTO.
\item 1996 WIPO Copyright Treaty -- member states must implement legal
  measures against circumvention of IPR protection measures.
\item 1998 DMCA -- legal protection of technical IPR protection measures in
  the USA (Digital Millenium Copyright Act).
\item 2001 -- EU guideline on the implementation of the WIPO specifications in
  national copyright law.
\item 2003 -- German UrhG amendment, basket 1 -- "German DMCA".
\end{itemize}
\end{frame}

\begin{frame}{On the Way to "Intellectual Property"}
\begin{itemize}
\item Further German debate:
  \url{https://dini.de/ag/ehemalige-arbeitsgruppen/urhg/}
\item Subjects:
  \begin{itemize}
  \item § 31a -- contracts for unknown types of use.
  \item § 52a, 52b -- availability to the public for teaching and research
    (later moved to a new § 60 and new sections 4, 5 and 5a).
  \item § 53 -- Reproductions for private use and other purposes.
  \end{itemize}
\item ACTA 2006--2012:
  \begin{itemize}
  \item With a vote on July 4, 2012, the EU Parliament decided not to ratify
    ACTA, which means that ACTA does not come into effect for the EU.
  \end{itemize}
\item TTIP since 2012 ... the next attempt.
\end{itemize}
\end{frame}

\begin{frame}{Counteractions from within Science}
October 2003 -- Berlin Declaration on Open Access to Knowledge in the Sciences
and Humanities \url{https://openaccess.mpg.de/Berlin-Declaration}
\begin{itemize}
\item Signed by well-known European and American Research organizations and
  universities.
  \begin{itemize}
  \item As of March 2011, more than 297 institutions from around the world
    were supporting the Berlin Declaration on Open Access.
  \end{itemize}
\item The signees commit themselves to further develop the idea of open
  access by e.g.  encouraging researchers to share their findings in Open
  Access publications.
\item Inclusion of the cultural heritage, i.e.  of the cultural assets kept in
  archives, libraries and museums, in the demand for Open Access.
\end{itemize}
\end{frame}

\begin{frame}{Counteractions from within Science}
2004 -- Göttingen Declaration on Copyright for Education and Research
\url{http://www.urheberrechtsbuendnis.de/ge.html.en}
\begin{itemize}
\item Foundation of the \emph{Coalition for Action “Copyright for Education
  and Research”} as a lobbying organization for science in the struggle around
  the amendment of the German UrhG.  \url{http://www.urheberrechtsbuendnis.de}
\item At the end of 2004 on the basis of the Göttingen Declaration the
  six major German science organizations Science Council, University Rectors'
  Conference, Max Planck Society, Helmholtz Association, Leibniz Association,
  Fraunhofer Society and almost 200 other institutions and 3\,000 individuals
  join forces in this alliance.
\item The Open Access principle is thus becoming increasingly important in the
  scientific field, conducive to the principle structures are established and
  institutionalised.
\end{itemize}\vskip2em
\end{frame}

\begin{frame}{A Slightly Wider Perspective}
  
Perspective yet around 2005: The (re)production conditions of the creatives
have changed dramatically in the last 20 years.  In a world that is more
restrictive and more immaterial, with ownership and and IPR the creatives have
bad cards and are largely defenseless and at the mercy of the owners and their
lawyers.  
\begin{quote}
  As, in the new digital society, creators establish genuinely free forms of
  economic activity, the dogma of bourgeois property comes into active
  conflict with the dogma of bourgeois freedom. (Eben Moglen, The dot
  Communist Manifesto, 2003)
\end{quote}

Visionaries like \emph{Richard Stallman} already envisaged such problems in
the early 1980s: the sustainable reproduction of the conditions of creativity
cannot and must not be left to the owners.
\end{frame}

\begin{frame}{A Slightly Wider Perspective}
  
When the freedom of access to the \emph{works} of others is an essential part
of the conditions of creative, then there \emph{must} be enforced an
appropriate legal weighing of the facts even against the will of the property
owners -- even if the monetary incentives are immense: "Be creative once and
then collect money forever".  
\begin{quote}\footnotesize
  “Free as in free speech not as in free beer” is a basic requirement of
  creative work, Richard Stallman never tires to repeat.
\end{quote}\vspace{-1em}
  
It is in the hands of the creative people themselves -- because they are
producing that "property" -- to organize their own conditions of production in
such a way that knowledge is freely available and everyone has access to it.
\begin{quote}\footnotesize
  Our time offers like no other a vast collection of knowledge in text form.
  The entire intellectual history of mankind is availably on CD-Roms, on
  internet sites, in second-hand bookshops and in book trade, everything is
  well networked and easily accessible, that it would be a shame not to use
  this material awake and with open senses. Because, to cite the smart Bacon
  once more: Knowledge is power. (Matthias Käther, Utopie kreativ, 2005)
\end{quote}
\end{frame}

\begin{frame}{A Slightly Wider Perspective}

With the \emph{GNU Project and Free Software}, that thought first emerged in
an area central to the digital society -- the area in which the tools of the
new society are built.

With the \emph{GNU Public License} (GPL) the meaning of an adequate
legal-technical regulation was early recognized and successfully
"implemented".

\emph{Creative Commons} extends this approach to other areas of culture and
creativity, \emph{Free Culture} (based on the book of the same name by
Lawrence Lessig) captures the cultural significance of such principles.

In this way, processual knowledge is developed to shape the own conditions of
creativity within the framework of the civil legal system.
\end{frame}

\begin{frame}{A Slightly Wider Perspective}

On December 13-14, 2010, the \textbf{International Expert conference "Open
  Access -- Open Data"} took place.  Six years after the first open access
conference in Cologne, it is time to sum up the state of development and
discuss the challenges for the next ten years. In addition, new ways for the
increasingly important open data movement are to be discussed.

The conference is organized by \textbf{Goportis} (now part of TIB Hannover).
Goportis is the name of the \emph{Leibniz library network of research
  information}, consisting of the three German central specialised libraries
TIB (Technical Information Library, Hanover), ZB MED (German Central Library
for medicine, Cologne/Bonn) and ZBW (German Central Library for Economics --
Leibniz Information Center for Economics, Kiel/Hamburg).  
\end{frame}

\begin{frame}{A Slightly Wider Perspective}

After all, with \emph{Open Access} the scientific community as a whole raised
the principle of free access to its own productions as one of their central
future projects. This is shown by the Conference \emph{Open Access and Open
  Data} once again.\vspace{-1em}
\begin{itemize}
\item On December 9, 2014, the Senate of Leipzig University passed a resolution
  "Open Access Policy"
\item With Qucosa \url{http://www.qucosa.de}, Saxony creates with ERDF funds
  (European Regional Development Fund) a Saxonian Open Access infrastructure
  for its academic institutions.
\end{itemize}\vspace{-1em}
The major scientific publishers as the previous advocates of restrictive IPR
can hardly withstand this pressure -- some of them, such as Springer, already
started with \emph{Springer Open Access} to develop appropriate business
models that take account of the new framework conditions.
\end{frame}
\end{document}
