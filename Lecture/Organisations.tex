\documentclass{beamer}
\usepackage{lsfolien}
\usepackage[english]{babel}
           
\myfootline{System Modelling and Semantic Web -- Summer Term 2022}{Hans-Gert
  Gräbe} 

\title{Modelling Sustainable Systems\\ and Semantic Web\\[6pt]
  Organisations as Systems\vskip1em}

\subtitle{Lecture in the Module 10-202-2312\\ for Master Computer Science}

\author{Prof. Dr. Hans-Gert Gräbe\\
\url{http://www.informatik.uni-leipzig.de/~graebe}}

\date{April 2022}
\begin{document}

{\setbeamertemplate{footline}{}
\begin{frame}
  \titlepage
\end{frame}}

\begin{frame}{What is a System?}
  \begin{block}{V. Petrov}
    A \emph{system} is a set of elements that are interconnected and interact
    with each other, forming a unified whole that possesses properties that
    are not already contained in the constituent elements considered
    individually.
  \end{block}

  \begin{block}{TRIZ Ontology Project}
    The necessity of the use of the term "system" occurs when it is required
    to emphasize that something is large, complex, immediately not wholly
    comprehensible, but at the same time a unified whole.

    Unlike the notions "set" or "aggregate", the concept of a system
    emphasises the ordering, the integrity, the regularity of construction,
    functioning and development.
  \end{block}
\end{frame}

\begin{frame}{What is a System?}
In all these definitions, the \emph{structuredness} and thus
\emph{decomposability} of the system in the analytic dimension is emphasised
on the one hand, and the \emph{interdependence} and thus
\emph{indecomposability} in the operating dimension on the other.

In addition to the components, the \emph{connecting elements} also play an
important role.  They mediate the \emph{flow of energy, material and
  information} that is required for the operation of each component.

\begin{block}{Viable Components}
  Viable components deliver processual services in \emph{guaranteed} quantity
  and quality during operation, if the \emph{external operational conditions}
  are guaranteed.
\end{block}

These processual services of the components altogether in combination form the
emergent function of the overall system.
\end{frame}

\begin{frame}{What is a System?}

\begin{block}{Self-similarity of the concept}
  Components can be considered as systems, where the upper system guarantees
  the required for operation throughput of energy, material and information.
\end{block}

\begin{block}{
    System concept as description of complicated real-world phenomena by
    reduction to the essentials.}
  \begin{itemize}
  \item[(1)] Outer demarcation of the system against an \emph{environment},
    reduction of these relationships to input/output relationships and
    guaranteed throughput.
  \item[(2)] Inner demarcation of the system by combining subareas to
    \emph{components}, whose functioning is reduced to “behavioural control”
    via input/output relations.
  \item[(3)] Reduction of the relations in the system itself to “causally
    essential” relationships.
  \end{itemize}
\end{block}

\end{frame}

\begin{frame}{What is a System?}
\begin{block}{
    Such a reductive description (explicitly or implicitly) exploits output from
    prior life:}
  \begin{itemize}
  \item[(1)] An at least vague idea about the (working) input/output services
    of the environment.
  \item[(2)] A clear idea of the inner workings of the components (beyond the
    pure specification).
  \item[(3)] An at least vague idea about causalities in the system itself,
    that precedes the detailed modelling.
  \end{itemize}
\end{block}

\end{frame}

\begin{frame}{Organisations as Systems}

This lecture topic aims to shed more light on the connection between the
concepts of system and (social) organisation.

\textbf{Social organisations} such as companies, associations, projects,
unions, parties, governments, states ... are undoubtedly theoretically
delimitable and practically delimited parts of reality with outwardly (and
inwardly) directed goals and purposes whose internal structure and dynamics
are driven by an external throughput of energy, material and information, and
which can therefore be studied from a systemic perspective.

These throughputs are already mentally charged in language form in a
\textbf{more complex social context} and in the form of interests, needs,
monetary flows and power relations.
\end{frame}
  
\begin{frame}{Systematic Management Basics}

Systemic development of social organisations is taking place in the unity and
difference of \emph{planned action} and \emph{experienced results}.

This requires a concept of \emph{action planning}, based on a \emph{conceptual
  understanding} of the process landscape within and around the organisation
in an appropriate explicit form of description and \emph{intelligible
  operational actions}.
  
  \begin{itemize}
  \item Planning and experience
  \item Dialectical relation of tension between plan and reality
  \item Practical real-world development and \emph{practice of thinking}
  \item Feedback relation between planning and experience
  \end{itemize}

\end{frame}

\begin{frame}{Organisation}

  \begin{block}{Formal Organisation (Wikipedia)}
    
    An organisation that is established as a \emph{means for achieving defined
      objectives} has been referred to as a formal organisation.\medskip

    Its design specifies how \emph{goals are subdivided and reflected} in
    subdivisions of the organisation. Divisions, departments, sections,
    positions, jobs, and tasks make up this work structure.\medskip

    Thus, the formal organisation is expected to \emph{behave impersonally} in
    regard to relationships with clients or with its members. [...]\medskip

    A \emph{bureaucratic structure} forms the basis for the appointment of
    heads or chiefs of administrative subdivisions in the organisation and
    endows them with the authority attached to their position.  (my emphasis)
  \end{block}
\end{frame}


\begin{frame}{Organisation}
  \begin{block}{Informal Organisation (Wikipedia)}
   
    [...] The informal organisation expresses the personal objectives and
    goals of the individual membership.

    Their objectives and goals may or may not coincide with those of the
    formal organisation. [...]
  \end{block}

  Digression to ORG and FOAF ontologies.
  \begin{itemize}
  \item ORG: \url{https://www.w3.org/TR/vocab-org}
  \item FOAF: \url{http://xmlns.com/foaf/spec/}
  \end{itemize}

  \begin{center}\Large Discussion  \end{center}
  
\end{frame}

\begin{frame}{The ORG Ontology}

  \texttt{org:OrganizationalUnit}, \texttt{org:FormalOrganization} and
  \texttt{org:OrganizationalCollaboration} as subconcepts of the concept
  \texttt{org:Organization}.

  \begin{block}{Organisation}
    ... represents a collection of people organised together into a community
    or other social, commercial or political structure.\medskip

    The group has some common purpose or reason for existence which goes
    beyond the set of people belonging to it and can act as an Agent.\medskip

    Organisations are often decomposable into hierarchical structures.
  \end{block}
\end{frame}

\begin{frame}{Organisation}

  \texttt{org:Organization} is related to \texttt{foaf:Agent}

  ... the class of agents; things that do stuff. A well known sub-class is
  \texttt{foaf:Person}, representing people. Other kinds of agents include
  \texttt{foaf:Organization} and \texttt{foaf:Group}. 

  A \texttt{foaf:Group}

  ... represents a collection of individual agents (and may itself play the
  role of a Agent, i.e. something that can perform actions).

  This concept is intentionally quite broad, covering informal and ad-hoc
  groups, long-lived communities, organisational groups within a workplace,
  etc. ...
\end{frame}

\begin{frame}{Organisation}

  While a Group has the characteristics of a Agent, it is also associated with
  a number of other Agents (typically people) who constitute the Group, its
  members. ...  The basic mechanism for saying that someone is to use the
  member property of the Group to indicate the agents that are members of the
  group.
  
  The terms Agent and Group thus introduce self-similar concepts of structures
  that are \emph{capable of action}. This corresponds to the legal
  construction of a \emph{juridical subject} in the sense of the Civil Code
  (BGB) if \emph{responsibility for the consequences of action} is added.

\end{frame}

\begin{frame}{Organisation as Socio-Technical System}

This corresponds closely with the \emph{system concept developed so far}:

  A \emph{system} is a \emph{delimited set of elements} (components, objects,
  resources) that are interconnected and interact with each other. Their
  interaction realises a \emph{qualitatively new function} (emergent function)
  and thus constitutes a \emph{new unified whole}.

  A system has a \emph{structural} and an \emph{operational} dimension which
  are in contradictory dialectical relation of decomposability and
  indecomposability.

  The \emph{operation of a system} requires a qualitatively and quantitatively
  defined throughput of energy, material and information.
  
\end{frame}

\begin{frame}{Organisation as Socio-Technical System}

  Ian Sommerville \emph{Software Engineering} also starts with the concept of
  a system and moves from there to the concept of \emph{organisation}.
  \vskip2em
  
\begin{block}{System}
  A system is a meaningful set of interconnected components that work together
  to achieve a specific goal.  
\end{block}
\end{frame}

\begin{frame}{Organisation as Socio-Technical System}
  \begin{block} {Technical computer-based systems}
    ... are systems that contain hardware and software components, but not
    procedures and processes. ... Individuals and organisations use technical
    systems for specific purposes, but knowledge of that purpose is not part
    of the system.\medskip

    For example, the word processor I use does not know that I am using it to
    write a book.
  \end{block}

\end{frame}

\begin{frame}{Organisation as Socio-Technical System}

  \begin{block}{Socio-technical systems}
    ... contain one or more technical systems, but beyond that -- and this is
    crucial -- the knowledge of how the system should be used to achieve a
    broader purpose.\medskip

    This means that these systems have \emph{defined work processes},
    \emph{human operators} as integral part of the system, are \emph{governed
      by organisational policies} and are \emph{affected by external
      constraints} such as national laws and regulations.
  \end{block}

\end{frame}

\begin{frame}{Essential Characteristics of Socio-Technical Systems}
\begin{itemize}
\item They have special properties that affect the system as a whole, and are
  not related to individual parts of the system. These special properties
  depend on the system components and the relationships between them. Because
  of this complexity, the system-specific properties can only be evaluated
  when the system is composed.\medskip
\item They are often not deterministic. The behaviour of the system depends on
  the human operators and on other people who do not always react in the same
  way. Also, the operation of the system can change the system itself.
\end{itemize}
\end{frame}

\begin{frame}{Essential Characteristics of Socio-Technical Systems}
\begin{itemize}
\item The extent to which the system supports organisational goals depends not
  only on the system itself. It also depends on the \emph{stability of the
    goals}, the relationships and \emph{conflicts between organisational
    goals}, and how people in the organisation \emph{interpret those goals}.
\end{itemize}

In this context, there is a clear shift on the scale of controllability from
direct control (technical systems) to indirect control (socio-technical
systems), which in \textbf{socio-economic systems} with a large number of
stakeholders or even \textbf{socio-ecological systems} shifts further in the
direction of movement according to intrinsic laws ("natural processes").

\begin{center}\Large Discussion  \end{center}
  \end{frame}

\begin{frame}{Shchedrovitsky on Organisations}
What is an organisation for Shchedrovitsky? He distinguishes three dimensions
of that notion
\begin{itemize}
\item Organisational work as practical activity.
\item Organisation as the result and means of organisational work.
\item Organisation as a form of life of the collective.
\end{itemize}

\end{frame}

\begin{frame}{Organisational Work}
When we organise we collect something. We need some structural elements ...
We must collect these elements in a particular way, and we must establish some
kind of connection and relations between them. When we are doing this sort of
work we must impose some organisational form on these elements. [\ldots]

\end{frame}

\begin{frame}{Organisation as the Result and Means\\ of Organisational Work}
Organisation as the result of organisational work can be regarded as both an
\textbf{artificial entity} and as \textbf{naturally living thing}.

The main question is: why does the organiser create a particular organisation?
The organisation acts here as an \textbf{artificial entity}. It has a purpose
(Zweck) and can be considered ... in terms of the functions that it, the
organisation (system!), must provide. ... These are all characteristics that
are seen from an artificial point of view.

As a tool, as a means, \textbf{as an artificial entity, the organisation does
  not and cannot have goals} (Ziele). Organisers can have goals. But for their
goals, in relation to their goals, the organisations they create are a means,
a means for them to achieve their goals.

\end{frame}

\begin{frame}{Organisation as a Form of Life of the Collective}
The organisation has been created, and it has begun to live its own life. And
then it turns out that, from a natural point of view, other goals may appear
in this organisation – the goals of the collective, which was
organised. Generally, something quite different begins, this
\textbf{organisation begins to live its own life}. Then we [\ldots] must seek
forms, methods, laws of the life of the groups and the collectives within
organisations.

When the organisation is seen from a natural viewpoint, it is not yet the
means, but the \textbf{form}, the \textbf{condition} of the life of the
collective (the people) who work in it.

\end{frame}
\end{document}
