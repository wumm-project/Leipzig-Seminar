\documentclass[11pt,a4paper]{article}
\usepackage{a4wide,url,enumitem}
\usepackage[english]{babel}
\usepackage[utf8]{inputenc}
\newcommand{\enquote}[1]{``#1''}
\setlist{noitemsep}
\parindent0pt
\parskip4pt

\title{Concept of the Lecture \\[.3em] \emph{Modelling Sustainable Systems and
    Semantic Web}}

\author{Hans-Gert Gr\"abe, InfAI Leipzig}

\date{July 29, 2022}

\begin{document}
\maketitle

\section*{General}

Until the summer semester 2022, the lecture was an integral part of the
courses offered by the working group \emph{Systematic Innovation
Methodologies}\footnote{\url{https://infai.org/systematische-innovationsmethodiken/}}
at the InfAI Leipzig for several years, but was recently attended only
irregularly and is therefore no longer offered in our regular schedule.

Since the lecture brings together essential concepts on the topics technology,
digital transformation, semantic web and on concept formation processes in
cooperative action, which have emerged over the years in our interdisciplinary
work, the lecture slides are collected here to have a uniform basis for their
further development.

The lecture consists of three parts.

In the first part the concept of a \emph{technical system} is explored and the
main concepts of TRIZ as an important systematic innovation methodology are
introduced.  In contrast to other creativity and innovation methodologies,
TRIZ focuses on the systematisation of engineering experiences.

In the second part, aspects of the creation of conceptual networks for data
models are studied more closely on the basis of the \emph{Resource Description
Frameworks} (RDF), the \emph{Linked Open Data Cloud}, the emerging \emph{Giant
Global Graph} and the importance of these developments for the organisation of
contexts of cooperative action and hence management of sustainable systems is
discussed.

Finally, in the third part, the role of data and information and the
generation of new language tools for the development of technical systems in
the context of a civil society are explored and, in particular, the importance
of concept formation processes in cooperative action are considered in more
detail.

A detailed plan of the lecture concept and a list of references can be found
in the \texttt{README.md} file in this repository.  Most of the material is
easily found on the internet. 
\newpage

\section*{Topics of the Lecture}

\begin{itemize}
\item Introduction. Concept of technology
\item Systems and their Development
\item Systems and Sustainability
\item Systemic Structures and Spaces of Action
\item Digital Spaces of Action
\item Internet Basics
\item RDF Basics
\item Modelling Conceptual Worlds
\item The Giant Global Graph
\item Data and Information
\item Storytelling, Conceptualisation, Information and Language
\item Knowledge and Action
\item On a Theory of Cooperative Action
\item Economic Network Structures and \enquote{Platform Capitalism}
\item The History of the .NET Project
\item Formation of an Open Culture
\end{itemize}

\end{document}
