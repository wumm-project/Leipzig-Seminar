\documentclass[11pt,a4paper]{article}
\usepackage{ls,enumitem}
\usepackage[english]{babel}

\setlist{noitemsep}

\title{Handout for the seminar \emph{Systems Science}}

\author{Hans-Gert Gr\"abe, Ken Pierre Kleemann, Lydie Laforet, Sabine
  Lautenschläger}

\date{Januar 11, 2020}

\begin{document}
\maketitle

\section{Aim and methodology of the seminar}

The concept of system plays a prominent role in computer science when it comes
to database systems, software systems, hardware systems, accounting systems,
access systems, etc.  In general computer science is regarded by a majority as
the "science of the \emph{systematic} representation, storage, processing and
transmission of information, especially the automated processing using digital
computers" (Wikipedia).  Also certain relevant professions such as
\emph{system architect} are held in high esteem by IT users.  

However, the importance of the concept of system extends far beyond the field
of computer science -- it is fundamental for all engineering sciences and as
\emph{Systems Engineering} with the ISO/IEC/IEEE-15288 standard "Systems and
Software Engineering", it is also the subject of international standardisation
processes.  Even more, the concept of systems also plays a role in the
description of complex natural and cultural processes -- for example in the
concept of an \emph{ecosystem}.

With the \emph{Semantic Web}, the analysis of the meaning of digital artefacts
becomes central, which are ultimately language artefacts and thus also
directly related to a meaningfully unfolded \emph{concept of a system} as
basis of any understanding of concrete systems.

The keyword \emph{Sustainability} finally refers to complex social processes
of coordination are addressed, which are accompanied by many information and
evaluation problems. Here the ability of the descriptive delimitation,
development and control of so-called systems on or across different
governance, spatial and temporal levels is of great importance. 

\begin{quote}
  \textbf{The objective of the seminar} is to gain a better understanding of
  this diversity of systemic concepts and to \emph{different system theories}
  as the subject of a \emph{Systems Science}.
\end{quote}

The seminar is an introductory course to systems science at the Master's
level, its development over time, ramification of approaches, key terms and
concepts.  \emph{Systems Science} is used here as an term for a field that
numerous scholars from a wide range of disciplines such as anthropology,
biology, chemistry, ecology, economics, mathematics, physics, psychology,
sociology and others have contributed. Developments such as cybernetics, chaos
theory or network analysis and network science can be seen as part of systems
science or are at least strongly related to it.  Some branches of systems
science are even regarded in Germany as new scientific fields with their own
rights such as synergetics or complexity science.

These developments have opened up new possibilities for improved analysis and
and decision-making in scientific, business and political areas. However, we
see on a daily basis that in complicated situations, especially in politics
and business, simple and direct decision-making processes still prevail,
leading to an increase in negative developments when the originally intended
effects do not materialise. Any unexpected side effect or counter-action, that
render the measures useless are a clear indication that the actors' mental
models were incomplete and broader systemic correlations have been neglected.
Systems thinking is therefore of particular importance for the transition
to a more sustainable society in Germany.  

This seminar we will trace the historical development of systems science (in
parts) and study relevant basic concepts. Course participants do not adhere to
any specific model (such as \emph{system dynamics}), but develop a deeper
understanding of systems science and a specific kind of "systemic thinking"
that can be used to address sustainability problems more successfully. We
achieve this by reading and discussing scientific papers and book chapters.

Students are expected to actively participate in the seminar through seminar
discussions, presentations, seminar papers and, last but not least by reading
the relevant material.  The course participants are asked and encouraged to
develop their own approach to the topic of sustainability.

\section{Course structure}

The course is held weekly according to the published schedule in a
presentation and discussion format.  Each week, seminar participants have to
study the assigned reading in advance and be prepared to discuss it in the
seminar. After a brief input from the seminar leader the student assigned as
\emph{discussion leader} is responsible for moderating the seminar discussion.
In preparation for the seminar, \emph{opponents} write short position papers.
The participants are expected to either lead the seminar or actively
participate in the discussion.

For weeks in which you \emph{lead} the seminar, you 
\begin{enumerate}
\item have completely studied the reading given for the session,
\item have to prepare a presentation and present it in about 20 minutes as an
  oral summary of the main points of the session,
\item have to prepare discussion questions to lead a group discussion for
  about 30 minutes.
\end{enumerate}
For weeks in which you are assigned \emph{as opponent}, you 
\begin{enumerate}
\item have studied in full the reading specified for the session, 
\item have to write a position paper of approximately 800 words (pdf, 11pt,
  single-spaced) and place it in the materials folder of the seminar.
\end{enumerate}

The presentations and the position papers will (later) be uploaded to github
and thus published.

\section{Position papers}

Each student writes three position papers and leads one seminar discussion.
You cannot also write a position paper for the discussion you are leading.
The position papers should clearly reference to the reading material of the
respective seminar. In the position papers, students briefly summarise the
main points of the reading material and add their own input. These positions
can take different forms. They can
\begin{itemize}
\item combine the respective theoretical aspect of systems science with
  questions of sustainable development at large,
\item compare or contrast the approach of the author with that of another
  author who has already been discussed in the seminar, or
\item comment on the scope of the seminar topic.
\end{itemize}
The fulfilment of these performances will not be assessed\footnote{We
  emphasise the academic nature of our project -- it is not about evaluating
  your performance, but about \emph{joint} acquisition of knowledge on a basis
  of equals.}, but is to be considered as prerequisite for admission to the
written examination that concludes the module.

\section{Literature}

\begin{itemize}
\item John M. Anderies, Marco A. Janssen, Elinor Ostrom (2004).  Framework to
  Analyze the Robustness of Social-ecological Systems from an Institutional
  Perspective. In: Ecology and Society 9 (1), 18.\\
  \url{https://www.ecologyandsociety.org/vol9/iss1/art18/}
\item William Ross Ashby (1958).  Requisite variety and its implications for
  the control of complex systems. In: Cybernetica 1:2, 83--99.\\
  \url{http://pcp.vub.ac.be/Books/AshbyReqVar.pdf}
\item Ludwig von Bertalanffy (1950). An outline of General System Theory,
  The British Journal for the Philosophy of Science, Volume I, Issue 2, 1
  August 1950, 134–165.\\ \url{https://doi.org/10.1093/bjps/I.2.134}
\item C.R. Binder, J. Hinkel, P.W. Bots, C. Pahl-Wostl (2013). Comparison of
  Frameworks for Analyzing Social-ecological Systems. Ecology and Society,
  18 (4), 26.\\ \url{https://www.ecologyandsociety.org/vol18/iss4/art26/}
\item Max Boisot, Bill McKelvey (2011). Complexity and
  Organization-Environment Relations: Revisiting Ashby’s Law of Requisite
  Variety. In: Allen, Peter, Steve Maguire and Bill McKelvey (eds.). The Sage
  Handbook of Complexity and Management, 279--298. (Available at
  \url{semanticscholar.org})
\item Fridolin Simon Brand, Kurt Jax (2007).  Focusing the Meaning(s) of
  Resilience: Resilience as a Descriptive Concept and a Boundary Object. In:
  Ecology and Society 12 (1), 23.
  \url{https://www.ecologyandsociety.org/vol12/iss1/art23/}
\item Leonhard Dobusch, Volker, Sigrid Quack (2011). Auf dem Weg zu einer
  Wissensallmende? Argumente Politik und Zeitgeschichte 28--30, S. 41--46.
\item T.J. Foxon, M.S. Reed, L.C. Stringer (2009). Governing long‐term
  social–ecological change: what can the adaptive management and transition
  management approaches learn from each other? Environmental Policy and
  Governance, 19 (1), 3--20.\\ \url{https://doi.org/10.1002/eet.496}
\item Frank W. Geels, Johan Schot (2007). Typology of Sociotechnical
  Transition Pathways. In: Research Policy 36 (2007), 399–417.\\
  \url{https://doi.org/10.1016/j.respol.2007.01.003} 
\item B.I. Goldovsky (1983). System der Gesetzmäßigkeiten des Aufbaus und der
  Entwicklung technischer Systeme.
  \url{https://wumm-project.github.io/Texts.html} 
\item Hans-Gert Gräbe (2019). Zur Entwicklung Technischer Systeme.
  Manuskript. \\ \url{https://wumm-project.github.io/Texts.html}
\item John H. Holland (2006). Studying complex adaptive systems. In: Journal
  of systems science and complexity, 19 (1),
  1–8.\\ \url{https://link.springer.com/article/10.1007/s11424-006-0001-z}
\item C.S. Holling (2000). Understanding the Complexity of Economic,
  Ecological, and Social Systems. In: Ecosystems (2001) 4, 390–405.
  \url{https://www.esf.edu/cue/documents/Holling_Complexity-EconEcol-SocialSys_2001.pdf}
\item Gisela Jacobasch (2019). Bienensterben -- Ursachen und Folgen.  Leibniz
  Online 37 (2019).
  \url{https://leibnizsozietaet.de/bienensterben-ursachen-und-folgen/}
\item Erich Jantsch (1992). Die Selbstorganisation des Universums. Vom Urknall
  zum menschlichen Geist.  Hanser, München.
\item Christian Jooß (2017). Selbstorganisation der Materie.  Verlag Neuer
  Weg, Essen.
\item Friedhart Klix, Karl Lanius (1999). Wege und Irrwege der
  Menschenartigen.  Kohlhammer, Stuttgart.
\item Karl Koltze, Valeri Souchkov (2017). Systematische Innovation.  2nd
  edition, Hanser, Munich.
\item Anton Kozhemyako (2019). Features of TRIZ applications for solving
  organizational and management problems: schematization of an inventive
  situation and working with models of contradictions. (In Russian).\\
  \url{https://matriz.org/kozhemyako/}
\item A. Lyubomirskiy, S. Litvin, S. Ikovenko, C.M. Thurnes, R. Adunka (2018).
  Trends of Engineering System Evolution (TESE).
\item Darrell Mann (2019).  Systematic innovation in complex
  environments. Proceedings of the TRIZ Summit 2019 Minsk.\\
  \url{https://triz-summit.ru/file.php/id/f304797-file-original.pdf}
\item C. Mele, J. Pels, F. Polese (2010). A brief review of systems theories
  and their managerial applications. Service Science, 2(1--2), 126--135.\\
  \url{https://doi.org/10.1287/serv.2.1_2.126}
\item John Mingers (1989). An Introduction to Autopoiesis -- Implications and
  Applications. In: Systems Practice, Vol. 2, No. 2, 1989.\\
  \url{https://link.springer.com/article/10.1007/BF01059497} 
\item Elinor Ostrom (2007). A diagnostic approach for going beyond panaceas.
  Proceedings of the national Academy of sciences, 104(39), 15181--15187.\\
  \url{https://doi.org/10.1073/pnas.0702288104}
\item Ilya Prigogine, Isabelle Stengers (1993). Das Pardox der Zeit. Piper,
  Munich, ch. 3--5.  
\item Günter Ropohl (2009). Allgemeine Technologie: eine Systemtheorie der
  Technik.  KIT Scientific Publishing.
  \url{https://books.openedition.org/ksp/3007} (Open Access) 
\item Mikhail Rubin (2019).  Zum Zusammenhang der Entwicklungsgesetze
  allgemeiner Systeme und der Entwicklungsgesetze technischer Systeme. \\
  \url{https://wumm-project.github.io/Texts.html}
\item Valeri Souchkov (2014). Breakthrough Thinking with TRIZ for Business
  and Management: An Overview. \url{https://www.semanticscholar.org}
\item Volker Stollorz (2011). Elinor Ostrom und die Wiederentdeckung der
  Allmende. Argumente Politik und Zeitgeschichte 28--30, S. 3--15. 
\item Clemens Szyperski (2002). Component Software. 2nd edition. Pearson
  Education. 
\item Robert E. Ulanowicz (2009). The dual nature of ecosystem dynamics.
  In: Ecological Modelling 220 (2009), 1886–1892.\\
  \url{https://people.clas.ufl.edu/ulan/files/Dual.pdf} 
\item V.I. Vernadsky (1997, Original 1936--38). Scientific Thought as a
  Planetary Phenomenon. \url{https://wumm-project.github.io/Texts.html}
\item Brian Walker, C. S. Holling, Stephen R. Carpenter, Ann Kinzig (2004).
  Resilience, Adaptability and Transformability in Social-ecological Systems. 
  In: Ecology and Society 9 (2).
  \url{https://www.ecologyandsociety.org/vol9/iss2/art5/}
\end{itemize}

\end{document}

\section{Zusatzliteratur}

Allgemeine Fachliteratur zum Thema Systemwissenschaft (x markiert leichte
Lesbarkeit).

Ggf. noch zusammenzustreichen (Laforet/Lautenschläger).

\begin{itemize}
\item Arthur, W. Brian (2009). The Nature of Technology. (x)
\item Arthur, Brian (2013). Complexity Economics: A Different Framework for
  Economic Thought. SFI working paper: 2013-04-012. 
\item Ashby, Ross (1956/2015). An Introduction to Cybernetics. 
\item Bateson, Gregory (1972/1987/2000). Ecology and Flexibility in Urban
  civilization. In: Bateson, Gregory. Steps to an Ecology of Mind.
\item Bednar, Jenna (2016). Robust Institutional Design – What Makes Some
  Institutions More Adaptable and Resilient to Changes in Their Environment
  Than Others? In: Wilson, David S., Alan Kirman (Eds.).  Complexity and
  Evolution -- Toward a New Synthesis in Economics, pp. 167--184, Strüngmann
  Forum Reports, MIT Press.
\item Beinhocker, Eric (2006/2007). The Origin of Wealth: Evolution,
  Complexity, And the Radical Remaking of Economics.  (x)
\item Bengtsson, Janne, Per Angelstam, Thomas Elmqvist, Urban Emanuelsson,
  Carl Folke, Margareta Ihse, Fredrik Moberg, Magnus Nyström (2003).
  Reserves, Resilience and Dynamic Landscapes. In: AMBIO: A Journal of the
  Human Environment, 32 (6), pp. 389--396.  
\item Bertalanffy, Ludwig von (1969/2006). General Systems Theory. 
\item Braitenberg, Valentino (1984). Vehicles – Experiments in Synthetic
  Psychology.  (x)
\item Cilliers, Paul (2001). Boundaries, Hierarchies and Networks in Complex
  Systems. In: International Journal of Innovation Management, Vol. 5, No. 2
  (June 2001), pp. 135–147.
\item Colander, David, Roland Kupers (2014). Complexity and the Art of Public
  Policy.  (x)
\item Erdi, Peter (2010). Complexity Explained. 
\item Fath, Brian D. (2017). Systems Ecology, Energy Networks, and a Path to
  Sustainability. In: Int. J. of Design \& Nature and Ecodynamics. Vol. 12,
  No. 1 (2017), pp. 1–15. 
\item Frischmann, Brett M. (2013). Two Enduring Lessons from Elinor Ostrom. In:
  Journal of Institutional Economics, 9, pp. 387--406.
\item Gowdy, John, Mariana Mazzucato, Jeroen C.J.M. van den Bergh, Sander E.
  van der Leeuw, David S. Wilson (2016). In: Wilson, David S., Alan Kirman
  (Eds.). Complexity and Evolution -- Toward a New Synthesis in Economics,
  pp. 327--350, Strüngmann Forum Reports, MIT press.
\item Gunderson, Lance H. (2000). Ecological Resilience -- In Theory and
  Application. In: Annual Review of Ecology and Systematics, Vol. 31 (2000),
  pp. 425--439. 
\item Hartmann, Dominik, Cristian Jara-Figueroa, Miguel Guevara, Alex Simoes,
  César A. Hidalgo (2016). The Structural Constraints of Income Inequality in
  Latin America. In: Integration \& Trade Journal, No. 40, June 2016,
  pp. 70--85. 
\item Hausmann et al. Atlas of Economic Complexity. (New version on sale at
  MIT press, free download of older versions) (x)
\item Heylighen, Francis (2008). Complexity and Self-organization. In: Marcia
  J. Bates and Mary Niles Maack (eds.). Encyclopedia of Library and
  Information Sciences.
\item Kauffman, Stuart A (1993). The Origins of Order. 
\item Kauffman, Stuart A (1996). At Home in the Universe. (x)
\item Kauffman, Stuart A (2008/2010). Reinventing the Sacred. 
\item Kharrazi, Ali, Elena Rovenskaya, Brian D. Fath, Masaru Yarime, Steven
  Kraines (2013). Quantifying the sustainability of economic resource
  networks: An ecological information-based approach. In: Ecological Economics
  90 (2013), pp. 177–186.
\item Latour, Bruno (1996). On actor-network theory: A few clarifications. In:
  Soziale Welt, 47. Jahrg., H. 4 (1996), pp. 369--381.\\
  \url{http://www.jstor.org/stable/40878163}
\item Lichtenstein, B., Bill McKelvey (2011). Four types of emergence: a
  typology of complexity and its implications for a science of management. In:
  Int. J. Complexity in Leadership and Management, Vol. 1, No. 4, 2011.
\item Maturana, Humberto (1975). The Organization of the Living: A Theory of
  the Living Organization. In: Int. J. Man-Machine Studies (1975) 7, 313--332.
\item McKelvey, Bill (2001). Energising Order-Creating Networks of Distributed
  Intelligence Improving the Corporate Brain. In: International Journal of
  Innovation Management, Vol. 5, No. 2 (June 2001), pp. 181–212.
\item Meadows, Donella H. (2008). Thinking in Systems. (x)
\item Mitchell, Melanie (2009). Complexity – A Guided Tour. (x)
\item Morin, Edgar (2008). On Complexity. 
\item Noe, Egon, Hugo Fjelsted Alrøe (2003). Combining Luhmann and
  Actor-Network Theory to see Farm Enterprises as Self-organizing Systems.
  Paper presented at \emph{The Opening of Systems Theory} in Copenhagen, May
  23--25, 2003.
\item Ostrom, Elinor (2009). A General Framework for Analyzing Sustainability
  of Social-Ecological Systems. In: Science 325, 419 (2009). 
\item Page, Scott (2011). Diversity and Complexity. (x)
\item Seidl, David (2004). Luhmann’s theory of autopoietic social systems.
\item Senge, Peter M. (1990/2006). The Fifth Discipline. (x)
\item Sterman, John (2000). Business Dynamics: Systems Thinking and Modeling
  for a Complex World.  (x)
\item Ulanowicz, Robert E. (2007).  Ecosystems becoming. In: Int. Journal of
  Ecodynamics. Vol 2, No. 3 (2007), pp. 153--164 
\item Ulanowicz, Robert E. (2009). A Third Window: Natural Life beyond Newton
  and Darwin.  (x)
\item van der Leeuw, Sander E. (2016). Adaptation and Maladaptation in the
  Past. In: Wilson, David S., Alan Kirman (Eds.) Complexity and Evolution --
  Toward a New Synthesis in Economics, pp. 239--269, Strüngmann Forum Reports,
  MIT Press.
\item Wilson, David S. (2016). Two Meanings of Complex Adaptive Systems. In:
  Wilson, David S., Alan Kirman (Eds.).  Complexity and Evolution -- Toward a
  New Synthesis in Economics, pp. 31–46, Strüngmann Forum Reports, MIT Press.
\item Wilson, David S., Alan Kirman (Editors, 2016). Complexity and Evolution:
  Toward a New Synthesis for Economics. (Strüngmann Forum Reports).  (x)\\
  \url{https://www.esforum.de/publications/sfr19/Complexity and Evolution.html}
\item Woermann, Minka (n.y.). What is Complexity Theory? Features and
  Implications. 
\end{itemize}

\end{document}
