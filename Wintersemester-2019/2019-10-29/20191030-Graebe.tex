\documentclass[11pt,a4paper]{article}
\usepackage{a4wide,url}
\usepackage[utf8]{inputenc}
\usepackage[german]{babel}

\parindent0pt
\parskip4pt
\title{Standpunktpapier zum Seminartermin am 29.10.2019}

\author{Hans-Gert Gr\"abe}

\date{30. Oktober 2019}

\begin{document}
\maketitle

In diesem Standpunktpapier werden einige Aspekte der Theorie dynamischer
Systeme (TDS) mit den von Lautenschläger und Laforet vorgetragenen
Systemtheorieansätzen abgeglichen und damit zugleich einige Punkte der TDS
genauer ausgeführt.

\section{Bertalanffys Allgemeine Systemtheorie}

Bertalanffy entwickelt in seinem Text (Bertalanffy 1950) zunächst genau die
Grundlagen der TDS im Verständnis jener Zeit.  Der Bezugstext steht damit ganz
am Anfang einer stürmischen Entwicklung der TDS in den 1960er und 1970er
Jahren, die zu fundamental neuen Einsichten in die Vielfalt von Formen der
Lösungen gewöhnlicher Differentialgleichungssysteme geführt haben.  Bereits in
diesem Gebiet\footnote{In den Gleichungen werden nur zeitabhängige Ableitungen
  zugelassen, keine partiellen Ableitungen nach auch noch anderen Parametern,
  das Gebiet der \emph{partiellen Differentialgleichungen} wird also noch
  nicht betreten.} finden sich erstaunliche Phänomene wie der Lorenzattraktor,
deterministisches Chaos, das Ende des Trajektorienbegriffs und fraktale
Gebilde. Mit partiellen Differentialgleichungen kommen noch
Solitonen\footnote{Auf dieses Phänomen bin ich in meinem Seminar nicht
  eingegangen, obwohl diese Strukturen, die in vielen Systemen partieller
  Differentialgleichungen als Lösungen auftreten, zu einem vollkommen neuen
  Verständnis des Welle-Teilchen-Dualismus führen. Siehe dazu
  \url{https://de.wikipedia.org/wiki/Soliton}. } hinzu. Bertalanffy hat also
nur eine erste Ahnung möglicher Phänomene. Seine mathematischen Betrachtungen
verwenden allein Taylorreihen und beschränken sich damit auf Phänomene nahe
einer Gleichgewichtslage, können also mathematisch auf steady state
Situationen (ohne wesentlich vereinfachende Annahmen) nicht einmal angewendet
werden.

Seine wissenschaftstheoretischen Überlegungen fußen auf der Analogie
entsprechender mathematischer Beschreibungsformen in verschiedenen
Wissenschaftsgebieten\footnote{Komplexe Systemtheorie stellt die Adäquatheit
  derartiger Beschreibungen heute selbst in Frage.} und stellen damit nach
meinem Verständnis auf \emph{methodologische} Ähnlichkeit von Zugängen und
\emph{nicht} auf Isomorphie von Strukturen (so Lautenschläger) ab. Dass
Bertalanffys Zugang \emph{deduktiv} sei, kann sich damit auch maximal auf den
mathematisch-deduktiven Kern seiner Argumentation beziehen, nicht aber auf die
weitergehenden wissenschaftstheoretischen Beobachtungen, bzw. dies wäre noch
genauer zu belegen.

\section{Der Raumbegriff der TDS}

Der Raumbegriff der TDS entwickelt sich aus dem physikalischen Begriff des
\emph{Phasenraums}. So „lebt“ ein klassisches Vier-Teilchen-System in einem
12-dimensionalen Phasenraum, der durch die $4\times 3$ Raumkoordinaten
aufgespannt wird. Derartige Phasenräume dienen zunächst der Koordinatisierung
der Bewegungsgleichungen, allerdings sieht bereits die Physik in solchen
Koordinatenabhängigkeiten einen Mangel, da die Gesetze unter
Koordinatentransformationen invariant sein müssen, also letztlich
koordinatenfreie Zusammenhangsbeschreibungen mehr Einsicht in bestehende
Zusammenhänge vermitteln. Damit steht zugleich die Frage, invariante
geometrische Strukturen in solchen höherdimensionalen Phasenräumen zu
beschreiben.

Derartige Fragen sind Gegenstand zum Beispiel der algebraischen Geometrie oder
der Differentialgeometrie. In diesen Beschreibungen (der invarianten
geometrischen Gebilde) treten ihrerseits Räume auf, die sich etwa im Konzept
der \emph{Vektorbündel} „materialisieren“ als \emph{Sprache}, um geometrische
Eigenschaften der betrachteten invarianten Strukturen zu beschreiben (wie
Fasern, Keime, Schnitte, Obstruktionen zur Fortsetzbarkeit von Schnitten,
Homologieklassen als Strukturen derartiger Obstruktionen usw.).

Im Bereich der Analysis wird der Raumbegriff weiter verallgemeinert zu
unendlich-dimensio"|nalen Banach- und Sobolev-Räumen, in denen sich gewisse
mathematische Konzepte (etwa das Lebesgue-Integral) überhaupt erst entfalten
lassen für Situationen, wo man mit „klassischen“ Lösungen nicht mehr
weiterkommt.  Theorien (wie etwa der Banachsche Fixpunktsatz) lassen sich
überhaupt erst auf der Basis derart verallgemeinerter Raumbegriffe konsistent
entwickeln.

\section{Steady State und Fließgleichgewichte}

Diese Begriffe entwickeln sich später zum Begriff des \emph{Attraktors}
weiter.  Zugleich wird erkannt, dass derartige Attraktoren extrem komplexe
Gestalt haben können, womit eine Unterscheidung zu chaotischem Verhalten
allein auf phänomenologischer Ebene schwierig wird.  Zugleich wird die Rolle
auch \emph{negativer Attraktoren} erkannt.  Derartige Strukturen und
Strukturbildungsprozesse sind typisch für dissipative Prozesse fern von
Gleichgewichtszuständen, die durch einen gewissen Durchsatz von Materie und
Energie getrieben werden. Der Durchsatz von Information spielt dabei keine
Rolle\footnote{Siehe dazu etwa noch einmal
  \url{https://de.wikipedia.org/wiki/Dissipative_Struktur}.}. Ich komme unten
auf diese Frage zurück.

\section{Komplexe und komplizierte Systeme}

Diese Unterscheidung habe ich überhaupt nicht begriffen. Sicher kann man einen
solchen Unterschied nicht an der Zerlegbarkeit eines technischen Artefakts
(„ein Auto ist kompliziert, nicht aber komplex“) festmachen, da ein
entsprechender Technikbegriff noch deutlich hinter dem des VDI (siehe meine
1. Vorlesung) zurückbliebe, der zum System wenigstens noch „Herstellung“ und
„Verwendung“ des Artefakts (oder -- dort bereits deutlich -- „Sachsysteme“)
rechnet.

Eine solche Unterscheidung lässt sich nach meinem Verständnis ausschließlich
an den Beschreibungsmethodiken festmachen, die etwa im Potsdamer Manifest (VDW
2005) als „mechanisch-materialistisch“ und „geistig-lebendig“ unterschieden
werden.  Damit kommen wir aber sofort auf grundlegende Fragen, welche Technik-
und Wissenschaftsverständnisse überhaupt nur Grundlage für „Nachhaltigkeit“
sein können und welchen Anteil das Wert-Nutzen-Denken des homo oeconomicus
oder auch nur des homo faber an der aktuellen Krise unserer fossil basierten
Produktionsweise hat.

Carlowitz hat vor 250 Jahren wenigstens noch über eine nachhaltige
Bewirtschaftung der nachwachsenden Ressource „Holz“ raisonniert\footnote{Dass
  Carlowitz' Probleme eng mit der aufkommenden kapitalistischen
  Produktionsweise zusammenhängen und vergleichbare Probleme der
  Bewirtschaftung von Infrastrukturen vorher mit den lokalen Allmendegesetzen
  stabil prozessiert werden konnten, hat Elinor Ostrom klar gezeigt, siehe
  etwa (Stollorz 2011). }. Unsere gesamte Technik und Wissenschaft hat sich
seither rasant weiterentwickelt, allerdings auf der Basis \emph{fossiler}
Rohstoffe, die sich definitiv \emph{nicht} in so kurzen Zeiten regenerieren
wie sie verbraucht werden.  Die damit verbundenen grundlegenden Probleme habe
ich bereits in der 2. Vorlesung („Peak Oil? Peak Everything!“) angeschaut.
Siehe dazu auch (Davis 2008), (Gräbe 2012).

\section{Informationsbegriff}

„Komplexe Systeme sind lernfähig“ (Laforet). Lernfähigkeit setzt nach meinem
Verständnis 1) Reflexionsfähigkeit und 2) Selbstreflexionsfähigkeit voraus.
Ich denke nicht, dass der Begriff „komplexes System“ derart eingeengt werden
sollte.  Insgesamt sind wir bei diesem Ansatz bei Informationstheorien auf dem
Stand der 1970er Jahre, etwa (Steinbuch 1969)\footnote{„Geschichte ist die uns
  überlieferte Information über frühere Versuche, die Zukunft zu gestalten.“
  (ebenda, S. 5)}, die Klaus Fuchs-Kittowski (Fuchs-Kittowski 2002) in der
Unterscheidung zwischen Kybernetik 1. und 2. Ordnung noch einmal resümierte.
Dieser Ansatz wurde bereits Ende der 1990er Jahre in Debatten zwischen Janich,
Capurro, Fleissner, Hofkirchner u.a. fundamental kritisiert. Dazu etwa (Janich
2006), (Capurro u.a. 1996), (Capurro 1998),  (Capurro 2002), (Klemm 2003).

\section*{Literatur}
\raggedright
\begin{itemize}
\item Bertalanffy, Ludwig von (1950). An outline of General System Theory,
  The British Journal for the Philosophy of Science, Volume I, Issue 2, 1
  August 1950, 134–165.
\item Capurro, Rafael, Peter Fleissner, Wolfgang Hofkirchner (1996). Is a
  unified theory of information feasible?
  \url{http://www.capurro.de/trialog.htm}
\item Capurro, Rafael (1998). Das Capurrosche Trilemma.
  \url{http://www.capurro.de/janich.htm}.
\item Capurro, Rafael (2002). Menschengerechte Information oder
  informationsgerechter Mensch? \url{http://www.capurro.de/gotha.htm}.
\item Davis, Mike (2008). Wer wird die Arche bauen?  Das Gebot zur Utopie im
  Zeitalter der Katastrophen.  Telepolis, 11.12.2008.
\item Fuchs-Kittowski, Klaus (2002). Wissens-Ko-Produktion.  Verarbeitung,
  Verteilung und Entstehung von Informationen in kreativ-lernenden
  Organisationen.  Festschrift zum 65. Geburtstag von Klaus Fuchs-Kittowski.
\item Gräbe, Hans-Gert (2012). Wie geht Fortschritt? LIFIS ONLINE [12.11.12]. 
\item Janich, Peter (2006). Was ist Information? Frankfurt/Main.
\item Klemm, Helmut (2003). Ein großes Elend. Informatik Spektrum,
  S. 267--273. 
\item Steinbuch, Karl (1969). Die informierte Gesellschaft.  Stuttgart,
  2. Auflage. 
\item Stollorz, Volker (2011). Elinor Ostrom und die Wiederentdeckung der
  Allmende. Aus Politik und Zeitgeschichte 28--30. Bundeszentrale für
  Politische Bildung. 
\item VDW -- Verein Deutscher Wissenschaftler (2005). „We have to learn to
  think in a new way“. Potsdamer Denkschrift.
\end{itemize}

\end{document}
