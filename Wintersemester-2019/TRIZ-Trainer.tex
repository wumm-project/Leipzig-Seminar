\documentclass[11pt,a4paper]{article}
\usepackage{a4wide,url}
\usepackage[utf8]{inputenc}
\usepackage[german]{babel}

\parindent0pt
\parskip4pt
\title{Handreichung zum Einsatz des TRIZ-Trainers}

\author{Hans-Gert Gr\"abe}

\date{21. Dezember 2019}

\begin{document}
\maketitle

\section{Allgemeines}

Im Rahmen einer internationalen Kooperation nutzen wir in diesem Semester im
Rahmen des Praktikums unseres Kurses „Semantic Web“ wie angekündigt probeweise
den von \emph{Target Invention} in Minsk (Belarus) entwickelten TRIZ-Trainer
\url{https://triztrainer.ru}.  Der TRIZ-Trainer ist eine leichtgewichtige
Version zur Unterstützung der Online-Phase von Blended
Learning\footnote{\url{https://de.wikipedia.org/wiki/Integriertes_Lernen}} als
methodischem Praktikumskonzept.  Ab 7. Januar 2020 werden wir Probleme sowohl
technischer als auch inhaltlicher Art sowie die Lernfortschritte in
wöchentlichen Besprechungen dienstags ab 17 Uhr genauer analysieren. 

Der TRIZ-Trainer konzentriert sich auf die Basiskonzepte des Einsatzes von
TRIZ an ausgewählten praktischen Beispielen -- die Analyse der jeweiligen
Problemsituation, die Identifizierung entsprechender Wirkfaktoren und
Widersprüche und die strukturierte Verwendung entsprechender Lösungsschemata.
Weitergehende TRIZ-Werkzeuge (strukturierte Analysen von
Stoff-Feld-Interaktionen, Funktionsanalyse, Prozessanalyse,
Root-Conflict-Analysis usw.), die in (Koltze/Souchkov 2017)\footnote{Karl
  Koltze, Valeri Souchkov (2017). Systematische Innovation.  2. Auf"|lage,
  Hanser, München.} ebenfalls besprochen werden, können eingesetzt werden,
sind aber nicht Teil des im TRIZ-Trainer eingebauten strukturierten Vorgehens,
das Ihnen auf der Hauptseite\footnote{Im nicht angemeldeten Zustand wird die
  Hauptseite durch eine (noch nicht ins Deutsche übersetzte) Steuerseite
  überlagert, der erste der drei Verweise zeigt auf die Eingangsseite,
  alternativ verwenden Sie den Link
  \url{https://triztrainer.ru/?view=default}.} angezeigt wird und in der
Fachwelt auch als „Weihnachtsbaum“ (christmas tree) bekannt ist.

Der TRIZ-Trainer ist selbst noch in Entwicklung, insbesondere der Ausbau
verschiedener Sprachversionen.  Im Rahmen unserer Kooperation haben wir die
Minsker Kollegen bei der Erstellung einer deutschsprachigen Version
unterstützt.  Dazu wurden die von Google Translate gelieferten Ergebnisse
editorisch überarbeitet, um einen einheitlichen Gebrauch der Terminologie zu
gewährleisten. Diese Arbeit ist aktuell zu 50\% umgesetzt, die restlichen
deutschen Texte sind die Rohversionen von Google Translate.  Einige Teile des
TRIZ-Trainers (insbesondere im Bereich „Ergänzendes“) sind noch nicht in einer
deutschen Version verfügbar. Dies sowie die weitere Konsolidierung der
Übersetzungen erfolgt im Zuge des weiteren Einsatzes des TRIZ-Trainers über
das dort eingebaute Redaktionssystems. Fortschritte in diesem Bereich werden
also unmittelbar wirksam.

\section{Registrierung und Aktivierung des Accounts}

Für die sechs Studierenden des Kurses wurden Accounts angelegt und der Rolle
„Student“ zugewiesen. Username ist die im Moodle hinterlegte Email-Adresse.
An diese Adresse sollten Sie einen Aktivierungslink geschickt bekommen haben,
mit dem Sie ein eigenes Passwort einrichten und damit Ihre Authentifizierung
am System (ganz rechts im Login-Feld) ermöglichen.  

Die weiteren Ausführungen gehen davon aus, dass Ihnen dies gelungen ist.  Die
beiden Felder daneben (mit den Tooltips „Notifications“ und „Settings“) dienen
der Steuerung Ihrer Aktivitäten. Wie das genau funktioniert, muss ich noch
klären.

Ich bin Ihnen als Trainer zugewiesen und kann auch die 6 Accounts sehen und
somit Ihre Aktivitäten verfolgen. Wie dies genau geht, muss ich ebenfalls noch
klären. 

\section{Was ist zu tun?}

Anton hat mir dazu folgendes geschrieben:
\begin{quote}
  Die Studenten entscheiden, welche Aufgaben sie lösen möchten. Von den
  verfüg\-baren 48 Aufgaben müssen unseres Erachtens mindestens 50\% gelöst
  werden, um das Zertifikat zu erhalten.
  
  Alle Aufgabenlösungen und alle Nachrichten in den Aufgabenchats werden zu
  Ihnen als Trainer weitergeleitet -- Sie erhalten entsprechende
  Benachrichtigungen.
\end{quote}

\end{document}
