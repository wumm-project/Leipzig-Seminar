\documentclass[11pt,a4paper]{article}
\usepackage{ls}
\usepackage[english]{babel}

\setcounter{secnumdepth}{-2}

\title{Concept of the Lecture \\[1em] \emph{Modelling Sustainable Systems and
    Semantic Web} \\[1em] Summer Term 2021}

\author{Hans-Gert Gr\"abe}

\date{March 15, 2021}

\begin{document}
\maketitle

\subsection{General}

The lecture will take place synchronously online (Thursdays 11-13 am.) and is
based on the Flipped Classroom concept. The lecture consists of three parts.

In the first part we explore the concept of a \emph{technical system} and
introduce the main concepts of TRIZ as the most important systematic
innovation methodology.  In contrast to other creativity and innovation
methodologies, TRIZ focuses on the systematisation of engineering experiences.

In the second part, we study more closely aspects of the creation of
conceptual networks for data models on the basis of the \emph{Resource
  Description Frameworks} (RDF), the \emph{Linked Open Data Cloud}, the
emerging \emph{Giant Global Graph} and the importance of these developments
for the organisation of contexts of cooperative action.

Finally, in the third part, we explore the role of data and information and
the generation of new language for the development of technical systems in the
context of a civil society and, in particular, the importance of concept
formation processes in cooperative action.

In addition to the general bibliography, each lecture will be accompanied by
literature for preparation which should be \textbf{studied before the
  lecture}, in order to be able to follow the explanations. In the lecture the
topics are presented only cursorily, but there is room to ask questions about
the literature and to discuss individual aspects.

Most of the material is available in the public material folder
\texttt{Material} in the github project \emph{Leipzig-Seminar}
\begin{center}
  \url{https://github.com/wumm-project/Leipzig-Seminar}
\end{center}
or is otherwise easily found on the internet. Nevertheless we will not
dispense on classical printed literature and your ability to obtain it.

With the opening for an international audience, we want to switch step by step
to English as lingua franca. Since most of the materials used so far are in
German, the lecture will take place in a mixed form, with English presentation
and slides, but mostly referencing to German-language literature for
self-study.

The progress of the lecture will be reported regularly in the above mentioned
github Repo. There you will also find the schedule and the slides of the
individual lectures.

\subsection{Digital Privacy}

We follow not only a theoretical but also a practical Open Culture approach
and make course materials publicly available.  This also applies to the
(annotated) chat recordings of the lecture, in which your names are mentioned.
We assume your consent to this procedure, if you do not explicitly object.
The discussions themselves will \textbf{not} be recorded.

\subsection{General Bibliography}

\begin{itemize}[noitemsep]
\item Robert Adunka (2020). TRIZ Anwendungsbeispiele. \\
  \url{https://www.triz-consulting.de/ueber-triz/triz-anwendungsbeispiele-2/} 
\item Iouri Belski (2020). Tools of TRIZ. A web repository of TRIZ materials
  on 12 simple TRIZ heuristics.
  \url{https://emedia.rmit.edu.au/triz/content/tools-triz}
\item Karl Koltze, Valeri Souchkov (2017). Systematische Innovationsmethoden.
  Hanser Verlag, München. ISBN 9783446451278
\item Andrei Kuryan, Dmitri Kucharavy (2018). The OTSM-TRIZ Heritage of
  Nikolai N. Khomenko. A General Theory of Powerful Thinking. Folien eines
  Vortrags auf dem TDS 2018 in St. Petersburg. As \texttt{OTSM-Folien.pdf} in
  the folder of materials.
\item Nikolai Khomenko, John Cooke (2007). Inventive problem solving using the
  OTSM-TRIZ “TONGS” model.  As \texttt{tongs-en.pdf} in the folder of
  materials.
\item Alex Lyubomirskiy, Simon Litvin, Sergei Ikovenko et al. (2018). Trends
  of Engineering System Evolution (TESE).  TRIZ Consulting Group. ISBN
  9783000598463.
\item Dietmar Zobel (2007). Kreatives Arbeiten. Expert Verlag, Renningen.\\
  ISBN 9783816927136.
\item Dietmar Zobel (2020). TRIZ für alle. Expert Verlag, Renningen. ISBN
  9783816985105.
\end{itemize}

\end{document}
