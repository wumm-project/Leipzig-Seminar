\documentclass{beamer}
\usepackage{lsfolien}
\usepackage[english]{babel}
\usepackage[utf8]{inputenc}

\myfootline{System Modelling and Semantic Web -- Spring 2021}{Hans-Gert Gräbe}

\newcommand{\ueberschrift}[1]{\begin{center}\bf #1\end{center}}

\title{Modelling Sustainable Systems\\ and Semantic Web\\[6pt]
  About the Notion of a Technical System
  \vskip1em}

\subtitle{Lecture in the Module 10-202-2309\\ for Master Computer Science}

\author{Prof. Dr. Hans-Gert Gräbe\\
\url{http://www.informatik.uni-leipzig.de/~graebe}}

\date{April 2021}
\begin{document}

{\setbeamertemplate{footline}{}
\begin{frame}
  \titlepage
\end{frame}}

\section{Basics}
\begin{frame}{Technical Systems in TRIZ}

„... a \emph{number of components} combined to a system by establishing
  \emph{specific interactions} between the components ... assigned to
  \emph{perform a controllable main useful function} ... within a particular
  context.“ (Glossary V. Souchkov) \vskip1em

A system is a set of elements ... that form a \emph{unified whole} that has
\emph{properties} that only emerge from the \emph{interaction} of the parts
(\emph{emergence} of the main useful function). (V. Petrov, 2020) \vskip1em

Significance of the system operator in TRIZ, non-triviality of the anti-system
(N. Feygenson, 2020)
\end{frame}

\begin{frame}{Basics}
Despite its importance, the definition of the concept remains vague.\vskip1em
  
How can the concept of a \emph{Technical System} (TS) be sharpened?\vskip1em
  
\begin{itemize}
\item [1.] Which aspects should be considered?
\item [2.] Four dimensions of the concept of a TS.
\item [3.] Technical Systems as Black Boxes.
\end{itemize}
\end{frame}
\begin{frame}{Aspects}
  \begin{block}{Differentiation between design time and runtime}
    \begin{itemize}
    \item During design time, the basic collaborative work is \emph{planned}.
    \item During runtime this \emph{plan is executed}.
    \end{itemize}
  \end{block}
  \begin{block}{This requires to distiguish between interpersonal} 
    \begin{itemize}
    \item forms of description that are communicated as \emph{justified
      expectations}, and
    \item embodiments, which \emph{experienced results} stand in a
      contradictory relation to the justified expectations.
    \end{itemize}
  \end{block}
\end{frame}
\begin{frame}{Aspects}
  \begin{block}{Aspect of Reuse}
    \begin{itemize}
    \item This does not apply to most large TS -- they are \emph{unique
      specimen}, even if they are assembled from standard components.
    \item Most computer specialists also create such \emph{unique specimen},
      because the IT systems that control the operation of such large TS are
      also unique.
    \item The same applies to government agencies, organzations, etc.
    \end{itemize}
  \end{block}
\end{frame}
\begin{frame}{Aspects}
  \begin{block}{Clear distinction between the professions}
    \begin{itemize}
    \item of Mechanical Engineering and Industrial Plant Engineering as well
      as
    \item the Supplier (specialist) and Master Builder (generalist) of such
      unique specimen.
    \end{itemize}
  \end{block}\vskip2em
  \begin{alertblock}{Thesis 1:}    
    The speciality of a technical system mainly lies in the \emph{interaction
      of its components} in a world of technical systems.

    \emph{Purposes} embed this relationality in human practices. 
\end{alertblock}
\end{frame}
\begin{frame}{First Approximation}
  \begin{block}{The four dimensions of the notion of a \emph{Technial System}}.
    \begin{itemize}
    \item [1.] The real-world unique specimen.
    \item [2.] The description of this real-world unique specimen.
    \end{itemize}
    For components that are manufactured in a larger number, additionally
    \begin{itemize}
    \item [3.] The description of the design of the system template.
    \item [4.] The description and functioning of delivery, assembly and
      operation of the real-world unique specimen that were produced according
      to this template (e.g. production plan, quality assurance plan, delivery
      plan, plans for operation, maintenance and service).
    \end{itemize}
  \end{block}
\end{frame}
\begin{frame}{First Approximation}
  \ueberschrift{TS as Black Box}

The basis of the concept is the \emph{notion of an Open System} from the more
general Theory of Dynamical Systems.\vskip1em

  Existing TS are normatively characterized 
  \begin{itemize}
  \item at the level of description through the \emph{specification of its
    interfaces} and
  \item on the level of embodiment through \emph{guaranteed functioning
    according to this specification}.
  \end{itemize}\vskip1em

A TS consists of components, which in turn are TS, their
specification-compliant functioning is assumed.\vskip1em

\end{frame}
\begin{frame}{Role of the concept of a TS}
  
The concept of a TS has an epistemic function of (functional) "reduction to
the essential".\vskip1em

Human practice is inseparably built into the concept of TS because the
concepts "essential", "guaranteed" and "functioning" can only be filled with
meaning from these practices.\vskip1em

This makes the widespread in TRIZ distinction between technical and
socio-technical systems problematic.
\end{frame}
  
\section{The Concept of a TS}
\begin{frame}{TS as White Box}
  \begin{itemize}
  \item [1.] Definition of the concept of a TS.
  \item [2.]  TS and the world of technical systems.
  \end{itemize}\vskip2em
  \begin{block}{The core of a technical system is ... }
    ... the description of concrete processes by reducing them to the
    essential with the goal of their practical application.
  \end{block}
\end{frame}
\begin{frame}{TS as White Box}
  \begin{block}{The reduction to the essential ... }
    ... focuses on the following three dimensions:
    \begin{itemize}
    \item[(1)] Delimitation of the TS from the outside against an
      \emph{environment}, reduction of this relationships to input/output
      relations and guaranteed throughput (Purpose and ability to work).
    \item[(2)] Delimitation of the TS from the inside by grouping parts as
      \emph{components}, reducing their functioning on a "behavior control"
      via their interfaces.
    \item[(3)] Reduction of the relationships in the TS itself to
      \emph{causally essential} ones.
    \end{itemize}
  \end{block}
\end{frame}
\begin{frame}{TS as White Box}
  \begin{block}{The TS in the World of Technical Systems}
    The description of a TS is only possible based on descriptions of other
    (explicitly or implicitly given) TS. The description is anteceded ...
    \begin{itemize}
    \item[(1)] ... by a vague idea of the (working) input/output
      characteristics of the environment.
    \item [(2)] ... by a clear understanding how the components work beyond
      their pure specification.
    \item [(3)] ... by a vague idea of cause and effect relationships in the
      system itself, that precedes the detailed modeling.
    \end{itemize}
  \end{block}
\end{frame}
\begin{frame}{TS as White Box}
  The concept is based on the availability of existing TS, which are present
  in (2) as components and in (3) as neighboring systems. \vskip1em

  Engineering practices thus take place in a \emph{World of Technical
    Systems}.\vskip1em

  Other systems -- components or neighboring systems -- are present in the
  description of a TS only by their specifications. \vskip1em

  A prerequisite for the smooth operation of a TS is therefore the guaranteed
  specification-compliant functioning of the corresponding infrastructure.
\end{frame}
\section{Components}
\begin{frame}{Components}
  \begin{itemize}
  \item [1.] The concept of a component according to Szyperski.
  \item [2.] Core concern, cross cutting concern.
  \item [3.] Components as functional connections.
  \item [4.] Components as function-object relationships between independent
    third parties.
  \item [5.] Components and infrastructure.
  \item [6.] Norms and Standards.
  \end{itemize}
\end{frame}

\begin{frame}{The World of Components}
  \begin{block}{The concept of a component according to Szyperski}
    What is a component?

    Szyperski gives a simple answer: "Components are for composition".
  \end{block}\vskip2em

  TS are assembled from existing components. Components can be purchased from
  third parties or developed in-house.
\end{frame}

\begin{frame}{The World of Components}

  \begin{block}{core concern, cross cutting concerns}
    Szyperski divides the world of component manufacturing (i.e. TS) in two
    partial worlds -- "design for component" and "design from component".
  \end{block}\vskip1em

  The first world is the world of component developers, that develop special
  component functions for business applications -- "core concern", this
  corresponds to the MUF -- as \emph{core system function}.\vskip1em
\end{frame}

\begin{frame}{The World of Components}

In addition to this core function, the operation of the component requires
\emph{supporting functions} (logging, data security, access control, printer
control, etc. -- "cross cutting concerns") that are based on the use of
\emph{established concepts} (description dimension) and services from other,
already \emph{prefabricated components} (application dimension), that
implement \emph{other technical principles} in other systems.\vskip1em

\begin{alertblock}{Thesis 2:} 
  In this sense, real-world components are always \emph{bundles of
    functionality} that bundle procedural knowledge from \emph{several} areas.
\end{alertblock}
\end{frame}

\begin{frame}{The World of Components}

The \emph{component developer} must master all such description forms of
functions of supporting components, at least on the abstraction level of their
specifications to build useful components.\vskip1em

The second world is the world of \emph{component assemblers}. They assemble
(following a pre-existing plan) the system from existing components, develop
or modify additional support functions ("glue code"), integrate and test the
complete system before releasing it to the customer.
\end{frame}

\section{Standardization}
\begin{frame}{Modularization and standardization}

This approach of division of labor between component developers and component
assemblers in the field of software engineering is also extensively used in
other engineering areas.\vskip1em

\emph{Modular systems} are widely used and allow the standardization of the
design of the unique real-world technical systems.\vskip1em
\end{frame}

\begin{frame}{Components and Frameworks}

This requires to connect the \emph{application logic} of the component as
"core concern" with the \emph{logic of infrastructural networking} as "cross
cutting concerns".\vskip1em

\begin{alertblock}{Thesis 3:} 
  The infrastructure logic is usually part of the \emph{component framework}
  that can only be used effectively if it is \emph{jointly owned by the actors
    of an entire area of technology}.
\end{alertblock}
\end{frame}

\begin{frame}{Standardization and Trends of TS Evolution}

Application logic and infrastructure logic are orthogonal to each other, which
means that the trends \emph{4.2 of increasing completeness of a system} and
\emph{4.4 of migration to the supersystem} practically counteract to each
other in the development of a TS. \vskip1em

\begin{alertblock}{Thesis 4:} 
  An improvement in the understanding of the \emph{infrastructure
    requirements} of interacting components (transition to the supersystem) as
  description form leads to a \emph{reduction of the level of requirements on
    the completeness} of individual components.
\end{alertblock}
\end{frame}

\begin{frame}{Standardization and Economies of Scale}

Standardization opens up the prospect of economies of scale for standard
components. Economies of scale lead to lower costs per unit and thus shift the
leading role of competition from competition for the \emph{better technical
  solution} to the competition for its \emph{cheaper economic manufacturing}.
\vskip1em

This means that the S-curve switches at the top of mature technical solutions
(including standardization) in the phase of general availability \emph{to
  another mode} in which the reduction of the economic cost of the
availability "state of the art" takes over as guiding function of further
development.
\end{frame}

\begin{frame}{Standardization and Economies of Scale}

\begin{alertblock}{Theseis 5:} 
  The technical "trend 4.1 of increasing (technical) value" changes on the
  third stage of the development on the S-curve to an economic "trend of
  decreasing (economic) value".
\end{alertblock}\vskip1em

Or, in economic terms: a demand-driven market turns into a supply-driven
market. The same (mature) use value has ever lower exchange value.
\end{frame}

\section{Conclusions}
\begin{frame}{Conclusions}
  \begin{alertblock}{Thesis 6:} 
    In the TRIZ theory of TS evolution a better distinction between young and
    mature technologies is required.
  \end{alertblock}\vskip1em
  \begin{block}{In mature technologies ... } 
    \begin{itemize}
    \item TS are \emph{bundles of technical principles},
    \item which in the descriptive form have the goal of \emph{unity in
      diversity} (think globally),
    \item from which in the practice form the \emph{diversity in real-world
      local application contexts} (act locally) has to be restored.
  \end{itemize}
  \end{block}
\end{frame}
\begin{frame}{Conclusions}
  \begin{alertblock}{Thesis 7:} 
    The directed graph of \emph{realized purposes} is the core of
    relationality in the world of technical systems.\vskip1em

    This graph is a global socio-technical artifact and is evolving in the
    contradictionality of description forms and execution forms.
  \end{alertblock}
\end{frame}
\end{document}
