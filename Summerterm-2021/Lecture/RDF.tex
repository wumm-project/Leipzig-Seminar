\documentclass{beamer}
\usepackage{lsfolien}
\usepackage[english]{babel}
\usepackage[utf8]{inputenc}

\myfootline{System Modelling and Semantic Web -- Spring 2021}{Hans-Gert Gräbe}

\def\lt{\texttt{<}}
\def\gt{\texttt{>}}
\def\deg{$^\circ$C}

\newcommand{\ueberschrift}[1]{\begin{center}\bf #1\end{center}}

\title{Modelling Sustainable Systems\\ and Semantic Web\\[6pt]
  RDF Basics
  \vskip1em}

\subtitle{Lecture in the Module 10-202-2309\\ for Master Computer Science}

\author{Prof. Dr. Hans-Gert Gräbe\\
\url{http://www.informatik.uni-leipzig.de/~graebe}}

\date{June 2021}
\begin{document}

{\setbeamertemplate{footline}{}
\begin{frame}
  \titlepage
\end{frame}}

\section{RDF Basics}
\begin{frame}{RDF Basics. Descriptions and Interpretations}

Information as interpreted data?\vspace{-1em}
\begin{itemize}
\item Measured values as data?
\end{itemize}
Language is full of implicit assumptions. An example:\vspace{-1em}
\begin{itemize}
\item On November 8th at the station Leipzig Airport at 5 p.m. was measured a
  temperature of 16{\deg}.
\item On {\lt}a type="Datum"{\gt}November 8th{\lt}/a{\gt} at the
  station\\ {\lt}a type="LocationInformation"{\gt}Leipzig
  Airport{\lt}/a{\gt}\\ at {\lt}a type="Time"{\gt}5 p.m.{\lt}/a{\gt} was
  measured\\ a {\lt}a
  type="PhysicalParameter"{\gt}temperature{\lt}/a{\gt}\\ of {\lt}a
  type="Temperature"{\gt}16 {\deg}{\lt}/a{\gt}.
\item Things and their names.
\end{itemize}
\end{frame}
\begin{frame}{RDF Basics. Example}

  {\small\tt \begin{tabbing}\hskip2em\=\kill
    @prefix s21: {\lt}http://od.fmi.uni-leipzig.de/s21/{\gt} .\\
    @prefix od: {\lt}http://od.fmi.uni-leipzig.de/model/{\gt}.\\
    @prefix rdfs: {\lt}http://www.w3.org/2000/01/rdf-schema\#{\gt}.\\
    @prefix odr: {\lt}http://od.fmi.uni-leipzig.de/rooms/{\gt}.\\
    @prefix odp: {\lt}http://od.fmi.uni-leipzig.de/personal/{\gt}.\\[4pt]

    s21:BIS.SemanticWeb.1\+\\
    rdf:type	od:English , od:LV , od:Vorlesung ;\\
    rdfs:label	"Modelling Substainable Systems ... " ;\\
    od:beginsAt	"11:15" ;\\
    od:dayOfWeek	"donnerstags" ;\\
    od:endsAt	"12:45" ;\\
    od:locatedAt	odr:online ;\\
    od:servedBy	odp:Graebe\_HansGert ;\\
    od:hasCType	"synchron" .
  \end{tabbing}}
  
  \begin{itemize}
  \item Identifiers and literals. Namespaces.
  \end{itemize}
\end{frame}
\begin{frame}{RDF Basics. Sentences}

Resolution in three-word sentences
\begin{center}\tt
  Subject Predicate Object .
\end{center}

  {\small\tt \begin{tabbing}
    s21:BIS.SemanticWeb.1 rdf:type od:Vorlesung .\\
    s21:BIS.SemanticWeb.1 rdfs:label	"Modelling ... " .\\
    s21:BIS.SemanticWeb.1 od:beginsAt	"11:15" .\\
    s21:BIS.SemanticWeb.1 od:dayOfWeek	"donnerstags" .\\
    s21:BIS.SemanticWeb.1 od:endsAt	"12:45" .\\
    s21:BIS.SemanticWeb.1 od:locatedAt	odr:online .\\
    s21:BIS.SemanticWeb.1 od:servedBy	odp:Graebe\_HansGert .\\
    s21:BIS.SemanticWeb.1 od:hasCType	"synchron" .
  \end{tabbing}}

  SPARQL-Schnittstelle für weitere Anfragen
  \url{http://od.fmi.uni-leipzig.de:8892/sparql}
\end{frame}
\begin{frame}{RDF Basics and the Internet of Things}
  \ueberschrift{Industry 4.0 and the Internet of Things}
  \begin{itemize}
  \item Fiction: There are no things on the Internet, only
    \emph{representations} of things, just like representations of people
    (digital identities).
  \item Descriptions as relational complexes between representations of real
    things or even just complexes of meaning.
  \item These things and complexes of meaning must also be assigned “Digital
    Identities” as textual representations to be able to formulate sentences
    about them in the Digital Universe.
  \end{itemize}
\end{frame}
\begin{frame}{RDF Basics. Conceptual Ingredients}
  \begin{itemize}  
  \item UTF-8 as a \textbf{uniform character base} for URIs and literals.
    \begin{itemize}  
    \item[] Best practice: URIs only made up of ASCII characters, no diacritics,
      special characters or similar.
    \end{itemize}
  \item URI as "digital identities" of resources, \emph{point} to resources. 
    \begin{itemize}  
    \item[] Like people's digital identities, these are \textbf{textual
      representations of "things"} in the text fragments circulating in the
      internet.
    \end{itemize}
  \item For computers, URIs are just strings, for people it is helpful if the
    URI already provides a suggestion about its semantics. 
    \begin{itemize}  
    \item[] Best practice: "Speaking names" as URIs
    \end{itemize}
  \end{itemize}
\end{frame}
\begin{frame}{RDF Basics. Best Practices}
  \begin{itemize}
  \item RDF -- Resource Description Framework
  \item Concept for writing down stories about "the world" as \textbf{sets} of
    three-word sentences
    \begin{center}
      {\lt}subject{\gt} {\lt}predicate{\gt} {\lt}object{\gt}.
    \end{center}
  \item Subject and predicate must be URIs. The object can be a URI or a
    literal (type \texttt{rdfs:Literal}). Literals can carry type and language
    markings.
  \item There are different notations for the same set of RDF sentences
    (Turtle, rdf/xml, json, ntriples) and tools to convert these notations.
    \begin{itemize}
    \item[] Redland RDF libraries \url{http://librdf.org/}
    \end{itemize}
  \item Pattern search as a powerful concept for analyzing such sets of
    sentences. SPARQL as query language.
  \end{itemize}
\end{frame}
\begin{frame}{RDF Basics. Example of a SPARQL Query}
Example of a request to the SPARQL endpoint
\url{http://od.fmi.uni-leipzig.de:8892/sparql}

Returns information about all teaching events (\texttt{od:LV}) with URI prefix
BIS
{\small\tt \begin{tabbing}\hskip1em\=\kill
  PREFIX od: {\lt}http://od.fmi.uni-leipzig.de/model/{\gt}\\
  SELECT distinct ?l ?name? ?d ?b\\ 
  from {\lt}http://od.fmi.uni-leipzig.de/s21/{\gt}\\
  WHERE \{\+\\
    ?l a od:LV .\\
    ?l rdfs:label ?name .\\
    ?l od:beginsAt ?b .\\
    ?l od:dayOfWeek ?d .\\
    filter regex(?l, 'BIS') .\-\\
  \}
\end{tabbing}}
\end{frame}
\begin{frame}{RDF Basics. Different Notations}
  \begin{itemize}
  \item \textbf{Turtle notation} -- collects together all sentences about the
    same subject. Such a set of predicate-object pairs can be interpreted as a
    set of key-value pairs that describes this subject.
    \begin{itemize}
    \item But: Here a key can have \emph{several} values!
    \item It is a particularly popular human readable notation.
    \item It is a subject-centered point of view, which well serves the
      specific point of view of "MY World" -- as discussed earlier.
    \item Computers prefer to work with sets of triples.
    \end{itemize}
  \item If the subjects and objects are interpreted as nodes and predicates as
    edges then a set of RDF sentences describes an \textbf{RDF graph} (and
    vice versa).
    \begin{itemize}
    \item[] A picture is often a better explanation than thousand words.
    \end{itemize}
  \end{itemize}
\end{frame}
\begin{frame}{RDF -- Sentences and Pattern}
  
Sentences are arranged following patterns:
\begin{itemize}
\item[1)] \textbf{Turtle:} Collect all sentences with the same subject.
  
  Interpretation of properties of an individual subject as key-value pairs.
  \begin{itemize}
  \item Key and value = attribute and attribute value
  \end{itemize}
\item[2)] Collect \textbf{all sentences with the same predicate}
  \begin{center}\tt    A od:beginsAt B  \end{center}
  \begin{itemize}
  \item \texttt{od:beginsAt} is not only a URI (\emph{syntax}), but also a
    predicate with two parameters (A and B) and a certain \emph{semantics}
    that is present in all sentences with this predicate as its
    \emph{instantiations}.
  \end{itemize}
\item[3)] Other patterns are possible, SPARQL as the general standard
  query language for pattern search in RDF sets of sentences.
\end{itemize}
See the file \textbf{Queries.txt} in the github Repo (with comments in
German).
\end{frame}
\begin{frame}{RDF -- Descriptions of Descriptions}
\begin{itemize}
\item \textbf{Self-similarity of the concept:} Also descriptions of
  descriptions can be formulated as RDF phrases. In particular, one can use
  RDF to describe RDF.
  \begin{itemize}
  \item A URI that appears as a predicate in a sentence can appear as subject
    or object in other sentences.
  
    Example:
    \begin{quote}\tt      
      od:beginsAt rdfs:domain od:LV .\\
      od:beginsAt rdfs:range rdfs:Literal .
    \end{quote}
  \end{itemize}
\item This means that also \emph{terms and concepts} can be described using
  RDF. $\to$ \textbf{Universals}
  \begin{itemize}
  \item What are universals? Ideas from Plato's \emph{heaven of ideas} or
    \emph{institutionalized conventions}, i.e. "fictions" in the earlier
    introduced meaning? 
  \end{itemize}
\end{itemize}
\end{frame}
\begin{frame}{RDF -- Basic Limitations}
\begin{itemize}
\item Set semantics, the order of the sentences does not matter.
  \begin{itemize}
  \item This is different in other approaches, such as the XML-based TEI (Text
    Encoding Initiative) which plays a central role in Digital Humanities.
  \end{itemize}
\item Problem of contextualization. In which spatio-temporal context the
  sentence has to be interpreted? There are several approaches here:
  \begin{itemize}
  \item Extend triples to quadruples with a fourth component containing the
    URI to the provenance (description).
  \item If the sentence contains an instantiation of a predicate, the context
    often can be inferred from the set of instantiations of that predicate.
  \item Often the context results more generally from the \emph{namespace} of
    the predicate and thus stands as an (explicit or implicit) \emph{model for
      a whole class of terms}.  But this shifts the problem only to the
    description of the model and thus an abstraction level upwards.
  \end{itemize}
\end{itemize}
\end{frame}
\begin{frame}{RDF -- Summary of the Central Concepts}
\begin{itemize}
\item \emph{Central idea:} Save textual descriptions in a uniform way as
  triples and use standard concepts and tools for the management of this data.
\item \emph{Resources:} URI, HTTP access
  \begin{itemize}
  \item URI = Unique Resource Identifier
  \item This can be used to access a worldwide distributed database in a
    uniform manner via a common protocol.
  \end{itemize}
\item \emph{Resource Descriptions:} Return on a HTTP request a useful piece of
  information in RDF format that can be combined with others such information
  units.
\item Operate an \emph{RDF Triple Stores} and \emph{SPARQL endpoints} as part
  of a worldwide distributed Data storage infrastructure,
  e.g. \url{http://od.fmi.uni-leipzig.de/} (note that only the SPARQL endpoint
  is publicly accessible).
\item SPARQL as language for (distributed) queries.
\end{itemize}
\end{frame}
\begin{frame}{RDF -- Language Forms and Practices}
  \ueberschrift{Processual Knowledge $\to$ Established Procedures}
  \vspace{-2em}\small
\begin{itemize}
\item Correspondence between the coherence of the language form and the
  coherence of practices.
\item The establishment of coherent practices as established procedures allows
  to substantiate predicates.

  “At the beginning of the school hour the break time sandwiches are being
  packed away.”

  Such normative sentences become possible only after such a transposition of
  the predicate to the subject position.
  \begin{itemize}
  \item[] Parallels to the concert example in the first lecture.
  \end{itemize}
\item Basic wisdom of computer science:
  \begin{itemize}
  \item[1.] A function can only be called after it has been defined.
  \item[2.] A function that is defined but not called indicates a design flaw.
  \end{itemize}
\item \emph{Processual knowledge} is a part of the description form,
  \emph{established procedures} are part of the execution form.
\end{itemize}\vfill
\end{frame}
\end{document}
