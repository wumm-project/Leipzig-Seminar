\documentclass{beamer}
\usepackage{lsfolien}
\usepackage[english]{babel}
\usepackage[utf8]{inputenc}

\myfootline{System Modelling and Semantic Web -- Spring 2021}{Hans-Gert Gräbe}

\newcommand{\ueberschrift}[1]{\begin{center}\bf #1\end{center}}

\title{Modelling Sustainable Systems\\ and Semantic Web\\[6pt] Information and
  Language \vskip1em}

\subtitle{Lecture in the Module 10-202-2309\\ for Master Computer Science}

\author{Prof. Dr. Hans-Gert Gräbe\\
\url{http://www.informatik.uni-leipzig.de/~graebe}}

\date{July 2021}
\begin{document}

{\setbeamertemplate{footline}{}
\begin{frame}
  \titlepage
\end{frame}}

\section{Technology and Culture}
\begin{frame}{Storytelling and Action}

  \ueberschrift{Previous Discussion}
  
Our actions are closely related to the stories which we are permanently
telling in parallel to each other.
\begin{itemize}
\item With these stories we transcend \emph{our own world of experience} that
  is only a small part of THE WORLD, that we can grasp only selectively.
\item Storytelling is the form in which we make other people's worlds of
  experience accessible for us.
\end{itemize}
\end{frame}
\begin{frame}{Storytelling and Action}

In the first plane locally emerges an image like in a theater: the stage (own
experiences) and backdrop (structured processing of the storytelling) as a
unit.
\begin{itemize}
\item Private action in this sense is acting on the stage, in front of this
  backdrop.
\end{itemize}

Cooperative Action:
\begin{itemize}
\item Is only possible on a \emph{common} stage, in front of a \emph{common}
  backdrop. Both are first to be created (communicatively, in language form).
\item Hence storytelling precedes cooperative action.
\item In particular this requires an intersubjective image of man.
\end{itemize}
\end{frame}
\begin{frame}{Storytelling and Action}
On the other hand, speaking is itself acting.
\begin{itemize}
\item Speech acts can contain prompt phrases, protocol sentences, judgement
  records, etc.
\item Storytelling is only possible in a cooperative context which is formed
  in cooperate action. Hence action also precedes storytelling.
\item Hermeneutic Circle.
\end{itemize}
The (cooperative) changes of the world are preceded by speaking about these
changes (the imagination of the change).
\begin{itemize}
\item Thinking and Doing: Justified expectations $\to$ World-changing action
  $\to$ Experienced results
\item In the tension between justified expectations and experienced results
  the ceasefire lines of the WORLD become visible.
\end{itemize}
\end{frame}
\begin{frame}{Storytelling and Action}
But how change a world that is also constantly changing by itself?
\begin{itemize}
\item Culture: Change the changes of the world (nature).
\item Technology (tool perspective) and storytelling (perspective of
  expectations and experiences) are two essential moments of culture.
\item Technology comes in here as \emph{processual knowledge}. 
\item Separation of goals and means as a specific type of storytelling.
\item The widespread separation of these two moments causes essential problems
  to understand the wholeness of the reality.
\end{itemize}
\end{frame}

\begin{frame}{Storytelling and Digital Change}
\ueberschrift{Previous Findings}
\begin{itemize}
\item Within the digital change new forms of storytelling are developing,
  which break up the previously institutionalized procedures of storytelling.
  \begin{itemize}
  \item How does progress work?
  \item Losses in advancing (Ernst Bloch)
  \end{itemize}
\item Web 1.0 -- Linked websites as a new form of storytelling.
\item Semantic Web -- RDF as a new basic technology to operate a certain kind
  of storytelling with computer support.
\item Digitization of important language artifacts.
  \begin{quote}\small
    Our time like no other offers a vast collection of knowledge in text form.
    The entire intellectual history of mankind is available on CD-ROMs, on
    websites, in second-hand bookshops and in bookstores, everything is well
    connected and easily accessible.  It would be a shame not to use this
    material awake and open minded. (Matthias Käther, 2004)
  \end{quote}
\end{itemize}\vspace*{2em}
\end{frame}

\begin{frame}{Storytelling and Digital Change}
What social conditions are required in order to develop this potential?
\begin{itemize}
\item \emph{Free} (as in free speech) access to the knowledge resources of the
  mankind, to communicate prospects of expectation and experience in an
  appropriate way.
\item Acting in a civil society as responsible \emph{private} action, in which
  the consequences of action are privately assigned is a cultural achievement.
  \begin{itemize}
  \item Ability to close contracts, liability, ownership, and
    institutionalised checks and balances in their historical evolution.
  \end{itemize}
\item The digital change requires a new balance between these two
  perspectives.
  \begin{itemize}
  \item In the legal context of a civil society that means above all
    readjustment of the legal constitution and framework.
  \end{itemize}
\end{itemize}
\end{frame}

\section{Information and Language}

\begin{frame}{Information and Language. Linguistics}

  \ueberschrift{What is Language?}
  
It is obviously about processes mediated by language (computer language). How
does language work?  What does linguistics say about this?
\vskip1em

\begin{quote}\small
  Language, a system of conventional spoken, manual (signed), or written
  symbols by means of which human beings, as members of a social group and
  participants in its culture, express themselves. The functions of language
  include communication, the expression of identity, play, imaginative
  expression, and emotional release.\\
  (\url{https://www.britannica.com/topic/language})
\end{quote}
\end{frame}

\begin{frame}{Information and Language. Linguistics}\small
  
\url{http://de.wikipedia.org/wiki/Sprachsystem}

The idea of how the \textbf{language system} is built depends on which
language or grammar theory as base. The different theories support mostly the
following assumptions about the components of the language system:
\begin{itemize}
\item There are \textbf{linguistic units that are organized hierarchically}
  and reach from the smallest units, the sounds, to the phonemes, morphemes,
  words, parts of sentences, sentences up to texts and possibly to discourses.
\item In this hierarchy, from the morphemes on the units have
  \textbf{additional to their form} a grammatical or lexical \textbf{meaning}.
\item At each level of the hierarchy there are \textbf{rules} that determine
  which positions and combinations of units are allowed and which are not.
  This applies to both the linguistic forms and their meanings.
\end{itemize}
\end{frame}

\begin{frame}{Information and Language. Linguistics}\small
  
\url{http://www.christianlehmann.eu/ling/lg_system/index.html}

\ueberschrift{Formative and Significative Subsystems}

The language system relates thoughts to sounds. This association is indirect
in several ways: A language system cannot associate thoughts ... and also not
sounds ... but only linguistic units with each other. These are on the one
hand \textbf{Significata} (the thought as content of the sign) and on the
other hand \textbf{Significantia} (the sound as expression of the sign).

Hence the language system contains \textbf{two formative subsystems}:
\begin{itemize}
\item In \textbf{Semantics}, the thought is formed into a significatum.
\item In \textbf{Phonology}, the sound is formed into a significant.
\end{itemize}
\end{frame}
\begin{frame}{Information and Language. Linguistics}\small
  
  (cont.)
  
In addition to these formative subsystems, there is the \textbf{Significant
  Subsystem}, which combines Significantia and Significata and thus creates
\textbf{Language Signs}. ... It is divided into two subsystems:
\begin{itemize}
\item Finalised language signs are stored in the \textbf{lexicon}.
\item New language signs are formed in the \textbf{grammar}.
\end{itemize}
\end{frame}

\begin{frame}{Information -- a new Phlogiston?}
\ueberschrift{What is Information?}
\begin{itemize}  
\item Inflationary use of the term information.
  \begin{itemize}
  \item Günter Ropohl remembers the times when there was a counter "Auskunft"
    at a German railway station. (Source: Klemm 2003)
  \end{itemize}
\item The computer scientists stick to an ontologizing (and ultimately a
  tangible) concept of information.
\item The linguists talk about language practices.
\end{itemize}
\end{frame}
\begin{frame}{Information -- a new Phlogiston?}

Another critical debate occured in the late 1990s
\begin{itemize}
\item Capurro's Trilemma
\item Trialog (Capurro, Fleissner, Hofkirchner): Is a unified theory of
  information feasible?
\item Heinz Klemm (2003): "A great misery"\\ (German: "Ein großes Elend")
\item Peter Janich: The concept of information has necessarily to refer to
  successful human communication.\vskip1em
  \begin{quote}\small
    However, for successful prompting practices it is fundamental that through
    them a successful connection is established for the involved people
    between the (language) act of prompting and (non-language) act of obeying.
    (Janich 1998)
    \end{quote}
\end{itemize}
\end{frame}

\begin{frame}{Information -- a new Phlogiston?}

Raphael Capurro:\vfill
\begin{quote}\small
  What I am criticizing is the idea to have by the reductionistic concept of
  information \textbf{a kind of phlogiston}: To mean that one comes through
  the different levels -- Aristotle called this logical error \emph{metabis
    eis allo genos} -- and thus to believe e.g. better to explain how life
  arises from matter. So we are not far from the use of the concept of form --
  \emph{informatio} originally goes back to \emph{forma} and \emph{eidos} --
  in relation to matter, life, soul, etc. We would be faced with a new or old
  form of metaphysics.
\end{quote}

The problem is once again: Where is the human being as an \emph{acting}
subject?

Klaus Fuchs-Kittowski stated already in the 1980s: \vfill
\begin{quote}\small
  The concept of \emph{unity of self-organization and generation of
    information} -- the \emph{information processing approach} neglects the
  \emph{formation of meaning} in the process of real life.
\end{quote}\vspace*{1em}
\end{frame}

\section{Summary}
\begin{frame}{Summary}\small

Starting point: RDF -- what is happening there and in general in the internet?
\begin{itemize}
\item It is a digital form of storytelling.
\item Storytelling accompanies our \emph{cooperative actions}. Cooperative
  actions are possible only in such an interpersonal language based context.
  
  Question (1): \emph{What} is here conveyed by language?
\item But: Storytelling is not reduced to its \emph{communicative} function.
  It has also a \emph{reflective} component.

  Question (2): How does \emph{theory building} work on such an empirical
  background?
\item There is an arc of tension: Justified expectations $\to$ World changing
  actions $\to$ Experienced results
  \begin{itemize}
  \item Interpersonally this arc of tension is to be explored only
    in language form and \emph{only in specific contexts}.
  \end{itemize}
\end{itemize}
\end{frame}

\begin{frame}{Summary}
Why act cooperatively? "Change the world".
\begin{itemize}
\item But how to change a world that is also constantly changing itself?  How
  to deal with the diversity and contradictions of the requests for change?
\item Approach "influence the changes of the world".
  \begin{itemize}
  \item "Doing" is embedded here, prior to it is the reality of life.
    Justified expectations can only be derived from this reality.
\item Experience: \emph{Practical} influence is (today) only possible through
  application of adequate processual knowledge and procedural skills.
  \end{itemize}
\item But why cooperate?
  \begin{itemize}
  \item Cooperative action is more powerful than individual action because
    synergistic effects are emerging.
  \item The whole is more than the sum of its parts.
  \end{itemize}
\end{itemize}
\end{frame}

\begin{frame}{Summary}\small

Cooperative action is only possible in a \emph{common context of meaning}.
\begin{itemize}
\item Experience: \emph{Understanding} language presupposes a common context
  of meaning on the one hand, and continues to rewrite it on the other.
\item This context of meaning expresses itself above all in the \emph{social
  use of common terminology in common activities}.

  Question (3): How can that itself be expressed in language form?
\end{itemize}

Experience: Such contexts of meaning are stabilised through
\emph{institutionalisation}. Meanings are tied to social practices as a
specific interaction between logos and telos.
\begin{itemize}
\item \emph{Practically proven actions} are socially secured as
  \emph{institutionalised practical procedures} and thus as technology.
\item Question (4): How to set up the notion of \emph{knowledge} in this
  context?
\end{itemize}\vspace*{2em}
\end{frame}
\begin{frame}{Summary}\small
Observation: Such institutionalized contexts of meaning are nested and
interlaced in many ways.
\begin{itemize}
\item Experience: Cooperation between cooperative structures requires
  translation between contexts of meaning.

  This is yet hardly understood in the field of semantic technologies.
\item People are involved in cooperative contexts with partial identities only
  $\to$ concept of roles.
\end{itemize}
The core of all four questions: How does such an institutionalisation of
contexts of meaning work?
\begin{itemize}
\item We also had identified this question as a core problem of semantic
  technologies.
\item Historically in the last 150 years there have been various attempts to
  this problem.
\end{itemize}
\end{frame}
\begin{frame}{Summary}
Attempts to develop a general language theory as Universal Theory.
\begin{itemize}
\item Logical positivism of the Vienna Circle (1920s).
\item Syntax, semantics, pragmatics (Charles W. Morris, 1940)
\item Continuation as semiotics and linguistics in the 1970s.
\item Noam Chomsky and his approach to a universal grammar.
\end{itemize}
At the same time, since 1920, increasing importance of evolutionary
approaches: Institutionalisations of contexts of meaning are hierarchically
complex and can be unterstood only in their historical-cultural development.
\begin{itemize}
\item Biosemantics: focus on coevolution of neural patterns and evolutionary
  patterns of contexts of meaning.
\end{itemize}
\end{frame}
\begin{frame}{Summary}
Pragmatics: Terms develop with their interactive use.  (Jacob L. Mey:
Pragmatics, 1993)
\begin{itemize}
\item The development of concepts cannot be detached from their practical use,
  in particular forms and practices of judgment and judgment.
\end{itemize}
\end{frame}
\end{document}
