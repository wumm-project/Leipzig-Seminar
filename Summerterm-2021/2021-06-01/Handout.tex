\documentclass[11pt,a4paper]{article}
\usepackage[utf8]{inputenc}
\usepackage[margin=1in]{geometry}
%\usepackage{amsmath,amsfonts,amssymb,mathtools,graphicx,float,url,enumitem}
\usepackage{url,enumitem}
\usepackage[english]{babel}

\setlist{noitemsep}
\parindent0pt
\parskip3pt

% Title content
\title{The S.M.A.R.T Approach by George T. Doran}
\author{Axel Schuster}
\date{31 May 2021}

\begin{document}
\maketitle
% Abstract
\section{Introduction}
The S.M.A.R.T. Approach \cite{doran1981there, eremit2016smart} is an attempt
to standardize the writing of effective objectives, to establish a controlled
workflow and therefore to ease management in the sense of management by
objectives. However one of the main problems of this approach is, that there
is no standardized definition of it. The acronym of smart objectives might
have occurred the first time in an Publication of George T. Doran
\cite{doran1981there} who claimed that „most managers still don't know what
objectives are and how they can be written" and that "characteristic of
management excellence is a climate in which company officers and managers talk
in terms of objectives" and therefore makes arguments for a standardized and
effective way to write objectives. He argues that the S.M.A.R.T. Approach can
fulfill this task and also describes requirements of an organization in form
of salary structure and job fit to fully realize the potential of this
method. Doran introduced S.M.A.R.T. as the acronym meaning Specific,
Measurable, Assignable, Realistic, Time-Related but one of the more common
ones is used in the form of Specific, Measurable, Achievable, Relevant,
Time-Bound.

Doran's publication being almost 40 years old, this method is widely known and
there are many alterations and derivatives of it in use. This work aims to
give an overview of the initial intention of the approach by Doran and its
most common usage.

\section{Goals vs. Objectives}

Objectives are seen as an important part of Organization communication by
Doran \cite{doran1981there}. There exist many different definitions of
objectives and goals in which they change its roles eventually. Since for
clear communication it is important to clarify vague or variably used. He
argues that the definition is not important as long it is used in the same way
in one organization, at least on executive level. For this reason two of the
more common usages of these terms which also align with the usage of the terms
which Doran used in his publication are being presented here.

\subsection*{Goals}
"Goals are the specific result or purpose expected from the project. The
project goals specify what will be accomplished over the entire project period
and should directly relate to the problem statement and vision.The goal is
achieved through the project objectives and activities." \cite{samhsa}

\subsection*{Objectives}
"Objectives are the specific steps that lead to the successful completion of
the project goals. Completion of objectives result in specific, measurable
outcomes that directly contribute to the achievement of the project goals."
\cite{samhsa}

\section{The S.M.A.R.T. Approach to write effective Objectives}
As mentioned before there exist many acronyms based on the term S.M.A.R.T.
floating around in publications and the internet. The exact definition can
vary and different versions may suit different tasks or organizations. For
this reason the most common acronym is being presented here in which
S.M.A.R.T.  stands for Specific, Measurable, Achievable, Relevant, Time-Bound
\cite{mindtools}. The chosen criteria are supposed to be applied on objectives
as tightly as possible but as abstract as needed according to Doran
\cite{doran1981there}.

\subsection*{Specific}
The objective should be clear and specific so it's tangible and therefore
easier to motivate the assigned people to achieve it. The following questions
can be useful to determine if the objective is specific enough:
\begin{itemize}
\item What do I want to be accomplished?
\item Why is this objective important?
\item Who is involved?
\item Where is it located?
\item Which resources or limits are involved?
\end{itemize}

\subsection*{Measurable}
It's important to have measurable objectives, meaning to be able to quantify
them or at least suggest an indicator of progress. This helps to stay focused
on the intended objective and meet the deadline. It also helps in evaluating
the objective and track the progress made on it. A measurable objective should
address questions such as:
\begin{itemize}
\item How much?
\item How many?
\item How will I know when it is accomplished?
\end{itemize}

\subsection*{Achievable}
An objective also needs to be realistic and attainable to be successful. In
other words, it should stretch the abilities but still remain possible. When
setting an achievable objective, it may be possible to identify previously
overlooked opportunities or resources. An achievable objective will usually
answer questions such as:
\begin{itemize}
\item How can the objective be accomplished? 
\item How realistic is the objective, based on constraints, such as financial
  factors and other resources?
\end{itemize}


\subsection*{Relevant}
This step is about ensuring that the objective matters, and that it also
aligns with other relevant objectives. Its supposed to brings progress for the
associated goal this objective is part of. A relevant objective can answer
"yes" to these questions:
\begin{itemize}
\item Does this seem worthwhile?  Is this the right time?
\item Does this match our other efforts/needs?
\item Are the assigned people the right ones to reach this objective?
\item Is it applicable in the current socio-economic environment?
\end{itemize}

\subsection*{Time-Bound}
An objective needs time frame. In this version it is requested in a form of a
target date, so that there is a deadline to focus on when the objective has to
be finished. This part of the SMART objective criteria helps to prevent
everyday tasks from taking priority over your longer-term goals on the one
hand and to make the objective more measurable on the other. A time-bound
objective will usually answer these questions:
\begin{itemize}
\item When?
\item What can I do six months from now?
\item What can I do six weeks from now?
\item What can I do today?
\end{itemize}

\section{The Right Conditions}
According to Doran \cite{doran1981there} it is very important to set the right
conditions in an organization to be able to establish an environment for
effective use of smart objectives to gain maximum productivity. In his opinion
the most important part is to find the right job fit for employees so they can
bring in their personal strengths where it is most effective. Since job
requirements can change quickly over time, job fit evaluation is needed on a
regular basis. He argues that therefore "radical change is needed in the
position evaluation and salary structure" so that it is possible to move up or
down in positions independent of the salary and reputation. Doran also claims
that it is the job of a excellent manger to move people to their job fit and
if not possible remove them. He argues that "if a manager has persons in the
wrong job, require him to face up to the reality of it, or be penalized. The
immorality lies in failing to tackle the problem, not in beeing soft about
it."

\begin{quote}
  important: as tight as possible and as abstract as needed\\
  critics today: performance indicator
\end{quote}

\section{Critics}
Searching for it there is much critic to be found about the
S.M.A.R.T. Approach. One of the more common ones is, that dividing goals into
smart objectives which fulfill all the criteria is too narrow and doesn't let
enough room for flexibility and lacks therefore agility
\cite{whysmartdoesnotwork, 6reasonswhy, fastgoals}. Another critic is, that
dividing every problem or task into objectives could lead to lose the focus on
the main goal \cite{6reasonswhy}.

Doran states that objectives may be clarified as abstract as needed and may be
divided into subobjectives to actually be worked on. So the argument of beeing
narrow gets inadequate in the approach described by him.

The problem is, which can also be viewed as the biggest point of criticism,
that there is no standard definition of that approach. It's not even certain
when the term has been used the first time. Some cite Doran's article
\cite{doran1981there} as the first occurrence \cite{wikipedia,
  managersorg}. The way he uses the term in his explanation of smart
objectives suggests that the term might has been used and therefore introduced
before though. Sometimes also Peter Drucker is named as the creator
\cite{managersorg}.

%\bibliographystyle{acm}
\raggedright
\begin{thebibliography}{xxx}

\bibitem{6reasonswhy} Lei Wang (2017). 6 Reasons Why "Smart" Goal Setting
  Does Not Work.  Foundry, Management \& Technology, Sept 21, 2017.
  \url{https://www.foundrymag.com/issues-and-ideas/article/21931645/6-reasons-why-smart-goalsetting-does-not-work}.
  \newblock [online resource].

\bibitem{fastgoals} Franziska Schneider (2018). Fast Goals -- why FAST is
  better than SMART for your goals. Workpath.com, Sep 19, 2018.
\newblock \url{https://www.workpath.com/magazine/fast-goals}.
\newblock [online resource].

\bibitem{samhsa} SAMSHA Native Connections (No Date).  {Setting Goals and
  Developing Specific, Measurable, Achievable, Relevant, and Time-bound
  Objectives}.  \newblock
  \url{https://www.samhsa.gov/sites/default/files/nc-smart-goals-fact-sheet.pdf}.
  \newblock [online resource].

\bibitem{managersorg} CMI -- Chartered Management Institute (2014).  Setting
  SMART Objectives. Checklist 231.  \newblock
  \url{https://www.managers.org.uk/wp-content/uploads/2020/03/CHK-231-Setting_Smart_Objectives.pdf}.
  \newblock [online resource].

\bibitem{wikipedia} Wikipedia: SMART criteria.  \newblock
  \url{https://en.wikipedia.org/wiki/SMART_criteria}.  \newblock [online
    resource].

\bibitem{mindtools} Mind Tools Content Team (2021). SMART Goals.  How to Make
  Your Goals Achievable.  \newblock
  \url{https://www.mindtools.com/pages/article/smart-goals.htm}.  \newblock
      [online resource].

\bibitem{whysmartdoesnotwork} Paul Marsh (2019). Why SMART objective setting
  is not working.  Personnel Today, January 18, 2019.  \newblock
  \url{https://www.personneltoday.com/hr/why-smart-objective-setting-does-not-work/}.
  \newblock [online resource].

\bibitem{doran1981there} George-T. Doran (1981) \newblock There’s a S.M.A.R.T.
  way to write management’s goals and objectives.  \newblock \emph{Management
    review 70}, 11, 35--36.

\bibitem{eremit2016smart} Britta Eremit, Kai F. Weber (2016).  \newblock
  S.M.A.R.T.-Methode – Specific Measurable Accepted Realistic Timely.
  \newblock In \emph{Individuelle Pers{\"o}nlichkeitsentwicklung: Growing by
    Transformation}.  Springer, pp.~93--99.

\end{thebibliography}

\end{document}
