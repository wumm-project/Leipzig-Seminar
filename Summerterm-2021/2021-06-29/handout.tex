\documentclass[a4paper,11pt]{article}
\usepackage[utf8]{inputenc}
\usepackage{a4wide,url}
\usepackage[english]{babel}

\parindent0pt
\parskip4pt

%\newcommand{\enquote}[1]{{\glqq#1\grqq{}}}
\newcommand{\enquote}[1]{``#1''}

\title{The Principles of Scientific Management}
\author{Stefan Grote}
\date{June 29, 2021}

\begin{document}

\maketitle

\section{Abstract}
In the late 19th century, Frederick Winslow Taylor was a foreman at a steel
plant in Pennsylvania, US. While observing his workmen, he identified multiple
reasons why workers would, sometimes even deliberately, stay behind Taylors
expectations of how fast a set of tasks could be completed. Believing that
major improvements in productivity were possible, Taylor started to develop a
new type of management, which aims to perfect and optimize how every task is
done. He formalized this new \enquote{Scientific Management}, sometimes also
referred to as \enquote{Taylorism}, in his book \textit{The Principles of
  Scientific Management}, which was published in 1911. Even though Scientific
Management received a lot of criticism over the years, the positive effects on
productivity and effectiveness were undeniable, resulting in aspects of the
management still being visible in today's production routines.

\section{Historical context}
In 1880, at age 25, Frederick Winslow Taylor became foreman at the Midvale
Steel plant in Pennsylvania, US. He was impressed by how much his workmen
would stay behind his expectations of how much should be possible in a single
day \cite{wikipedia}. While investigating, he identified several reasons for
that:
\begin{itemize}
\item Each worker was responsible for a complete production routine, and there
  was no specification of how to do it exactly.
\item Employers were looking for \enquote{the perfect man} for the job,
  instead of telling workers how do to it.
\item Responsibility of how the work was done lay with the worker.
\item Workers would choose the easiest way or the way of least resistance.
\end{itemize}
Taylor was convinced that he could minimize losses due to these reasons by
applying scientific methods to manufacturing.

\section{The Principles of Scientific Management}
\subsection{Fundamentals}
\textit{The Fundamentals of Scientific Management} is the first chapter in
\cite{book} after the introduction. Here, Taylor prepares the basis for his
principles by showing how both the task of the management and why workers work
less than they could.

According to Taylor, the main task of management is to make maximum prosperity
possible for both the employer and the employee. This can be achieved by a
simple chain of events: By increasing the productivity of workers, the company
makes more profit. More profit then results in higher wages. This is only
possible, if both sides do their absolute best. Therefore, the views of
employers and employees are not necessarily contradictory.

On the other hand, Taylor identified three reasons, why employees would
deliberately work less than they could (\enquote{to soldier}):
\begin{enumerate}
\item Belief, that an increase in output of a single worker or machine would
  result in the dismissal of then obsolete other workers.
\item Workers are paid for work they done. Not showing, how fast it can
  actually be done, results in higher wages for less work.
\item \enquote{Rule-of-thumb} methods still had more importance than
  scientific approaches to solving a problem.
\end{enumerate}

\subsection{Principles}
\textit{The Principles of Scientific Management} is the next and last chapter
of \cite{book}. Before presenting his principles, Taylor gives an overview
about a different and widely used type of management: \enquote{Initiative and
  Incentive}. The initiative of a worker describes \enquote{his best
  endeavors, his hardest work, all his traditional knowledge, his skill, his
  ingenuity, and his good-will} \cite{book}, and it's the task of the
management to make workers use their whole initiative. The management
accomplishes this by giving incentive, for examples promotions, raising wages
and similar aspects.

In Taylors new Scientific Management, incentive is only given in an indirect
way: Only by sticking to his principles, an increase in productivity and
therefore wages can be achieved. Taylors four principles of scientific
management are:
\begin{enumerate}
\item They develop a science for each element of a man's work, which replaces
  the old rule-of-thumb method.
\item They scientifically select and then train, teach, and develop the
  workman, whereas in the past he chose his own work and trained himself as
  best he could.
\item They heartily cooperate with the men so as to insure all of the work
  being done in accordance with the principles of the science which has been
  developed.
\item There is an almost equal division of the work and the responsibility
  between the management and the workmen. The management take over all work
  for which they are better fitted than the workmen, while in the past almost
  all of the work and the greater part of the responsibility were thrown upon
  the men.
\end{enumerate}

According to Taylor, the advantage of Scientific Management over Initiative
and Incentive lies in combining the initiative of the worker and the new tasks
of the management. This results in an even higher productivity and
effectiveness.

\section{Scientific Management}
When put into practice, Scientific Management actually yielded good
results. After finding the \enquote{one-best-way} to do something (principle
1), workers would be trained to do that exact method over and over again
(principle 2). Workers would also be supervised and checked using different
methods (principle 3). For example, stopwatches were used to measure a workers
performance, until they were forbidden in 1916 due to the Hoxie-report
\cite{hoxie}. By selecting the methods, the management automatically took on
more responsibility, because if those methods were not good, it's now the
managements fault and no longer the workers (principle 4).

But even though productivity increased, Scientific Management received
criticism for multiple core aspects. After strikes in weapon factories, a
special committee, which commissioned the Hoxie-report \cite{hoxie}, was
formed to examine Scientific Management and its methods \cite{wikipediade}.
Criticism regarding Scientific Management includes, but is not limited to, the
results of the Hoxie-report and the following bullet points:
\begin{itemize}
\item Measuring time and fatigue is too inaccurate, and destroys the solidariy
  between workers.
\item Work is now split up in physical and mental work, which results in few
  highly qualified and many under qualified workers.
\item Physical work is split up in to many small parts, resulting in
  monotonous repetition.
\item Scientific management itself results in outsourcing and lower wages,
  because now even unskilled workers can complete more complex tasks by simply
  following instructions.
\end{itemize}
In spite of this, the ideas of Scientific Management spread internationally
and influenced others to implement or extend them.

\section{Scientific Management today}
While Scientific Management in it's original form is not used today anymore
(because of the aforementioned criticism), some aspects are still visible in
today's industry. An example for this is the production line for new cars: the
same steps have to performed on every car, and while some can be completed by
machines, workers are still needed. Delays caused by mistakes or not following
the best procedure slow down the whole process and can potentially result in
losses.

Even though it was initially developed for the secondary sector of the
economy, sometimes it can be found in the tertiary sector as well: Fast food
production and templates for documents or other objects are just two examples
for how in the beginning, the best way to do something was found, and then
reused over and over again by the employees.

\bibliographystyle{unsrt}\raggedright
\begin{thebibliography}{xxx}
\bibitem{book} Frederick Winslow Taylor. The Principles of Scientific
  Management, 1911.\\
  \url{https://archive.org/details/principlesofscie00taylrich/}
\bibitem{hoxie} Robert Franklin Hoxie. Scientific management and
  labor. Appleton, New York 1915.\\
  \url{https://archive.org/details/scientificmanage00hoxi/}
\bibitem{wikipedia} Wikipedia. Scientific Management\\
  \url{https://wikipedia.org/wiki/Scientific_management}
\bibitem{wikipediade} Wikipedia. Taylorismus\\
  \url{https://de.wikipedia.org/wiki/Taylorismus}
\end{thebibliography}
%\bibliography{quellen}
\end{document}
