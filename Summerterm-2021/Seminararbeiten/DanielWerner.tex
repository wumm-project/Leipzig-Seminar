\documentclass[a4paper,12pt]{scrartcl}

\usepackage{hyperref}
\usepackage{url}
\usepackage{a4wide}

\usepackage[scaled=1]{helvet}
\usepackage{mathpazo}
\usepackage{enumitem}
\selectfont
\linespread{1.25}

\usepackage[english]{babel}
\usepackage[utf8]{inputenc}
\usepackage[T1]{fontenc}
\usepackage{graphicx}
\usepackage[font=small, labelfont=bf]{caption}
\usepackage[list=true, font=small, labelfont=bf, 
labelformat=brace, position=top]{subcaption}

\usepackage{amstext}

\usepackage{courier}
\usepackage{caption}

\title{{\LARGE Selected aspects of Russell Ackoff's approach of Interactive Planning from the perspective of the WUMM system theory approach}}

\author{\\\vspace{-0.4em}{\normalsize Daniel Werner} }
\date{{\normalsize Leipzig, September 27, 2021}}


\begin{document}
\maketitle
\bigskip

\vfill

\noindent 
Leipzig University \hfill Faculty for Mathematics and Computer Science \\
Summer semester 2021 \hfill Seminar Complex Systems and Co-Operative Action \\
\thispagestyle{empty}
\pagenumbering{gobble}
\clearpage
\pagenumbering{roman}

%%%----------------------------------------------------------
\tableofcontents
%%%----------------------------------------------------------
\newpage

\pagenumbering{arabic}
\section{Introduction}

Interactive Planning is a methodology derived by Russel L. Ackoff, in his  book "\textit{Creating the Corporate Future: Plan or Be Planned for}"\cite{ackoff:1981}. It is build upon the basic concept that the future of a company depends on what actions and events it realizes in the present, aiming towards an ideal  future. Interactive Planning, and trough its execution their planner, wants to design a desirable present which it then tries to approximate as good as possible. The approach clearly sees the interdependence of problems and incorporates this into its planning.
Ackoff based this idea on the ideal of his so called "Interactivist", who wants to actively shape the future of the organization. The "Interactivist" accepts that the future cannot be predicted  and planned for in an all-encompassing way.\cite{ackoff:2001}

The methodology will be briefly introduced in the course of the paper, but Interactive Planning itself will not be the subject of this paper. Its aim rather is to compare Ackoff's ideas to the system theory approaches discussed in the WUMM project.\cite{wumm:2021} But before doing that, the fundamentals and definitions have to be laid out. Russell Ackoff's approach of Interactive Planning is not only a perfect example for the popular management styles of the 1970s and 1980s, but it also reveals a lot about contemporary ways of system theory and its development up until that point in time. One has to understand the development of the engineer-manager-relationship, the huge influences of cybernetics starting in the late 1940s and also the resulting system theories of this era.
Only then we can pick up selected aspects of the Interactive Planning methodology and compare, evaluate and order it inside the vast field of not only management but system theories.

The goal here is not to evaluate Interactive Planning from the perspective of other management theories, albeit their importance in understanding the significance of it, but rather from a systemic point of view with certain aspects of management in mind. 

\clearpage



\section{History of engineering and management}

\subsection{Motivation}
To understand for whom Ackoff's methodology is intended for, one has to look at the historical development of management and its roots in the engineering discipline. 
This profession has undergone major changes in the past decades. The first step, therefore, is to define the term "manager". In its original form, it refers to the "salaried entrepreneurs" and managers who became increasingly common with the Industrial Revolution in the 19th century. In contrast to this are the "owner-entrepreneurs". Furthermore, in a systemic context, a distinction should actually be made between "management" and "leadership". In the relevant literature, management refers in particular to the inward-looking management of a company and the tactical decision-making dimensions.\cite{kocka:1999} 

Management is also described as "thing management", since it primarily deals with objects and processes, not with people. Leadership, on the other hand, deals with the strategic decisions of the entrepreneur. It is not action in general that is expected, but "leadership action" - motivating, directing, promoting and criticizing people are all part of a leader's job here.\cite{niermann:2017} 

Sometimes it is even claimed that "leaders" can only be recruited from a certain group of people, on the basis of an almost transcendent, unobservable ability.\cite{boehmer:2014}
However, we do not want to make such a precise differentiation for this chapter and instead make do with the comprehensive term "management".

\subsection{Co-development of the engineer and the manager}
Let us first look at the Industrial Revolution. In Germany at that time, there were only a few "managerial enterprises", i.e. companies with a manager who did not own a large share of the capital of the enterprise. Exceptions were the coal and steel industry, major banks and railway companies. The professional group of engineers also underwent a change. Following the example of the French "Ecole Polytechnique", "Technische Hochschulen" were established in Germany, as well as technical and trade schools for an intermediate level of education.\cite{kaiser:2006} This strong academization and differentiation did not quite take place in the USA and England. Basically, practical, empirical engineering training was still preferred here instead of purely school-based training. In this context, the approaches were more practice-oriented in contrast to the theoretical and scientific foundations taught in Germany and France. A distinction between engineers and technicians was also made here primarily through prestige, reputation and rank, such as memberships in engineering associations or leading positions, not through profession.

However, in contrast to the English cradle of industrialization and the accompanying availability of trained engineers, the growing US economy manifested a shortage of qualified skilled workers. This led to the introduction of single-purpose machine tools, standardization and the formation of large-scale enterprises. Also, an engineering school system increasingly emerged in the US along continental European lines - from fewer than ten educational institutions before the Civil War to 180 schools and universities for engineers in 1945.\cite{gispen:2006}
This growth trend was not limited to the US. In Germany, too, companies grew, making them more and more complex. Personal and direct management methods as well as nepotism were often no longer sufficient; instead, individual functional departments were differentiated and managed by specialists. In mechanical engineering factories, for example, design departments were created specifically for engineers, separate from workshops and the management of the company. Nevertheless, the increasing complexity ensured that management difficulties in particular hampered the growth of companies. Such companies, which were run by former craftsmen or technicians, now had a high demand for commercial and managerial skills. There was a lack of systematically trained management personnel.\cite{kocka:1999}

Particularly in bureaucratized American industry, many lower to middle management positions emerged quite early from the ranks of the engineering profession, beginning in 1880. Also, more and more higher industrial management positions, as well as engineering consultants, emerged from them.\cite{gispen:2006}
With the beginning of the turn of the century, at the latest after World War I, the trend toward large-scale company formation was also in full swing.The number of people working in small businesses (up to ten people) halved between 1882 and 1925 in Germany. The dawn of the age of organization is often proclaimed, and the German social democratic theorist Rudolf Hilferding speaks of "organized capitalism".  Large-scale organization developed a high demand for managers, which further emphasized the need for systematic management methods.\cite{gispen:2006}

In response to this and as a pioneer for further management theories, the American engineer Frederick Taylor published his "\textit{Principles of Scientific Management}" in 1911, paving the way for the "enterprise as a machine based on the division of labor."\cite{oelsnitz:2009} Engineers and entrepreneurs took educational trips to the U.S., and renowned engineering associations such as the VDI began to include management topics in their publications.\cite{kocka:1999} In Germany, professional expertise, personified by a graduated chemist, physicist, or engineer, was considered the basis of the claim to authority. In the United States, professional managers were increasingly trained in the sense of "business administration" studies. This type of management education ("Betriebswirtschaftslehre"), on the other hand, did not find its way into German universities until the 1920s.\cite{kocka:1999}

\subsection{Development of management theories}

\subsubsection{Classical management movement}
Like the profession of the manager, the first theory behind it, the classical management movement, has its origins in the late course of the Industrial Revolution, with particularly high activity between 1885 and 1940. The reasons for this are primarily the larger production scales and complexities, which led to the need for more efficient planning, organization and control of workers.\cite{pindur:1995} In his famous book "\textit{An Inquiry into the Nature and Causes of the Wealth of Nations}", economist Adam Smith wrote in 1776 that efficiency gains could be achieved through the division of labour and specialization.\cite{smith:2008} The emergence of a managerial caste is ultimately an expression of this principle and rational and scientific approaches made their way into the organization.

The classical management movement has two bases. On the one hand, there is "scientific management", shaped by minds such as Frederick Taylor, who aimed to improve productivity. On the other hand, there was "general administrative management" or bureaucratic management, with representatives such as the German sociologist Max Weber, who understood and analysed the company as an organised bureaucracy and wanted to improve it as a whole.\cite{oelsnitz:2009}

\subsubsection{Behavioral management movement}
In the 1920s and 1930s, the behavioral management movement followed as an extension to, and in some cases reduction to, the human and inter-human aspects of organization. The already discussed concept of "leadership" came to the foreground along with topics such as motivation and psychology.\cite{pindur:1995} The Australian sociologist Elton Mayo, with his "Hawthorne Experiments", is one of the founders of the sub-movement of "Human Relations" theory, which, especially in the 1930s and 1940s, examined work in the company not only in the context of monetary incentives, but also the effects of social phenomena. According to this theory, human attention and a "social home" at the workplace contribute significantly to productivity.\cite{oelsnitz:2009}

\subsubsection{Quantitative management theory}
Another movement is Quantitative Management Theory, which took a very algorithmic and technical approach. Through the use of statistics, models and later also computer-aided simulations, resources were to be allocated efficiently and inventories and schedules optimally designed, for example. In general, decisions and plans were to be algorithmised as much as possible. This approach was pursued especially during and after the World War II and is still influential today, not only in management and system theories, but also forms the basis for professions such as systems engineer.\cite{hossain:2019}

\subsubsection{Modern management theory and systems approach}
After the Second World War, the integration of different methods and an interdisciplinary approach to management started to increase. Modern management theory began to adopt aspects of cybernetics and systems theory, resulting in the systems approach. The theories mentioned will be looked at in more detail later, but for now we will focus on management developments.
Organizations, as Frederick Taylor saw them, can be equated with closed machines - a closed system that does not interact with its environment and has intrinsic, routine functions. In contrast, open systems often have non-routine tasks, the organization as a whole is involved in problem solving, interaction takes place vertically and horizontally. Open systems operate under unstable conditions and require a constant flow of inputs and outputs. 
The Swiss economist Hans Ulrich had a similar idea with his approach to socio-technical systems. According to him, artificially created systems must have similar characteristics to natural systems if they are to remain stable. Technical considerations on cybernetics are transferred by him into a socio-dynamic concept in that the organization has a life of its own. Managers, it is said, can only act meaningfully in such a constantly changing system if they are aware of the limitations of their effectiveness.\cite{oelsnitz:2009}

These considerations have generated much praise, further additions, but also criticism. Another guideline for action was elaborated by Fredmund Malik, a student of Ulrichs. In his main work "\textit{Strategie des Managements komplexer Systeme}", he describes the necessity of the range of action ("Varietät") of the company, which must be at least as large as the possible environmental influences on it, which is based on the work of W. Ross Ashby in his book "\textit{An Introduction to Cybernetics}".\cite{jackson:1991} He goes on to say that over-complexity is harmful and that the operational vertical range of manufacture and thus complexity can be reduced through possibilities such as outsourcing.\cite{malik:1984}
An analogous consideration of the dynamic development of systems can be found, for example, in Holling's "Adaptive Cycle". Here, in addition to a phase of exploitation, there is also a phase of reorganization, a necessity for the cycle to be closed and the system, and thus also the organization, to continue to exist.\cite{holling:2001}

Criticism of the systems approach, on the other hand, was voiced mainly because of the necessary high level of abstraction, which therefore resembles more of a formal framework than a concrete recommendation for practitioners.
Furthermore, the suitability of the reference to nature through cybernetics was partly questioned. Unlike living beings exposed to Darwinian processes, which can hardly change their environment and have many hundreds of years to adapt, companies can certainly significantly influence their environment and often have little time to adapt to change.\cite{oelsnitz:2009}

\subsubsection{Strategic management theory}
Lastly, there are the developments in strategic management of the 1950s and 1960s. Here, similar to quantitative management, the decision-making process and the actions of the organization are analyzed, but with a specific focus on monitoring, evaluation of external and internal environmental influences and a resulting strategy definition. Elements of such a strategy are e.g. mission statements and comparative advantages. Often discussed and defined terms like "goals" and "objectives" as well as "critical success factors" and "corporate identity and culture" are increasingly used in this context.\cite{pindur:1995}

According to Dess, four basic components form the basis of Strategic Management:\cite{dess:1993}
\begin{enumerate}
\item environmental scanning,
\item strategy formulation,
\item strategy implementation,
\item evaluation and control.
\end{enumerate}

A prominent contributor to this movement is Peter Drucker with his work "\textit{Practice of Management}", where strategy serves as a means to analyze and possibly change the current situation. Also to be mentioned is Igor Ansoff, who after Drucker in 1965 took a more rational approach to the subject with his work "\textit{Corporate Strategy}", introducing new vocabulary and rules as well as concepts such as the "Gap Analysis".\cite{ansoff:1965}
Finally, in addition to Russell Ackoff, who will be discussed later, the Canadian academic Henry Mintzberg should be mentioned here with his work "\textit{The Structuring of Organizations}". For him, strategy is a mediating force between an organization and its environment and he defined consistent patterns in the strategic decision-making process.\cite{pindur:1995}

\clearpage

\section{Systems Theory}

A system refers to a complex whole, as a collection of individual components, whose function is based on the interaction of these very parts. Or more mathematically expressed, it is a complex of elements $ p_{1},...,p_{n} $, which are connected or interact by means of different relations $ R_{i} $.\cite{bertalanffy:1950}

Systems theory is thus the interdisciplinary study of such systems.
Traditionally, systems, as in the work of Taylor, but also, for example, in biology, have been analyzed only in terms of their individual parts. This reductionism wants to reach an understanding of the whole system by the isolated understanding of each single part. 
The alternative to this approach is holism. Here it is assumed that the system has emergent functions, which become apparent not by studying its individual parts, but only by considering it as a whole. According freely to Aristotle, the whole is more than the sum of its parts, produced by synergistic effects.

This idea of holism was not new at the beginning of the 20th century, but it became a popular subject of research and publication by pioneers like Ludwig von Bertalanffy with his "General System Theory" and by the field of cybernetics with people like Norbert Wiener, John von Neumann and Heinz von Foerster until the end of the last quarter of the century.
Especially physics, biology, control engineering and the social sciences dealt with systems theory in a more or less overlapping way.\cite{jackson:2003}

\subsection{Cybernetics and General System Theory}	
The beginnings of the scientific field of cybernetics date back to around the middle of the 20th century, more precisely to the developments during the Macy Conferences between 1946 and 1953. Scientists from a wide range of disciplines met here to work out, validate and refine what had already been compiled in interdisciplinary research in the years before, especially during the Second World War. The product of these conferences and the Macy Group was the book "\textit{Cybernetics: Or Control and Communication in the Animal and the Machine}" published in 1948 by Norbert Wiener. This brought together three main components, a "set of models" - the logical calculus of neural activity, Claude Shannon's information theory and behavioural theory with the central theme of feedback.\cite{pias:2004}

Wiener sees in this work the necessity of unifying communication, control and statistical mechanics, both of living and artificial systems. It aims to create a common terminology for interdisciplinary problem solving. This also gave rise to the term "cybernetics", after the Greek word for steersman.
It goes on to say that an organization in the anthropological and sociological sense, based on information and communication, should transcend the individual in favour of the community. Fundamentally, the dynamic and circular processes of feedback loops are an essential component of social systems, according to Wiener.
In 1961, the book was expanded by two further chapters dealing with self-replication and self-organization of systems. John von Neumann's work on the cellular automaton and on game theory receive particular attention here.\cite{wiener:1948}

In summary, the study object of cybernetics are complex systems, i.e. multi-component systems with non-linear interactions between elements. The general mechanism for controlling such a system is the (negative) feedback loop - the current output of the process that is to be controlled is registered, compared with a desired goal and, if necessary, the input is adjusted so that the goal is achieved. In the positive feedback loop, on the other hand, the deviation from the goal is amplified.
This also shows the strong technical influence of cybernetics, which is reflected in the developments in control engineering ("Regelungstechnik").\cite{jackson:2003}

Another important publication of this time is "\textit{General System Theory: Foundations, Development, Applications}" by the Austrian biologist Ludwig von Bertalanffy in 1968, even though his reflections on general system theory date back to his work in 1937.
His thoughts stem from biology. According to him, organisms should be studied as a whole, as an "open system" with input and output; the prevailing reductionism should be abandoned if the system is to be understood as a holistic whole.\cite{jackson:1991}

He is convinced that these considerations can also be applied to other disciplines, since they consider systems as abstract entities, not in the context of biology - "we also find formally identical or isomorphic laws in completely different fields", he writes in one of his works on general system theory.\cite{bertalanffy:1950}

\subsubsection{Development of Cybernetics in the 1970s}
In the context of cybernetics, there was a move away from the part/whole scheme of classical systems theory towards a system/environment view with different levels of complexity.\cite{kneer:2009}

The vagueness of the research field, already noticeable at the beginning of cybernetics, expanded to the extent that interest seemed to drop off in the mid-1970s. Research funds were lacking and the subject was up for debate. This moment in cybernetics is also reflected in Heinz von Foester's 1974 work "\textit{Cybernetics of Cybernetics}". In this work a kind of conclusion of the "classical" cybernetics, with its roots in technology, is sought and at the same time an attempt is made to open up something new by introducing a turn towards the social and societal. The theory of second-order cybernetics is established, which deals on a meta-level with cybernetics applied to itself. New terminologies and definitions are sought, including for the concept of cybernetics itself. One of them says, for example, that cybernetics in its original form is too narrowly defined. As a heuristic it considers problems by means of parameters of communication and control. Second-order cybernetics could then be applied as a general heuristic to any kind of problem.\cite{mueller:2008}

\subsection{Systems Thinking}

In its most general sense, the term "systems thinking" describes the ability to solve problems in complex systems in the context of systems theory considerations - the quasi-practical application of the insights of systems theory and cybernetics.

According to Russell L. Ackoff, there are three types of system perspectives - mechnical, organismic and social. The company, according to Ackoff, can now be analyzed in the context of all these three different perspectives. The mechanical system here represents the closed, reductionist view of the Industrial Revolution, analogous to Taylor's scientific management.
The organismic view, on the other hand, incorporates a holistic view of the organization, similar to Bertalanffy's general system theory. Ultimately, for him, the contemporary view is the social one, since it additionally takes into account the purposes and interests of its system parts, i.e. the employees, but also the company as a whole. Employees are at the same time individuals and collectives of an inseparable system.\cite{ackoff:1994}

\subsubsection{Hard Systems Thinking and Management Cybernetics}

According to the British systems scientist Mike Jackson, hard systems deal with problems that are well-defined and optimally solvable. In the context of the history of cybernetics, a large part of technical problems (belonging to physics, mathematics, computer science and control engineering) are of this nature and belong to this category of hard systems.
Especially in and after the Second World War similar methods for system-applied problem solving were developed. So-called "Systems Methodologies" emerged in all conceivable application areas, such as Operational Research, Systems Analysis or Systems Engineering.
The system as an exact modeling of reality comes to the fore here, as well as an algorithmization of the problem solution by predefined methods.\cite{jackson:2003}

Management cybernetics represents the beginnings of the incorporation of aspects such as negative feedback loops into management theory. It is built on the same technical-scientific considerations as cybernetics itself, and is therefore not a true development of Hard Systems Thinking, but rather a parallel, if not equivalent, theory. As described by Ackoff in the context of systems thinking, management cybernetics also tend to adopt mechanical or organismic perspectives on the organization - often a more or less simple "input-transformation-output" scheme with simple first-order feedback loops. The adjustment of the input in case of a deviation of a function from the goal is represented by the management. Methods such as the black box approach or techniques analogous to those from operational research or systems analysis were provided as a toolbox.\cite{jackson:1991}

Criticism of Hard Systems Thinking comes mainly from the assumption about the existence of a clearly defined system - for engineering problems this seems to be no problem, but for social phenomena and management theory this is hardly applicable. The human aspect of systems is often ignored and incorporated, if at all, as part of the mechanical model. 
Also, different versions of reality could only be considered to a limited extent due to the strongly structured model-like nature of the methods. The same criticism, as well as the lack of subjectivity, can be applies to Management Cybernetics.\cite{jackson:1991}

\subsubsection{Soft Systems Thinking  and Organizational Cybernetics}

Especially in the 1960s, the idea of applying systems theory and especially cybernetics to management theory became very popular. A hollistic view of the organization as a "cooperative system" was also considered necessary and useful in this domain.\cite{jackson:1991} Stafford Beer, a pioneer of this development, redefines these terms in his book "\textit{Cybernetics and Management}". According to Beer, management is the science of control and cybernetics the science of effective control. The entire tradition of management, in which people in the upper ranks of the organizational hierarchy have the highest decision-making power, was outdated for him.\cite{beer:1959} Although his theory has its origins in management cybernetics, it evolved around Ashby's concept of variety and the social and metaheuristic developments of second-order cybernetics. Beers' "viable system model" (VSM) became particularly popular, serving as a reference model for describing, diagnosing and realizing management in an organizational framework, taking into account the relationships within and between each level of the organization. Beer also sees the recursiveness in the system of the organization itself, in that each level as a part can also be seen as a system again. It is therefore perhaps even more closely related to the system approaches of the time than to classical cybernetics. It can also serve more complex systems and models.\cite{jackson:2003}

Besides the Organizational Cybernetics movement was also Soft Systems Thinking. Historically, Soft Systems methodology is considered an emergent practice of systems engineering and goes back to Peter Checkland. Based on Hard Systems Thinking and the well-defined problem-solving methods of Systems Engineering, Checkland now sought to enable greater complexity and the ambiguity of social organizations to be applied to management. The result of this work was methods for obtaining knowledge about systems and structured processes that intervene in and change these systems.\cite{jackson:1991}

Soft Systems are thus a direct development of Hard Systems in that they integrate a subjective component and thus offer the possibility of taking individual perceptions, values and interests into account - according to Ackoff's System Thinking, this enables a social-systemic perspective. Soft Systems Thinking allows for different world views ("Weltanschauung"), is not purely functionalist in structure, like systems engineering or systems analysis, and, in contrast to organizational cybernetics, is not rigidly structural but rather interpretative. The latter is essential in order to incorporate "social facts" into a system. If hard systems approaches are a means of optimizing systems, soft systems approaches are a means of learning about systems, according to Checkland.\cite{checkland:1983}


\subsection{The Systems Engineer as Symbiosis}	

Even if systems engineering in the context of systems theories belongs to the field of hard systems thinking, and is thus supposedly not well suited for managing complex social systems, this profession ultimately represents a symbiosis of many different aspects of management theory, cybernetics and systems theory, unlike all other methods and professions, which have also outgrown these fields. If the roots of the profession of the control engineer are still very clearly recognizable in classical cybernetics, the discipline of the systems engineer manages to achieve a clear crossover into modern systems theory.

Strategy management concepts, such as the analysis of the environment, a clear formulation of the strategy as well as continuous evaluation and control can be found in this discipline. Also black box and white box considerations, systems of systems, as well as the cybernetics of cybernetics concepts are present here - the transfers are very similar. 

To fully understand a complex system, a systems engineer must be able to view it from three perspectives: technical, economical, and social.
The overlap between systems engineers and managers also follows immediately from the fact that here, too, a system must be viewed holistically. "Management has a design and operation function, as does engineering."\cite{goode:1957} The engineer is to ensure that a complex, man-made system, selected from a multitude of other possibilities, satisfies the stakeholders in the long run.\cite{hossain:2019}

Product management in particular has many parallels to systems engineering, as does software engineering. In his book "Software Engineering", Ian Sommerville states:
\begin{quote}
"Understanding and managing the software specification and requirements [...] are important. You have to know what different customers and users of the system expect from it and you have to manage their expectations so that a useful system can be delivered."\cite{sommerville:2011}
\end{quote}
Hereby software engineering is also promoted into the ranks of management, as well as into the systemic context, in which different components in a complex software product must be considered holistically in order to generate a satisfactory product.

\clearpage

\section{Interactive Planning methodology by R. L. Ackoff}

\subsection{The theory behind Interactive Planning}

Similar to Checkland's thoughts on Soft Systems methodology, the work on Interactive Planning methodology has its origins in Ackoff's studies on Hard Systems Thinking and Operational Research. For him, both fields have become too rigid, structure-oriented, and mathematical - they  have lost touch with the real-world problems of managers. According to Ackoff, the so-called "systems age" has more to offer than optimization and objectivity. Instead, adapting and learning should be at the forefront of a constantly changing world and therefore of management methods.\cite{ackoff:1981}

Under this newly emerging banner of "social systems sciences" ($S^3$), Ackoff researched and taught at the University of Pennsylvania.
Meanwhile, his doctoral supervisor and colleague, C. W. Churchman, also developed a program on "social systems design" at the University of Berkeley.\cite{jackson:1991}

According to Ackoff's approach, objectivity could not be achieved through value-freedom, as is usually assumed, but, on the contrary, only through the interaction of different groups with diverse opinions. These individuals must themselves participate in the planning process and help shape it. An organization as a social system serves three purposes: its own, that of its parts, and its superordinate system. Interactive Planning is the practical application of these theories.\cite{jackson:2003}


\subsection{The three principles of Interactive Planning}
In the context to Ackoff's "social systems sciences", there are three principles in the background of Interactive Planning:
\begin{itemize}
\item The participative principle: \\Involvement of as many different stakeholders in the process as possible. All planning members should learn to understand the organization as a whole.

\item The principle of continuity:\\Since \textit{IP} is not based on future predictions, the plans that have been developed must always be monitored, evaluated and modified.

\item The holistic principle:\\Relevance of simultaneous and interdependent planning across all levels of the organization. Coordination (all parts across the same organization level) and Integration (one part of the system across all levels of organization) always have to be kept in mind while planning.
\end{itemize}


\subsection{The Phases of Interactive Planning}

Interactive Planning consists of two parts, Idealization and Realization, which each consist of planning phases. Idealization has two phases, Formulating the Mess and Ends Planning, whereas Realization has the remaining four phases, Means Planning, Resource Planning, Design of Implementation and Design of Controls.

\subsubsection{Idealization}
\paragraph{Formulating the Mess} 

This first phase can be viewed as a situational analysis. The term "mess" describest the multiple, interacting threats the organization will face in the future (unless it changes). The goal is to find the reasons for the organization's potential decline if it does not adapt to its environment. It consists of four sub-activities: 

\begin{itemize}
\item System Analysis:\\
A detailed description of how the system (organization + environment) functions and operates, of the organizational structure, policies, strategies, and practices.

\item Obstruction Analysis:\\
Identification of attributes and characteristics that impede the organization's development and identification of conflicts within the entire system itself.

\item Reference Projections:\\
Extrapolation of current organizational data and performance characteristics into the future, assuming no changes are happening in the organization or its environment.

\item Reference Scenario:\\
A detailed, procative, and possibly even shocking description of how the organization would end itself (self-destruct) on its current path, assuming the previous analysis is to be true. Systematically it is the synthesis of the previous three steps that yields the "mess," the disorder in which the organization finds itself. 
The objectives of this reference scenario are to highlight implications of current behavior and draw attention to relevant problems. Also its aim is to motivate all stakeholders to change and improve the organization.
\end{itemize}


\paragraph{Ends Planning} 

This second phase is probably the most complex and important one. The planners define what the organization would like to be at the present time. It aims at identifying the discrepancies between the developed Reference Scenario and the desired present. The "End" is then the goal to be achieved, the formulation of the ideal of the organization. 
For this purpose, Ackoff describes the methodology of "Idealized Design".


\subsubsection{Idealized Design}
The basic assumption of the approach is, that the organization to be planned was destroyed last night, but its environment in which it was embedded remains intact. On this base the planners should design an organization to replace the current (destroyed) one \textbf{right now}.
Every possible organization is conceivable, except for two contrains and one prerequisite:

\begin{itemize}
\item Technological Feasibility: The design should only use technologies that are usable at the current time.

\item Operational Viability: The system, should it begin to exist, is able to survive in the current environment.

\item Learning and Adaptation: The organization should have the ability to continuously adapt to internal and external changes that potentially affect it. This also implies  the requirement for adaptability of internal and external stakeholders.
\end{itemize}

The goal of "Idealized Design" is \textbf{not} an ideal organization, but a possible, best result at the time of development.
Idealized Design consists of three phases:

\begin{enumerate}
\item formulation of a mission statement
\item specification of the characteristics that the organization to be designed should possess
\item design of an organization with these characteristics.
\end{enumerate}

\subsubsection{Realization}

\paragraph{Means Planning} 

This phase is the first of the realization step. It aims at the development of means/opportunities to close or at least reduce the gaps identified before.  Therefore, it can be seen as the correlation of "Ends" and "Idealized Design". Here, the planners elaborate and select courses of action, projects, program, and new policies that drive the organization closer to the ideal.
The planned ideal present should be approximated as best as possible for the near future.

Problems in the Reference Scenario can be handled by either "resolving, solving or absolving".\cite{ackoff:1981} Absolving ( justifying) should rarely be chosen and should be done only under certain circumstances.
It is better to find a solution to the problem (solve) or at least to eliminate it (resolve).  For each problem, several alternative solutions can be discussed, which are then prioritized and selected by means of questioning, experiments, models or simulations.\cite{ackoff:1981}


\paragraph{Resource  Planning} 

The means and possibilities developed are now considered in the context of economic and business aspects, specifically under the following questions:
\begin{itemize}
\item What  and how many resources are needed to implement the Means? Where are they needed?
\item When will the resources be needed and how much will be available?
\item What should happen in the event of a shortage or surplus of resources?
\end{itemize}

Ackoff identifies five relevant categories of resources for planning: inputs (e.g., materials, supplies, energy and services), facilities and equipment, personnel, money and data (information, knowledge, understanding and wisdom).



\paragraph{Design of Implementation} 

This phase aims at planning and executing of the previously developed Means (in the context of resources). Decisions of the previous phases are translated into a set of instructions and schedules. Those responsible for planning should fully and holistically coordinate this process, and be available as contact persons. Or simply put by Ackoff: \glqq
Determining who is to do what, when and where
\grqq.\cite{ackoff:2001}


\paragraph{Design of Control} 

This final phase runs in parallel with the previous one as a control and monitoring instance. Criteria are identified and selected that allow evaluation of the success of planning decisions. Using these metrics, the instructions and flowcharts are then to be monitored, checked for effectiveness, and adjusted if necessary in the event of errors.


\subsubsection{Execution of Interactive Planning}

According to Ackoff, these six phases should be initialized in this order, but need not be explicitly performed in this order. Due to their strong dependence on each other, they often take place simultaneously and interactively. Interactive planning is therefore also continuous planning, in which no phase is ever completed.	


\subsection{Analysis on Ackoff's approaches}

\subsubsection{Regarding Management Theory}	

Although Interactive Planning belongs more to the systems approach in the context of the delimitations of the previous chapters on management theories, the parallel to the strategic management movement is without doubt very clear. The abstract strategic planning process described by Gregory G. Dess in "\textit{Strategic Management}"\cite{dess:1993} is found almost unchanged in Ackoff's Interactive Planning methodology.

The analysis of the systemic problem situation, or according to Dess "environmental scanning", takes place in Ackoff's approach in the form of the first phase of "Formulating the Mess" in exactly the same way as the "strategy formulation" in Ackoff's phases of "Ends and Means Planning". Implementation as well as continuous evaluation and control of the strategies and plans also take place afterwards. The principle of continuity, as Ackoff calls it, is a characteristic of strategic management - strategy development is an evolutionary process. The "gap analysis" made prominent by Igor Ansoff is also found in Interactive Planning, in that "ends" and "means" are to be bridged. Structurally, Interactive Planning seems to be of a strategic nature. 

It is only when one looks at the underlying philosophy, the social contexts and the social planning details that one notices the proximity to the systems approach. While strategic management strives for a clear top-down view, Interactive Planning, with its participative and Hollistic principle, takes a different approach. The organization's environment directly influences the individual's perception of empowerment and, conversely, empowered individuals with proactive behaviour influence their systemic    \linebreak        environment.\cite{wilbon:2012} This is a feeback loop that only complex systems, not simple linear top-down approaches, can produce.

Interactive Planning as a management methodology is thus completely in line with systems engineering, in that complex systems have to be analysed, planned and realised holistically in their environment. The roots of this discipline, as already presented in this work, make it clear that only a symbiosis of technical understanding, the inclusion of involved individual views of stakeholders, a holistic view of the system in its environment, but also the embedding of the components in the system, as a sub-system, as well as the interaction with each other and the emergent functions can lead to a true understanding of the system. These same principles and perspectives are also required in Interactive Planning.

\subsubsection{Regarding Systems Theory}	

Ackoff builds his Interactive Planning approach on the basis of a purposeful system model, since social systems, according to him, are always also purposeful systems, and these in turn are embedded in higher-level purposeful systems.\cite{jackson:2003}

This approach of "purpose", as well as the notions of "mess", "ends" and "goal" can also be found in earlier works of Ackoff and Churchman as a commentary on cybernetics. They state, that purpose and purposive systems are particularly relevant to the connection between management and cybernetics.
A definitional scheme would be needed which is not based on servo-mechanics or neural behavior, but on purpose and methods of organization. The purposive object must be linked to certain aspects of the environment and oriented by and guided by the Goal. It further shows properties like "choice" in the form of a selection process. Thereby choice alone is not sufficient, but necessary to characterize purposive Beauvoir. Classical cybernetics is "too crude" for social sciences, which require a more sophisticated analysis, according to Churchman and Ackoff.\cite{chruchman:1950}

Finally, they emphasize that the purposive object is in some way the producer of some end result (end, objective, goal): "The man who writes a poem, the machine that computes, the social group that averts disaster, are all acting in such a way that their acts [...] produce some end result."\cite{chruchman:1950}
"Means" are established here as alternative types of behavior having the same function, and "ends", "goals" and "objectives" are defined as the purpose-giving product.\cite{chruchman:1950}

Thus, Ackoff anticipates the classification of Interactive Planning with respect to classical cybernetics. Although there are clearly parallels in the method, such as a continuous improvement of the organization by means of a negative feedback loop, these are not far-reaching enough and, as in Hard System Thinking, lack the necessary subjectivity, the social structures and the self-propagation of the system.
The influences of Soft System Thinking, as a social science development of the system approach, are clearly found here, in that each individual in the organization is given a purpose, an own will, an own world view.

We can also see clear analogies to Georgy Shchedrovitsky's reflections on methodological work and thinking. He also sees a need to do something about the "increasing differentiation between the sciences and the professions".\cite{shchedrovitsky:1981} Furthermore, the role of organization and related management activities plays an increasingly important role in our daily social life. These need to be supported by science. Like cybernetics in the 1950s, the systems approach offers the hope of solving these problems, bringing together disintegrated science and technology disciplines and developing a basis for common understanding and working. 

Although contemporary approaches to the systems approach failed in this huge task according to Shchedrovitsky, just as cybernetics did at the time, the conditions and methods that emerged through the systems approach are a sensible basis for further considerations. For him, systemic problems, and thus systemic thinking, are not of an objective nature, but the subject of multiple, subjective perspectives, as in Ackoff's approaches in the participative principle. The aim of methodological work should have a practical character - "constructions, projects, norms". These should not be verified with facts and checked for correctness, but, like the plans produced by Interactive Planning, tested for feasibility and the associated research should always serve design and standardization.\cite{shchedrovitsky:1981}

Ackoff's inclusion of a hollistic meta-level, as demanded for example by second-order cybernetics, can also be found here. Shchedrovitsky writes that "methodology creates and uses knowledge of knowledge, it is always aware of itself".\cite{shchedrovitsky:1981} Similarly, in  Interactive Planning, the planning context and process itself is the most important component, not the plan produced.\cite{wilbon:2012}


\subsubsection{Critique to Interactive Planning}	

The critique on Interactive Planning will only be brief here, since the aim of the work is not the evaluation of the methodology, but rather the classification of it. Nevertheless, a well-founded critique of the method also gives new approaches for critique of theoretical considerations, from which the method has sprung.

The criticism is mostly directed at the overlying concept of Soft Systems Thinking or Ackoff's version of the "social systems model". First of all, it is discussed whether the world view of an always unifying group of diverse individuals, which is the basis of soft systems thinking, is truly realistic. According to this view, deep-seated conflict is always connected with organizations and society as such. Thus, a diverse group of stakeholders cannot simply agree on a "basic community of interests at a higher level of desirability"\cite{ackoff:1975} in case of disagreement, as stated in Interactive Planning. It is not always possible to find a higher meta-system that ultimately unites all points of view.\cite{jackson:1991}

Furthermore, it is criticized that even if all stakeholders could agree, there will still be inequalities in the weightings. Less privileged stakeholders could therefore not be involved in the planning process as extensively as would be necessary for an objective consensus from different world views. Thus, the participative principle would not be conducive, but would further exacerbate already existing power imbalances, as less privileged stakeholders would be intimidated by the massive resources of the more powerful ones, thus reducing their horizon of expectations to more "realistic" aspirations.

Ultimately, all of these criticisms grow out of the contrast between Ackoff's perspective of a consensual social world and the principle of conflicting social systems of critical systems thinking.\cite{jackson:2003}

\newpage

\section{Conclusion}
In the end, this work has shown how multilateral the field of management and systems theories is, using the example of the Interactive Planning approach. Their development is at least as interesting as their insights themselves. In fact, their development is, in my opinion, absolutely necessary to understand the theories in their entirety, because just as in systems theory and cybernetics, which attempt to gain insights from interdisciplinary research, it is not the direct and isolated consideration of an approach, but rather the detailed embedding within many different disciplines of knowledge that is conducive to understanding.

Russell L. Ackoff's Interactive Planning approach shows many influences from management and systems theory. Concepts of strategic management, parts of cybernetics and a social science system approach come together and form the theoretical basis for Ackoff's methodology.
Both the approach and its critique point to how much theoretical grounding in these fields remains to be evaluated, debated, and synthesized. And subsequently, how many methodologies, practical applications, or even disciplines still grow out of these considerations.


\newpage

\bibliographystyle{acm}
\begin{thebibliography}{xxx}

\bibitem{wumm:2021}  The WUMM Project.
\newblock\url{https://wumm-project.github.io/}.
\newblock Accessed 22 September 2021.

\bibitem{ackoff:1975}  Ackoff, Russell L. (1975).
\newblock {\em A reply to the comments of Chesterton}.
\newblock Goodsman, Rosenhead and Thunhurst, ORQ 26:96.

\bibitem{ackoff:1981}  Ackoff, Russell L. (1981).
\newblock {\em Creating the Corporate Future: Plan or Be Planned for}.
\newblock New York:	Wiley.

\bibitem{ackoff:1994}  Ackoff, Russell L. (1994).
\newblock {\em Systems thinking and thinking systems}.
\newblock System Dynamics Review, 10(2‐3), 175-188.

\bibitem{ackoff:2001}  Ackoff, Russell L. (2001).
\newblock {\em A Brief Guide to Interactive Planning and Idealized Design}.  
\newblock IDA Publishing, May 31, 2001.
\newblock \url{https://www.ida.liu.se/~steho87/und/htdd01/AckoffGuidetoIdealizedRedesign.pdf},
\newblock Accessed 22 June, 2021.

\bibitem{ansoff:1965}  Ansoff, Harry Igor (1965).
\newblock {\em Corporate Strategy: An Analytic Approach to Business Policy for Growth and Expansion}.
\newblock McGraw-Hill, New York.

\bibitem{beer:1959} Beer, S. (1959).
\newblock {\em  Cybernetics and management}.
\newblock London: English Universities Press.

\bibitem{boehmer:2014}  Böhmer, M. (2014).
\newblock {\em Die Form(en) von Führung, Leadership und Management. Eine differenztheoretische Explizierung}.
\newblock Carl-Auer Verlag.

\bibitem{chruchman:1950}  Churchman, C. W. and Ackoff, R. L. (1950).
\newblock {\em Purposive Behavior and Cybernetics}.
\newblock Social Forces, 29(1), 32–39.
\newblock \url{https://doi.org/10.2307/2572754}

\bibitem{checkland:1983} Checkland, P.B. (1983).
\newblock {\em OR and the systems movement: Mappings and conflicts.}.
\newblock Journal of the Operational Research Society, 34, 661-675.

\bibitem{dess:1993}  Dess, G.G. and Miller, A. (1993).
\newblock {\em Strategic Management}.
\newblock McGraw-Hill, New York.

\bibitem{gispen:2006}  Gispen, Kees (2006).
\newblock {\em Gemeinsamkeiten und Gegensätze. Die Entwicklung des Ingenieurberufs in Großbritannien und den USA 1800 bis 1950}.
\newblock TG Technikgeschichte, Jahrgang 73, Heft 1 , Seite 47 - 60.
\newblock \url{doi.org/10.5771/0040-117X-2006-1-47}

\bibitem{goode:1957} Goode, Harry H.  and Machol, Robert Engel  (1957).
\newblock {\em System Engineering: An Introduction to the Design of Large-scale Systems}.
\newblock McGraw-Hill.

\bibitem{holling:2001}  Holling, C (2001).
\newblock {\em Understanding the Complexity of Economic, Ecological, and Social Systems}.
\newblock Ecosystems volume 4, pages 390–405.
\newblock \url{https://doi.org/10.1007/s10021-001-0101-5}

\bibitem{hossain:2019}  Ibne Hossain, Niamat Ullah \& Jaradat, Raed \& Hamilton, Michael \& Keating, Charles \& Goerger, Simon. (2019)
\newblock {\em A Historical Perspective on Development of Systems Engineering Discipline: A Review and Analysis.}.
\newblock  Journal of Systems Science and Systems Engineering, Volume 29.
\newblock \url{https://doi.org/10.1007/s11518-019-5440-x}

\bibitem{jackson:1991}  Jackson, Michael (1991).
\newblock {\em Systems Methodology for the Management Sciences}.
\newblock Springer US.

\bibitem{jackson:2003}  Jackson, Michael (2003).
\newblock {\em Systems Thinking: Creative Holism for Managers}.
\newblock John Wiley \& Sons.

\bibitem{kaiser:2006}  Kaiser, Walter and König, Wolfgang  (2006).
\newblock {\em Geschichte des Ingenieurs - Ein Beruf in sechs Jahrtausenden}.
\newblock Carl Hanser Verlag, München / Wien.

\bibitem{kneer:2009}  Kneer, G., \& Schroer, M. (2009).
\newblock {\em Handbuch Soziologische Theorien}.
\newblock VS Verlag für Sozialwissenschaften, Wiesbaden.

\bibitem{kocka:1999}  Kocka, J. (1999).
\newblock {\em Management in der Industrialisierung Die Entstehung und Entwicklung des klassischen Musters}.
\newblock Zeitschrift für Unternehmensgeschichte, 44(2), 135-149.
\newblock \url{https://doi.org/10.1515/zug-1999-0202}.

\bibitem{malik:1984}  Malik, Fredmund (1984).
\newblock {\em Strategie des Managements komplexer Systeme}.
\newblock Haupt Verlag.

\bibitem{mueller:2008}  Müller, A. (2008).
\newblock {\em Zur Geschichte der Kybernetik: Ein Zwischenstand}.
\newblock Österreichische Zeitschrift für Geschichtswissenschaften, 19(4), 6–27.
\newblock \url{https://doi.org/10.25365/oezg-2008-19-4-2}.

\bibitem{niermann:2017}  Niermann P.FJ., Schmutte A.M. (2017).
\newblock {\em Managemententscheidungen}.
\newblock Springer Gabler, Wiesbaden.

\bibitem{pias:2004}  Pias, Claus (2004).
\newblock {\em Zeit der Kybernetik – Eine Einstimmung}.
\newblock Cybernetics. The Macy-Conferences 1946–1953. Bd. II. Berlin: Diaphanes, S. 9–41.

\bibitem{pindur:1995}  Pindur, W., Rogers, S.E. and Suk Kim, P. (1995).
\newblock {\em  The history of management: a global perspective}.
\newblock Journal of Management History (Archive), Vol. 1 No. 1, pp. 59-77.
\newblock \url{https://doi.org/10.1108/13552529510082831}

\bibitem{shchedrovitsky:1981}  Shchedrovitsky, G. P. (1981).
\newblock {\em Principles and General Scheme of the Methodological Organization of System-Structural Research and Development}.
\newblock Translated from the Russian original by Hans-Gert Gräbe, Leipzig.
\newblock \url{https://wumm-project.github.io/Texts/Principles-1981-en.pdf}.
\newblock Accessed 17 September, 2021.

\bibitem{smith:2008} Smith, Adam (2008).
\newblock {\em An inquiry into the nature and causes of the wealth of nations}.
\newblock Oxford University Press.

\bibitem{sommerville:2011} Sommerville, I. (2011).
\newblock {\em Software engineering 9th Edition}.
\newblock Pearson.

\bibitem{bertalanffy:1950}  Von Bertalanffy, L. (1950).
\newblock {\em An outline of general system theory}.
\newblock British Journal for the Philosophy of Science, 1, 134–165.
\newblock \url{https://doi.org/10.1093/bjps/I.2.134}

\bibitem{oelsnitz:2009}  Von der Oelsnitz, Dietrich (2009).
\newblock {\em Management: Geschichte, Aufgaben, Beruf}.
\newblock C.H.Beck.

\bibitem{wiener:1948}  Wiener, Norbert (1948).
\newblock {\em Cybernetics: Or Control and Communication in the Animal and the Machine}.
\newblock MIT Press, second edition, 1961.

\bibitem{wilbon:2012} Wilbon, Anthony D. (2012).
\newblock {\em  Interactive planning for strategy development in academic-based cooperative research enterprises}.
\newblock Technology Analysis \& Strategic Management, 24:1, 89-105.
\newblock \url{https://doi.org/10.1080/09537325.2012.643564}


\end{thebibliography}


\end{document}

\cleardoublepage
