\documentclass[11pt,a4paper]{article}
\usepackage{ls}
\usepackage[english]{babel}

\title{Concept for the Research Seminar\\ "Complex Systems and Co-Operative
  Action"}  

\author{Hans-Gert Gr\"abe, Ken Pierre Kleemann, Sabine Lautenschläger (all
  Uni Leipzig),\\ Ralf Laue (WH Zwickau) }

\date{March 31, 2021}

\begin{document}
\maketitle

\section{Aim and methodology of the seminar}

The concept of a \emph{system} plays a prominent role in computer science when
it comes to database systems, software systems, hardware systems, accounting
systems, access systems, etc.  In general, computer science is regarded by a
majority as the "science of the \emph{systematic} representation, storage,
processing and transmission of information, especially their automatic
processing using digital computers" (German Wikipedia).  Also certain relevant
professions such as the \emph{system architect} are in high esteem by IT
users.

However, the significance of the concept of system extends far beyond the
field of computer science -- it is fundamental for all engineering sciences
and as \emph{Systems Engineering} with the ISO/IEC/IEEE-15288 standard
"Systems and Software Engineering", it is also the subject of international
standardisation processes.  Even more, the concept of systems also plays an
important role in the description of complex natural and cultural processes --
for instance in the concept of an \emph{ecosystem}.

While classical TRIZ focuses strongly on instrumentally feasible engineering
solutions, Systems Engineering "is an interdisciplinary field of engineering
and engineering management that focuses on how to design, integrate, and
manage complex systems over their life cycles. At its core, systems
engineering utilizes systems thinking principles to organize this body of
knowledge. The individual outcome of such efforts, an engineered system, can
be defined as a combination of components that work in synergy to collectively
perform a useful function." (English Wikipedia). 

In the winter semester 2019/20, we had already studied more intensively the
concept of system and, in particular, examined its application in complex
socio-ecological, socio-economic and socio-technical contexts, see
\cite{Graebe2020}. The central concepts of transition management and activity
management stood for two different perspectives on structural change
processes. In the \emph{transition management} approach, the
structural-transitional challenges are in the foreground, the \emph{activity
  management} approach studies the access to structural changes via the
actions and co-actions of actors and stakeholders.

In both approaches, however, the focus was on a holistic-structural and
analytical view of a decision preparation rather than on procedural management
approaches of decision-making and decision implementation in complex and
contradictory real-world situations.

The WUMM project\footnote{WUMM stands in German for "Widersprüche und
  Managementmethoden" (contradictions and management methods).} aims at a
better understanding of such management processes. The starting point is TRIZ
as a systematic innovation methodology derived from engineering experience
with contradictory requirement situations. Today, similar demands for an
experience-based systematic approach are being made in the field of
management, which means that engineering approaches and admissions are also on
the agenda there.

With the field of "Business TRIZ", which has been unfolding for about 20
years, this transfer of experience is being actively promoted, which is
embedded in older management cultures and approaches.  In recent years,
co-operative action by differently specialised experts has become increasingly
important. In such interdisciplinary work contexts, the development of
\emph{common conceptual systems} of sufficient performance proves to be a
difficult problem that can be supported by digital semantic technologies.
Parallel to these challenges agile approaches play a major role in recent
years, not only in the field of management, but also increasingly in the
solution of socio-technical and engineering problems concerning ongoing
co-operative actions in multi-stakeholder contexts -- for example with the
concept of \emph{technical ecosystems}.

\textbf{In the seminar}, we want to learn more about traditional appoaches to
management theories (F. Taylor, R. Ackoff, P. Drucker, H. Mintzberg) and
relate this to developments in the area of Business TRIZ.  We are particularly
interested in the connection between the dialectical resolution of
contradictory requirement situations in the sense of TRIZ methodology and the
emergence of common conceptual and notational worlds as result of the
application of suitable semantic web technologies.  A special emphasis will be
put on the work of the \emph{Methodological School of Management} and the
Moscow Methodological Circle around G.P. Shchedrovitsky. 

The seminar is a \textbf{research seminar} in which we jointly explore
different aspects of co-operative action in different management concepts.

The students are expected to actively participate in the seminar through
seminar discussions, presentations and last but not least by reading the
relevant materials.  For the successful completion of the seminar, a topic has
to be presented as discussion leader and a paper of 2-3 pages on the topic has
to be submitted in advance.

All materials and seminar reports that can be made publicly available, will be
published in the github repo
\begin{center}
  \url{https://github.com/wumm-project/Leipzig-Seminar}
\end{center}
in the \texttt{Summerterm-2021} folder.

\section{Seminar Organisation}

The seminar will be held weekly on Tuesdays 9-11 am (Leipzig time)
synchronously online.  Prior to each appointment participants have to study
the assigned reading to be in a position to discuss the problems in the
seminar.  The seminar is moderated by a \emph{discussion leader}, who prepares
a short workout of 2-3 pager and makes it available to the participants in
advance \emph{before the seminar} (by Sunday evening).

More about the seminar (for students of Leipzig University) can be found in
OPAL\footnote{\url{https://bildungsportal.sachsen.de/opal/} -- Course
  S21.BIS.SIM.}.  The \textbf{primary source for the seminar plan} is the file
\texttt{Seminarplan.md} in the \texttt{Summerterm-2021} folder of the github
repository \emph{Leipzig-Seminar}.

Upon request, for external seminar participants, access to internal materials
will be provided that cannot be made publicly available in the github repo
\emph{Leipzig-Seminar}.

\section{Examination. Topics for seminar papers}

In order to be admitted to the examination, the seminar must be successfully
completed, one of the seminars has to be moderated as discussion leader and
for this seminar a short workout has to be prepared and made available to the
participants.

Students who are enrolled in the 10-LP module "Semantic Web" must also
successfully complete the TRIZ lab and then take an oral examination (30
minutes) about the acquired knowledge of concepts of systematic innovation
methodologies and Semantic Web.

Students who are enrolled in the 5-LP seminar module "Applied Computer
Science" have to prepare a seminar paper as examination.  More detailed topics
will be announced in the second half of the summer term. The seminar paper has
to be completed until the end of the semester on September 30, 2021.

\section{Privacy}

We follow an Open Culture approach not only theoretically, but also
practically and make course materials publicly available.  This also applies
to the course materials you have produced (presentations, seminar papers) as
well as to (annotated) chat sessions of the seminar discussions, in which your
names are also mentioned.  We assume your consent to this procedure if you do
not explicitly object.  The seminar discussions themselves are \textbf{not}
recorded.

To simplify the further use of the materials and texts, the papers are asked
to be compiled using {\LaTeX}.  Also the {\LaTeX} source should be provided
under the terms of the
CC-0\footnote{\url{https://creativecommons.org/publicdomain/zero/1.0}} license
in order to create a corresponding corpus of texts that can be used to
accompany similar efforts in the OpenDiscovery project. Of course, this cannot
be "decreed". \textbf{Please inform the seminar instructor if you do not
  wish to make your work available for this exchange}.

\section{Seminar plan}

The exact topics and themes will be defined at the beginning of the seminar,
when the number of participants can be estimated more precisely.  So far
interest was expressed in including the following seminar topics:
\begin{itemize}
\item Gr\"abe: Methodological School of Management \cite{Khristenko2014,
  Shchedrovitsky1981}.  A compact presentation of the approaches of the Moscow
  Methodological Circle to questions of a systematic management methodology,
  which had considerable influence on the shaping of the TRIZ approaches.
\item Gr\"abe: Co-operative Action \cite{Goodwin2018, Krug2019}. From Krug's
  abstract: "Charles Goodwin is considered one of the pioneers of social
  interaction research. In his latest book he rearranges his previous
  publications in terms of a concept that could lead to a radical turn in
  anthropology, because his conception of co-operative action covers not only
  the practices of moment-by-moment actions in face-to-face
  interactions. Rather, his approach also encompasses actions with so-called
  absent predecessors, whose previous actions in the form of materiality or
  bodies of knowledge left behind have an impact on the actions of the
  interactants in the here and now. \ldots"
\item Gr\"abe: Schematization of an inventive situation \cite{Kozhemyako2019}.
  The classical systems approach of TRIZ\footnote{A \emph{system} is a set of
    elements in relationship and connection with each other, which forms a
    certain integrity, unity. The need to use the term «system» arises when it
    is necessary to emphasize that something is large, complex, not fully
    immediately understandable, yet whole, unified. [\ldots] the concept of a
    system emphasises order, integrity, regularities of construction,
    functioning and development.  \url{https://triz-summit.ru/onto_triz/100/}}
  is not very well developed to grasp hierarchies of system abstractions. This
  is attempted to be tackled anew with the approach of schematisation in the
  context of management approaches, since in the management field especially
  methods of indirect control have to work with such abstractions. The basic
  concepts are also closely related to approaches of the Moscow Methodological
  Circle.
\item Kleemann: Development of the concept of a philosophy of technology in
  the historical media discourse.  Ernst Kapp \cite{Kapp1877}, Ernst Cassirer
  \cite{Cassirer1930}, Marshall McLuhan \cite{McLuhan1964}, André
  Leroi-Gourhan \cite{LeroiGourhan1993}.
\item Laue: Goal-Models and the i* modelling method, see
  \url{http://www.cs.toronto.edu/km/istar/} and \cite{Yu2010}.  "Much of the
  difficulty in creating information technology systems that truly meet
  people's needs lies in the problem of pinning down system requirements. This
  book offers a new approach to the requirements challenge, based on modeling
  and analyzing the relationships among stakeholders. Although the importance
  of the system-environment relationship has long been recognized in the
  requirements engineering field, most requirements modeling techniques
  express the relationship in mechanistic and behavioral terms." (From the
  summary in \cite{Yu2010})
\item Lautenschläger: Management of socio-ecological transformations. 
\end{itemize}

The seminar starts on April 13 with a kick-off meeting (in German). At
subsequent appointments concepts of the Methodologial School of Management
will be presented. We plan that student's presentations start at the beginning
of May with more detailed explanations of traditional management theories and
approaches as 
\begin{itemize}[noitemsep]
\item Russell Ackoff. System Thinking and Management. 
\item Russell Ackoff. Interactive Planning.
\item Peter F. Drucker (1975). The Practice of Management.
\item Henry Mintzberg (1991). Mintzberg on Management.
\item Frederick W. Taylor (1911).  The Principles of Scientific Management.
\item MBO -- Management by objectives.
\item The SMART approach -- specific, measurable, achievable, realistic,
  time-based. 
\end{itemize}
and continue with more advanced topics as
\begin{itemize}[noitemsep]
\item Business TRIZ and systematic management apporaches \cite{Souchkov2010}. 
\item Schematization \cite{Kozhemyako2019}.
\item Goal-Models and the i* modelling method.
\end{itemize}

The up to date and continuously updated seminar plan can be found in the
github repo of the
seminar\footnote{\url{https://github.com/wumm-project/Leipzig-Seminar/blob/master/Summerterm-2021}}.

\begin{thebibliography}{xxx}
\bibitem{Cassirer1930} Ernst Cassirer (1930). Form und Technik.  
\bibitem{Goodwin2018} Charles Goodwin (2018). Co-operative Action.  Cambridge
  University Press. ISBN 978-1-108-71477-8.
  
  Available as e-book \url{https://doi.org/10.1017/9781139016735} at UB
  Leipzig using your shibboleth credentials at UL.
\bibitem{Graebe2020} Hans-Gert Gräbe, Ken Pierre Kleemann (2020). Seminar
  Systemtheorie. Universität Leipzig. Wintersemester 2019/20 (in German).
  Rohrbacher Manuskripte, Heft 22. ISBN 9783752620023.
\bibitem{Kapp1877} Ernst Kapp (1877). Grundlinien einer Philosophie der
  Technik.
\bibitem{Khristenko2014} Viktor B. Khristenko, Andrei G. Reus, Alexander
  P. Zinchenko et al. (2014). Methodological School of Management. Bloomsbury
  Publishing.  ISBN 978-1-4729-1029-5.

  Available as e-book at UB Leipzig\\
  \url{https://ebookcentral.proquest.com/lib/leip/detail.action?docID=6159470}
\bibitem{Kozhemyako2019} Anton Kozhemyako (2019). Features of TRIZ
  applications for solving organizational and management problems:
  schematization of an inventive situation and working with models of
  contradictions (in Russian).  Dissertation for application for the degree of
  a TRIZ Master. \url{https://matriz.org/kozhemyako/}

  An English translation is in preparation.
\bibitem{Krug2019} Maximilian Krug (2019). Review: Charles Goodwin (2018).
  Co-Operative Action (in German). Forum Qualitative Sozialforschung, 20(1),
  1-7.  \\ \url{https://doi.org/10.17169/fqs-20.1.3197}.
\bibitem{LeroiGourhan1993} André Leroi-Gourhan (1993).  Gesture and speech. 
\bibitem{McLuhan1964} Marshall McLuhan (1964).  Understanding media. The
  extension of man. 
\bibitem{Shchedrovitsky1981} Georgi P. Shchedrovitsky (1981). Principles and
  General Scheme of the Methodological Organization of System-Structural
  Research and Development.  \\
  \url{https://wumm-project.github.io/Texts/Principles-1981-en.pdf}
\bibitem{Souchkov2010} Valeri Souchkov (2010).  TRIZ and Systematic Business
  Model Innovation.  In: Proceedings TRIZ Future Conference 2010, Bergamo,
  Italy.   Available at ResearchGate.
\bibitem{Vernadsky1938} Vladimir I. Vernadsky (1936-38): Scientific Thought as
  a Planetary Phenomenon.\\
  \url{https://wumm-project.github.io/Texts/Vernadsky1938-en.pdf}
\bibitem{Yu2010} Eric Yu, Paolo Giorgini, Neil Maiden, John Mylopoulos (2010).
  Social Modeling for Requirements Engineering. MIT Press.  ISBN
  978-0262240550.
  
  Available as e-book at UB Leipzig\\
  \url{https://ebookcentral.proquest.com/lib/leip/detail.action?docID=3339201}
\end{thebibliography}

\end{document}

