\documentclass[11pt,a4paper]{article}
\usepackage{a4wide,url,enumitem}
\usepackage[utf8]{inputenc}
\usepackage[english]{babel}

\setlist{noitemsep}
\parindent0pt
\parskip4pt
\setcounter{tocdepth}{2}
\title{Seminar Notes}

\author{Hans-Gert Gr\"abe}

\date{Summer Term 2021}

\begin{document}
\maketitle
\tableofcontents
\newpage

\section{Management by Incentive}

In the presentation and discussion, it became clear that incentives offer a
very comprehensive toolbox to improve the fit and performance of "human
resources" (HR) in an organisation, but that this requires a very precise
understanding of the \emph{places} in the organisation where these HR are
"docking" with the organisation. We identified these places earlier as
\emph{roles} and \emph{role descriptions} in the context of the organisation's
structure and process description. If role-specific effects are to be achieved
through incentives, both the specificity of the role and the specificity of
the person to take that role must be considered. Accordingly, for incentives
with emergent effects, the strategic goals of the organisation are a
determining component. Somewhat underexposed remained the question that
incentives presuppose the \emph{availability} of the corresponding resources
in the company, i.e., they presuppose an (ultimately financial)
\emph{throughput} through the company (as a system) in order to be able to
influence the efficiency of the use of resources with this management
instrument. In this sense, management by incentives is a management instrument
"of abundance".

\section{Russell Ackoff. System Thinking and Management}

Ackoff's systems theoretical approaches to management theory date back to the
1970s and are strongly influenced by the developments at that time, especially
in cybernetics. With the availability of the first computers at that time,
questions of \emph{control theory} were newly discussed, which in the
pre-computer age could often only be solved by filigree mechanical
constructions such as the pressure regulator on steam engines, the clock
pendulum, etc. The computer opened up completely new dimensions of management
and control through data collected and processed at the right time, even
though technology at that time was still far from being able to realise the
mature theoretical concepts that had already been conceived. The roots also
reach far back into the first decades of the 20th century to the beginnings of
\emph{assembly line} production, the associated standardisation of operating
procedures and the possibilities and realisation of \emph{automation
  technology} emerging in parallel. In the cybernetics wave of the 1960s,
practical applications of control circuits in various domains played an
important role, right up to the "BMSR technology professional" as a newly
emerging profession in the GDR economy. So much about the background of
Ackoff's theoretical approaches. See also \cite{Steinbuch1966},
\cite{Steinbuch1971} or \cite{Prigogine1981}.

It is interesting to note the convergence of Ackoff's concept of system with
those introduced earlier in the lecture and the seminar, but also their
differences. Ackoff, too, sees the constitutive property of a system in the
interaction of its components and the emergent functionality resulting from
it. This phenomenon, which is also called \emph{synergy}, leads from
concurrency and, in the worst case, opposition of the components to
cooperation. In contrast to TRIZ, where the creation of such a cooperation is
a constructive engineering achievement, Ackoff takes up the idea with the
concepts of organismic and social systems that such \emph{synergetic} or
\emph{symbiotic} (as a higher form of synergy) structuring phenomena also
occur "spontaneously" and without "constructive" intention under
natural-biological (organismic) or socio-cultural conditions of humans'
co-operative actions. In the field of management, especially systematic
management, the idea then arises to influence such "natural" processes also in
\emph{socio-economic systems}, namely in \emph{organisations}, in order to
push them in a certain direction through active management action (ultimately
in a technical understanding of "social engineering" and management as a kind
of engineering profession).

An essential contradiction arose from the attempt to gain influence on
systemic processes with such individual management action, the emergent
character of which had just been postulated, i.e. which precisely does
\emph{not} develop from the influence of one component alone, even if it is at
the top of a management hierarchy and concentrates great decision-making
powers in its hands.

\subsection{"A whole cannot be divided into independent parts." (Ackoff)}

However, this is exactly what is done in system's analysis -- first
\emph{structural analysis} and then \emph{procedural analysis} are performed,
the analysis of the interaction of the parts. These contradictory views are
present both in the system concept of the lecture and in Shchedrovitsky's
argumentation \cite[p. 61 cont.]{MSM}. Shchedrovitsky further asks (p. 58)
what is the significance of a "human component" in such a system (i.e. in an
organisation), which on the one hand is a "cog in the system" (ibid.) when it
is about functioning of the system as a whole, while on the other hand the
(formal) organisation as a "living organisation" has an "informal structure"
(ibid.). Shchedrovitsky goes on to ask what it means for a "manager component"
to be an element of two subsystems of the organisation and thus of the system
as a whole, on the one hand in the management circle of the company and on the
other hand as the head of his own department (p. 61). Is this manager then not
rather the \emph{link}, the "transmission rod" (ibid.) between these two
subsystems?

Is Ackoff's thesis therefore self-contradictory? Is the answer to the question
"Can a system be broken down into parts?" therefore perhaps "No, but we have
no choice but to try it if we want better to understand the system"?

In particular, what does this mean for another thesis of Ackoff "if each part
is managed well, the whole will be"? For M. Rubin, not only analysis and
synthesis play a role here. Both have to be complemented by \emph{evaluation},
in which the \emph{justified expectations} that have grown up from a system
description are compared with the \emph{experienced results}, which closes the
circle of system descriptions to an evolving dynamic that relates it with
reality that is evolving, too.

\subsection{Mechanical, Organismic and Social Systems}

We already analysed this subdivision in more detail a year ago in the seminar,
see \cite{Graebe2020}, and exposed the roots of such a subdivision in the
history of ideas. Briefly speaking, they consist in the fact that 400 years
ago, with the gradual transition from scholasticism to an experimentally based
understanding of science, the success of an initially mechanical technology
became the basis of generalised scientific world views. Attempts to explain
biological phenomena mechanistically quickly came to their limits. The first
major critique of such a kind of explanation is certainly Offray's "Man a
Machine" \cite{Offray1747}, even though such explanatory approaches are still
widely used today, not least in certain explanatory approaches in the Human
Brain Project or in the field of AI, when their mechanical constructs are now
finally supposed to obtain a "divine spark" of intelligence breathed into them
thanks to advances in computer technology.

Criticism led as early as in the 18th century to the demarcation of
(technologically accessible) \emph{mechanical} systems from (technologically
inaccessible at those times) \emph{organismic} ones. With the further
development of chemistry (from phlogiston to modern analytical methods, which
would not have been possible without the developments of physics and precision
mechanical technologies based on it) and biology (from a theory of the
development of species according to Darwin and Haeckel to modern molecular
genetic methods, which in turn would not have been possible without those
scientific and technological developments in chemistry), lines of tradition
have shifted here, but the qualitative picture has not changed.

The concept of \emph{social systems} in its diversity as socio-technical,
socio-economic, socio-cultural and socio-ecological systems is, however, new
and relatively recent in this phalanx of broader reflection, although already
clearly articulated and developed in a relatively strong materialist reading
in Hegel, Feuerbach and Marx/Engels. Its breakthrough is also linked to the
cybernetics debate of the 1960s and more complex approaches to control
circuits, which have since been investigated in greater mathematical detail.
The properties of systems discovered in that research process (up to the
"strange attractors" discussed in the lecture) revealed extremely complicated
forms of progression of systems, which can be described deterministically by
simple differential equations. This showed that classical mechanistic
approaches can only be exploited to adequately describe a very small section
of reality. These argumentations and insights influenced even political
writings such as the Club of Rome's "Limits to Growth" (1972) or Gorbachev's
demand for a "new political thinking" \cite{Gorbachev1987}, but resulted also
in the demand for a new thinking in science as a transition "from the
materialistic-mechanistic worldview to a mental-vital cosmos" in the Potsdam
Manifesto 2005 \cite{PotsdamManifesto2005}, there also the demand "We have to
learn to think in a new way.“ The latter goes back to the politically
motivated Einstein-Russell Manifesto 1955) up to practical approaches of an
Open Culture (open source, software ecosystems, energy ecosystems, models of
distributed autonomous agents, etc.). It remains to be explored which
qualitatively new approaches are hidden behind such thought figures.

\section{MBO – Management by Objectives}

In contrast to Management by Incentives (MBI), which is primarily oriented
towards the activation of forces of self-control, Management by Objectives
(MBO) focuses on controlling the implementation of objectives in the
combination of planning and monitoring.

The unit of implementation is again thought systemically as a delimitable and
delimited management unit with an externally given objective (target), i.e. it
can be interpreted in the context of our system concept. However, this system
concept requires for such a unit to work not only \emph{objectives} but also a
corresponding \emph{throughput of resources}, without which a real world
system is not viable, no matter how hard one tries in managing objectives.
This aspect remained underexposed in the presented MBO concept.

In contrast to MBI, which (implicitly) assumes a comfortable throughput and
asks which incentive systems can be used to stimulate the system's internal
self-organisation and self-structuring forces, MBO takes a much more
structured approach. The \emph{objective} as an external, unquestioned
requirement is detailed in and by the systemic structural unit into an
\emph{action plan}, which consists of individual \emph{activities} that are
linked to each other via \emph{work products} as output and subsequent input,
which must be produced on time and in quality and are subjected to quality
assurance (QA) at milestones. This QA is carried out against quality measures
that can be product- or process-related and form the basis for progress
control.

This remains also within the framework of our system concept, which provides
precisely such a white box analysis of structural and process organisation of
the system. In addition to the flow of substance and energy between the
activities, the instrumentation with quality measures also initiates an
\emph{flow of information} in the system, which, however, remained unspecified
-- beyond the SMART requirement -- and was only addressed as management
instrument. This corresponds quite well with the TRIZ systematics, when flows
of substance and energy as well as actions of tools on objects (in the
activities) are described in more detail, but the control structure as another
component of the system still remain hidden.

The systemic self-similarity structure of the approach remained underexposed,
but immediately emerges when the systemic "white box" -- "black box" principle
developed in the lecture is applied to the action plans as a
component-container pattern. The \emph{action plan} as a systemic "white box"
contains activities as systemic "black box" components that enter the planning
solely through their IO characteristics. For the overall process, the
components must work at runtime in accordance with their specification
assuming the required throughput is guaranteed. However, this throughput is
guaranteed by the scheduled provision of the required work products by other
components (through their activities). This organisational principle can be
repeated at the activity level, but also at the level of the strategic
interaction of the different action plans at the enterprise level. Multi-level
hierarchical management systems can be built according to this principle of
action planning, which will mainly differ in the eigentimes of the built-in
evaluation cycles. Action planning at different levels does not necessarily
have to follow the same rules.

Problems in such a system, which can be recognised via the reports on process
parameters informally passed on to the (human) control component (a TRIZ
system concept), are transformed in this control component into a process
modification output, insofar it remains within the limits and tolerances of
the system (more precisely: in the system description). Hence such problems
can be compensated within the system itself. If the problems exceed the
self-regulation capacity of the system, they must be propagated to the
supersystem.  Such regulation processes are bound to managers and
responsibilities. The abilities, competencies, skills and formal requirements
for occupying such a position are fixed in the corresponding \emph{role
  definition}. At runtime of the system, the \emph{role-adequate staffing} of
this post is part of the "throughput" required for the system to function.

The \emph{control component} of the system -- in the simplest case given by
the responsible manager -- is thus at the same time its "brain", which holds a
description (a picture) of the system and regularly updates it via the
incoming information (the "sensors" of the system), which forms the basis of
the control (the "management") of the system. It is necessary to distinguish
the dynamics of the real system and the dynamics of this image of the system.

In this understanding, management means \emph{exerting influence on the
  self-movement of the system} (Shchedrovitsky: "You can only manage something
that is in motion.") Therefore, not only a metabolism of substance and energy
must be organised in the system, but also a metabolism of information. This
information turnover is the core ingredient to process the difference between
description and reality of the system dynamics.

What remains problematic for the whole approach is the basically sacrosanct
\emph{objective}, which leads to a rigid hierarchical management structure.
Such a structure has the advantage of exploiting the local compensation
potentials to the maximum in order to limit necessary system adjustments to
the lowest possible components. Also, cumbersome feedback structures lead to
inadequate objectives being recognised late and wrongly assigned to
responsibilities. Comprehensive systemic reorganisations require a significant
deviation from regular operations and thus a considerable use of resources.
In companies organised in this way, they can usually only be implemented with
massive support from strategic management that can override the usual rules. 

Further flexibility is achieved above all from action plans which are built up
from parameterised company-wide mandatory process modules that can be adapted
(taylored) to special application scenarios, thus systematising company-wide
project experience (CMMI maturity level 3 "defined"). Uniformly instrumented
process modules of this kind also allow to establish a company-wide "brain"
for such management experiences (CMMI maturity level 4 "quantitatively
managed"). For both approaches, however, a "system of systems" is required
that integrates the action planning systems into a company-wide systemic
framework at a higher level of abstraction.

\section{Business Process Definition Metamodel – BPDM}

\subsection{Current Trends in Business Process Language Standardisations}

Presentation and discussion focused on a closer look at concepts, tools and
models for the description of Business Processes (BP). This field has been
massively evolving over the last 20 years.  This evolution is characterised by
a theory-practice cycle between development of appropriate concepts and
testing their practical suitability in applications in a co-operative process
of practice. This clearly goes beyond the sometimes speculative
conceptualisations of Russell Ackoff or Peter Drucker.

The result is a process of development and consolidation of the terminology
itself, which was illustrated in the presentation by means of the historical
development of various BP language systems (in addition to BPMN, 3LGM and EPK
were considered). The need of cross-concept coordination processes for
\emph{practical} interoperability led to cooperate structures between the
various groups around the different emerging standards in a \emph{Business
  Process Management Initiative} (BPMI, until 2005) and a \emph{Business
  Modeling and Integration Domain Task Force} (BMI DTF, until 2008). Since
2008 the standardisation was coordinated by the \emph{Object Management Group}
(OMG). In this course, BPMN consolidated and is nowadays widely accepted as
leading standard, while the other alternatives lost significance.

BPDM emerged as the product (in TRIZ terms) of the process of formally fixing
the meaning of terms that had previously been used in the different standards
with a wider range of variation.

In this process, the cross-company infrastructural significance of such
descriptive systems becomes evident. Only on the basis of clearly agreed
models with a sufficient number of degrees of freedom (parameters)
cross-company workflows can be effectively coordinated. Only in such a
cross-company infrastructure digitally executable "smart contracts" will be
available for recurring and clearly defined business transactions. A similar
development is well known from computer science with the formalisation of
repeating functional sequences in function definitions.

The development path of the potential of the modelling language covers several
levels of maturity -- from simple communication in concrete cooperations
(model) to the development of structured communication on the basis of
emerging common concepts (meta-model) to the (currently still informal)
development of the expressive capacity of that language itself at the level of
a meta-meta-model. This evolution illustrates once more the dynamic character
of the thesis "the meaning of terms \emph{is} their use" developed in the
lecture.

Ralf Laue additionally explained the meanwhile dominant position of BPMN 2.0
with the fact that it is the only standard that also defines a \emph{process
  semantics}. This aspect remains to be explored further, especially the
significance and position of such a process- and workflow-oriented
\emph{process semantics} compared to a taxonomically oriented concept
formation process of a \emph{structural semantics} within the language
development process. Process semantics presuppose structural semantics, but
the modelling focus is shifted from static to dynamic models. It remains to be
clarified to what extent the required conceptual system can be understood as a
simple extension of the structural model or whether more fundamental
qualitative changes in the modelling itself are present here (Shchedrovitsky
once more: "You can only manage systems that are in motion").

BPDM is part of that taxonomically oriented component of BPMN. This raises the
question whether it is really out of date as claimed in the presentation or it
is nowadays already an integral part of the established taxonomic-structural
component of that language universe containing notational systems such as
BPMN, CMMN or DMN. In any case, Ralf Laue is aware of such a shift of the
frontier of current BP concept formation processes in the direction of
variabilities of dynamic process description schemes.

\subsection{The Relation to a Systemic Approach} 

Let's discuss the relation of the presented BP concepts to our systemic
context.

The definition "A business process consists of a sequence of coordinated
activities. These are either tasks or subprocesses" clearly shows that we are
in the continuity of the concepts presented in the last seminar on MBO (action
and activity) and that the self-similarity of the approach still assumed there
(action as activity in a more comprehensive action) is clearly presupposed.

This allows to interpret the concepts of BP modelling in our systemic context.
The main problem from the last seminar -- a clear distinction between design
time and runtime of the respective systemic description -- was addressed
rudimentarily in the presentation with the distinction between BP and BP
instance, but not developed consistently. The (simple) distinction between
class and instance (or object) from OO programming takes place on another
level, between \emph{template} and \emph{specific expression}, both of which
are still forms of description.

This is already clear from the position of compiler and interpreter as tools
-- as it is well known, that both are translators from a high-level language
into a machine language, hence input and output are in language form. Only the
\emph{execution} of that detailed programme \emph{on a robot infrastructure},
for example, has real-world consequences. The relationship of BP and BP
instances as a high-level and detailed description form of (potential)
business transactions to the \emph{real} business transactions themselves is
similar. The essential link between both is the \emph{solution of the resource
  question}, without which the potentiality of the form of description cannot
be transformed into the reality of the form of execution. This fundamental
question remained unilluminated once more.
\enlargethispage{1em}

\section{The S.M.A.R.T. Approach}

One of the central questions of the discussion was once again the relationship
between description and execution forms in the context of management action.
SMART comes with the claim to support managers in formulating effective
objectives. Like comparable approaches, it is a \emph{methodological handout}
that has to be tailored to the domain-specific context of management planning
in order to produce context-specific management tools and documents for
management action (description level), which then have to prove themselves
efficient in the concrete practice of management action.

Of these three levels, the lowest one of special management action is
ultimately decisive for practical performance. The other two levels -- that of
a special management strategy and that of the methodological foundation of
such a strategy -- clearly move on at two more comprehensive spatio-temporal
scales. This suggests that three system levels are intertwined here.

The first system level is that of the manager with its assigned area of
responsibility and in which he or she represents an essential element of the
control component of that system. In addition to a primarily output-oriented
view of "objectives", such a system also needs an \emph{external throughput}
to function, which must be guaranteed by the system environment. In this
sense, the manager in that system has not only an inwardly directed
\emph{control function}, but also an outwardly directed \emph{security
  function}.  The effect that a system itself exerts an active influence on
securing its conditions of existence is under-illuminated in the management
approaches considered so far.

This influence must be exerted and negotiated in the supersystem, in which the
management strategies are coordinated company-wide among the managers. This
supersystem has not only a processual but also a structural dimension,
concerning available tools and institutionalisation. On the first system level
cooperative action appears primarily in a form based on the division of labour
as cooperation between the manager as individual leader and the domain experts
(to be understood here in a very broad interpretation) of the respective area
of application under secured resource throughput. On the second system level
it is about securing precisely this throughput of resources. The throughput,
which appears as a \emph{contextual prerequisite} on the first level, is the
object of management action on the second level in the sense of an overarching
primary "objective" to "keep the business running" as the first prerequisite
for any further strategic objective. These aspects link directly to the
debates a year ago \cite{Graebe2020}, \cite{Holling2001}, \cite{Foxon2009}
about resilience or systematic transformations of corporate contexts in which
this resilience has been exhausted. This goes far beyond SMART approaches and
thus marks an area of sufficiently "resilient" systems in which SMART
management action is even possible.

While at the lowest system level, management activities are primarily directed
towards operationally producing the expected output, at higher management
levels it is primarily a matter of reproducing the \emph{ability} to produce
output in the necessary quantity and quality.
\enlargethispage{1em}

ISO 9000 and CCMI are structured in such a way: They are not primarily about
\emph{product quality} of an operational output, but about \emph{process
  quality} in the respective companies with the proviso that process quality
is a necessary, but not sufficient, prerequisite for product quality. Process
quality, however, includes appropriate management measures for product quality
and thus builds up a \emph{feedback loop} of quality experiences (positive and
negative) on production conditions, which in turn has an impact on process
quality.

This raises the question how the SMART approach fits into more complex
strategies of corporate development and the prerequisites that must be in
place for a fruitful implementation of the approach. At the same time, this
question relativises a context-free bashing of the methodology. Above we
identified the system of corporate development as the supersystem of the
system of operative control of the production. In this hierarchy the system of
cross-company methodologies, to which the SMART approach belongs, is a
super-supersystem. It has in turn an effect primarily through feedback loops
in the system of corporate development. SMART is therefore not only and not so
much a methodological tool for individual managers to reflect on their own
management experience, but part of the organisation of a company-wide process
of applying the methodology. However, this presupposes that the activities in
the company are structured in such a way that SMART principles can be applied
at all. In CMMI, for example, this is only the case from level 2 onwards. In
addition to a cooperative feedback loop at system level 2 of the company-wide
management experiences, there is therefore another loop between the systematic
implementation of structural requirements and the increasing possibility of
effective use of the SMART principles themselves.

\section{Goal-Models and the i$^\ast$ Modelling Method}

\subsection{Systemic Structuring Processes. Theory and Practice}

Systemic concepts have proven useful in engineering applications and are the
basis of the design of technical systems from components. Systemic concepts
are thus the core of systematic innovation methodologies such as TRIZ. They
allow to delimit different levels of analysis and synthesis on the one hand
and on the other to connect them alternating black box and white box modes.
The dialectical conceptual pair of \emph{system} and \emph{context} plays a
central role here.

The mental systemic structure of analysis and synthesis as forms of
description has a clear influence also on the \emph{practical} structuring of
the world as "reality for us". Systemically based forms of reflection are
transferred to real-world structures in the course of action, even if in this
process "the material" more or less "resists" based on own laws of motion.
Thus the establishing form of relationship in this process is to be understood
as a dialectically shaped co-evolutionary relationship. Such co-evolutionary
relationships between real-world and reflexive processes of formation of
structure are in no way a privilege of anthropomorphic action, but are typical
for coupled flow equilibria in other developed, metabolising biological
systems as well.

\subsection{Structural, Functional and Processual Systemic Forms of
  Description} 

Engineering approaches in general and TRIZ in particular have difficulties to
switch from structural and functional to processual forms of description. For
example, TRIZ knows a larger selection of function-oriented methods and tools,
but flow-oriented approaches are weakly developed. This has much to do with
the fact that systemic composition methods can exploit functional
specifications of interfaces, but in rare cases processual performance
parameters are specified as interfaces. Such imbalances are also widespread in
computer science -- functional tests start early in integration scenarios in
software testing, including the use of test drivers and mock objects, which
are \emph{functionally} designed, whereas load, stress and performance tests
usually start only at the system test level.
\enlargethispage{1em}

\subsection{Systemic Approaches in Engineering and Management}

As central goal of our seminar we try to identify systemic approaches in
different management theories and to investigate parallels and differences
between engineering and management action. We found that the context of
management action can be well captured with self-similar systemic methods, but
the associated hierarchical scaling remains underexposed in most management
theories.

The major management theories we have looked at so far focus on
\emph{practical} management action. Descriptive forms and theoretical
approaches are considered merely as support and tools for this main focus. The
context for such management actions are systemically structured "living
organisations" (Shchedrovitsky) in the mode of operation. Thus the context of
management action differs fundamentally from engineering action, which -- at
least in the horizon of the experience of most TRIZ practitioners -- refers to
systems in \emph{maintenance} or even \emph{design mode}. This is, however,
only a provisional demarcation, because the majority of engineers are
\emph{production engineers} and thus are concerned with the \emph{mode of
  operation} of large-scale technical systems. Moreover, the maintenance mode
of a technical system is part of the operating mode of the (socio-technical)
supersystem.

Embedding management action in an already \emph{well developed systemically
  structured real-world context} and thus in an advanced structured "living
organisation" also makes sense under another aspect. We consider objectives,
rules and frameworks -- including the authorisation of the managers themselves
-- as \emph{given} existing conditions of the management action and
consequently fade out the analysis of their historical genesis as part of a
"reduction to essentials".

From a TRIZ perspective, management in this sense is a \emph{function} of the
control component of the corresponding system. It remains open to what extent
this function can still be personalised under modern technical conditions as
in classical management theories, if methodological knowledge (of the manager)
and deep domain-specific special knowledge (of the production engineer) are
both required for appropriate management. It is also open to what extent
management action under modern technical conditions can be primarily directed
at the inside of the system. It may well be that division of management as in
SCRUM (between product owner, SCRUM master and team) are methodologically more
appropriate.

\subsection{i$^\ast$ Models and Business Process (BP) Modelling}

The long introduction served primarily once again to mark the place of the
explanations on i$^\ast$ models in the overall theoretical building of
management theories.  The repeated comment that various of the discussed
management approaches are of academic interest only without significant
practical impact indicates above all the multi-layered nature of the
corresponding (interpersonal) reflection structures of production, selection
and self-regulating dissemination of the corresponding explanatory patterns.

With an actor-centered view, i$^\ast$ models consistently implement the
approach of "areas of responsibility" (lane, performer role and actor in BPDM)
built into other BP models. It is implemented in a more strict way as it
points in the direction of a systemic closure by a feedback loop between
expectations and experiences (not yet included in the i$^\ast$ concept). The
areas of action of the individual actors highlighted in grey in the i$^\ast$
diagrams have many parallels to the concept of action spaces in the lecture.
Compared to classical management approaches, this requires a concept of
interaction \emph{between} such management areas and thus opens the door to
self-similar system concepts. It was not further addressed in the presentation
to what extent this is already realised in the current i$^\ast$ concept,
i.e. if system-supersystem structures are conceptualised. However, it must be
taken into account that the approach does not claim to be a complete BP
modelling, but focuses on requirements engineering and thus on a first phase
of a detailed modelling.

i$^\ast$ modelling consists of two essentially different dimensions -- the
design of \emph{dependency relations} between areas of action and the
modelling of internals of the "grey areas". From a systemic point of view, the
former is comparable to the black box specification of dependencies. On the
one hand import and export interfaces can be distinguished by specifying such
couplings and directions. On the other hand it preserves not only the
functional "what?" but also the causal "for whom?". In modern component
concepts in computer science, such as the \emph{CORBA Component Model} CCM
\cite{CCM}, these import and export interfaces play a central role as
receptacles and facets, and the couplings are dropped under the aspect of
(broader) reuse in favour of a formal specification of the interface.

Inside the "grey areas", process structures are modelled as directed graphs
from predefined building blocks. This models a white box approach.  It remains
unclear to what extent structural, functional and process-related aspects are
differentiated. In the examples given, the structuring of \emph{goals} -- the
central notion in an i$^\ast$ model -- is used in all three ways. It is also
clear that the notions \emph{task} and \emph{resource} conceal further
requirements (a task must be implemented, a resource must be provided), which
are not further developed in the diagram. The distinction between these two
types of (masked) requirement was not touched in the presentation and
discussion.

It remains to note that the approach goes beyond those considered so far in
one special direction. It considers the interaction of independent third
parties (the actors) in a service area and thus \emph{contractually bound
  management structures} without a "central manager" with correspondingly
authorised possibilities of intervention. It remains open to what extent such
modelling approaches are typical for a service-oriented industry and whether
we are or since when we really arrived in a "service society" in which such
economic relationships are dominant.

\section{Mintzberg on Management}

The goal of our seminar is to gain insights into different approaches of
management theories and to interpret them in a system-theoretical context.
This was outlined in more detail in the first seminar. Management in this
understanding is always directed towards \emph{managing an organisational
  context}, even if this is not explicitly present in some management
approaches.

This refers especially to approaches that identify management largely with
management of people, such as "management by incentive". Also the concept of
"systematic management" \cite{Hill2015} mentioned in the first seminar, which
focuses on management as process, points in the direction of such an
understanding of management that is primarily oriented towards tactical
concerns.  Shchedrovitsky \cite[p. 66]{MSM} states that management must not be
reduced to such a perspective: "Programme design is what management is all
about. Someone who cannot design development programmes cannot manage people".

In the sense of our notion of technology with its three levels
\begin{itemize}
\item socially available processual knowledge,
\item institutionalised practical procedures and 
\item private procedural skills
\end{itemize}
such approaches in fact neglect the second level of targeted organisational
development. In the first seminar the interplay of personal development
possibilities and organisational structuring was addressed above all by a
reference to Ian Sommerville \cite[ch. 19]{Sommerville2015} with his version
of the concept of a socio-technical system. Shchedrovitsky points out that
designed structuring processes within organisations, however, have their
limits and, especially in the case of changing framework conditions, easily
come into contradiction with self-movement structures of a "living
organisation". The distinction between formal and informal organisations
according to the gradual expression of such a designed structuring component
takes up this idea. It can be assumed that such informal organisations, which
are typical for multi-stakeholder contexts, are to be managed by different
principles than formal organisations with their clear structures of
authorisation and responsibility.

The general connection between practical activity and conceptual reflection
described by Shchedrovitsky in \cite[p. 70 cont.]{MSM} unfolds for management
and in management theories on three levels:
\begin{itemize}
\item The practical activity of a manager in the conceptual context of his or
  her own experience and of a concrete organisation with little separation of
  the concrete and the abstract.
\item The practical reflection of this action in groups of such managers in
  the conceptual context of generalised own experience and typified
  organisational structures with little separation of domain-specific and
  methodological aspects (the term "group" is used here in the sense of
  Shchedrovitsky).
\item The academic reflection of these reflections across domain boundaries in
  order to advance to general methodological insights.
\end{itemize}
Each of these three levels is associated with a specific spatio-temporal
dimension of a feedback loop between justified expectations and experienced
results. This feedback is objective insofar as the evaluation of the
experienced results is not solely individual-subjective, but -- in the sense
of Berger/Luckmann's "legitimate interpretation of meaning" \cite{Berger1966}
-- broken by the social behaviour of others. In this sense, managers and
management theories are themselves "managed".

Managers on the first level are evaluated by the extent to which their actions
have been "successful". Managing at this level is a private procedural skill
that can only be improved in practice. However, since "success" is evaluated
on the conceptual standards of the second level, these conceptual standards
must be acquired in a learning process, for which there are a large number of
educational offers and certification structures for "management
professionals".

In this process, strange structures emerge, as described in "Mintzberg on
Management" -- an orientation structure in language form which the management
novices have to follow and to adopt, while those rules do not seem mandatory
to be applied by the same management gurus that teach the novices. In any
case, (not only) Mintzberg repeatedly points out that management rules are
made to disregard them. Good managers know above all which rules in which
contexts are better not applied (and are allowed not to apply them).

This strange evaluative relationship of development forms between level 1 (of
practical management) and level 2 (of systematisation of practical management
experience) does repeat itself in the relationship of levels 2 and 3 when it
is emphasised that most academic approaches in management theories have little
practical relevance. The alleged ineffectiveness of level 2 and 3 approaches
for the respective upstream level is opposed by the dense institutional
structures in which new generations of managers are growing on just these
rules. This is in sharp contrast to Mintzberg's insistence on the purported
lack of usefulness of advice from these training structures.

\section{The Toyota Management System}

The Toyota Management System, like other management approaches, has its roots
at the turn of the 20th century. Management thus appears as a specific aspect
of further differentiation of an industrial production process whose
beginnings lie in the second half of the 19th century. The essential
prerequisites for this were
\begin{itemize}
\item the production-technical provision of energy sources, which in their
  power went far beyond the previously available human and animal energy
  sources,
\item the mass production of standardised intermediate products and
  semi-finished goods on the basis of advanced scientific and technical
  knowledge (e.g. iron processing and steel production in the second half of
  the 19th century),
\item the inventive use of these new principles of action and materials in
  new, standardised products suitable for the mass market.
\end{itemize}
This unfolding industrial mode of production was shaped in many places by new
entrepreneu\-rial personalities who were distinguished by both technical
expertise and entrepreneurial skill and thus combined two essential components
of economic success -- access to generally available processual knowledge as
private processual skills and a feeling for market demands -- which later
diverged in the professions of engineer and manager. However, in the
biographical literature, engineering requirements and their scientific
foundations are in the foreground; there is no mention of special management
techniques developed and generalised at that time.

This was also the case with Toyoda senior and Henry Ford around 1900. However,
there were the scientific methods so successfully applied in the technical
field and the challenge to apply them also to business organisation. The
vision was to organise a factory to run as well and smoothly as a machine.

The system-theoretically founded design principles of hierarchical composition
of a whole organising the functional interaction of viable technical artefacts
as components were transferred to the organisation of production. However, in
that situation the principles must not to be applied in a constructive way as
in the design of a machine, but as reorganisation of an already viable
business system, a "living organisation". Among all components of such a
business system the "human" component turned out to be the most unwieldy and
least accessible to "specification-compliant functioning".

Major transitions in this direction took place in the first quarter of the
20th century. They are primarily associated with the names Frederick Taylor
and Henry Ford. Ford introduced assembly lines with his car factory as early
as 1913 and thus switched to a highly disaggregated form of organisation of
production. Interestingly (according to Wikipedia), the initiative for this
reorganisation did not originate from Ford himself, but from leading employees
of the Ford Motor Company.

Fordism is also usually associated with a second insight, that the entire
economic cycle requires not only supply but also demand with purchasing power
in order to close the product cycle and thus also the cycle of capital. This
requires an appropriate wage policy. In this way, labour turnover can also be
dampened and thus important private processual skills can be kept in the
company. Of course, this is only possible if corresponding profits are
generated in the company, i.e. if the systemic context ensures not only the
throughput of energy and material, but also of capital (that can be converted
into social energy).

We studied such management approaches several times in the seminar. The
repetition here is mainly to highlight similarities and contrasts with the
Toyota system.  The Toyota Motor Corporation, which started car production in
1936, was not only in competition with General Motors and the Ford Motor
Company, which dominated the American car market of that time, but also in
tension with the first emerging American management theories based mostly on
\emph{structural} production-organisational approaches. Toyota, on the other
hand, relies on a \emph{process-oriented} model of \emph{networking} systemic
approaches. This is particularly visible in the principle of "just in time".
The throughput of a \emph{resource} to be guaranteed in a system as a
contextual condition is coupled to the \emph{resource provision process} ("the
right part at the right time in the right amount") of another system. In this
approach, the focus is on the networked structure of flows of interweaving
processes and thus the coordination of cycles and rhythms instead of static
structures and a stronger orientation towards quantities. The Toyota
Management System and the closely related Total Quality Management (TQM)
approach differ significantly in their model-theoretical approaches to process
modelling, which is reflected, for example, in the different guidelines ISO
9001 (for classical process models) and ISO 9004 (for TQM-based process
models).

It remains open how such a process oriented coordination works beyond the
boundaries of the company. For example, the expansion of a worldwide digital
shipment tracking system comes with completely new networking possibilities in
the area of Supply Chain Management (SCM) for just-in-time production.  It is
currently confronted with increasing real-world logistical problems due to
disruptions in connection with the Corona pandemic. Forms of description and
execution forms are in tension. However, major problems do not arise
\emph{within} the individual systems of logistics or production, but \emph{at
  their interface}.  Short-term real-world disruptions in the logistics chain
can be communicated, but they influence real-world execution of longer-term
planning in production, which in turn has impact on the chain of distribution.
Such problems at the interface of two systems -- the resource demand of
production and the resource provision by logistics -- the disruption of
coordinated rhythms -- must be intercepted by corresponding robustness of the
production organisation of the target system to avoid further propagation to
other systems. The Toyota system of exhausting all productive reserves has
therefore its limits.

All relies on the stabilisation of existing systemic processes \emph{and} the
stabilisation of their interaction. Gradual changes are possible above all in
stable environments and thus in a context that is today much associated with
the term \emph{resilience}. One question remains: How does the Toyota System
handle such more disruptive change management, e.g. caused by deep
technological changes as the Digital Change?

\section{Taylor's Principles of Scientific Management}

Taylor's approach throws a spotlight on practical production-organisational
developments and their reflection in the early phase of the 20th century.

We are faced with a beginning production-organisational differentiation, which
in the further course leads to the formation of the professions of
\emph{engineer} and \emph{manager}. Both professions are (well-paid) wage
labourers in the sense that they are usually don't own the companies for which
they work.

The emergence of the \emph{Principles of Scientific Management} is embedded in
the technological upheavals of that time, which led to the devaluation of
previous private processual skills (the "rules of thumb"). New processual
skills are to be built up. In that context the "scientific" methods that have
been successfully applied in the technical field are also to be applied to the
organisation of production. In contrast to the scientific background
knowledge, on which engineering solutions were formed and based at that time,
such a background did not exist in the production-organisational context.
Taylor therefore generalised above all his own production-organisational
experiences within the context of an increasing algorithmisation of
production.

This strenghtens the production-organisational description form. Weights shift
from the workers' private processual skills to institutionalised procedural
methods. When such methods are applied it is assumed that only a small amount
of conditioning for workers is required, i.e. a short training is sufficient
in order to make them to function within the "living organisation".

This marginalises the essential feedback loop between justified expectations
and experienced results at the base of the individual worker and shifts it to
the cooperative space of action of the enterprise as a whole. There, the two
professions -- engineer and manager -- take over the coordination of this
feedback at the technical and production-organisational level.

It is also noteworthy that these differentiations have their roots in the
differentiation within wage labour itself in the second half of the 19th
century. Taylor begins as a foreman in a steel plant, knows very well the
motives and methods of his colleagues to resist too much work pressure and
ultimately acts against them with his methodical approach. The differentiation
of professions thus leads to contradictions and tensions in the workforce
itself, which later experience a conceptual consolidation in the distinction
between \emph{blue collar} and \emph{white collar}.

Taylor's principles are to be considered on the plane of the further
development of the organisation of production. With the emerging "assembly
line society", the further division of the production-organisational process
continues and the processual skills of a largely unskilled workforce are less
and less important. In addition to the \emph{profession} for the few, there is
now also the \emph{job} for the many.

With the invention of the computer, this "trivialisation of production" is
pushed even further on the one hand and culminates today in the image of the
automatic "Factory 4.0", in which the "renitent human element" can apparently
be completely eliminated. On the other hand, the increasing technisation of
production leads to a revitalisation of suitable non-trivial "processual
skills". As simple tasks are more and more carried out by automated systems,
this aspect is gaining increasing importance again.

\section{Russell Ackoff. Interactive Planning}

Ackoff's Interactive Planning has a high affinity to TRIZ concepts and
methods.  The relation to Business TRIZ need to be explored further, but this
is a topic for future seminars.

In my remarks I concentrate once more on a question that played a subordinate
role in the analysis of management tasks so far: The relationship between
\emph{resource use} and \emph{resource provision}.

Let me start first with a relation of this problem to the context of the
conceptual system developed in the lecture. A central question was "What does
it mean to change a world that is itself also constantly changing?" This can
only mean to \emph{get influence} on the development of the world. But why get
influence on the development of the world? Because the way as the world
developments "by itself" is problematic. Management therefore means to solve
problems. "Where there are no problems, there is no need for management". The
solution of problems is preceded by their delimitation and analysis. In a
systemic context, this is done by internal demarcation of elements, external
demarcation of the system against an environment and reduction of the
relationships between the elements to essential ones.

A systemic approach thus assumes that a contextually delimited area is
accessible to description as a white box and thus to management. Management in
this understanding is an \emph{internal function} of the system, which in the
TRIZ context is assigned to the control component of the system. In this
context, the interaction of the various systemic resources must be organised.
Resources appear conceptually in TRIZ under various names -- as components,
tools, processed objects, etc. All these resources have in common that they
appear in the system with a functionally (components and tools) or
structurally (processed object as preliminary product) significant role, but
the \emph{reproduction} of these resources takes place in neighbouring
systems.

In this sense, components are also neighbouring systems rather than
subsystems, because the emergent properties of a system result from the
\emph{interaction} of the component properties. Since only the
\emph{interfaces} of the components are accessed here, components thus appear
as a black box just like the services of neighbouring systems in the
environment.  An immersive system approach makes a clear distinction between
"inside" and "outside". This looks different in a submersive system approach
-- a component appears in the system as a \emph{reference} to the
implementation of its functionality.  The same for services from neighbouring
systems.  A similar conceptual arrangement is known from the theory of
Component Software.

Thus in a systemic context \emph{resource use} and \emph{resource
  reproduction} are in a contradictory relationship, since both occur in
different systems. For resource utilisation, the interface to which the
resource couples must be described in more detail. Since a system description
is always a reduction to essentials, this interface description can and must
work with a \emph{fiction}\footnote{The notion \emph{fiction} is used in this
  place in the meaning developed in the lecture.} -- an abbreviated way of
speaking about a social normality. In a world of labour division, this
"production of normality" as reproduction of resources is outsourced to
another systemic context.

The central resource in cross-system socio-technical processes is the human
being. Resource description and resource reproduction for "human resources"
use the concepts of \emph{role description} and \emph{role occupation}. A
problem (even a contradiction) arises when no suitable candidates are found
to fill a role.  Approaches such as Management by Incentive or Management by
Objective completely ignore these questions and assume that sufficient
qualified personnel is available. F. Taylor's Scientific Management includes
a qualification programme for appropriate role appointments, which is still
significant today in the concept of trainee programmes. However, even then
this approach only workes for unskilled and semi-skilled workers, but not for
high qualified technical personnel.

What has been explained here for human resources also applies to other
resources. A system that consumes a certain resource depends on the provision
and thus reproduction of this resource, which corresponds to a coupling of two
systems. The contradictory nature of the description form results from the
fact that in each of the two perspectives the other system is seen as a black
box. This leads to inconsistencies in the coupling of justified expectations
and experienced results in the common execution form.

Although in the Toyota system such interfaces are considered in more detail,
the question remains open how exactly the contradictions of a "just in time"
coupling are processed. The approach to provide "the right thing at the right
time in the right quality" propagates a \emph{push concept} and externalises
the responsibility for solving the mentioned contradiction to the (other)
system of resource reproduction. However, this system is dependent on a
certain external throughput in order to function adequately. The question how
the resource-consuming system affects this systemic existence condition of the
resource-producing system remains open in the Toyota system approach. The
overarching principle of the coordination of cooperative action is the push
principle, according to which everything happens "by itself".

Ackoff's approach of Interactive Planning moves a step further and formulates
the coordination problem at least as a planning problem. From a systemic point
of view, the contradiction of resource mediation can only be solved in a
further system whose emergent function is to secure precisely this resource
mediation. In an immersive system approach, such a system would be called an
supersystem. In a submersive systems approach, however, this system stands
alongside the systems it connects, because it has to fulfill a specific task
that has to do with the \emph{functioning} of the subsystems and to a less
extend with their \emph{function}.

Management in this sense therefore does not only mean leadership as a personal
characteristic or control as a functional component of a system, but is a
central element of system development itself.

\section{Anton Kozhemyako. Contradictory Business Processes\\ and
  Schematization}

During our seminar we observed that the term "management", especially as a
profession in its own right, emerged only in the course of the 20th century.
Moreover, we established a close connection to the growing penetration of
production processes by technology thus deepening the division of labour. Like
the technical development in the 20th century, these production-organisational
innovations did not form a linear process, but were characterised by at least
two upheavals -- the \emph{transition to assembly line production} and a
largely increasing degree of dissection of work processes since the 1930s, and
\emph{digitally based measurement and control processes} since the 1960s. The
latter are gaining a new dimension of significance in the digital
transformation where they are widespread introduced as SCM and CRM in
production-organisational processes \emph{between companies}. This also
influences the structure of management processes. Moreover, this requires the
development of common conceptual systems, at least in areas that are linked by
such common supply chains.

Accordingly, management theories in the 20th century focused on different
focal points. Taylor's \emph{Scientific Management} (1911) is still entirely
under the impression of the principal possibilities of large-scale machine
production on a "scientific" basis and focuses above all on the preparation of
the worker for such an industrial process, which is driven in its descriptive
form by "one clever head" but (still) requires "a thousand hands" for
practical realisation.

Even in such an algorithmisation of production, as is well known, the devil is
in the detail. In computer science, the difference between abstract high-level
language programs (up to generative programming) and their implementation in
machine code is well known and has been the subject of a lengthy process of
development. Everything remains relatively simple as long as the computer
program does not leave the computer as runtime environment and produces at
most figures on the screen. The story gets more difficult when the algorithm
is designed to control and manage real-world processes.

The story also becomes more difficult when the algorithm leaves the
single-user computer built according to von Neumann principles and starts to
make programs interact in a distributed environment. Ackoff's
\emph{Interactive Planning} addresses the algorithmisation of production
processes in such distributed structures.

However, the dependencies and interaction of actors in \emph{multi-stakeholder
  contexts} can no longer be captured in a mechanical-algorithmic concept.
\emph{Agent-based programming} in complex authorisation and mediation
structures are developed, in which the individual "agents" operate as actors,
each with its own \emph{fictions}, i.e. abbreviated ways of speaking about
"social normality" that is produced by others of the involved actors. In
addition to the goals and interests of the actors, the \emph{mastery of means}
moves further into the foreground. It is more and more a limiting factor and
at the same time a trigger for further systemic differentiation.

Ackoff's self-similar approach of \emph{Systemic Thinking} should not be
misunderstood as conceptualising a "system of systems" as "supersystem". The
"system of systems" is reduced in its descriptive dimension, also in Ackoff's
case, to the interaction of the subsystems as \emph{working units}, while in
the descriptions of the subsystems their \emph{functioning} in the sense of a
means perspective is in the foreground. The forms of description themselves
use different conceptual systems on both levels and thus ultimately also
different languages. Both forms of description meet only in the
\emph{interaction of a real-world practice}, which, however, also
\emph{structures itself systemically} under the influence of systemic planning
-- proven thoughts solidify in systemic structures that can be reused as
templates. This is part of the feedback loop between justified expectations
and experienced results. Ackoff's proposal is therefore to concentrate the
development of new systemic boundaries in the description form along the
narrower contexts ("cohesion") of practical cooperation of already established
systemic structures that \emph{emerge in practice} in this process of
experience.  In this way emerging along those boundaries systemic
"superstructures" are additionally strengthened in terms of description. But
this "superstructure" is by no means a "supersystem", since it operates with
other "fictions" as abbreviated ways of speaking than each of the subsystems.

Management as problem solving enters precisely this field of \emph{system
  formation} by transforming problems into solutions and thus developing
systems further -- from the "system as it is" to the "system as it should be"
and finally to the "system as it has become". Management theories claim to
provide conceptual and methodological tools for this. \textbf{Ackoff} sees a
systemically based methodology as an important general approach to such a
task.  \textbf{Shchedrovitsky} develops a conceptual system specifically
oriented towards the manager's field of activity. \textbf{Mintzberg} probably
assumes that such a uniform methodological approach is not possible due to the
structural diversity of his \emph{basic types}. He claims that at best a
toolbox of basic procedures can be given to be applied based on experience and
situation.

\textbf{Business Process Modelling}, whether as SMART approach, as Goal Model,
in MBO or in Business TRIZ, however, assumes that a \emph{process-oriented}
conceptual system is well suited for modelling business processes and thus
supporting real-world management processes. The focus, however, has always
been on \emph{inner-system interrelationships}. The throughput of material,
energy (including the social energy of real employees) and information
\emph{driving} the inner system structuring appears in these methodologies in
the term \emph{resource} as an interface. This seems to be a difficult issue
because in the \emph{BPDM glossary} the term "resource" is used in many places
but not defined.  V. Souchkov summarises the term in his glossary
\cite{SouchkovGlossary} as "Any type of tangible or intangible matter that can
be used to solve an inventive problem: time, space, substances, fields, their
properties and parameters, etc." The \emph{TDS-100 glossary}
\cite{TDSGlossary} defines only the term "SuField resource", but writes about
it in more detail "These are fields, substances, time, space, neutral or
harmful functions and relationships that are available in the system,
supersystem, subsystem and can be used to realise useful functions. For the
realisation of useful functions, derived resources can be used that are formed
by transformation or combination of original resources. Universal resources
are emptiness and periods (pauses) in time. ...". In any of these definitions,
\emph{resources are given as things ready for use}. Resource production and
provision is not addressed.

However, we have already seen that in the term \emph{resource} two processes
of different systems are coupled -- the \emph{use} of the provided resource
and the \emph{process of its production or reproduction}. In the case of
"human resources", the concept of "role" is used, which breaks down into the
sub-processes of "role definition" and "role occupation" as two complex forms
of description with a clearly different focus. Anton Kozhemyako takes up this
question in his text with the terms "generalised object" and "filling" and
follows approaches proposed by Shchedrovitsky.  However, the exact modelling
in the case under consideration remains controversial ...

\begin{thebibliography}{xxx}
\bibitem{Berger1966} Peter L. Berger, Thomas Luckmann (1966). The Social
  Construction of Reality: A Treatise in the Sociology of Knowledge. ISBN
  0-385-05898-5.
\bibitem{CCM} CORBA Component Model Specification.  Version 4.0, chapter 6.
  OMG, April 2006.  \url{https://www.omg.org/spec/CCM/4.0/PDF}
\bibitem{Foxon2009} Timothy J. Foxon, Mark S. Reed, Lindsay C. Stringer
  (2009). Governing long‐term social-ecological change: what can the adaptive
  management and transition management approaches learn from each other?
  Environmental Policy and Governance, 19 (1), 3--20.
  \url{https://doi.org/10.1002/eet.496}
\bibitem{Gorbachev1987} Mikhail S. Gorbachev (1987). Perestroika and new
  thinking for our country and the world.
\bibitem{Graebe2020} Hans-Gert Gräbe, Ken Kleemann (2020). Seminar
  Systemtheorie, Universität Leipzig, Wintersemester 2019/20. Rohrbacher
  Manuskripte, Heft 22. LIFIS, Berlin.
  \url{https://nbn-resolving.org/urn:nbndebsz:15-qucosa2-748430}
\bibitem{Hill2015} Rebekiah Hill (2015).  Systematic Management: Theory,
  Overview.
  \url{https://study.com/academy/lesson/systematic-management-theory-lesson-quiz.html}
\bibitem{Holling2001} Crawford S. Holling (2001). Understanding the Complexity
  of Economic, Ecological, and Social Systems. In: Ecosystems (2001) 4,
  390–405.
  \url{https://www.esf.edu/cue/documents/Holling_Complexity-EconEcol-SocialSys_2001.pdf}
\bibitem{MSM} Viktor B. Khristenko, Andrei G. Reus, Alexander P. Zinchenko et
  al. (2014). Methodological School of Management. Bloomsbury Publishing. ISBN
  978-1-4729-1029-5.
\bibitem{Offray1747} Julien Offray de La Mettrie (1747). Man a Machine.
  \url{https://en.wikiquote.org/wiki/Man_a_Machine}. In German:
  \url{http://www.lsr-projekt.de}
\bibitem{PotsdamManifesto2005} Potsdam Manifesto (2005).
  \url{https://en.wikipedia.org/wiki/Potsdam_Denkschrift}
\bibitem{Prigogine1981} Ilya Prigogine, Isabelle Stengers (1981). Order Out of
  Chaos. Man’s New Dialogue with Nature. (Dialog mit der Natur). Piper,
  München.
\bibitem{Sommerville2015} Ian Sommerville (2015). Software Engineering.
\bibitem{SouchkovGlossary} Valeri Souchkov (2014). Glossary of TRIZ and
  TRIZ-related terms. Version 1.2.
  \url{https://matriz.org/wp-content/uploads/2016/11/TRIZGlossaryVersion1_2.pdf} 
\bibitem{Steinbuch1966} Karl Steinbuch (1966). The Informed Society. History
  and Future of Telecommunications. (Die informierte Gesellschaft).
  Stuttgart.
\bibitem{Steinbuch1971} Karl Steinbuch (1971). Machine and Man. On the way to
  a cybernetic anthropology. (Automat und Mensch). Springer, Berlin.
\bibitem{TDSGlossary} TDS Ontology Project (2021). A short glossary of basic
  TRIZ concepts and terms. (in Russian)
  \url{https://triz-summit.ru/onto_triz/100/}
\end{thebibliography}

\end{document}
